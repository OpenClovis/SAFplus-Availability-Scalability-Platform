\hypertarget{group__group37}{
\chapter{Functional Overview}
\label{group__group37}
}

\begin{flushleft}
The OpenClovis CheckSum library provides a set of APIs for computing the CheckSum on the input byte stream. The computation can be done for a 16-bit or a 
32-bit number, depending on the need of the application.
\section{Interaction with other components} This component is used by all the components of ASP, as well as the application that needs the CheckSum of
some data.

\chapter{Service Model}
TBD


\chapter{Service APIs}





\subsection{clCksm16bitCompute}
\index{clCksm16bitCompute@{clCksm16bitCompute}}
 \hypertarget{pagecksm101}{}\paragraph{cl\-Cksm16bit\-Compute}\label{pagecksm101}
\begin{Desc}
\item[Synopsis:]Computes a 16-bit CheckSum.\end{Desc}
\begin{Desc}
\item[Header File:]clCksmApi.h\end{Desc}
\begin{Desc}
\item[Syntax:]

\footnotesize\begin{verbatim}   ClRcT clCksm16bitCompute(
              				CL_IN  ClUint8T *pData,
              				CL_IN  ClUint32T length,
              				CL_OUT ClUint16T *pCheckSum);
\end{verbatim}
\normalsize
\end{Desc}
\begin{Desc}
\item[Parameters:]
\begin{description}
\item[{\em p\-Data:}](in) Data for which CheckSum is being computed. This must be a valid pointer and cannot be NULL.
\item[{\em length:}](in) Length of the data for which the CheckSum is to be computed.
\item[{\em p\-Check\-Sum:}](out) Location where the computed CheckSum is stored. This must be a valid pointer and cannot be NULL.\end{description}
\end{Desc}
\begin{Desc}
\item[Return values:]
\begin{description}
\item[{\em CL\_\-OK:}]The function executed successfully. 
\item[{\em CL\_\-ERR\_\-NULL\_\-POINTER:}]{\tt{pData}} or {\tt{pCheckSum}} contains a NULL pointer.\end{description}
\end{Desc}
\begin{Desc}
\item[Description:]This function is used to compute a 16-bit Check\-Sum.\end{Desc}
\begin{Desc}
\item[Library File:]lib\-Cl\-Utils\end{Desc}
\begin{Desc}
\item[Related Function(s):]\hyperlink{pagecksm102}{cl\-Cksm32bit\-Compute} \end{Desc}
\newpage

\subsection{clCksm32bitCompute}
\index{clCksm32bitCompute@{clCksm32bitCompute}}
\hypertarget{pagecksm102}{}\paragraph{cl\-Cksm32bit\-Compute}\label{pagecksm102}
\begin{Desc}
\item[Synopsis:]Computes a 32-bit Check\-Sum.\end{Desc}
\begin{Desc}
\item[Header File:]clCksmApi.h\end{Desc}
\begin{Desc}
\item[Syntax:]

\footnotesize\begin{verbatim}   ClRcT clCksm32bitCompute(
           				CL_IN  ClUint8T *pData,
           				CL_IN  ClUint32T length,
           				CL_OUT ClUint32T *pCheckSum)
\end{verbatim}
\normalsize
\end{Desc}
\begin{Desc}
\item[Parameters:]
\begin{description}
\item[{\em p\-Data:}](in) Data for which Check\-Sum is being computed. This must be a valid pointer and cannot be NULL.
\item[{\em length:}](in) Length of the data for which the checksum is to be computed.
\item[{\em p\-Check\-Sum:}](out) Location where the computed CheckSum is stored. This must be a valid pointer and cannot be NULL.\end{description}
\end{Desc}
\begin{Desc}
\item[Return values:]
\begin{description}
\item[{\em CL\_\-OK:}]The function executed successfully. 
\item[{\em CL\_\-ERR\_\-NULL\_\-POINTER:}]{\tt{pData}} or {\tt{pCheckSum}} contains a NULL pointer.\end{description}
\end{Desc}
\begin{Desc}
\item[Description:]This function is used to compute a 32-bit Check\-Sum.\end{Desc}
\begin{Desc}
\item[Library File:]lib\-Cl\-Utils\end{Desc}
\begin{Desc}
\item[Related Function(s):]\hyperlink{pagecksm101}{cl\-Cksm16bit\-Compute} \end{Desc}


\chapter{Service Management Information Model}
TBD
\chapter{Service Notifications}
TBD
\chapter{Debug CLIs}
TBD
\end{flushleft}