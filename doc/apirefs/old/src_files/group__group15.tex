\hypertarget{group__group15}{
\chapter{Functional Overview}
\label{group__group26}
}
\begin{flushleft}

The OpenClovis Database Abstraction Layer (DBAL) provides a standard interface for any Open\-Clovis ASP infrastructure component or application 
to interface with the commonly used relational database. \par
 \par
 Open\-Clovis ASP R2.3 currently supports the following databases: \par
 \par
 \begin{itemize}
\item GNU Database Manager \item Berkeley Database \item SOLID Database\end{itemize}

DBAL Library can be used by any software component that requires huge data storage. As an abstraction layer,
it interacts with different databases. 
 \par
 The primary user of this interface is the COR component that can dump or read its object repository to and from a database, for either persistent
 storage or offline processing. DBAL is also a standalone library with no dependencies on any other OpenClovis ASP components.
 
\chapter{Service Model}
	TBD

\chapter{Service APIs}



\section{Type Definitions}


\subsection{ClDBFileT}
\index{ClDBFileT@{ClDBFileT}}
\textit{typedef char*  ClDBFileT;}
\newline
\newline
The type of an identifier for the file name.


\subsection{ClDBNameT}
\index{ClDBNameT@{ClDBNameT}}
\textit{typedef char* ClDBNameT;}
\newline
\newline
The type of an indentifier for the name of the database.

\subsection{cDBType\_\-e}
\index{cDBType\_\-e@{cDBType\_\-e}}
\begin{tabbing}
xx\=xx\=xx\=xx\=xx\=xx\=xx\=xx\=xx\=\kill
\textit{typedef enum cDBType\_e\{}\\
\>\>\>\>\textit{CL\_DB\_TYPE\_HASH=0,}\\
\>\>\>\>\textit{CL\_DB\_TYPE\_BTREE,}\\
\>\>\>\>\textit{CL\_DB\_MAX\_TYPE}\\
\textit{\};}\end{tabbing}
The {\tt {cDBType\_\-e}} enumeration contains the definition for the type of the database.
\begin{itemize}
\item
\textit{CL\_\-DB\_\-TYPE\_\-HASH} - Hash table.
\item
\textit{CL\_\-DB\_\-TYPE\_\-BTREE} - B-tree.
\end{itemize}

\subsection{cDBFlag\_\-e}
\index{cDBFlag\_\-e@{cDBFlag\_\-e}}
\begin{tabbing}
xx\=xx\=xx\=xx\=xx\=xx\=xx\=xx\=xx\=\kill
\textit{typedef enum \{}\\
\>\>\>\>\textit{CL\_DB\_CREAT=0,}\\
\>\>\>\>\textit{CL\_DB\_OPEN,}\\
\>\>\>\>\textit{CL\_DB\_APPEND,}\\
\>\>\>\>\textit{CL\_DB\_MAX\_FLAG}\\
\textit{\}cDBFlag\_e;;}\end{tabbing}
The values of the {\tt{cDBFlag\_\-e}} enumeration type contains the definition of the database flag.


\subsection{ClDBFlagT}
\index{ClDBFlagT@{ClDBFlagT}}
\textit{typedef enum cDBFlag\_\-e ClDBFlagT;}
\newline
\newline
The type of the flag specifying the opening type.


\subsection{ClDBHandleT}
\index{ClDBHandleT@{ClDBHandleT}}
\textit{typedef ClHandleT ClDBHandleT;}
\newline
\newline
The type of the handle for the the database.


\subsection{ClDBRecordT}
\index{ClDBRecordT@{ClDBRecordT}}
\textit{typedef ClUint8T* ClDBRecordT;}
\newline
\newline
The type of the handle for a record.


\subsection{ClDBKeyT}
\index{ClDBKeyT@{ClDBKeyT}}
\textit{typedef ClUint8T* ClDBKeyT;}
\newline
\newline
The type of the  handle for the key.

\newpage

\section{Library Life Cycle APIs}
\subsection{clDbalLibInitialize}
\index{clDbalLibInitialize@{clDbalLibInitialize}}
\hypertarget{pagedbal201}{}\paragraph{cl\-Dbal\-Lib\-Initialize}\label{pagedbal201}
\begin{Desc}
\item[Synopsis:]Contains DBAL configuration module entry-point.\end{Desc}
\begin{Desc}
\item[Header File:]clDbalCfg.h\end{Desc}
\begin{Desc}
\item[Syntax:]

\footnotesize\begin{verbatim}     ClRcT clDbalInitialize(void);
\end{verbatim}
\normalsize
\end{Desc}
\begin{Desc}
\item[Parameters:]None.\end{Desc}
\begin{Desc}
\item[Return values:]
\begin{description}
\item[{\em CL\_\-OK:}]The function executed successfully.\end{description}
\end{Desc}
\begin{Desc}
\item[Description:]This function contains the Database Abstraction Layer configuration module entry-point. This will be used to set the configuration 
parameters of the DBAL.\end{Desc}
\begin{Desc}
\item[Library File:]lib\-Cl\-Dbal.a\end{Desc}
\begin{Desc}
\item[Related Function(s):]\hyperlink{pagedbal202}{cl\-Dbal\-Lib\-Finalize} \end{Desc}


\newpage
\subsection{clDbalLibFinalize}
\index{clDbalLibFinalize@{clDbalLibFinalize}}
\hypertarget{pagedbal202}{}\paragraph{cl\-Dbal\-Lib\-Finalize}\label{pagedbal202}
\begin{Desc}
\item[Synopsis:]Contains DBAL configuration module exit-point.\end{Desc}
\begin{Desc}
\item[Header File:]clDbalCfg.h\end{Desc}
\begin{Desc}
\item[Syntax:]

\footnotesize\begin{verbatim}     ClRcT clDbalLibFinalize(void);
\end{verbatim}
\normalsize
\end{Desc}
\begin{Desc}
\item[Parameters:]None.\end{Desc}
\begin{Desc}
\item[Return values:]
\begin{description}
\item[{\em CL\_\-OK:}]The function executed successfully.\end{description}
\end{Desc}
\begin{Desc}
\item[Description:]This function contains the Database Abstraction Layer configuration module exit-point.\end{Desc}
\begin{Desc}
\item[Library File:]lib\-Cl\-Dbal.a\end{Desc}
\begin{Desc}
\item[Related Function(s):]\hyperlink{pagedbal201}{cl\-Dbal\-Lib\-Initialize } \end{Desc}

\newpage


\section{Functional APIs}
\subsection{clDbalOpen}
\index{clDbalOpen@{clDbalOpen}}
\hypertarget{pagedbal103}{}\paragraph{cl\-Dbal\-Open}\label{pagedbal103}
\begin{Desc}
\item[Synopsis:]Opens a database.\end{Desc}
\begin{Desc}
\item[Header File:]clDbalApi.h\end{Desc}
\begin{Desc}
\item[Syntax:]

\footnotesize\begin{verbatim}   ClRcT  clDbalOpen(
                        		CL_IN ClDBFileT dbFile,
                			CL_IN ClDBNameT dbName,
                			CL_IN ClDBFlagT dbFlag,
                		        CL_IN ClUint32T maxKeySize,
                		        CL_IN ClUint32T maxRecordSize,
                		        CL_OUT ClDBHandleT* pDBHandle);
\end{verbatim}
\normalsize
\end{Desc}
\begin{Desc}
\item[Parameters:]
\begin{description}
\item[{\em db\-File:}](in) This parameter specifies the file name and path, where the database is to be created. If this parameter is NULL and the
database is Berkeley Database, then an in-memory database is created. This parameter is ignored, if the database ID is GNU DBM or SOLID.
\item[{\em db\-Name:}](in) Name of the database to open.
\item[{\em db\-Flag:}](in) This flag can accept the following two values: \begin{itemize}
\item If {\tt{CL\_\-DB\_\-CREAT}} is specified and the database already exists, then the existing database is deleted and a new database is created. 
If the database does not exist, then a new one is created. 
\item If {\tt{CL\_\-DB\_\-OPEN}} is specified and the existing database is opened.
\end{itemize}
\item[{\em max\-Key\-Size:}](in) Maximum size of the key (in bytes) to be stored in the database. This parameter is valid only for SOLID database.
\item[{\em max\-Record\-Size:}](in) Maximum size of the record (in bytes) to be stored in the database. This parameter is valid only for SOLID database.
\item[{\em p\-DBHandle:}](out)Pointer to the variable of type {\tt{Cl\-DBHandle\-T}} in which the newly created database handle is returned.
\end{description}
\end{Desc}
\begin{Desc}
\item[Return values:]
\begin{description}
\item[{\em CL\_\-OK:}]The function executed successfully. 
\item[{\em CL\_\-ERR\_\-NULL\_\-POINTER:}]{\tt{pDBHandle}} contains a NULL pointer. 
\item[{\em CL\_\-ERR\_\-INVALID\_\-PARAMETER:}]An invalid parameter has been passed to the function. A parameter is not set correctly.
\item[{\em CL\_\-ERR\_\-NO\_\-MEMORY:}]Memory allocation failure. 
\item[{\em CL\_\-ERR\_\-DB\_\-ERROR:}]Failure in the underlying database.\end{description}
\end{Desc}
\begin{Desc}
\item[Description:]This function opens a database instance on the specified database. Handle returned by this function will be used for further DBAL 
operations.\end{Desc}
\begin{Desc}
\item[Library File:]Cl\-Dbal\end{Desc}
\begin{Desc}
\item[Related Function(s):]\hyperlink{pagedbal104}{cl\-Dbal\-Close} \end{Desc}


\newpage
\subsection{clDbalClose}
\index{clDbalClose@{clDbalClose}}
\hypertarget{pagedbal104}{}\paragraph{cl\-Dbal\-Close}\label{pagedbal104}
\begin{Desc}
\item[Synopsis:]Closes a database.\end{Desc}
\begin{Desc}
\item[Header File:]clDbalApi.h\end{Desc}
\begin{Desc}
\item[Syntax:]

\footnotesize\begin{verbatim}   ClRcT  clDbalClose(
                    			CL_IN ClDBHandleT dbHandle);
\end{verbatim}
\normalsize
\end{Desc}
\begin{Desc}
\item[Parameters:]
\begin{description}
\item[{\em db\-Handle:}](in) Handle of the database being closed.\end{description}
\end{Desc}
\begin{Desc}
\item[Return values:]
\begin{description}
\item[{\em CL\_\-OK:}]The function executed successfully. 
\item[{\em CL\_\-ERR\_\-INVALID\_\-HANDLE:}]{\tt{dbHandle}} is an invalid handle.\end{description}
\end{Desc}
\begin{Desc}
\item[Description:]This function closes a database instance using its handle. In case of Berkeley Database, if the database being closed is an in-memory 
database, then all the data is lost once the database is closed.\end{Desc}
\begin{Desc}
\item[Library File:]Cl\-Dbal\end{Desc}
\begin{Desc}
\item[Related Function(s):]\hyperlink{pagedbal103}{cl\-Dbal\-Open} \end{Desc}


\newpage
\subsection{clDbalRecordInsert}
\index{clDbalRecordInsert@{clDbalRecordInsert}}
\hypertarget{pagedbal105}{}\paragraph{cl\-Dbal\-Record\-Insert}\label{pagedbal105}
\begin{Desc}
\item[Synopsis:]Adds a record into the database.\end{Desc}
\begin{Desc}
\item[Header File:]clDbalApi.h\end{Desc}
\begin{Desc}
\item[Syntax:]

\footnotesize\begin{verbatim}   ClRcT clDbalRecordInsert(
                                	CL_IN  ClDBHandleT  dbHandle,
                    			CL_IN  ClDBKeyT     dbKey,
                    			CL_IN  ClUint32T    keySize,
                  			CL_IN  ClDBRecordT  dbRec,
                  			CL_IN  ClUint32T    recSize);
\end{verbatim}
\normalsize
\end{Desc}
\begin{Desc}
\item[Parameters:]
\begin{description}
\item[{\em db\-Handle:}](in) Handle to the database into which the record is being added. 
\item[{\em db\-Key:}](in) Record's key handle. In case of GDBM, the key must be a string. 
\item[{\em key\-Size:}](in) Size of the key. 
\item[{\em db\-Rec:}](in) Record handle. In case of GDBM, the record must be a string. 
\item[{\em rec\-Size:}](in) Size of the record.\end{description}
\end{Desc}
\begin{Desc}
\item[Return values:]
\begin{description}
\item[{\em CL\_\-OK:}]The function executed successfully. 
\item[{\em CL\_\-ERR\_\-INVALID\_\-HANDLE:}]{\tt{dbHandle}} is an invalid handle. 
\item[{\em CL\_\-ERR\_\-DUPLICATE:}]If an already existing key is added. 
\item[{\em CL\_\-ERR\_\-DB\_\-ERROR:}]Failure in the underlying database.\end{description}
\end{Desc}
\begin{Desc}
\item[Description:]This function adds a record to a database known by its handle. If the specified key already exists in the database, then an error is 
returned.\end{Desc}
\begin{Desc}
\item[Library File:]Cl\-Dbal\end{Desc}
\begin{Desc}
\item[Related Function(s):]\hyperlink{pagedbal103}{cl\-Dbal\-Open},\hyperlink{pagedbal104}{cl\-Dbal\-Close},
\hyperlink{pagedbal107}{cl\-Dbal\-Record\-Get}, \hyperlink{pagedbal106}{cl\-Dbal\-Record\-Replace},
\hyperlink{pagedbal108}{cl\-Dbal\-Record\-Delete} \end{Desc}


\newpage
\subsection{clDbalRecordReplace}
\index{clDbalRecordReplace@{clDbalRecordReplace}}
\hypertarget{pagedbal106}{}\paragraph{cl\-Dbal\-Record\-Replace}\label{pagedbal106}
\begin{Desc}
\item[Synopsis:]Replaces a record in the database.\end{Desc}
\begin{Desc}
\item[Header File:]clDbalApi.h\end{Desc}
\begin{Desc}
\item[Syntax:]

\footnotesize\begin{verbatim}   ClRcT clDbalRecordReplace(
                                	CL_IN ClDBHandleT dbHandle,
                                	CL_IN ClDBKeyT   dbKey,
                                	CL_IN ClUint32T  keySize,
                                	CL_IN ClDBRecordT  dbRec,
                                	CL_IN ClUint32T  recSize);
\end{verbatim}
\normalsize
\end{Desc}
\begin{Desc}
\item[Parameters:]
\begin{description}
\item[{\em db\-Handle:}](in) Handle to the database to which the record is being added. \item[{\em db\-Key:}](in) Record's key handle.
\item[{\em key\-Size:}](in) Size of the key. \item[{\em db\-Rec:}](in) Record handle. \item[{\em rec\-Size:}](in) Size of the record.\end{description}
\end{Desc}
\begin{Desc}
\item[Return values:]
\begin{description}
\item[{\em CL\_\-OK:}]The function executed successfully. 
\item[{\em CL\_\-ERR\_\-INVALID\_\-HANDLE:}]{\tt{dbHandle}} is an invalid handle. 
\item[{\em CL\_\-ERR\_\-DB\_\-ERROR:}]Failure in the underlying database.
\end{description}
\end{Desc}
\begin{Desc}
\item[Description:]This function replaces a record in a database known by its handle. If the specified key already exists in the database, then the 
record associated with it is replaced with the current record. If the key does not exist, then the record is added.\end{Desc}
\begin{Desc}
\item[Library File:]Cl\-Dbal\end{Desc}
\begin{Desc}
\item[Related Function(s):]\hyperlink{pagedbal103}{cl\-Dbal\-Open},\hyperlink{pagedbal104}{cl\-Dbal\-Close},
\hyperlink{pagedbal107}{cl\-Dbal\-Record\-Get}, \hyperlink{pagedbal105}{cl\-Dbal\-Record\-Insert}, 
\hyperlink{pagedbal108}{cl\-Dbal\-Record\-Delete} \end{Desc}


\newpage

\subsection{clDbalRecordGet}
\index{clDbalRecordGet@{clDbalRecordGet}}
\hypertarget{pagedbal107}{}\paragraph{cl\-Dbal\-Record\-Get}\label{pagedbal107}
\begin{Desc}
\item[Synopsis:]Retrieves a record from the database.\end{Desc}
\begin{Desc}
\item[Header File:]clDbalApi.h\end{Desc}
\begin{Desc}
\item[Syntax:]

\footnotesize\begin{verbatim}   ClRcT clDbalRecordGet(
                                	CL_IN ClDBHandleT dbHandle,
                                	CL_IN ClDBKeyT  dbKey,
                                	CL_IN ClUint32T  keySize,
                                	CL_OUT ClDBRecordT* pDBRec,
                                	CL_OUT ClUint32T* pRecSize);
\end{verbatim}
\normalsize
\end{Desc}
\begin{Desc}
\item[Parameters:]
\begin{description}
\item[{\em db\-Handle:}](in) Handle to the database from which the record is being retrieved. 
\item[{\em db\-Key:}](in) Key handle, whose associated 
record is being retrieved. \item[{\em key\-Size:}](in) Size of the key. 
\item[{\em p\-DBRec:}](out) Pointer to the record handle, in which the record is 
being returned. Memory allocation is done by DBAL but the user must free this memory using {\tt{cl\-Dbal\-Record\-Free()}} function. 
\item[{\em p\-Rec\-Size:}](out) Pointer to the variable in which size of the record is being returned.\end{description}
\end{Desc}
\begin{Desc}
\item[Return values:]
\begin{description}
\item[{\em CL\_\-OK:}]The function executed successfully. 
\item[{\em CL\_\-ERR\_\-NULL\_\-POINTER:}]{\tt{pDBRec}} or {\tt{pRecSize}} contains a NULL pointer. 
\item[{\em CL\_\-ERR\_\-INVALID\_\-HANDLE:}]{\tt{dbHandle}} is an invalid handle. 
\item[{\em CL\_\-ERR\_\-DB\_\-ERROR:}]Failure in the underlying database. 
\item[{\em CL\_\-ERR\_\-NOT\_\-EXIST:}]The key does not exist.\end{description}
\end{Desc}
\begin{Desc}
\item[Description:]This function retrieves a record from the database using the database handle and the record's key. If no record is found on the key 
of the specified record, an error is returned.\end{Desc}
\begin{Desc}
\item[Library File:]Cl\-Dbal\end{Desc}
\begin{Desc}
\item[Related Function(s):]\hyperlink{pagedbal103}{cl\-Dbal\-Open}, \hyperlink{pagedbal104}{cl\-Dbal\-Close}, 
\hyperlink{pagedbal106}{cl\-Dbal\-Record\-Replace}, \hyperlink{pagedbal105}{cl\-Dbal\-Record\-Insert}, 
\hyperlink{pagedbal108}{cl\-Dbal\-Record\-Delete} \end{Desc}


\newpage

\subsection{clDbalRecordDelete}
\index{clDbalRecordDelete@{clDbalRecordDelete}}
\hypertarget{pagedbal108}{}\paragraph{cl\-Dbal\-Record\-Delete}\label{pagedbal108}
\begin{Desc}
\item[Synopsis:]Deletes a record from the database.\end{Desc}
\begin{Desc}
\item[Header File:]clDbalApi.h\end{Desc}
\begin{Desc}
\item[Syntax:]

\footnotesize\begin{verbatim}  ClRcT clDbalRecordDelete(
					CL_IN ClDBHandleT dbHandle,
					CL_IN ClDBKeyT dbKey,
					CL_IN ClUint32T keySize);
\end{verbatim}
\normalsize
\end{Desc}
\begin{Desc}
\item[Parameters:]
\begin{description}
\item[{\em db\-Handle:}](in) Handle to the database from which the record is being deleted. \item[{\em db\-Key:}](in) Key handle of the record. 
\item[{\em key\-Size:}](in) Size of the key.\end{description}
\end{Desc}
\begin{Desc}
\item[Return values:]
\begin{description}
\item[{\em CL\_\-OK:}]The function executed successfully. 
\item[{\em CL\_\-ERR\_\-INVALID\_\-HANDLE:}]{\tt{dbHandle}} is an invalid handle. 
\item[{\em CL\_\-ERR\_\-DB\_\-ERROR:}]Failure in the underlying database. 
\item[{\em CL\_\-ERR\_\-NOT\_\-EXIST:}]The key does not exist.\end{description}
\end{Desc}
\begin{Desc}
\item[Description:]This function is used to delete a record from a database using its handle and record's key. If the record is not found, the function 
returns an error.\end{Desc}
\begin{Desc}
\item[Library File:]Cl\-Dbal\end{Desc}
\begin{Desc}
\item[Related Function(s):]\hyperlink{pagedbal103}{cl\-Dbal\-Open},\hyperlink{pagedbal104}{cl\-Dbal\-Close},
\hyperlink{pagedbal106}{cl\-Dbal\-Record\-Replace}, \hyperlink{pagedbal105}{cl\-Dbal\-Record\-Insert}, 
\hyperlink{pagedbal107}{cl\-Dbal\-Record\-Get} \end{Desc}


\newpage
\subsection{clDbalFirstRecordGet}
\index{clDbalFirstRecordGet@{clDbalFirstRecordGet}}
\hypertarget{pagedbal109}{}\paragraph{cl\-Dbal\-First\-Record\-Get}\label{pagedbal109}
\begin{Desc}
\item[Synopsis:]Returns the first key and the associated record from a database.\end{Desc}
\begin{Desc}
\item[Header File:]clDbalApi.h\end{Desc}
\begin{Desc}
\item[Syntax:]

\footnotesize\begin{verbatim}   ClRcT clDbalFirstRecordGet(
                                	CL_IN ClDBHandleT dbHandle,
                                	CL_OUT ClDBKeyT*  pDBKey,
                                	CL_OUT ClUint32T* pKeySize,
                                	CL_OUT ClDBRecordT* pDBRec,
                                	CL_OUT ClUint32T* pRecSize);
\end{verbatim}
\normalsize
\end{Desc}
\begin{Desc}
\item[Parameters:]
\begin{description}
\item[{\em db\-Handle:}](in) Handle to the database from where the record is being retrieved.
\item[{\em p\-DBKey:}](out) Pointer to the key handle, in which 
the first record's key is returned. Memory allocation is done by DBAL but the user must free this memory using {\tt{cl\-Dbal\-Key\-Free()}} function.
\item[{\em p\-Key\-Size:}](out) Pointer to the variable in which size of the key is being returned.
\item[{\em p\-DBRec:}](out) Pointer to the record handle in which the first record is returned. Memory allocation is done by DBAL but the user must free 
this memory using {\tt{cl\-Dbal\-Record\-Free()}} function.
\item[{\em p\-Rec\-Size:}](out) Pointer to the variable in which size of the record is being returned.\end{description}
\end{Desc}
\begin{Desc}
\item[Return values:]
\begin{description}
\item[{\em CL\_\-OK:}]The function executed successfully. 
\item[{\em CL\_\-ERR\_\-NULL\_\-POINTER:}]{\tt{pDBKey}}, {\tt{pKeySize}}, {\tt{pDBRec}}, or {\tt{pRecSize}} contains a NULL pointer. 
\item[{\em CL\_\-ERR\_\-INVALID\_\-HANDLE:}]{\tt{dbHandle}} is an invalid handle. 
\item[{\em CL\_\-ERR\_\-DB\_\-ERROR:}]Failure in the underlying database. 
\item[{\em CL\_\-ERR\_\-NOT\_\-EXIST:}]The first record does not exist.\end{description}
\end{Desc}
\begin{Desc}
\item[Description:]This function is used to return the first key and the associated record from a database. If no records are found in the specified 
database, an error is returned.\end{Desc}
\begin{Desc}
\item[Library File:]Cl\-Dbal\end{Desc}
\begin{Desc}
\item[Related Function(s):]\hyperlink{pagedbal103}{cl\-Dbal\-Open}, \hyperlink{pagedbal104}{cl\-Dbal\-Close}, 
\hyperlink{pagedbal106}{cl\-Dbal\-Record\-Replace}, \hyperlink{pagedbal105}{cl\-Dbal\-Record\-Insert},
\hyperlink{pagedbal107}{cl\-Dbal\-Record\-Get}, \hyperlink{pagedbal108}{cl\-Dbal\-Record\-Delete}, 
\hyperlink{pagedbal110}{cl\-Dbal\-Next\-Record\-Get} \end{Desc}


\newpage

\subsection{clDbalNextRecordGet}
\index{clDbalNextRecordGet@{clDbalNextRecordGet}}
\hypertarget{pagedbal110}{}\paragraph{cl\-Dbal\-Next\-Record\-Get}\label{pagedbal110}
\begin{Desc}
\item[Synopsis:]Returns the next key and the associated record from a database.\end{Desc}
\begin{Desc}
\item[Header File:]clDbalApi.h\end{Desc}
\begin{Desc}
\item[Syntax:]

\footnotesize\begin{verbatim}   ClRcT clDbalNextRecordGet(
					CL_IN ClDBHandleT dbHandle,
                                        CL_IN ClDBKeyT currentKey,
                                        CL_IN ClUint32T currentKeySize,
                            		CL_OUT ClDBKeyT* pDBNextKey,
                            		CL_OUT ClUint32T* pNextKeySize,
                            		CL_OUT ClDBRecordT* pDBNextRec,
                                        CL_OUT ClUint32T* pNextRecSize);
\end{verbatim}
\normalsize
\end{Desc}
\begin{Desc}
\item[Parameters:]
\begin{description}
\item[{\em db\-Handle:}](in) Handle to the database from which the next record is being retrieved. 
\item[{\em current\-Key:}](in) Key handle of the current record. 
\item[{\em current\-Key\-Size:}](in) Size of the current key.
\item[{\em p\-DBNext\-Key:}](out) Pointer to the key handle, in which the next record's key is being returned. Memory allocation is done by DBAL but the 
user must free this memory using {\tt{cl\-Dbal\-Key\-Free()}} function.
\item[{\em p\-Next\-Key\-Size:}](out) Pointer to the variable in which size of the key is being returned.
\item[{\em p\-DBNext\-Rec:}](out) Pointer to the record handle, in which handle of the next
record is returned. Memory allocation is done by DBAL but the user must free this memory using {\tt{cl\-Dbal\-Record\-Free()}} function.
\item[{\em p\-Next\-Rec\-Size:}](out) Pointer to the variable in which size of the record is being returned.\end{description}
\end{Desc}
\begin{Desc}
\item[Return values:]
\begin{description}
\item[{\em CL\_\-OK:}]The function executed successfully. 
\item[{\em CL\_\-ERR\_\-NULL\_\-POINTER:}]{\tt{pDBNextRec}}, {\tt{pNextRecSize}}, {\tt{pNextKeySize}}, or {\tt{pDBNextKey}} contains a NULL pointer. 
\item[{\em CL\_\-ERR\_\-INVALID\_\-HANDLE:}]dbHandle contains an invalid handle. 
\item[{\em CL\_\-ERR\_\-DB\_\-ERROR:}]Failure in the underlying database. 
\item[{\em CL\_\-ERR\_\-NOT\_\-EXIST:}]The key does not exist.\end{description}
\end{Desc}
\begin{Desc}
\item[Description:]This function is used to retrieve the next key and the associated record from a database using the current record's key. If the record 
is not available, an error is returned.\end{Desc}
\begin{Desc}
\item[Library File:]Cl\-Dbal\end{Desc}
\begin{Desc}
\item[Related Function(s):]\hyperlink{pagedbal103}{cl\-Dbal\-Open}, \hyperlink{pagedbal104}{cl\-Dbal\-Close}, 
\hyperlink{pagedbal106}{cl\-Dbal\-Record\-Replace}, \hyperlink{pagedbal105}{cl\-Dbal\-Record\-Insert}, 
\hyperlink{pagedbal107}{cl\-Dbal\-Record\-Get}, \hyperlink{pagedbal108}{cl\-Dbal\-Record\-Delete}, \end{Desc}


\newpage

\subsection{clDbalTransactionBegin}
\index{clDbalTransactionBegin@{clDbalTransactionBegin}}
\hypertarget{pagedbal111}{}\paragraph{cl\-Dbal\-Transaction\-Begin}\label{pagedbal111}
\begin{Desc}
\item[Synopsis:]Begins a transaction.\end{Desc}
\begin{Desc}
\item[Header File:]clDbalApi.h\end{Desc}
\begin{Desc}
\item[Syntax:]

\footnotesize\begin{verbatim}   ClRcT clDbalTransactionBegin(
                          		CL_IN ClDBHandleT  dbHandle);
\end{verbatim}
\normalsize
\end{Desc}
\begin{Desc}
\item[Parameters:]
\begin{description}
\item[{\em db\-Handle:}](in) Handle to the database in which the transaction is to be started.\end{description}
\end{Desc}
\begin{Desc}
\item[Return values:]
\begin{description}
\item[{\em CL\_\-OK:}]The function executed successfully. 
\item[{\em CL\_\-ERR\_\-INVALID\_\-HANDLE:}]{\tt{dbHandle}} is an invalid handle. 
\item[{\em CL\_\-ERR\_\-NOT\_\-SUPPORTED:}]The underlying database does not support transactions.\end{description}
\end{Desc}
\begin{Desc}
\item[Description:]This function marks the beginning of a transaction. This function must be called before calling further DBAL transaction related
functions. This function returns {\tt{transaction\-Id}}, which will be passed for further operations as part of the {\tt{db\-Handle}}.\end{Desc}
\begin{Desc}
\item[Library File:]Cl\-Dbal\end{Desc}
\begin{Desc}
\item[Note:]Transactions may only span threads, if they do so serially. That is, each transaction must be active in only a single thread of control at a
given instant.\end{Desc}
\begin{Desc}
\item[Related Function(s):]\hyperlink{pagedbal112}{cl\-Dbal\-Transaction\-Commit}, 
\hyperlink{pagedbal113}{cl\-Dbal\-Transaction\-Abort} \end{Desc}


\newpage
\subsection{clDbalTransactionCommit}
\index{clDbalTransactionCommit@{clDbalTransactionCommit}}
\hypertarget{pagedbal112}{}\paragraph{cl\-Dbal\-Transaction\-Commit}\label{pagedbal112}
\begin{Desc}
\item[Synopsis:]Commits a transaction.\end{Desc}
\begin{Desc}
\item[Header File:]clDbalApi.h\end{Desc}
\begin{Desc}
\item[Syntax:]

\footnotesize\begin{verbatim}   ClRcT clDbalTransactionCommit(
                          		CL_IN ClDBHandleT  dbHandle);
\end{verbatim}
\normalsize
\end{Desc}
\begin{Desc}
\item[Parameters:]
\begin{description}
\item[{\em db\-Handle:}](in) Handle to the database in which the transaction is being commited.\end{description}
\end{Desc}
\begin{Desc}
\item[Return values:]
\begin{description}
\item[{\em CL\_\-OK:}]The function executed successfully. 
\item[{\em CL\_\-ERR\_\-NULL\_\-POINTER:}]On passing a NULL pointer. 
\item[{\em CL\_\-ERR\_\-INVALID\_\-HANDLE:}]{\tt{dbHandle}} is an invalid handle. 
\item[{\em CL\_\-ERR\_\-COMMIT\_\-FAILED:}]Commit of underlying database failed. 
\item[{\em CL\_\-ERR\_\-NOT\_\-SUPPORTED:}]The underlying database does not support transactions.\end{description}
\end{Desc}
\begin{Desc}
\item[Description:]This function commits the recently started transaction. All operations performed after calling {\tt{cl\-Dbal\-Transaction\-Begin}}, 
will be committed in the database.\end{Desc}
\begin{Desc}
\item[Library File:]Cl\-Dbal\end{Desc}
\begin{Desc}
\item[Related Function(s):]\hyperlink{pagedbal111}{cl\-Dbal\-Transaction\-Begin}, \hyperlink{pagedbal113}{cl\-Dbal\-Transaction\-Abort} \end{Desc}


\newpage
\subsection{clDbalTransactionAbort}
\index{clDbalTransactionAbort@{clDbalTransactionAbort}}
\hypertarget{pagedbal113}{}\paragraph{cl\-Dbal\-Transaction\-Abort}\label{pagedbal113}
\begin{Desc}
\item[Synopsis:]Aborts a transaction.\end{Desc}
\begin{Desc}
\item[Header File:]clDbalApi.h\end{Desc}
\begin{Desc}
\item[Syntax:]

\footnotesize\begin{verbatim}   ClRcT clDbalTransactionAbort(
                              		ClDBHandleT  dbHandle);
\end{verbatim}
\normalsize
\end{Desc}
\begin{Desc}
\item[Parameters:]
\begin{description}
\item[{\em db\-Handle:}](in) Handle to the database in which the transaction is being aborted.\end{description}
\end{Desc}
\begin{Desc}
\item[Return values:]
\begin{description}
\item[{\em CL\_\-OK:}]The function executed successfully. 
\item[{\em CL\_\-ERR\_\-INVALID\_\-HANDLE:}]{\tt{dbHandle}} is an invalid handle. 
\item[{\em CL\_\-ERR\_\-ABORT\_\-FAILED:}]Failure in the abort of the underlying database.
\item[{\em CL\_\-ERR\_\-NOT\_\-SUPPORTED:}]The underlying database does not support transactions.\end{description}
\end{Desc}
\begin{Desc}
\item[Description:]This function is used to abort the recently started transaction. This removes the transaction ID specified in the handle.\end{Desc}
\begin{Desc}
\item[Library File:]Cl\-Dbal\end{Desc}
\begin{Desc}
\item[Related Function(s):]\hyperlink{pagedbal111}{cl\-Dbal\-Transaction\-Begin}, \hyperlink{pagedbal112}{cl\-Dbal\-Transaction\-Commit} \end{Desc}


\newpage
\subsection{clDbalRecordFree}
\index{clDbalRecordFree@{clDbalRecordFree}}
\hypertarget{pagedbal114}{}\paragraph{cl\-Dbal\-Record\-Free}\label{pagedbal114}
\begin{Desc}
\item[Synopsis:]Frees the database record.\end{Desc}
\begin{Desc}
\item[Header File:]clDbalApi.h\end{Desc}
\begin{Desc}
\item[Syntax:]

\footnotesize\begin{verbatim}   ClRcT clDbalRecordFree(
                          		CL_IN ClDBHandleT dbHandle
                          		CL_IN ClDBRecordT dbRec);
\end{verbatim}
\normalsize
\end{Desc}
\begin{Desc}
\item[Parameters:]
\begin{description}
\item[{\em db\-Handle:}](in) Handle to the database. \item[{\em db\-Rec:}](in) Record to be freed.\end{description}
\end{Desc}
\begin{Desc}
\item[Return values:]
\begin{description}
\item[{\em CL\_\-OK:}]The function executed successfully. 
\item[{\em CL\_\-ERR\_\-INVALID\_\-HANDLE:}]{\tt{dbHandle}} is an invalid handle.\end{description}
\end{Desc}
\begin{Desc}
\item[Description:]This function is used to free the record returned by {\tt{clDbalFirstRecordGet()}} and {\tt{clDbalNextRecordGet()}} APIs. The 
record is created by {\tt{clDbalRecordInsert()}} and cannot be freed by {\tt{cl\-Heap\-Free()}}. So to clean the records, this 
function must be called.\end{Desc}
\begin{Desc}
\item[Library File:]Cl\-Dbal\end{Desc}
\begin{Desc}
\item[Related Function(s):]\hyperlink{pagedbal107}{cl\-Dbal\-Record\-Get}, \hyperlink{pagedbal109}{cl\-Dbal\-First\-Record\-Get}, 
\hyperlink{pagedbal110}{cl\-Dbal\-Next\-Record\-Get}, \hyperlink{pagedbal115}{cl\-Dbal\-Key\-Free} \end{Desc}


\newpage
\subsection{clDbalKeyFree}
\index{clDbalKeyFree@{clDbalKeyFree}}
\hypertarget{pagedbal115}{}\paragraph{cl\-Dbal\-Key\-Free}\label{pagedbal115}
\begin{Desc}
\item[Synopsis:]Frees the database key.\end{Desc}
\begin{Desc}
\item[Header File:]clDbalApi.h\end{Desc}
\begin{Desc}
\item[Syntax:]

\footnotesize\begin{verbatim}   ClRcT clDbalKeyFree(
                          		CL_IN ClDBHandleT dbHandle
                            		CL_IN ClDBKeyT dbKey);
\end{verbatim}
\normalsize
\end{Desc}
\begin{Desc}
\item[Parameters:]
\begin{description}
\item[{\em db\-Handle:}](in) Handle to the database. \item[{\em db\-Key:}](in) Key to be freed.\end{description}
\end{Desc}
\begin{Desc}
\item[Return values:]
\begin{description}
\item[{\em CL\_\-OK:}]The function executed successfully. 
\item[{\em CL\_\-ERR\_\-INVALID\_\-HANDLE:}]{\tt{dbHandle}} is an invalid handle.\end{description}
\end{Desc}
\begin{Desc}
\item[Description:]This function is used to free the Key returned by {\tt{cl\-Dbal\-First\-Record\-Get()}} and {\tt{cl\-Dbal\-Next\-Record\-Get}}. The 
key is created by {\tt{clDbalRecordInsert()}} cannot be freed by {\tt{cl\-Heap\-Free()}}. So, this function needs to be called to free the keys, .\end{Desc}
\begin{Desc}
\item[Library File:]Cl\-Dbal\end{Desc}
\begin{Desc}
\item[Related Function(s):]\hyperlink{pagedbal109}{cl\-Dbal\-First\-Record\-Get}, \hyperlink{pagedbal110}{cl\-Dbal\-Next\-Record\-Get} \end{Desc}


\chapter{Service Management Information Model}
TBD
\chapter{Service Notifications}
TBD
\chapter{Debug CLIs}
TBD
\end{flushleft}