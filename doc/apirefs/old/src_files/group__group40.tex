\hypertarget{group__group40}{
\chapter{Functional Overview}
\label{group__group40}
}

\begin{flushleft}

The OpenClovis Queue Library provides implementation for an ordered list. It supports enqueuing, dequeuing, retrieval of a node, and 
walk through the queue. \par
 \par
 FIFO, another name for a queue, is an acronym for the way it operates, First-In-First-Out. The standard interface to the FIFO queue consists of 
 {\tt {ENQUEUE()}} function for adding new elements, and a DEQUEUE() function for removing the oldest element. Enqueue adds an element at the rear (end) of 
 the queue. Dequeue removes an element from the front (start) of the queue. The following operations are supported on queues: \par

\begin{itemize}
\item Enqueues an element into the queue. \item Dequeues an element from the queue. \item Walks through the queue. \item Returns the number of elements 
in the queue. \item Deletes the Queue. \par
 \par
 Before peforming any of the mentioned operations, you must create a queue. When creating a queue, you need to specify the maximum size for the queue. If 
 the maximum size is specified as 0, then you can enqueue any number of elements. Otherwise, the number of elements you can enqueue is limited to the 
 maximum size. That is, at any instant, the queue can have a maximum of {\tt{max\-Size}} number of elements, specified when the queue is created. \par
\end{itemize}

\section{Interaction with other components} Queue APIs depend on Heap for memory allocation and functions to free the memory. 

\chapter{Service Model}
TBD


\chapter{Service APIs}


\section{Type Definitions}
\subsection{ClQueueDequeueCallbackT}
\index{ClQueueDequeueCallbackT@{ClQueueDequeueCallbackT}}
\textit{typedef void       (*ClQueueDequeueCallbackT)(}
\newline \textit{ClQueueDataT userData);}
\newline
\newline
The type of the callback function to dequeue user-data.


\subsection{ClQueueT}
\index{ClQueueT@{ClQueueT}}
\textit{typedef ClHandleT ClQueueT;}
\newline
\newline
 The type of the handle for the queue.

\subsection{ClQueueDataT}
\index{ClQueueDataT@{ClQueueDataT}}
\textit{typedef ClHandleT ClQueueDataT;}
\newline
\newline
 The type of the handle for the user-data.


\subsection{ClQueueWalkCallbackT}
\index{ClQueueWalkCallbackT@{ClQueueWalkCallbackT}}
\textit{typedef void       (*ClQueueWalkCallbackT)(}
\newline \textit{ClQueueDataT userData,}
\newline \textit{void* userArg);}
\newline
\newline
 The type of the callback function for walking through the queue.

\newpage

\section{Functional APIs}
\subsection{clQueueCreate}
\index{clQueueCreate@{clQueueCreate}}
\hypertarget{pageq101}{}\paragraph{cl\-Queue\-Create}\label{pageq101}
\begin{Desc}
\item[Synopsis:]Creates a queue.\end{Desc}
\begin{Desc}
\item[Header File:]clQueueApi.h\end{Desc}
\begin{Desc}
\item[Syntax:]

\footnotesize\begin{verbatim}   ClRcT clQueueCreate(
                           		CL_IN  ClUint32T maxSize,
                           		CL_IN  ClQueueDequeueCallbackT fpUserDequeueCallBack,
                           		CL_IN  ClQueueDequeueCallbackT fpUserDestroyCallBack,
                           		CL_OUT  ClQueueT* pQueueHead);
\end{verbatim}
\normalsize
\end{Desc}
\begin{Desc}
\item[Parameters:]
\begin{description}
\item[{\em max\-Size:}](in) Maximum size of the Queue. It specifies the maximum number of elements that can exist at any instant in the Queue. 
This must be an unsigned integer. You can enqueue elements into the queue until this maximum limit is reached. If you specify this parameter as 0, then 
there is no limit on the size of the Queue.
\item[{\em fp\-User\-Dequeue\-Call\-Back:}](in) Pointer to the dequeue callback function. This function
accepts a parameter of type {\tt{Cl\-Queue\-Data\-T}}. The user-data of the removed element from the queue is passed as an argument to the callback 
function. 
\item[{\em fp\-User\-Destroy\-Call\-Back:}](in) Pointer to the destroy callback function. This function accepts a parameter of type 
{\tt{Cl\-Queue\-Data\-T}}. This function is called for every element in the queue.
\item[{\em p\-Queue\-Head:}](out) Pointer to the variable of type {\tt{Cl\-Queue\-T}} in which the function returns a valid Queue handle.\end{description}
\end{Desc}
\begin{Desc}
\item[Return values:]
\begin{description}
\item[{\em CL\_\-OK:}]The function executed successfully. \item[{\em CL\_\-ERR\_\-NO\_\-MEMORY:}]Memory allocation failure. 
\item[{\em CL\_\-ERR\_\-NULL\_\-POINTER:}]{\tt{pQueueHead}} contains a NULL pointer.\end{description}
\end{Desc}
\begin{Desc}
\item[Note:]Returned error is a combination of the component ID and error code. Use {\tt{CL\_\-GET\_\-ERROR\_\-CODE(RC)}}
defined in the {\tt{clCommonErrors.h}} file to get the error code.\end{Desc}
\begin{Desc}
\item[Description:]This function is used to create and initialize a queue.\end{Desc}
\begin{Desc}
\item[Library File:]lib\-Cl\-Utils\end{Desc}
\begin{Desc}
\item[Related Function(s):]\hyperlink{pageq106}{cl\-Queue\-Delete} \end{Desc}
\newpage


\subsection{clQueueNodeInsert}
\index{clQueueNodeInsert@{clQueueNodeInsert}}
\hypertarget{pageq102}{}\paragraph{cl\-Queue\-Node\-Insert}\label{pageq102}
\begin{Desc}
\item[Synopsis:]Enqueues an element (user-data) into the Queue.\end{Desc}
\begin{Desc}
\item[Header File:]clQueueApi.h\end{Desc}
\begin{Desc}
\item[Syntax:]

\footnotesize\begin{verbatim}   ClRcT clQueueNodeInsert(
                       			CL_IN  ClQueueT queueHead,
                       			CL_IN  ClQueueDataT userData);
\end{verbatim}
\normalsize
\end{Desc}
\begin{Desc}
\item[Parameters:]
\begin{description}
\item[{\em queue\-Head:}](in) Handle of the queue returned by {\tt{cl\-Queue\-Create()}} function. \item[{\em user\-Data:}](in) User-data. You must 
allocate 
and de-allocate memory for the user-data.\end{description}
\end{Desc}
\begin{Desc}
\item[Return values:]
\begin{description}
\item[{\em CL\_\-OK:}]The function executed successfully. 
\item[{\em CL\_\-ERR\_\-NO\_\-MEMORY:}]Memory allocation failure. 
\item[{\em CL\_\-ERR\_\-MAXSIZE\_\-REACHED:}]The maximum size is reached. 
\item[{\em CL\_\-ERR\_\-INVALID\_\-HANDLE:}]An invalid handle has been passed to the function.\end{description}
\end{Desc}
\begin{Desc}
\item[Note:]Returned error is a combination of the component ID and error code. Use 
{\tt {CL\_\-GET\_\-ERROR\_\-CODE(RET\_\-CODE)}} defined in the {\tt{clCommonErrors.h}} file to get the error code.\end{Desc}
\begin{Desc}
\item[Description:]This function is used to enqueue an element (user-data) into the queue.\end{Desc}
\begin{Desc}
\item[Library File:]lib\-Cl\-Utils\end{Desc}
\begin{Desc}
\item[Related Function(s):]\hyperlink{pageq103}{cl\-Queue\-Node\-Delete} \end{Desc}
\newpage


\subsection{clQueueNodeDelete}
\index{clQueueNodeDelete@{clQueueNodeDelete}}
\hypertarget{pageq103}{}\paragraph{cl\-Queue\-Node\-Delete}\label{pageq103}
\begin{Desc}
\item[Synopsis:]Removes an element from the queue.\end{Desc}
\begin{Desc}
\item[Header File:]clQueueApi.h\end{Desc}
\begin{Desc}
\item[Syntax:]

\footnotesize\begin{verbatim}   ClRcT clQueueNodeDelete(
                       			CL_IN  ClQueueT queueHead,
                       			CL_OUT  ClQueueDataT* userData);
\end{verbatim}
\normalsize
\end{Desc}
\begin{Desc}
\item[Parameters:]
\begin{description}
\item[{\em queue\-Head:}](in) Handle of the queue returned by {\tt{cl\-Queue\-Create()}} function. \item[{\em user\-Data:}](out) Handle of the user-data. 
The user-data of the removed node is returned.\end{description}
\end{Desc}
\begin{Desc}
\item[Return values:]
\begin{description}
\item[{\em CL\_\-OK:}]The function executed successfully. 
\item[{\em CL\_\-ERR\_\-INVALID\_\-HANDLE:}]An invalid handle has been passed to the function. 
\item[{\em CL\_\-ERR\_\-NULL\_\-POINTER:}]{\tt{userData}} contains a NULL pointer.
\item[{\em CL\_\-ERR\_\-NOT\_\-EXIST:}]The queue is empty.\end{description}
\end{Desc}
\begin{Desc}
\item[Note:]Returned error is a combination of the component ID and error code. Use
{\tt {CL\_\-GET\_\-ERROR\_\-CODE(RET\_\-CODE)}} defined in the {\tt{clCommonErrors.h}} file to get the error code.\end{Desc}
\begin{Desc}
\item[Description:]This function is used to remove an element from the front of the queue. The user delete callback, registered during creation, is 
called with the removed element (user-data).\end{Desc}
\begin{Desc}
\item[Library File:]lib\-Cl\-Utils\end{Desc}
\begin{Desc}
\item[Related Function(s):]\hyperlink{pageq102}{cl\-Queue\-Node\-Insert} \end{Desc}
\newpage

\subsection{clQueueWalk}
\index{clQueueWalk@{clQueueWalk}}
\hypertarget{pageq104}{}\paragraph{cl\-Queue\-Walk}\label{pageq104}
\begin{Desc}
\item[Synopsis:]Walks through the queue.\end{Desc}
\begin{Desc}
\item[Header File:]clQueueApi.h\end{Desc}
\begin{Desc}
\item[Syntax:]

\footnotesize\begin{verbatim}   ClRcT clQueueWalk(
                       			CL_IN  ClQueueT queueHead,
                       			CL_IN  ClQueueWalkCallbackT fpUserWalkFunction,
                       			CL_IN  void* userArg);
\end{verbatim}
\normalsize
\end{Desc}
\begin{Desc}
\item[Parameters:]
\begin{description}
\item[{\em queue\-Head:}](in) Handle of the queue returned by {\tt{cl\-Queue\-Create}} function.
\item[{\em fp\-User\-Walk\-Function:}](in) Pointer to the callback function. It accepts the following two parameters: \begin{itemize}
\item Cl\-Queue\-Data\-T \item void $\ast$ - Each element in the queue is passed as the first argument to the callback function.\end{itemize}
\item[{\em user\-Arg:}](in) User-specified argument. This variable is passed as the second argument to the user's callback function.\end{description}
\end{Desc}
\begin{Desc}
\item[Return values:]
\begin{description}
\item[{\em CL\_\-OK:}]The function executed successfully. 
\item[{\em CL\_\-ERR\_\-NULL\_\-POINTER:}]{\tt{userArg}} contains a NULL pointer. 
\item[{\em CL\_\-ERR\_\-INVALID\_\-HANDLE:}]An invalid handle has been passed to the function.\end{description}
\end{Desc}
\begin{Desc}
\item[Note:]Returned error is a combination of the component ID and error code. Use 
{\tt {CL\_\-GET\_\-ERROR\_\-CODE(RET\_\-CODE)}} defined in the {\tt{clCommonErrors.h}} file to get the error code.\end{Desc}
\begin{Desc}
\item[Description:]This function is used to perform a walk on the queue. The user-specified callback function is called with every element (user-data) in the 
queue.\end{Desc}
\begin{Desc}
\item[Library File:]lib\-Cl\-Utils\end{Desc}
\begin{Desc}
\item[Related Function(s):]None. \end{Desc}
\newpage


\subsection{clQueueSizeGet}
\index{clQueueSizeGet@{clQueueSizeGet}}
\hypertarget{pageq105}{}\paragraph{cl\-Queue\-Size\-Get}\label{pageq105}
\begin{Desc}
\item[Synopsis:]Retrieves the number of data elements in the queue.\end{Desc}
\begin{Desc}
\item[Header File:]clQueueApi.h\end{Desc}
\begin{Desc}
\item[Syntax:]

\footnotesize\begin{verbatim}   ClRcT clQueueSizeGet(
                       			CL_IN  ClQueueT queueHead,
                       			CL_OUT  ClUint32T* pSize);
\end{verbatim}
\normalsize
\end{Desc}
\begin{Desc}
\item[Parameters:]
\begin{description}
\item[{\em queue\-Head:}](in) Handle of the queue returned by the {\tt{cl\-Queue\-Create()}} function. 
\item[{\em p\-Size:}](out) Pointer to variable of type {\tt{Cl\-Uint32T}}, in which the size of the queue is returned.\end{description}
\end{Desc}
\begin{Desc}
\item[Return values:]
\begin{description}
\item[{\em CL\_\-OK:}]The function executed successfully. 
\item[{\em CL\_\-ERR\_\-NULL\_\-POINTER:}]{\tt{pSize}} contains a NULL pointer. 
\item[{\em CL\_\-ERR\_\-INVALID\_\-HANDLE:}]An invalid handle has been passed to the function.
\end{description}
\end{Desc}
\begin{Desc}
\item[Note:]Returned error is a combination of the component ID and error code. Use {\tt {CL\_\-GET\_\-ERROR\_\-CODE(RET\_\-CODE)}} defined in the
{\tt{clCommonErrors.h}} file to get the error code.\end{Desc}
\begin{Desc}
\item[Description:]This function is used to retrieve the number of data elements in the queue.\end{Desc}
\begin{Desc}
\item[Library File:]lib\-Cl\-Utils\end{Desc}
\begin{Desc}
\item[Related Function(s):]None. \end{Desc}
\newpage


\subsection{clQueueDelete}
\index{clQueueDelete@{clQueueDelete}}
\hypertarget{pageq106}{}\paragraph{cl\-Queue\-Delete}\label{pageq106}
\begin{Desc}
\item[Synopsis:]Deletes the queue.\end{Desc}
\begin{Desc}
\item[Header File:]clQueueApi.h\end{Desc}
\begin{Desc}
\item[Syntax:]

\footnotesize\begin{verbatim}   ClRcT clQueueDelete(
                       			CL_IN  ClQueueT* pQueueHead);
\end{verbatim}
\normalsize
\end{Desc}
\begin{Desc}
\item[Parameters:]
\begin{description}
\item[{\em p\-Queue\-Head:}](in) Pointer to the queue handle returned by {\tt{cl\-Queue\-Create}} function.\end{description}
\end{Desc}
\begin{Desc}
\item[Return values:]
\begin{description}
\item[{\em CL\_\-OK:}]The function executed successfully. 
\item[{\em CL\_\-ERR\_\-NULL\_\-POINTER:}]{\tt{pQueueHead}} contains a NULL pointer. 
\end{description}
\end{Desc}
\begin{Desc}
\item[Note:]Returned error is a combination of the component ID and error code. Use {\tt {CL\_\-GET\_\-ERROR\_\-CODE(RET\_\-CODE)}} defined in the
{\tt{clCommonErrors.h}} file to get the error code.\end{Desc}
\begin{Desc}
\item[Description:]This function is used to delete all the elements in the queue. The destroy callback function, registered during creation is called for 
every element in the queue.\end{Desc}
\begin{Desc}
\item[Library File:]lib\-Cl\-Utils\end{Desc}
\begin{Desc}
\item[Related Function(s):]\hyperlink{pageq101}{cl\-Queue\-Create} \end{Desc}

\chapter{Service Management Information Model}
TBD
\chapter{Service Notifications}
TBD
\chapter{Debug CLIs}
TBD
\end{flushleft}