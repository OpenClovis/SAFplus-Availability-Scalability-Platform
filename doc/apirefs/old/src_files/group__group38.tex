\hypertarget{group__group38}{
\chapter{Functional Overview}
\label{group__group38}
}
\begin{flushleft}

A Circular list is a chain of nodes, without any termination. The last node is linked back to the first node. A node is a data structure which contains 
user data. The following are the operations supported by Circular List implementation: 
 \par
 \begin{itemize}
\item Adds a node at the beginning of the list. \item Adds a node at the end of the list. \item Adds a node before a specific node in the list. \item 
Adds a node after a specific node in the list. \item Returns the first node from the list. \item Returns the last node from the list. \item Returns the
previous node from the list. \item Returns the next node from the list. \item Walks through the list, starting from a specific node. \item Deletes a node 
from the list. \item Returns the number of nodes from the list. \item Retrieves the data from a node from the list. \item Destroys the list. 
 \par
\end{itemize}
Before performing any of the above mentioned operations, you must create a list. While creating a list, you need to specify the maximum size for the list. If the 
maximum size is specified as 0, then you can add any number of nodes. Otherwise, the number of nodes you can add is limited to the maximum size. At any 
instant, the list can have a maximum of {\tt{max\-Size}} number of nodes, specified when the list is created. 
 \par
 This file contains the definitions and API prototypes for Circular Linked List.
\section{Interaction with other components} Circular List APIs depend on Heap for memory allocation and functions to free the allocated memory. 

\chapter{Service Model}
TBD


\chapter{Service APIs}


\section{Type Definitions}

\subsection{ClClistDropPolicyT}
\index{ClClistDropPolicyT@{ClClistDropPolicyT}}
\begin{tabbing}
xx\=xx\=xx\=xx\=xx\=xx\=xx\=xx\=xx\=\kill
\textit{typedef enum \{}\\
\>\>\>\>\textit{CL\_NO\_DROP,}\\
\>\>\>\>\textit{CL\_DROP\_FIRST,}\\
\>\>\>\>\textit{CL\_DROP\_MAX\_TYPE}\\
\textit{\} ClClistDropPolicyT;}\end{tabbing}
The values of the \textit{ClClistDropPolicyT} enumeration contains/decides the action to be taken when the linked list is full.
The values of this enumeration have the following interpretation:
\begin{itemize}
\item
\textit{CL\_\-NO\_\-DROP} - The first node must not be dropped even if the list is full.
\item
\textit{CL\_\-DROP\_\-FIRST} - Drops the first node, if the list is full.
\item
\textit{CL\_\-DROP\_\-MAX\_\-TYPE} - Adds a new drop policy type before the first node.
\end{itemize}


\subsection{ClClistDeleteCallbackT}
\index{ClClistDeleteCallbackT@{ClClistDeleteCallbackT}}
\textit{typedef void(*ClClistDeleteCallbackT)(}
\newline \textit{ClClistDataT userData);}
\newline
\newline
The type of the callback function that is invoked when a node is deleted from the container. It deletes the callback function pointer.


\subsection{ClClistT}
\index{ClClistT@{ClClistT}}
\textit{typedef ClHandleT ClClistT;}
\newline
\newline
The type of the handle for the circular list.

\subsection{ClClistDataT}
\index{ClClistDataT@{ClClistDataT}}
\textit{typedef ClHandleT ClClistDataT;}
\newline
\newline
The type of the handle for the user-data.


\subsection{ClClistNodeT}
\index{ClClistNodeT@{ClClistNodeT}}
\textit{typedef ClHandleT ClClistNodeT;}
\newline
\newline
The type of the handle for the circular node.



\subsection{ClClistWalkCallbackT}
\index{ClClistWalkCallbackT@{ClClistWalkCallbackT}}
\textit{typedef void(*ClClistWalkCallbackT)(}
\newline \textit{ClClistDataT userData, }
\newline \textit{void *userArg);}
\newline
\newline
The type of the callback function to be used when iterating/walking through the elements of a container. 


\newpage

\section{Functional APIs}
\subsection{clClistCreate}
\index{clClistCreate@{clClistCreate}}
\hypertarget{pagecl101}{}\paragraph{cl\-Clist\-Create}\label{pagecl101}
\begin{Desc}
\item[Synopsis:]Creates a Circular Linked list.\end{Desc}
\begin{Desc}
\item[Header File:]clClistApi.h\end{Desc}
\begin{Desc}
\item[Syntax:]

\footnotesize\begin{verbatim}   ClRcT clClistCreate(
                      				CL_IN  ClUint32T maxSize,
                      				CL_IN  ClClistDropPolicyT dropPolicy,
                      				CL_IN  ClClistDeleteCallbackT fpUserDeleteCallBack,
                      				CL_IN   ClClistDeleteCallbackT fpUserDestroyCallBack,
                      				CL_OUT  ClClistT* pListHead);
                      
                    
\end{verbatim}
\normalsize
\end{Desc}
\begin{Desc}
\item[Parameters:]
\begin{description}
\item[{\em max\-Size:}](in) Maximum size of the list. It specifies the maximum number of nodes that can exist at any instant in the list.
This must be an unsigned integer. You can add nodes to the list until this maximum limit is reached. If the value of this parameter is set to 0, there 
is no limit on the size of the list.
\item[{\em drop\-Policy:}](in) Drop policy for the list. If you try to add a node into a list that has reached it maximum limit, and if the
drop policy is set to {\tt{CL\_\-DROP\_\-FIRST}}, the first node is dropped and the specified node is added. If this parameter is set to
{\tt{CL\_\-NO\_\-DROP}}, an error is returned. This parameter is valid, only if the list is of fixed size.
\item[{\em fp\-User\-Delete\-Call\-Back:}](in) Pointer to the delete callback function of the user. This function accepts a parameter of type 
{\tt Cl\-Clist\-Data\-T\/}. After deleting the specified node, the user-data stored in that node is passed as an argument to the callback function.
\item[{\em fp\-User\-Destroy\-Call\-Bac\-K:}](in) Pointer to the destroy callback function. This function accepts a parameter of type 
{\tt Cl\-Clist\-Data\-T}. When destroying, the user-data stored in each node is passed as an argument to the callback function, one by one.
\item[{\em p\-List\-Head:}](out) Pointer to the variable of type {\tt{Cl\-Clist\-T}} in 
which the function returns a valid list handle on successful creation of the list.\end{description}
\end{Desc}
\begin{Desc}
\item[Return values:]
\begin{description}
\item[{\em CL\_\-OK:}]The function executed successfully. 
\item[{\em CL\_\-ERR\_\-NULL\_\-POINTER:}]{\tt{pListHead}} contains a NULL pointer. 
\item[{\em CL\_\-ERR\_\-NO\_\-MEMORY:}]Memory allocation failure.\end{description}
\end{Desc}
\begin{Desc}
\item[Note:]Returned error is a combination of the component ID and error code. Use {\tt{CL\_\-GET\_\-ERROR\_\-CODE(RET\_\-CODE)}} defined in
the {\tt{clCommonErrors.h}} file to get the error code.\end{Desc}
\begin{Desc}
\item[Description:]This function creates and initializes the Circular Linked list.\end{Desc}
\begin{Desc}
\item[Library File:]lib\-Cl\-Utils\end{Desc}
\begin{Desc}
\item[Related Function(s):]\hyperlink{pagecl114}{cl\-Clist\-Delete} \end{Desc}
\newpage

\subsection{clClistFirstNodeAdd}
\index{clClistFirstNodeAdd@{clClistFirstNodeAdd}}
\hypertarget{pagecl102}{}\paragraph{cl\-Clist\-First\-Node\-Add}\label{pagecl102}
\begin{Desc}
\item[Synopsis:]Adds a node at the beginning of the list.\end{Desc}
\begin{Desc}
\item[Header File:]clClistApi.h\end{Desc}
\begin{Desc}
\item[Syntax:]

\footnotesize\begin{verbatim}   ClRcT clClistFirstNodeAdd(
                         		CL_IN  ClClistT listHead,
                         		CL_IN  ClClistDataT userData);
\end{verbatim}
\normalsize
\end{Desc}
\begin{Desc}
\item[Parameters:]
\begin{description}
\item[{\em list\-Head:}](in) Handle of the list returned by the {\tt{clClistCreate}} API. \item[{\em user\-Data:}](in) User-data. 
You must allocate and de-allocate memory for this parameter.\end{description}
\end{Desc}
\begin{Desc}
\item[Return values:]
\begin{description}
\item[{\em CL\_\-OK:}]The function executed successfully. 
\item[{\em CL\_\-ERR\_\-NO\_\-MEMORY:}]Memory allocation failure. 
\item[{\em CL\_\-ERR\_\-MAXSIZE\_\-REACHED:}]The maximum size is reached. 
\item[{\em CL\_\-ERR\_\-INVALID\_\-HANDLE:}]An invalid handle has been passed to the function.\end{description}
\end{Desc}
\begin{Desc}
\item[Note:]Returned error is a combination of the component ID and error code. Use {\tt{CL\_\-GET\_\-ERROR\_\-CODE(RET\_\-CODE)}} defined in
the {\tt{clCommonErrors.h}} file to get the error code.\end{Desc}
\begin{Desc}
\item[Description:]This function adds a node at the beginning of the Circular Linked list.\end{Desc}
\begin{Desc}
\item[Library File:]lib\-Cl\-Utils\end{Desc}
\begin{Desc}
\item[Related Function(s):]\hyperlink{pagecl103}{cl\-Clist\-Last\-Node\-Add}, \hyperlink{pagecl104}{cl\-Clist\-After\-Node\-Add},
\hyperlink{pagecl105}{cl\-Clist\-Before\-Node\-Add} \end{Desc}
\newpage


\subsection{clClistLastNodeAdd}
\index{clClistLastNodeAdd@{clClistLastNodeAdd}}
\hypertarget{pagecl103}{}\paragraph{cl\-Clist\-Last\-Node\-Add}\label{pagecl103}
\begin{Desc}
\item[Synopsis:]Adds a node at the end of the list.\end{Desc}
\begin{Desc}
\item[Header File:]clClistApi.h\end{Desc}
\begin{Desc}
\item[Syntax:]

\footnotesize\begin{verbatim}   ClRcT clClistLastNodeAdd(
                          		CL_IN  ClClistT listHead,
                          		CL_IN  ClClistDataT userData);
\end{verbatim}
\normalsize
\end{Desc}
\begin{Desc}
\item[Parameters:]
\begin{description}
\item[{\em list\-Head:}](in) Handle of the list returned by the create API. \item[{\em user\-Data:}](in) User-data. 
You must allocate and de-allocate memory for this parameter.\end{description}
\end{Desc}
\begin{Desc}
\item[Return values:]
\begin{description}
\item[{\em CL\_\-OK:}]The function executed successfully. \item[{\em CL\_\-ERR\_\-NO\_\-MEMORY:}]Memory allocation failure. 
\item[{\em CL\_\-ERR\_\-MAXSIZE\_\-REACHED:}]The maximum size is reached. 
\item[{\em CL\_\-ERR\_\-INVALID\_\-HANDLE:}]An invalid handle has been passed to the function.\end{description}
\end{Desc}
\begin{Desc}
\item[Note:]Returned error is a combination of the component ID and error code. Use {\tt{CL\_\-GET\_\-ERROR\_\-CODE(RET\_\-CODE)}} defined in 
the {\tt{clCommonErrors.h}} file to get the error code.\end{Desc}
\begin{Desc}
\item[Description:]This function adds a node at the end of the Circular Linked list.\end{Desc}
\begin{Desc}
\item[Library File:]lib\-Cl\-Utils\end{Desc}
\begin{Desc}
\item[Related Function(s):]\hyperlink{pagecl104}{cl\-Clist\-After\-Node\-Add}, \hyperlink{pagecl105}{cl\-Clist\-Before\-Node\-Add} \end{Desc}
\newpage


\subsection{clClistAfterNodeAdd}
\index{clClistAfterNodeAdd@{clClistAfterNodeAdd}}
\hypertarget{pagecl104}{}\paragraph{cl\-Clist\-After\-Node\-Add}\label{pagecl104}
\begin{Desc}
\item[Synopsis:]Adds a node after a specified node in the list.\end{Desc}
\begin{Desc}
\item[Header File:]clClistApi.h\end{Desc}
\begin{Desc}
\item[Syntax:]

\footnotesize\begin{verbatim}   ClRcT clClistAfterNodeAdd(
                         		CL_IN  ClClistT listHead,
                         		CL_IN  ClClistNodeT currentNode,
                         		CL_IN  ClClistDataT userData);
\end{verbatim}
\normalsize
\end{Desc}
\begin{Desc}
\item[Parameters:]
\begin{description}
\item[{\em list\-Head:}](in) Handle of the list returned by the create API. \item[{\em current\-Node:}](in) Handle of the node after which the specified 
data must be added. \item[{\em user\-Data:}](in) User-data. You must allocate and de-allocate memory for this parameter.\end{description}
\end{Desc}
\begin{Desc}
\item[Return values:]
\begin{description}
\item[{\em CL\_\-OK:}]The function executed successfully. \item[{\em CL\_\-ERR\_\-NO\_\-MEMORY:}]Memory allocation failure. 
\item[{\em CL\_\-ERR\_\-MAXSIZE\_\-REACHED:}]The maximum size is reached. 
\item[{\em CL\_\-ERR\_\-INVALID\_\-HANDLE:}]An invalid handle has been passed to the function.\end{description}
\end{Desc}
\begin{Desc}
\item[Note:]Returned error is a combination of the component ID and error code. Use {\tt{CL\_\-GET\_\-ERROR\_\-CODE(RET\_\-CODE)}} defined in
the {\tt{clCommonErrors.h}} file to get the error code.\end{Desc}
\begin{Desc}
\item[Description:]This function adds a node after a specified node in the Circular Linked list.\end{Desc}
\begin{Desc}
\item[Library File:]lib\-Cl\-Utils\end{Desc}
\begin{Desc}
\item[Related Function(s):]\hyperlink{pagecl103}{cl\-Clist\-Last\-Node\-Add}, \hyperlink{pagecl105}{cl\-Clist\-Before\-Node\-Add} \end{Desc}
\newpage

\subsection{clClistBeforeNodeAdd}
\index{clClistBeforeNodeAdd@{clClistBeforeNodeAdd}}
\hypertarget{pagecl105}{}\paragraph{cl\-Clist\-Before\-Node\-Add}\label{pagecl105}
\begin{Desc}
\item[Synopsis:]Adds a node before a specified node in the list.\end{Desc}
\begin{Desc}
\item[Header File:]clClistApi.h\end{Desc}
\begin{Desc}
\item[Syntax:]

\footnotesize\begin{verbatim}   ClRcT  clClistBeforeNodeAdd(
                         		CL_IN  ClClistT listHead,
                         		CL_IN  ClClistNodeT currentNode,
                         		CL_IN  ClClistDataT userData);
\end{verbatim}
\normalsize
\end{Desc}
\begin{Desc}
\item[Parameters:]
\begin{description}
\item[{\em list\-Head:}](in) Handle of the list returned by the create API. \item[{\em current\-Node:}](in) Handle of the node before which the specified
data must be added. 
\item[{\em user\-Data:}](in) User-data. You must perform memory allocation and de-allocation for this parameter.\end{description}
\end{Desc}
\begin{Desc}
\item[Return values:]
\begin{description}
\item[{\em CL\_\-OK:}]The function executed successfully. \item[{\em CL\_\-ERR\_\-NO\_\-MEMORY:}]Memory allocation failure. 
\item[{\em CL\_\-ERR\_\-MAXSIZE\_\-REACHED:}]The maximum size is reached. 
\item[{\em CL\_\-ERR\_\-INVALID\_\-HANDLE:}]An invalid handle has been passed to the function.\end{description}
\end{Desc}
\begin{Desc}
\item[Note:]Returned error is a combination of the component ID and error code. Use {\tt{CL\_\-GET\_\-ERROR\_\-CODE(RET\_\-CODE)}} defined in
the {\tt{clCommonErrors.h}} file to get the error code.\end{Desc}
\begin{Desc}
\item[Description:]This function adds a node before a specified node in the Circular Linked list.\end{Desc}
\begin{Desc}
\item[Library File:]lib\-Cl\-Utils\end{Desc}
\begin{Desc}
\item[Related Function(s):]\hyperlink{pagecl103}{cl\-Clist\-Last\-Node\-Add}, \hyperlink{pagecl104}{cl\-Clist\-After\-Node\-Add} \end{Desc}
\newpage


\subsection{clClistNodeDelete}
\index{clClistNodeDelete@{clClistNodeDelete}}
\hypertarget{pagecl106}{}\paragraph{cl\-Clist\-Node\-Delete}\label{pagecl106}
\begin{Desc}
\item[Synopsis:]Deletes a node from the list.\end{Desc}
\begin{Desc}
\item[Header File:]clClistApi.h\end{Desc}
\begin{Desc}
\item[Syntax:]

\footnotesize\begin{verbatim}              
                   	ClRcT clClistNodeDelete(
                   		CL_IN ClClistT listHead,
                   		CL_IN ClClistNodeT node);
\end{verbatim}
\normalsize
\end{Desc}
\begin{Desc}
\item[Parameters:]
\begin{description}
\item[{\em list\-Head:}](in) Handle of the list returned by the create API. \item[{\em node:}](in) Handle of the node which is to be deleted.\end{description}
\end{Desc}
\begin{Desc}
\item[Return values:]
\begin{description}
\item[{\em CL\_\-OK:}]The function executed successfully. \item[{\em CL\_\-ERR\_\-INVALID\_\-HANDLE:}]An invalid handle has been passed to the function.
\end{description}
\end{Desc}
\begin{Desc}
\item[Note:]Returned error is a combination of the component ID and error code. Use {\tt{CL\_\-GET\_\-ERROR\_\-CODE(RET\_\-CODE)}} defined in
the {\tt{clCommonErrors.h}} file to get the error code.\end{Desc}
\begin{Desc}
\item[Description:]This function deletes the specified node from the Circular Linked list. The delete callback function, registered during the creation is
called with the data in the node to be deleted.\end{Desc}
\begin{Desc}
\item[Library File:]lib\-Cl\-Utils\end{Desc}
\begin{Desc}
\item[Related Function(s):]\hyperlink{pagecl102}{cl\-Clist\-First\-Node\-Add}, \hyperlink{pagecl103}{cl\-Clist\-Last\-Node\-Add},
\hyperlink{pagecl104}{cl\-Clist\-After\-Node\-Add}, \hyperlink{pagecl105}{cl\-Clist\-Before\-Node\-Add} \end{Desc}
\newpage


\subsection{clClistFirstNodeGet}
\index{clClistFirstNodeGet@{clClistFirstNodeGet}}
\hypertarget{pagecl107}{}\paragraph{cl\-Clist\-First\-Node\-Get}\label{pagecl107}
\begin{Desc}
\item[Synopsis:]Returns the first node from the list.\end{Desc}
\begin{Desc}
\item[Header File:]clClistApi.h\end{Desc}
\begin{Desc}
\item[Syntax:]

\footnotesize\begin{verbatim}   ClRcT clClistFirstNodeGet(
                           		CL_IN  ClClistT listHead,
                           		CL_OUT  ClClistNodeT* pFirstNode);
\end{verbatim}
\normalsize
\end{Desc}
\begin{Desc}
\item[Parameters:]
\begin{description}
\item[{\em list\-Head:}](in) Handle of the list returned by the create API. 
\item[{\em p\-First\-Node:}](out) Pointer to the variable of type {\tt{Cl\-Clist\-Node\-T}}.\end{description}
\end{Desc}
\begin{Desc}
\item[Return values:]
\begin{description}
\item[{\em CL\_\-OK:}]The function executed successfully. 
\item[{\em CL\_\-ERR\_\-NULL\_\-POINTER:}]{\tt{pFirstNode}} contains a NULL pointer. 
\item[{\em CL\_\-ERR\_\-INVALID\_\-HANDLE:}]An invalid handle has been passed to the function. 
\item[{\em CL\_\-ERR\_\-NOT\_\-EXIST:}]The list is empty.\end{description}
\end{Desc}
\begin{Desc}
\item[Note:]Returned error is a combination of the component ID and error code. Use {\tt{CL\_\-GET\_\-ERROR\_\-CODE(RET\_\-CODE)}} defined in 
the {\tt{clCommonErrors.h}} file to get the error code.\end{Desc}
\begin{Desc}
\item[Description:]This function returns the first node from the Circular Linked list.\end{Desc}
\begin{Desc}
\item[Library File:]lib\-Cl\-Utils\end{Desc}
\begin{Desc}
\item[Related Function(s):]\hyperlink{pagecl108}{cl\-Clist\-Last\-Node\-Get}, \hyperlink{pagecl109}{cl\-Clist\-Next\-Node\-Get}, 
\hyperlink{pagecl110}{cl\-Clist\-Previous\-Node\-Get} \end{Desc}
\newpage

\subsection{clClistLastNodeGet}
\index{clClistLastNodeGet@{clClistLastNodeGet}}
\hypertarget{pagecl108}{}\paragraph{cl\-Clist\-Last\-Node\-Get}\label{pagecl108}
\begin{Desc}
\item[Synopsis:]Returns the last node from the list.\end{Desc}
\begin{Desc}
\item[Header File:]clClistApi.h\end{Desc}
\begin{Desc}
\item[Syntax:]

\footnotesize\begin{verbatim}   ClRcT clClistLastNodeGet(
                          		CL_IN  ClClistT listHead,
                          		CL_OUT  ClClistNodeT* pLastNode);
\end{verbatim}
\normalsize
\end{Desc}
\begin{Desc}
\item[Parameters:]
\begin{description}
\item[{\em list\-Head:}](in) Handle of the list returned by the create API. 
\item[{\em p\-Last\-Node:}](out) Pointer to the variable of type {\tt{Cl\-Clist\-Node\-T}}.\end{description}
\end{Desc}
\begin{Desc}
\item[Return values:]
\begin{description}
\item[{\em CL\_\-OK:}]The function executed successfully. 
\item[{\em CL\_\-ERR\_\-NULL\_\-POINTER:}]{\tt{pLastNode}} contains a NULL pointer. 
\item[{\em CL\_\-ERR\_\-INVALID\_\-HANDLE:}]An invalid handle has been passed to the function.
\item[{\em CL\_\-ERR\_\-NOT\_\-EXIST:}]The list is empty.\end{description}
\end{Desc}
\begin{Desc}
\item[Note:]Returned error is a combination of the component ID and error code. Use {\tt{CL\_\-GET\_\-ERROR\_\-CODE(RET\_\-CODE)}} defined in
the {\tt{clCommonErrors.h}} file to get the error code.\end{Desc}
\begin{Desc}
\item[Description:]This function is used to return the last node from the Circular Linked list.\end{Desc}
\begin{Desc}
\item[Library File:]lib\-Cl\-Utils\end{Desc}
\begin{Desc}
\item[Related Function(s):]\hyperlink{pagecl109}{cl\-Clist\-Next\-Node\-Get}, \hyperlink{pagecl110}{cl\-Clist\-Previous\-Node\-Get} \end{Desc}
\newpage

\subsection{clClistNextNodeGet}
\index{clClistNextNodeGet@{clClistNextNodeGet}}
\hypertarget{pagecl109}{}\paragraph{cl\-Clist\-Next\-Node\-Get}\label{pagecl109}
\begin{Desc}
\item[Synopsis:]Returns the next node from the list.\end{Desc}
\begin{Desc}
\item[Header File:]clClistApi.h\end{Desc}
\begin{Desc}
\item[Syntax:]

\footnotesize\begin{verbatim}   ClRcT clClistNextNodeGet(
                         		CL_IN  ClClistT listHead,
                         		CL_IN  ClClistNodeT currentNode,
                         		CL_OUT  ClClistNodeT* pNextNode);
\end{verbatim}
\normalsize
\end{Desc}
\begin{Desc}
\item[Parameters:]
\begin{description}
\item[{\em list\-Head:}](in) Handle of the list returned by the create API. \item[{\em current\-Node:}](in) Handle of the current node. 
\item[{\em p\-Next\-Node:}](out) Pointer to the variable of type {\tt{Cl\-Clist\-Node\-T}}.\end{description}
\end{Desc}
\begin{Desc}
\item[Return values:]
\begin{description}
\item[{\em CL\_\-OK:}]The function executed successfully. 
\item[{\em CL\_\-ERR\_\-NULL\_\-POINTER:}]{\tt{pNextNode}} contains a NULL pointer. 
\item[{\em CL\_\-ERR\_\-INVALID\_\-HANDLE:}]An invalid handle has been passed to the function. 
\item[{\em CL\_\-ERR\_\-NOT\_\-EXIST:}]The list is empty.\end{description}
\end{Desc}
\begin{Desc}
\item[Note:]Returned error is a combination of the component ID and error code. Use {\tt{CL\_\-GET\_\-ERROR\_\-CODE(RET\_\-CODE)}} defined in 
the {\tt{clCommonErrors.h}} file to get the error code.\end{Desc}
\begin{Desc}
\item[Description:]This function is used to return the next node of the specified node from the Circular Linked list.\end{Desc}
\begin{Desc}
\item[Library File:]lib\-Cl\-Utils\end{Desc}
\begin{Desc}
\item[Related Function(s):]\hyperlink{pagecl108}{cl\-Clist\-Last\-Node\-Get}, \hyperlink{pagecl110}{cl\-Clist\-Previous\-Node\-Get} \end{Desc}
\newpage


\subsection{clClistPreviousNodeGet}
\index{clClistPreviousNodeGet@{clClistPreviousNodeGet}}
\hypertarget{pagecl110}{}\paragraph{cl\-Clist\-Previous\-Node\-Get}\label{pagecl110}
\begin{Desc}
\item[Synopsis:]Returns the previous node from the list.\end{Desc}
\begin{Desc}
\item[Header File:]clClistApi.h\end{Desc}
\begin{Desc}
\item[Syntax:]

\footnotesize\begin{verbatim}   ClRcT clClistPreviousNodeGet(
                             		CL_IN  ClClistT listHead,
                             		CL_IN  ClClistNodeT currentNode,
                             		CL_OUT  ClClistNodeT* pPreviousNode);
\end{verbatim}
\normalsize
\end{Desc}
\begin{Desc}
\item[Parameters:]
\begin{description}
\item[{\em list\-Head:}](in) Handle of the list returned by the create API. \item[{\em current\-Node:}](in) Handle of the current node. 
\item[{\em p\-Previous\-Node:}](out) Pointer to the variable of type {\tt{Cl\-Clist\-Node\-T}}.\end{description}
\end{Desc}
\begin{Desc}
\item[Return values:]
\begin{description}
\item[{\em CL\_\-OK:}]The function executed successfully. \item[{\em CL\_\-ERR\_\-NULL\_\-POINTER:}]{\tt{pPreviousNode}} contains a NULL pointer. 
\item[{\em CL\_\-ERR\_\-INVALID\_\-HANDLE:}]An invalid handle has been passed to the function. 
\item[{\em CL\_\-ERR\_\-NOT\_\-EXIST:}]The list is empty.\end{description}
\end{Desc}
\begin{Desc}
\item[Note:]Returned error is a combination of the component ID and error code. Use {\tt{CL\_\-GET\_\-ERROR\_\-CODE(RET\_\-CODE)}} defined in 
the {\tt{clCommonErrors.h}} file to get the error code.\end{Desc}
\begin{Desc}
\item[Description:]This function is used to return the previous node of the specified node from the Circular Linked list.\end{Desc}
\begin{Desc}
\item[Library File:]lib\-Cl\-Utils\end{Desc}
\begin{Desc}
\item[Related Function(s):]\hyperlink{pagecl108}{cl\-Clist\-Last\-Node\-Get}, \hyperlink{pagecl109}{cl\-Clist\-Next\-Node\-Get} \end{Desc}
\newpage

\subsection{clClistWalk}
\index{clClistWalk@{clClistWalk}}
\hypertarget{pagecl111}{}\paragraph{cl\-Clist\-Walk}\label{pagecl111}
\begin{Desc}
\item[Synopsis:]Walks through the list.\end{Desc}
\begin{Desc}
\item[Header File:]clClistApi.h\end{Desc}
\begin{Desc}
\item[Syntax:]

\footnotesize\begin{verbatim}   ClRcT clClistWalk(
                   			CL_IN  ClClistT listHead,
                   			CL_IN  ClClistWalkCallbackT fpUserWalkCallBack,
                   			CL_IN  void* userArg);
\end{verbatim}
\normalsize
\end{Desc}
\begin{Desc}
\item[Parameters:]
\begin{description}
\item[{\em list\-Head:}](in) Handle of the list returned by the create API.
\item[{\em fp\-User\-Walk\-Call\-Back:}](in) Pointer to the user callback function. It can have two values: \begin{itemize}
\item The first parameter must be of type {\tt{Cl\-Clist\-Data\-T}}. \item The second parameter must be of type void $\ast$. \par
 The user-data stored in each node is passed one-by-one as the first argument to the callback function.\end{itemize}
\item[{\em user\-Arg:}](in) User-specified argument. This variable is passed as the second argument to the callback function.\end{description}
\end{Desc}
\begin{Desc}
\item[Return values:]
\begin{description}
\item[{\em CL\_\-OK:}]The function executed successfully. 
\item[{\em CL\_\-ERR\_\-NULL\_\-POINTER:}]{\tt{userArg}} contains a NULL pointer. 
\item[{\em CL\_\-ERR\_\-INVALID\_\-HANDLE:}]An invalid handle has been passed to the function.\end{description}
\end{Desc}
\begin{Desc}
\item[Note:]Returned error is a combination of the component ID and error code. Use {\tt{CL\_\-GET\_\-ERROR\_\-CODE(RET\_\-CODE)}} defined in 
the {\tt{clCommonErrors.h}} file to get the error code.\end{Desc}
\begin{Desc}
\item[Description:]This function performs a walk through on the Circular Linked list. The user-specified callback function is called with every node's data.\end{Desc}
\begin{Desc}
\item[Library File:]lib\-Cl\-Utils\end{Desc}
\begin{Desc}
\item[Related Function(s):]None. \end{Desc}
\newpage

\subsection{clClistDataGet}
\index{clClistDataGet@{clClistDataGet}}
\hypertarget{pagecl112}{}\paragraph{cl\-Clist\-Data\-Get}\label{pagecl112}
\begin{Desc}
\item[Synopsis:]Retrieves data from a node in the list.\end{Desc}
\begin{Desc}
\item[Header File:]clClistApi.h\end{Desc}
\begin{Desc}
\item[Syntax:]

\footnotesize\begin{verbatim}   ClRcT clClistDataGet(
                     			CL_IN  ClClistT listHead,
                     			CL_IN  ClClistNodeT node,
                     			CL_OUT  ClClistDataT* pUserData);
\end{verbatim}
\normalsize
\end{Desc}
\begin{Desc}
\item[Parameters:]
\begin{description}
\item[{\em list\-Head:}](in) Handle of the list returned by the create API. 
\item[{\em node:}](in) Handle of the node. 
\item[{\em p\-User\-Data:}](out) Pointer to the variable of type {\tt{Cl\-Clist\-Data\-T}}.\end{description}
\end{Desc}
\begin{Desc}
\item[Return values:]
\begin{description}
\item[{\em CL\_\-OK:}]The function executed successfully. \item[{\em CL\_\-ERR\_\-NULL\_\-POINTER:}]{\tt{pUserData}} contains a NULL pointer. 
\item[{\em CL\_\-ERR\_\-INVALID\_\-HANDLE:}]An invalid handle has been passed to the function.\end{description}
\end{Desc}
\begin{Desc}
\item[Note:]Returned error is a combination of the component ID and error code. Use {\tt{CL\_\-GET\_\-ERROR\_\-CODE(RET\_\-CODE)}} defined in 
the {\tt{clCommonErrors.h}} file to get the error code.\end{Desc}
\begin{Desc}
\item[Description:]This function retrieves data from the specified node in the list.\end{Desc}
\begin{Desc}
\item[Library File:]lib\-Cl\-Utils\end{Desc}
\begin{Desc}
\item[Related Function(s):]None. \end{Desc}
\newpage


\subsection{clClistSizeGet}
\index{clClistSizeGet@{clClistSizeGet}}
\hypertarget{pagecl113}{}\paragraph{cl\-Clist\-Size\-Get}\label{pagecl113}
\begin{Desc}
\item[Synopsis:]Returns the number of data elements (nodes) in the list.\end{Desc}
\begin{Desc}
\item[Header File:]clClistApi.h\end{Desc}
\begin{Desc}
\item[Syntax:]

\footnotesize\begin{verbatim}   ClRcT clClistSizeGet(
                     			CL_IN  ClClistT listHead,
                     			CL_OUT  ClUint32T* pSize);
\end{verbatim}
\normalsize
\end{Desc}
\begin{Desc}
\item[Parameters:]
\begin{description}
\item[{\em list\-Head:}](in) Handle of the list returned by the create API. 
\item[{\em p\-Size:}](out) Pointer to the variable of type {\em Cl\-Uint32T\/}.\end{description}
\end{Desc}
\begin{Desc}
\item[Return values:]
\begin{description}
\item[{\em CL\_\-OK:}]The function executed successfully. 
\item[{\em CL\_\-ERR\_\-NULL\_\-POINTER:}]{\tt{pSize}} contains a NULL pointer. 
\item[{\em CL\_\-ERR\_\-INVALID\_\-HANDLE:}]An invalid handle has been passed to the function.
\end{description}
\end{Desc}
\begin{Desc}
\item[Note:]Returned error is a combination of the component ID and error code. Use {\tt{CL\_\-GET\_\-ERROR\_\-CODE(RET\_\-CODE)}} defined in
the {\tt{clCommonErrors.h}} file to get the error code.\end{Desc}
\begin{Desc}
\item[Description:]This function is used to return the number of data elements (nodes) in the list.\end{Desc}
\begin{Desc}
\item[Library File:]lib\-Cl\-Utils\end{Desc}
\begin{Desc}
\item[Related Function(s):]None. \end{Desc}
\newpage


\subsection{clClistDelete}
\index{clClistDelete@{clClistDelete}}
\hypertarget{pagecl114}{}\paragraph{cl\-Clist\-Delete}\label{pagecl114}
\begin{Desc}
\item[Synopsis:]Destroys the list.\end{Desc}
\begin{Desc}
\item[Header File:]clClistApi.h\end{Desc}
\begin{Desc}
\item[Syntax:]

\footnotesize\begin{verbatim}   ClRcT clClistDelete(
                     			CL_IN  ClClistT* pListHead);
\end{verbatim}
\normalsize
\end{Desc}
\begin{Desc}
\item[Parameters:]
\begin{description}
\item[{\em p\-List\-Head:}](in) Handle of the list returned by the create API.\end{description}
\end{Desc}
\begin{Desc}
\item[Return values:]
\begin{description}
\item[{\em CL\_\-OK:}]The function executed successfully. 
\item[{\em CL\_\-ERR\_\-NULL\_\-POINTER:}]{\tt{pListHead}} contains a NULL pointer. 
\item[{\em CL\_\-ERR\_\-INVALID\_\-HANDLE:}]An invalid handle has been passed to the function.\end{description}
\end{Desc}
\begin{Desc}
\item[Note:]Returned error is a combination of the component ID and error code. Use {\tt{CL\_\-GET\_\-ERROR\_\-CODE(RET\_\-CODE)}} defined in the
{\tt{clCommonErrors.h}} file to get the error code.\end{Desc}
\begin{Desc}
\item[Description:]This function deletes all the nodes in the list. The destroy callback function, registered during creation, is called for every node in the
list with the corresponding data.\end{Desc}
\begin{Desc}
\item[Library File:]lib\-Cl\-Utils\end{Desc}
\begin{Desc}
\item[Related Function(s):]\hyperlink{pagecl101}{cl\-Clist\-Create} \end{Desc}


\chapter{Service Management Information Model}
TBD
\chapter{Service Notifications}
TBD
\chapter{Debug CLIs}
TBD
\end{flushleft}