
\hypertarget{group__group11}{
\chapter{Functional Overview}
\label{group__group11}
}

\begin{flushleft}

The OpenClovis Chassis Manager (CM) communicates with the chassis to obtain and control hardware platform status information. The CM uses Hardware 
Platform Interface (HPI) to: 
\begin{itemize}
\item
Interact with the hardware platform and thereby, monitor its status.
\item
Control the entities contained in it and Field Replaceable Units (FRUs), such as blades, AMCs, PMCs, and so on. 
\end{itemize}

Chassis Manager is designed to use HPI B.01.01 specification's implementation of OpenHPI (Open source of HPI implementation), and {\tt{libhcl}} (Radisys HPI 
Client Library), and provide the platform management services to ASP. CM depends on the Shelf Manager to monitor and control operations on the
chassis. The Shelf Manager is responsible for monitoring and control to ensure correct operation of boards and other Shelf components. CM resides on 
the Shelf Manager to provide this system management functionality. The Chassis Manager controls the activation and de-activation of the blades based on 
Shelf Manager's implementation during hotswap state transition, when a blade is inserted and removed.
\par
Chassis Manager detects arrival and departure of cards through HPI/IPMI events and other platform hardware related events, and reports it to the 
appropriate ASP components. 
\par
It also takes management requests from various sources such as Availability Management. For instance, CM supports hot-swapping, where the 
entities running on a node can shutdown before the node is physically removed.

\chapter{Service Model}
TBD

\chapter{Service APIs}

\section{Type Definitions}


\subsection{ClCmFruOperationT}
\index{ClCmFruOperationT@{ClCmFruOperationT}}
\begin{tabbing}
xx\=xx\=xx\=xx\=xx\=xx\=xx\=xx\=xx\=\kill
\textit{typedef enum \{}\\
\>\>\>\>\textit{CL\_CM\_POWERON\_REQUEST,}\\
\>\>\>\>\textit{CL\_CM\_POWEROFF\_REQUEST,}\\
\>\>\>\>\textit{CL\_CM\_POWER\_CYCLE\_REQUEST,}\\
\>\>\>\>\textit{CL\_CM\_INSERT\_REQUEST,}\\
\>\>\>\>\textit{CL\_CM\_EXTRACT\_REQUEST,}\\
\>\>\>\>\textit{CL\_CM\_RESET\_REQUEST,}\\
\>\>\>\>\textit{CL\_CM\_WARM\_RESET\_REQUEST,}\\
\>\>\>\>\textit{CL\_CM\_RESET\_ASSERT\_REQUEST,}\\
\>\>\>\>\textit{CL\_CM\_RESET\_DEASSERT\_REQUEST}\\
\textit{\} ClCmFruOperationT;}\end{tabbing}
The enumeration, {\tt{ClCmFruOperationT}}, contains the various chassis manager operations.


\newpage

\section{Functional APIs}


\subsection{clCmVersionVerify}
\index{clCmVersionVerify@{clCmVersionVerify}}
\hypertarget{pagecm202}{}\paragraph{cl\-Cm\-Version\-Verify}\label{pagecm202}
\begin{Desc}
\item[Synopsis:]Verifies the version of the library supported by the CM.\end{Desc}
\begin{Desc}
\item[Header File:]clCmApi.h\end{Desc}
\begin{Desc}
\item[Syntax:]

\footnotesize\begin{verbatim}        ClRcT clCmVersionVerify(
                         			ClVersionT *version  );
\end{verbatim}
\normalsize
\end{Desc}
\begin{Desc}
\item[Parameters:]
\begin{description}
\item[{\em version}](in/out): As an input parameter, it accepts the version being verified. As an output parameter, it returns the version that is 
supported.
\end{description}
\end{Desc}
\begin{Desc}
\item[Return values:]
\begin{description}
\item[{\em CL\_\-OK:}]The API executed successfully. \item[{\em CL\_\-ERR\_\-VERSION\_\-MISMATCH:}]The version provided is not supported.\end{description}
\end{Desc}
\begin{Desc}
\item[Description:]This function is used to verify if the version of the client is supported. If the version is not supported, an error is returned 
with the value of the supported version populated in the {\tt{out}} parameter. Typically, this function must be invoked at the initialization of the application.
\end{Desc}
\begin{Desc}
\item[Library File:]Cl\-Cm\end{Desc}
\begin{Desc}
\item[Related Function(s):]None. \end{Desc}


\newpage
\subsection{clCmBladeOperationRequest}
\index{clCmBladeOperationRequest@{clCmBladeOperationRequest}}
\hypertarget{pagecm202}{}\paragraph{cl\-Cm\-Blade\-Operation\-Request}\label{pagecm202}
\begin{Desc}
\item[Synopsis:]Performs an operation on the blade or any platform hardware.\end{Desc}
\begin{Desc}
\item[Header File:]clCmApi.h\end{Desc}
\begin{Desc}
\item[Syntax:]

\footnotesize\begin{verbatim}   ClRcT clCmBladeOperationRequest(
              				CL_IN  ClUint32T ChassisId,
              				CL_IN  ClUint32T SlotId,
              				CL_IN ClCmFruOperationT request);
\end{verbatim}	
\normalsize
\end{Desc}
\begin{Desc}
\item[Parameters:]
\begin{description}
\item[{\em Chassis\-Id:}](in) {\tt{Chassis\-Id}} of the blade. 
\item[{\em Slot\-Id:}](in) {\tt{Slot\-Id}} of the blade. 
\item[{\em request:}](in) Operation requested on the hardware.\end{description}
\end{Desc}
\begin{Desc}
\item[Return values:]
\begin{description}
\item[{\em CL\_\-OK:}]The API executed successfully. 
\item[{\em CL\_\-ERR\_\-INVALID\_\-PARAMETER:}]An invalid parameter has been passed to this function. A parameter is not set correctly.
\item[{\em CL\_\-ERR\_\-VERSION\_\-MISMATCH:}]The version of the client and server incompatible. 
\item[{\em CL\_\-ERR\_\-CM\_\-HPI\_\-ERROR:}]An HPI error has occurred at the server. 
\item[{\em CL\_\-ERR\_\-TRY\_\-AGAIN:}]CM is busy and cannot service the request immediately.\end{description}
\end{Desc}
\begin{Desc}
\item[Description:]This API is used to operate on a BLADE or platform hardware. It accepts three arguments: {\tt{Chassis\-Id}}, {\tt{Slot\-Id}}
of the FRU to operate, and the requested operation. The various operations are listed in the enumeration, {\tt{Cl\-Cm\-Fru\-Operation\-T}}.\end{Desc}
\begin{Desc}
\item[Warning:]Setting-up the hotswap state of the BLADE must conform to the HPI hotswap state transition sequence. Only valid state transitions are 
allowed and any invalid state transition requests are rejected as {\tt{invalid}} requests.\end{Desc}
\begin{Desc}
\item[Note:]Refer to {\tt{DBG\_\-PRINTS}} on the Chassis Manager console or Log file for HPI errors.\end{Desc}
\begin{Desc}
\item[Library File:]Cl\-Cm\end{Desc}
\begin{Desc}
\item[Related Function(s):]None. \end{Desc}

\chapter{Service Management Information Model}
TBD

\chapter{Service Notifications}
TBD

\chapter{Configuration}
TBD

\chapter{Debug CLI}
TBD

\end{flushleft}