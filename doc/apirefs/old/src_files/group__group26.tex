\hypertarget{group__group26}{
\chapter{Functional Overview}
\label{group__group26}
\index{Log Service@{Log Service}}
}


\begin{flushleft}

Computer programs record various events for further analysis. The purpose of recording events could be to debug various mis-configurations, operational
issues, and so on.
The ASP Log Service provides the facility to record the information about these events. 
A program on the cluster can use the Log Service to record information. 
\par
Log Service persists the information recorded by its clients so that the information is available for viewing.
The viewer may be an offline viewer or an online viewer. An offline viewer views the information in the future and an online viewer may view the 
information as soon as it is generated. 
\par
Log Service does not interpret the information recorded by its clients or apply any 
semantic meaning to it. It treats the information as an octet stream.
\par
All information related to an event of interest is stored as one unit called the Log Record. Log Records are grouped together based on a certain 
theme defined by the client. A group of Log Records is called a Log Stream. These Log Streams can be shared by various components of an application or 
of different applications. The theme and the users of a Log Stream are defined by the application. The ASP Log Service does not impose any 
restrictions on the same. 
\par
The Log Streams flow into Log File, where the Log Records are physically stored and persisted. As the clients of Log Service may be perpetual and 
the lifetime of the Log Records may be forever, any amount of physical storage for persisting the Log Records may be insufficient. 
Thus, the logged records need to be siphoned off to some other form of storage to make way for new records. Log Service requires the presence of an 
archiving utility that siphons off the records to another form of persistent storage. 
\par
Log Service is designed to provide highly efficient and reliable logging service. Log Service provides interfaces for:
\begin{itemize}
\item
Creating a Log Stream, opening an existing Log Stream, recording an event into an opened Log Stream, and closing a Log Stream. 
\item
Viewing logged records and for siphoning them to some other form of storage.
\item
Changing the properties of Log Streams.
\item
Creating Log Files and persist Log Streams in the Log Files, periodically.
\item
Recording binary buffer, strings, and data in TLV format. 
\item
Filtering the Log Records, thereby disallowing certain Log Records, on the basis of filter criteria, from entering the Log Stream.
\end{itemize}

\par
A program can use any number of Log Streams and a Log Stream can be used by multiple programs. 
A program can have the Log Service block Log Records and MetaData files for archiving to create space for more Log Records in the Log Files. 
An application can use Log Service for the following:
\begin{itemize}
\item
To debug by logging the trace records on to the Log Streams.
\item
Open the Log Stream, store the Log Records into the Log Stream, and close the Log Stream. 
Log Stream can be assigned as: 
\begin{enumerate}
\item Cluster wide resources - Programs across different nodes can use the same Log Stream. 
\item Node specific resource - Programs on a specific node only can make use of that Log Stream. 
\end{enumerate}
\end{itemize}

\par
Log Service requires that all Log Records in a Log File have the same length. This can be specified at the time of creation of Log Files. 
There is no restriction on the format of the Log Record. 
Rendering the Log Records is not in the scope of Log Service. 



\chapter{Service Model}
\section{Usage Model}
\index{Usage Model@{Usage Model}}

The usage model of the Log Service is a Producer-Consumer model (the users of Log Services are Producers and Consumers). Loggers produce the Log Records.
Log Readers and Log Viewers consume the Log Records. Multiple Loggers may log into a specific Log Stream. Similarly, a specific record may be 
consumed by many Log Readers and may be viewed many times. The Log Viewer may be online or offline. Thus the records may be consumed any time after
being produced. Archiever is a special kind of consumer that consumes every record only once and siphons off the records to another storage.
\begin{Desc}
\item[Usage Scenarios] 
\begin{enumerate}
\item Health monitoring: Various components of the system periodically log messages recording their core state. A system administrator or a monitoring
program periodically analyzes the records to determine if the system is running as expected. If any abnormal activity is detected corrective action can be
taken.
\item Failure analysis: If there is a failure in the system, the administrator can check the Log Records to determine the cause of failure.
\item Realtime troubleshooting: The troubleshooter can check the Log Records as they are generated. The troubleshooter can identify and rectify the problem.
Troubleshooter has the option of setting the filters for relevant records only.
\item Debugging and Testing: Log Service provides the facility of logging ASCII messages and binary buffer. Applications can use Log Services to record 
data during testing and debugging phase.
\item Data Mining: Log Records are an important source of information regarding system usage scenarios. MIS can check the Log Records to determine 
the usage pattern and can use the data for subsequent planning.   
\item 	Speciality Log: Applications can create their own special log. For example, Logs having encrypted data. Applications can define their own message 
identifiers and format. Log Service does not interpret the log data. So, the applications can use Log Service for logging these records.
\item Archiving: Log Records can be archived for offline viewing or for future use. Log Service can provide a block of records from the Log File.
\end{enumerate}
\end{Desc}

\section{Functional Description}
\index{Functional Description@{Functional Description}}
The purpose of the Log Service is to record information provided by the Logger for future use. This information is provided in the form of Log Records.
Log Records flow through a Log Stream and are persisted by the Log Service in the Log File.
The producer of Log Records is called the Logger which uses \textit{clLogWriteAsync()} API to write a Log Record into a Log Stream. 
\par
Before using the services of Log Service, the Logger has to initialize the client library of the Log Service by invoking \textit{clLogInitialize()}. 
\textit{clLogInitialize()} returns a \textit{logServiceHandle} that can be used in subsequent operations on the Log Service to identify the
initialization. When this handle is no more required, the association can be closed by invoking \textit{clLogFinalize()}. As a result, the Logger gains 
access to the Log Stream by invoking \textit{clLogStreamOpen()}. \textit{clLogStreamOpen()} returns a \textit{logStreamHandle} that can be used in 
\textit{clLogWriteAsync()} API to identify the Log Stream. When a Log Stream is no longer required by a Logger, it can be closed by invoking \textit{clLogStreamClose()}. 
\par
There are two types of consumers of Log Records:
\begin{itemize}
\item
Consumers that consume the records through the Log Service.
\item
Consumers that consume the records by directly reading the file. 
\end{itemize}
The first type of consumers are those that are online. For example, the Log Reader or online Viewer or Archiever. 
The second type of consumers are offline viewers who may not be present in the cluster itself. The offline
consumers are out of scope of Log Service.



\begin{Desc}
\item [Log Service:] Framework that provides logging services to applications. Log Service is responsible for:
\begin{itemize}
\item Creation, maintenance, persistence, and closure of Log Streams.
\item 	Creation, maintenance, and closure of Log Files.
\item 	Propagating the filter criteria to all interested Log Service elements.
\item 	Filtering Log Records.
\item 	Providing Log Records for archival purpose.
\item 	Writing a Log Record to a specific Log Stream.
\end{itemize}
\end{Desc}

\begin{Desc}
\item [Logger:] A client of the Log Service that uses the interface provided by the Log Service to write the Log Records to specific Log Streams.
\begin{itemize}
\item 	Logger uses \textit{clLogInitialize()} API to initialize the log services for itself.
\item	Logger uses \textit{clLogStreamOpen()} API to gain access to a Log Stream.
\item	Logger can create a new Log Stream using \textit{clLogStreamOpen()} API.
\item	Logger uses \textit{clLogWriteRecord()} API to introduce a Log Record to a specific Log Stream.
\item	To avoid blocking, logger can use an asynchronous log write using the \textit{clLogWriteRecordAsync()} API.
\item	Logger has option of specifying the data in
	\begin{itemize}
\item TLV format
\item	C printf style
\item	Binary format
\end{itemize}
\item	Logger uses the \textit{clLogStreamClose()} API to terminate its relationship with the Log Stream.
\item	Logger uses the \textit{clLogFinalize()} API to terminate its relationship with the Log Service.
\end{itemize}
\end{Desc}


\begin{Desc}	
\item 
[LogStream:] A collection of similar Log Records or a conceptual flow of Log Records.
\begin{itemize}
\item	Log Streams have associated scope. They can be local to the node where only the loggers can access them, or they can be global 
to a cluster where any logger can access the Log Stream.
\item	A Log Stream is uniquely identified by its scope and its name.
\item	Each Log Stream has an associated Log File which is used for persisting.
\item	Log Service does not define any type of Log Streams.
\item	Currently, Log Service does not provide any pre-created Log Streams to the loggers.
\item	Log Streams are created when a call is made to \textit{clLogStreamOpen()/clLogSreamOpenAsync()} API with all the required parameters  
(stream attributes) and the Log Stream should not preexist. 
\item	Multiple loggers can use the same Log Stream and a logger can open multiple Log Streams.
\item	If all the loggers using a Log Stream invoke the \textit{clLogStreamClose()} API, the Log Service tries to persist the Log Stream 
and then destroys it.
\end{itemize}
\end{Desc}

\begin{Desc}	
\item
[Log File:] A collection of Log Streams stored on a permanent storage media.
\begin{itemize}
\item	Log Files can reside on any node having persistent storage.
\item	A Log File can be shared by multiple Log Streams.
\item	A Log File is created when the first Log Stream that specifies the Log File as its output destination is created. 
\item	A Log File can be uniquely identified by its name and location.
\item	All the Log Records in a Log File have a fixed length, which is specified at the time of creating the Log File.
\item	Maximum size of the Log File is fixed and is specified at creation time.
\item	When the maximum size of a Log File is reached, the action taken by Log Service is dependent on the policy specified at the creation time. The 
possible actions are:
\begin{itemize}
\item   Wrap: Overwrite the oldest records.
\item	Halt: No more records are allowed to be added.
\item	Rotate: Create new Log File.
\end{itemize}
\item	Each Log File has an associated Metadata file containing data regarding the Log File and Log Stream.
\item	A Log File closes when all the associated Log Streams are closed.
\end{itemize}
\end{Desc}



\begin{Desc}	
\item
[Log Record:] An ordered set of information provided by the logger. It is the basic unit that is written to the specified Log Stream.
\begin{itemize}
\item Log Service defines Log Record as having 2 parts:
\begin{itemize}
\item Header
\item	Data
\end{itemize}
\item The header part has a fixed format with the following fields:
\begin{itemize}
\item	Endianess
\item	Severity Level
\item	TimeStamp
\item	Context Identifier
\item	Service Identifier
\item	Stream Identifier
\item	Message Identifier
\end{itemize}
\item	Log Records can be written to a specific Log Stream using \textit{clLogWriteRecord()/clLogWriteRecordAsync()} API. Logger is required to provide the 
severity level, service identifier, and message identifier as the parameters. The remaining of the header fields are filled by the Log Service.
\item	Logger can specify the data part as a binary buffer, in printf style, or in TLV format.
\item	Two message identifiers are reserved:
\begin{itemize}
\item	0xFFFFFFFF: Implies that data is specified as binary buffer along with the length of buffer.
\item	0xFFFFFFFE: Implies the data is specified as C printf style format.  
\end{itemize}
\item	If the message identifier does not match with the mentioned two identifiers, then data is specified in TLV format.
\item	Log Service does not interpret the data. 
\item	Log Service does not define the format for the contents of the data.
\item	Logger and the Log Renderer should agree on the format but that is outside the scope of the Log Service.
\item	By design Log Service will try to minimize the logging latency.
\item	Log Service may not be able to log a record, if it is rejected due to filter or currently unavailable resources. For example, 
Log File is full.
\end{itemize}
\end{Desc}


\begin{Desc}
\item
[Archiver:] An entity that uses Log Service to archive a block of Log Records.
\begin{itemize}
\item Archiver uses \textit{clLogArchiveInitialze()} API to associate itself with a Log File to obtain the required data to retrieve records.
\item 	As part of initialization, Archiver is provided information regarding:
\begin{enumerate} \item The last record that was archived.
\item	The current record number.
\item	Associated meta data file size.
\end{enumerate}
\item	Archiver uses this information to create a buffer and invokes \textit{clLogGetRecords()} to obtain the records.
\item	Archiver uses \textit{clLogGetMetadata()} API to obtain the associated meta data.
\item	Archiver uses \textit{clLogArchiveFinalize()} API to update the archiving information maintained by the Log Service and 
to close the association with the Log File.
\end{itemize}
\end{Desc}


\begin{Desc}
\item
[LogFilter:] A pattern used by logger and Log Service to restrict entry of Log Records in the LogStreams.
\begin{itemize}
\item Filter criteria is set through administrative means. 
\item Log Service provides \textit{clLogFilterSet()} API to set the filter, this API is invoked on the owner Log Service element. 
\item Multicasting is used to propagate the filter criteria to other Log Service elements. The responsibility of the owner is to propagate the 
filter criteria changes.
\item The application obtains the filter criteria through \textit{clLogFilterSetCallback()} function.
\item Log Service can interpret only the header of the Log Record. So, filtering can be performed only on header fields. Currently, 
filtering is supported on:
\begin{itemize}
\item
Severity Levels 
\item
Context Identifier 
\item
Message Identifier
\end{itemize}
\end{itemize}
\end{Desc}





\chapter{Service APIs}

\section{Type Definitions}
\index{Type Definitions@{Type Definitions}}

\subsection{ClLogHandleT}
\index{ClLogHandleT@{ClLogHandleT}}
\textit{typedef ClHandleT ClLogHandleT;}
\newline
\newline
The type of the handle that Log Service provides to a process when the Log Service library is initialized. 
A process uses this handle when it invokes Log service functions so that the Log Service can recognize the process.


\subsection{ClLogStreamHandleT}
\index{ClLogStreamHandleT@{ClLogStreamHandleT}}
\textit{typedef ClHandleT ClLogStreamHandleT;}
\newline
\newline
The type of handle of a Log Stream that the Log Service provides to a process that has opened a stream for Logging or registered itself as a Handler for
a stream. A process uses this handle while invoking functions of Log Service API on a Log Stream.


\subsection{ClLogFileHandleT}
\index{ClLogFileHandleT@{ClLogFileHandleT}}
\textit{typedef ClHandleT ClLogFileHandleT;}
\newline
\newline
The type of handle of a Log File that the Log Service provides to a process that has opened a Log File for consuming Log Records.
A process uses this handle while invoking functions of Log Service API on a Log File.



\subsection{ClLogCallbacksT}
\index{ClLogCallbacksT@{ClLogCallbacksT}}
\begin{tabbing}
xx\=xx\=xx\=xx\=xx\=xx\=xx\=xx\=xx\=\kill
\textit{typedef struct \{}\\
\>\>\>\>\textit{ClLogStreamOpenCallbackT    clLogStreamOpenCallback;}\\
\>\>\>\>\textit{ClLogFilterSetCallbackT          clLogFilterSetCallback;}\\
\>\>\>\>\textit{ClLogWriteLogCallbackT         clLogStreamOpenCallback;}\\
\>\>\>\>\textit{ClLogRecordDeliverCallbackT	clLogRecordDeliverCallback;}\\
\textit{\} ClLogCallbacksT;}\end{tabbing}

The callback structure that the process provides to the Log Service. Currently, 
\textit{clLogWriteCallback} is not supported and setting this field does not 
make any difference.


\subsection{ClLogStreamScopeT}
\index{ClLogStreamScopeT@{ClLogStreamScopeT}}
\begin{tabbing}
xx\=xx\=xx\=xx\=xx\=xx\=xx\=xx\=xx\=\kill
\textit{typedef enum \{}\\
\>\>\>\>\textit{CL\_LOG\_STREAM\_GLOBAL = 0,}\\
\>\>\>\>\textit{CL\_LOG\_STREAM\_LOCAL = 1,}\\
\textit{\} ClLogStreamScopeT;}\end{tabbing}
The enumeration \textit{ClLogStreamScopeT} contains the scope of the Log Stream. You can specify the scope of a Log Stream using this enumeration.
Log Stream can be local to a node (Local Log Stream) or global to the cluster (Global Log Stream).


\subsection{ClLogFileFullActionT}
\index{ClLogFileFullActionT@{ClLogFileFullActionT}}
\begin{tabbing}
xx\=xx\=xx\=xx\=xx\=xx\=xx\=xx\=xx\=\kill
\textit{typedef enum\{}\\
\>\>\>\>\textit{CL\_LOG\_FILE\_FULL\_ACTION\_ROTATE = 0,}\\
\>\>\>\>\textit{CL\_LOG\_FILE\_FULL\_ACTION\_WRAP = 1,}\\
\>\>\>\>\textit{CL\_LOG\_FILE\_FULL\_ACTION\_HALT = 2,}\\
\textit{\} ClLogFileFullActionT;}\end{tabbing}
You can specify the action Log Service must take when the Log File becomes full using this enumeration. 
\begin{itemize}
\item \textit{CL\_\-LOG\_\-FILE\_\-FULL\_\-ACTION\_\-ROTATE} - Creates a new Log File Unit when the current Log File Unit 
becomes full. The maxFilesRotated attribute of the Log Stream specifies the number of maximum Log File Units that can simultaneously exist.
The Log service deletes the oldest Log File Unit and creates a new Log File Unit.
\item
\textit{CL\_\-LOG\_\-FILE\_\-FULL\_\-ACTION\_\-WRAP} - Treats the Log File as a circular buffer. Log Service starts overwriting the oldest records
when the Log File becomes full.
\item	\textit{CL\_\-LOG\_\-FILE\_\-FULL\_\-ACTION\_\-HALT} - Log Service stops inserting records into the
Log File.
\end{itemize}




\subsection{ClLogStreamAttributesT}
\index{ClLogStreamAttributesT@{ClLogStreamAttributesT}}
\begin{tabbing}
xx\=xx\=xx\=xx\=xx\=xx\=xx\=xx\=xx\=\kill
\textit{typedef struct\{}\\
\>\>\>\>\textit{ClCharT		*fileName;}\\
\>\>\>\>\textit{ClCharT		*filePath;}\\
\>\>\>\>\textit{ClUint32T		maxFileSize;}\\
\>\>\>\>\textit{ClUint32T		recordSize;}\\
\>\>\>\>\textit{ClBoolT		haProperty;}\\
\>\>\>\>\textit{ClFileFullActionT	fileFullAction;}\\
\>\>\>\>\textit{ClUint32T		maxFilesRotated;}\\
\>\>\>\>\textit{ClUint32T		flushFreq;}\\
\>\>\>\>\textit{ClTimeT		flushInterval;}\\
\>\>\>\>\textit{ClWatermarkT	highWatermark;}\\
\textit{\} ClLogStreamAttributesT;}\end{tabbing}} 
The structure \textit{ClLogStreamAttributesT}contains the attributes of a Log Stream. The fields of this structure have the following interpretation.
\begin{itemize}
\item
\textit{fileName} - Prefix for the names of the Log File Units to be created. The name of a Log File Unit is \tt{fileName_<creationTime>} where creation
time is the wall clock time when the File Unit is created. This name is also used to create the configuration file name fileName.cfg.
\item
\textit{filePath} - Path where Log File Units and configuration files are stored. This is a string with a pattern \tt{<nodeName>:<absolutePath>}, where
the <nodeName> is the logical node name in the cluster where the file is stored and <absolutePath> is the absolute path on that node. This path must
be pre-created. Log Service does not create any missing directories.
\item
\textit{maxFileSize} - Size of each Log File Unit for this Log Stream in bytes. If it is set to zero, there is no previously defined limit on 
Log File Unit size.
\item
\textit{recordSize} - Size of each Log Record in this Log Stream in bytes.
\item
\textit{haProperty} - If this attribute is set to \tt{CL\_\-TRUE}, the Log File where this Log Stream is persistent and must always be available. 
If it is set to \tt{CL\_\-FALSE} and if the node hosting this Log File fails, the Log File is not retained.
\item
\textit{fileFullAction} - Action that the Log Service should take when the Log File Unit becomes full. Refer to the enumeration 
\textit{ClLogFileFullActionT} for details on this field.
\item
\textit{maxFileRotated} - Indicates the number of Log File Units in a Log File if the fileFullAction is set to 
CL\_\-LOG\_\-FILE\_\-FULL\_\-ACTION\_\-ROTATE. If fileFullAction is set to any other value, this field is ignored. 
A value of zero indicates no previously defined upper limit on the number of Log File Units in the Log File.
\item
\textit{flushFreq} - Number of Log Records after which the Log Stream must be flushed. This is the maximum number of records on a node which are still
not persisted at any given point in time. Log Service guarantees that no more than flushFreq number of records will be lost if a node fails. A value of 
zero indicates that this field is ignored and Log Records are flushed based on flushInterval. At least one of flushFreq and flushInterval must be 
non-zero for a Log Stream.
\item
\textit{flushInterval} - Time in nanoseconds after which the Log Stream must be flushed. This is the maximum duration of time a Log Record can stay 
in a Log Stream without being persisted. Log Service guarantees that Log Records generated in \textit{flushInterval} time are not lost on a node failure. 
A value of zero indicates that this field is ignored and Log Records are flushed based on flushFreq. 
Either flushFreq or flushInterval must be non-zero for a Log Stream.
\item
\textit{highWatermark} - If the Log File size reaches the higher value of this limit, an event is published and an alarm is issued. The archiver 
combines the oldest Log Records into an archive to free up space in Log File before it becomes full. No new event is generated until the 
Log File utilization goes below lower value of this limit. Both the values are specified in terms of percentage of Log File size.




\subsection{ClLogStreamOpenFlagsT}
\index{ClLogStreamOpenFlagsT@{ClLogStreamOpenFlagsT}}
\textit{typedef ClUint8T ClLogStreamOpenFlagsT;}
\newline
\newline
\textit{ClLogStreamOpenFlagsT} specifies the open attributes used in clLogStreamOpen() and clLogStreamOpenAsync() while opening a Log Stream for Logging.
\newline 
\newline
\textit{\#define CL\_\-LOG\_\-STREAM\_\-CREATE 0x1} - A value or parameter of type ClLogStreamOpenFlagsT. It is either zero or bitwise OR of the following: 
\begin{itemize}
\item
\textit{CL\_\-LOG\_\-STREAM\_\-CREATE} - This flag requests the creation of a Log Stream if the identified Log Stream does not exist.
\end{itemize}
	


\subsection{ClLogStreamInfoT}
\index{ClLogStreamInfoT@{ClLogStreamInfoT}}
\begin{tabbing}
xx\=xx\=xx\=xx\=xx\=xx\=xx\=xx\=xx\=\kill
\textit{typedef struct\{}\\
\>\>\>\>\textit{ClNameT			streamName;}\\
\>\>\>\>\textit{ClLogStreamScopeT		streamSope;}\\
\>\>\>\>\textit{ClCharT			*pNodeName;}\\
\>\>\>\>\textit{ClStreamAttributesT	streamAttributes;}\\
\textit{\} ClLogStreamInfoT;}\end{tabbing}} 
The structure \textit{ClLogStreamInfoT} contains information about a Log Stream. The fields of this structure have the following interpretation:
\begin{itemize}
\item
\textit{streamName} - Name of the Log Stream that is unique within its namespace.
\item
\textit{streamScope} -Scope of the stream. It can take the values specified in \textit{ClLogStreamScopeT}.
\item
\textit{pNodeName} - Name of the node on which the Log Stream exists. This is valid if the streamScope is set to CL_LOG_STREAM_LOCAL. If 
streamScope is set to CL_LOG_STREAM_GLOBAL, this field is ignored. It is a string that terminates with NULL.
\item
\textit{streamAttributes} - Attributes of the Log Stream identified by 3-tuple streamName, streamScope and pNodeName.
\end{itemize}


\subsection{ClLogStreamHandlerFlagsT}
\index{ClLogStreamHandlerFlagsT@{ClLogStreamHandlerFlagsT}}
\textit{typedef ClUint8T ClLogStreamHandlerFlagsT;}
\newline
\newline
The following values specify the Log Handler options used in clLogStreamregister() while registering as a handler for a Log Stream.
\newline 
\newline
\textit{#define CL_LOG_HANDLER_WILL_ACK	0x1} - A value or parameter of type ClLogStreamHandlerFlagsT. It is either zero or bitwise OR of the following: 
\begin{itemize}
\item
\textit{CL_LOG_HANDLER_WILL_ACK} - This flag informs the Log Service that this handler acknowledges all the Log Records receives.
\end{itemize}




\subsection{ClLogSeverityT}
\index{ClLogSeverityT@{ClLogSeverityT}}
\textit{typedef ClUint8T ClLogSeverityT;}
\newline
\newline
\textit{ClLogSeverityT} specifies the log severity levels supported by the Log Service. It can take one of the following values:
\begin{itemize}
\item \textit{CL\_\-LOG\_\-SEV\_\-EMERGENCY 0x1}
	\item\textit{#define CL\_\-LOG\_\-SEV\_\-ALERT 0x2}
	\item\textit{#define CL\_\-LOG\_\-SEV\_\-CRITICAL 0x3}
	\item\textit{#define CL\_\-LOG\_\-SEV\_\-ERROR 0x4}
	\item\textit{#define CL\_\-LOG\_\-SEV\_\-WARNING 0x5}
	\item\textit{#define CL\_\-LOG\_\-SEV\_\-NOTICE 0x6}
	\item\textit{#define CL\_\-LOG\_\-SEV\_\-INFO 0x7}
	\item\textit{#define CL\_\-LOG\_\-SEV\_\-DEBUG 0x8}
	\item\textit{#define CL\_\-LOG\_\-SEV\_\-MAX 0x15}
\end{itemize}	


\subsection{ClLogSeverityFilterT}
\index{ClLogSeverityFilterT@{ClLogSeverityFilterT}}
\textit{typedef ClUint16T ClLogSeverityFilterT;}
\newline
\newline
A variable of this type is used as a bitmap. Values from ClLogSeverityT are used to set individual bits in this bitmap.




\subsection{ClLogFilterFlagsT}
\index{ClLogFilterFlagsT@{ClLogFilterFlagsT}}
\textit{typedef ClUint8T ClLogFilterFlagsT;}
\newline
\newline
\textit{ClLogFilterFlagsT} is used to determine if the new filter set is used to overwrite old filter or to
modify the old filter. It can take one of the following values:
\begin{itemize}
\item
\textit{#define CL_LOG_FILTER_ASSIGN 0x1} - Discards the old filter settings and uses the new settings.
\item
\textit{#define CL_LOG_FILTER_MERGE_ADD	0x2} - Adds to the old filter settings as per filter settings provided. This is achieved by ORing the new filter
with old filter.
\item
\textit{#define CL_LOG_FILTER_MERGE_DELETE 0x3} - Deletes the old filter settings as per filter settings provided. This is achieved by clearing the 
new filter bits from old filter.
\end{itemize}	


\subsection{ClLogFilterT}
\index{ClLogFilterT@{ClLogFilterT}}
\begin{tabbing}
xx\=xx\=xx\=xx\=xx\=xx\=xx\=xx\=xx\=\kill
\textit{typedef struct \{}\\
\>\>\>\>\textit{ClLogSeverityFilterT	logSeverityFilter;}\\
\>\>\>\>\textit{ClUint16T			logMsgIdSetLength;}\\
\>\>\>\>\textit{ClUint8T			*pLogMsgIdSet;}\\
\>\>\>\>\textit{ClUint16T			logCompIdSetLength;}\\
\>\>\>\>\textit{ClUint8T			*pLogCompIdSet;}\\
\textit{\} ClLogFilterT;}\end{tabbing}
The structure contains the filter settings. The members of this structure have the following interpretation:
\begin{itemize}
\item
\textit{logSeverityFilter} - This field identifies the severity levels to be masked. Each severity level has a bit identified in this
bitmap. These bits identify the severity levels which must be masked or prevented from entering the Log Stream.
\item
\textit{logMsgIdSetLength} - Number of octets in the memory chunk pointed by pLogMsgIdSet field.
\item
\textit{pLogMsgIdSet} - Pointer to the memory chunk which is treated as a bitmap and set bits identify the message IDs to be masked. Bit numbers
in this bitmap are same as the message ID.
\item
\textit{logCompIdSetLength} - Number of octets in the memory chunk pointed by pLogCompIdSet field.
\item
\textit{pLogCompIdSet} - Pointer to a memory chunk which is treated as a bitmap. The bits that are set identify the component IDs to be masked. 
The lower octet of the component ID identifies the class of the component and higher octet identifies the node on which 
it is running. Log Service sets the filter on the class of the component and so, only the lower octet is used. The bit number maps to the lower
octet. This implies that the Log Records from a class of components are turned on or off.
\end{itemize}	


















\subsection{ClLogServiceHandleT}
\textit{typedef ClHanldleT ClLog ServiceHandleT;}
\newline
\newline
The type of the handle of the log service. It is returned to the process when the log service is initialized.
It captures the Log Service and process association.
	


\subsection{ClLog FileHandleT}
\textit{typedef ClHanldleT ClLog FileHandleT;}
\newline
\newline
The type of the handle for a Log File.





\subsection{ClLogSeverityFlagsT}
\textit{typedef ClUint16T ClLogSeverityFlagsT;}
\newline
\newline
The type of the bitmap for filtering records.
\begin{itemize}
\item
\textit{CL\_\-LOG\_\-SEV\_\-<level>} identifies a bit position in bitmap to allow or disallow a Log Record of a specific severity.
\end{itemize}	


\subsection{ClLogAckFlagsT}
\textit{typedef ClUint32T ClLogAckFlagsT;}
\newline
\newline
The type of the acknowledgement required for \textit{clLogWriteRecordAsync} calls.
\newline 
\newline
\textit{\#define CL\_\-LOG\_\-RECORD\_\-WRITE\_\-ACK    0x1}
\newline 
\newline
\textit{CL\_\-LOG\_\-RECORD\_\-WRITE\_\-ACK} implies that the logging application requires an acknowledgement to confirm if the Log Service 
 wrote the Log Record successfully.







\subsection{ClLogFilterT}
\begin{tabbing}
xx\=xx\=xx\=xx\=xx\=xx\=xx\=xx\=xx\=\kill
\textit{typedef struct ClLogFileFullActionT\{}\\
\>\>\>\>\textit{ClLogSeverityFlagsT *logSeverityMap;}\\
\>\>\>\>\textit{ClUint32T                     *logMsgIdSet;}\\
\>\>\>\>\textit{ClUint32T                     *logCtxIdSet;}\\
\>\>\>\>\textit{ClUint8T                        logFilterFlags;}\\
\textit{\} ClLogFilter;}\end{tabbing}
The structure \textit{ClLogFilterT} contains the new filter criteria. A set of values is specified for filtering by message ID and context ID.
The context ID is used to identify the application. The \textit{logFilterFlags} determine the way in which the existing filtering criteria is merged
with the specified filtering criteria.
\begin{itemize}
\item
\textit{CL\_\-LOG\_\-FILTER\_\-ASSIGN} - Overwrite the previous values of the filter.
\item
\textit{CL\_\-LOG\_\-FILTER\_\-MERGE} - Merge the existing filtering criteria with the specified criteria.
\end{itemize}



\subsection{ClLogStreamCreateAttributesT}
\begin{tabbing}
xx\=xx\=xx\=xx\=xx\=xx\=xx\=xx\=xx\=\kill
\textit{typedef struct ClLogStreamCreateAttributes\{}\\
\>\>\>\>\textit{ClStringT                       logFileName;}\\
\>\>\>\>\textit{ClStringT                       logFileURL;}\\
\>\>\>\>\textit{ClUint32T                      maxLogFileSize;}\\
\>\>\>\>\textit{ClUint32T                      maxLog RecordSize;}\\
\>\>\>\>\textit{ClBoolT                          haProperty;}\\
\>\>\>\>\textit{ClLog FileFullActionT    logFileFullAction;}\\
\>\>\>\>\textit{ClUint16T                       maxFilesRotated;}\\
\>\>\>\>\textit{ClUint32T                       flushFreq;}\\
\>\>\>\>\textit{ClTimeT                          flushInterval;}\\
\>\>\>\>\textit{ClUint32T                       highWatermark;}\\
\textit{\} ClLogStreamCreateAttributesT;}\end{tabbing}
The structure \textit{ClLogStreamCreateAttributesT} contains the Log File information required to create a new Log file.
\begin{itemize}
\item \textit{logFileName} - POSIX file name of the desired Log File.
\item \textit{logFilePath} - URL specifying the location of the file.
\item	\textit{maxLog FileSize} - Maximum size of a Log File.
\item \textit{maxLog RecordSize} - Maximum size of the Log Record that can be written to this file. Bigger size records are truncated 
to this size and smaller size records are padded to this size.
\item \textit{haProperty} - This SAF specified attribute is currently ignored.
\item	\textit{logFileFullAction} - Specifies the Log Service behavior when the maximum size of a file is reached.
\item \textit{maxFilesRotated} - Specifies the number of files maintained at a time if the \textit{logFileFullAction} is 
\textit{CL\_\-LOG\_\-FILE\_\-FULL\_\-ACTION\_\-ROTATE}, otherwise ignored.
\item \textit{flushFreq} - Number of records after which flushing action should be performed.
\item \textit{flushInterval} - Time period after which flushing action should be performed.
\item \textit{highWatermark} - Notification is sent when the file size crosses the threshold.
\end{itemize}





\subsection{ClLogRecNumT}
\begin{tabbing}
xx\=xx\=xx\=xx\=xx\=xx\=xx\=xx\=xx\=\kill
\textit{typedef struct ClLogRecNum\{}\\
\>\>\>\>\textit{ClUint32T    fileNum;}\\
\>\>\>\>\textit{ClUint32T    recNum;}\\
\textit{\} ClLogRecNumT;}\end{tabbing}
The structure \textit{ClLogRecNumT} contains the record number. More than one file can exist, so, \textit{fileNum} indicates the file number and the
\textit{recNum} indicates the record number in the file.



\newpage
\section{Library Life Cycle Functions}
\subsection{clLogInitialize}
\index{clLogInitialize@{clLogInitialize}}
\hypertarget{pagelog101}{}\paragraph{cl\-Log\-Initialize}\label{pagelog101}
\begin{Desc}
\item[Synopsis:]Initializes the Log client library.\end{Desc}
\begin{Desc}
\item[Header File:] clLogApi.h \end{Desc}
\begin{Desc}
\item[Syntax:]

\footnotesize\begin{verbatim}        ClRcT clLogInitialize(
           							CL_OUT ClLogHandleT *logHandle,
           							CL_IN const ClLogCallbacksT *logCallbacks,
           							CL_INOUT ClVersionT *version
							  )

\end{verbatim}
\normalsize
\end{Desc}
\begin{Desc}
\item[Parameters:] \begin{description}
\item[{\em logHandle}]:(out) Handle obtained through a previous call to clLogInitialize.
\item[{\em logCallbacks}]:(in) Specifies callback function of the application that the Log Service may invoke. Only non-null callback functions in the 
structure are registered.
\item[{\em version}]:(in/out) As an input parameter, it points to the required Log Service version. As an output parameter, it is the version 
supported by the Log. For details on this datatype, refer to the chapter, Common Type Definitions in the \textit{OpenClovis\_\-ApiRef\_\-Overview-2.3.pdf}.

\end{description}
\end{Desc}
\begin{Desc}
\item[Return values:]
\begin{description}
\item[{\em CL\_\-OK}]: Function completed successfully.
\item[{\em CL\_\-LOG\_\-ERR\_\-VERSION\_\-MISMATCH}]: The version parameter is not compatible with the version of the Log Service implementation.
\item[{\em CL\_\-LOG\_\-ERR\_\-TIMEOUT}]: Timeout occurred before the call could complete.
\item[{\em CL\_\-LOG\_\-ERR\_\-TRY\_\-AGAIN}]: The service cannot be provided at this time. The process can try later.
\item[{\em CL\_\-LOG\_\-ERR\_\-INVALID\_\-PARAM}]: The parameter passed is invalid.
\item[{\em CL\_\-LOG\_\-ERR\_\-NO\_\-MEMORY}]: Out of memory. Unable to provide the service.
\item[{\em CL\_\-LOG\_\-ERR\_\-NO\_\-RESOURCE}]: Insufficient resources (other than memory) to provide the service.
\item[{\em CL\_\-LOG\_\-ERR\_\-NULL\_\-POINTER}]: Expecting pointer to valid memory space.
\end{description}
\end{Desc}
\begin{Desc}
\item[Description:]This function initializes the Log Service. It registers the callback functions if specified.
This function must be invoked before any Log Service function can be used. 
\end{Desc}
\begin{Desc}
\item[Library File:] libLogClient\end{Desc}
\begin{Desc}
\item[Related Function(s):]\hyperlink{pagelog102}{clLogSelectionObjectGet()}, \hyperlink{pagelog103}{clLogDispatch()}, 
\hyperlink{pagelog104}{clLogFinalize()}\end{Desc}
\newpage



\subsection{clLogSelectionObjectGet}
\index{clLogSelectionObjectGet@{clLogSelectionObjectGet}}
\hypertarget{pagelog102}{}\paragraph{cl\-Log\-Selection\-Object\-Get}\label{pagelog101}
\begin{Desc}
\item[Synopsis:] \end{Desc}
\begin{Desc}
\item[Header File:] clLogApi.h \end{Desc}
\begin{Desc}
\item[Syntax:]

\footnotesize\begin{verbatim}        	ClRcT clLogSelectionObjectGet(
           						CL_IN ClLogHandleT logHandle,
           						CL_OUT ClSelectionObjectT *selectionObject
								     )
\end{verbatim}
\normalsize
\end{Desc}
\begin{Desc}
\item[Parameters:] \begin{description}
\item[{\em logHandle}]:(in) Handle obtained through a previous call to clLogInitialize.
\item[{\em selectionObject}]:(out) Pointer to operating system handle that the application uses to detect the pending callbacks.
\end{description}
\end{Desc}
\begin{Desc}
\item[Return values:]
\begin{description}
\item[{\em CL\_\-OK}]: Function completed successfully.
\item[{\em CL\_\-LOG\_\-ERR\_\-INVALID\_\-HANDLE}]: The handle is invalid.
\item[{\em CL\_\-LOG\_\-ERR\_\-TRY\_\-AGAIN}]: The service cannot be provided at this time. The process can try later.
\item[{\em CL\_\-LOG\_\-ERR\_\-INVALID\_\-PARAM}]: The parameter passed is not correct.
\item[{\em CL\_\-LOG\_\-ERR\_\-NO\_\-MEMORY}]: Out of memory. Unable to provide the service.
\item[{\em CL\_\-LOG\_\-ERR\_\-NO\_\-RESOURCE}]: Insufficient resources (other than memory) to provide the service.
\item[{\em CL\_\-LOG\_\-ERR\_\-NULL\_\-POINTER}]: Expecting pointer to valid memory space.
\end{description}
\end{Desc}
\begin{Desc}
\item[Description:] This function returns the OS handle associated with the logHandle. In a POSIX environment, it is a file descriptor that can be used
with select() and poll() to detect pending callbacks. \textit{selectionObject} is valid until \textit{clLogFinalize()} is invoked .
\end{Desc}
\begin{Desc}
\item[Library File:] libLogClient\end{Desc}
\begin{Desc}
\item[Related Function(s):]\hyperlink{pagelog101}{clLogInitialize()}, \hyperlink{pagelog103}{clLogDispatch()}, \hyperlink{pagelog104}{clLogFinalize()}\end{Desc}
\newpage


\subsection{clLogDispatch}
\index{clLogDispatch@{clLogDispatch}}
\hypertarget{pagelog103}{}\paragraph{cl\-Log\-Dispatch}\label{pagelog101}
\begin{Desc}
\item[Synopsis:] \end{Desc}
\begin{Desc}
\item[Header File:] clLogApi.h \end{Desc}
\begin{Desc}
\item[Syntax:]
\footnotesize\begin{verbatim}        	ClRcT clLogDispatch(
           					CL_IN    ClLogHandleT        logHandle,
           					CL_IN    ClDispatchFlagsT    dispatchFlags
							   )
\end{verbatim}
\normalsize
\end{Desc}
\begin{Desc}
\item[Parameters:] \begin{description}
\item[{\em logHandle}]:(in) Handle obtained through a previous call to clLogInitialize.
\item[{\em dispatchFlags}]:(in) Flags specifying the callback execution behavior of  the \textit{clLogDispatch()} function. Value can be one
of the following:
\begin{itemize}
\item
\textit{CL\_\-DISPATCH\_\-ONE}
\item
\textit{CL\_\-DISPATCH\_\-ALL} or 
\item
\textit{CL\_\-DISPATCH\_\-BLOCKING}
\end{itemize}
\end{description}
\end{Desc}
\begin{Desc}
\item[Return values:]
\begin{description}
\item[{\em CL\_\-OK}]: Function completed successfully.
\item[{\em CL\_\-LOG\_\-ERR\_\-INVALID\_\-HANDLE}]: The handle is invalid.
\item[{\em CL\_\-LOG\_\-ERR\_\-TRY\_\-AGAIN}]: The service cannot be provided at this time. The process can try later.
\item[{\em CL\_\-LOG\_\-ERR\_\-INVALID\_\-PARAM}]: The parameter passed is not correct.
\item[{\em CL\_\-LOG\_\-ERR\_\-NO\_\-MEMORY}]: Out of memory. Unable to provide the service.
\item[{\em CL\_\-LOG\_\-ERR\_\-NO\_\-RESOURCE}]: Insufficient resources (other than memory) to provide the service.
\end{description}
\end{Desc}
\begin{Desc}
\item[Description:] This function invokes pending callbacks for the \textit{logHandle} in the context of the calling thread. This is 
specified by the \textit{dispatchFlags} parameter.
\end{Desc}
\begin{Desc}
\item[Library File:] libLogClient\end{Desc}
\begin{Desc}
\item[Related Function(s):]\hyperlink{pagelog101}{clLogInitialize()}, \hyperlink{pagelog102}{clLogSelectionObjectGet()}\end{Desc}
\newpage


\subsection{clLogFinalize}
\index{clLogFinalize@{clLogFinalize}}
\hypertarget{pagelog104}{}\paragraph{cl\-Log\-Finalize}\label{pagelog101}
\begin{Desc}
\item[Synopsis:]Finalizes the Log client library.\end{Desc}
\begin{Desc}
\item[Header File:] clLogApi.h \end{Desc}
\begin{Desc}
\item[Syntax:]
\footnotesize\begin{verbatim}        	
						ClRcT clLogFinalize(
           						CL_IN    ClLogHandleT        logHandle
								   )
\end{verbatim}
\normalsize
\end{Desc}
\begin{Desc}
\item[Parameters:] \begin{description}
\item[{\em logHandle}]:(in) Handle obtained through a previous call to clLogInitialize.
\end{description}
\end{Desc}
\begin{Desc}
\item[Return values:]
\begin{description}
\item[{\em CL\_\-OK}]: Function completed successfully.
\item[{\em CL\_\-LOG\_\-ERR\_\-INVALID\_\-HANDLE}]: The handle is invalid.
\item[{\em CL\_\-LOG\_\-ERR\_\-TRY\_\-AGAIN}]: The service cannot be provided at this time. The process can try later.
\item[{\em CL\_\-LOG\_\-ERR\_\-INVALID\_\-PARAM}]: The parameter passed is not correct.
\item[{\em CL\_\-LOG\_\-ERR\_\-NO\_\-MEMORY}]: Out of memory. Unable to provide the service.
\item[{\em CL\_\-LOG\_\-ERR\_\-NO\_\-RESOURCE}]: Insufficient resources (other than memory) to provide the service.
\item[{\em CL\_\-LOG\_\-ERR\_\-TIMEOUT}]: Timeout occurred before the call could complete.
\end{description}
\end{Desc}
\begin{Desc}
\item[Description:]This function closes the association between the application and the Log Service captured by \textit{logHandle}.
The application must invoke this function once for each handle acquired by calling \textit{clLogInitialize()}. This should be performed before 
the existing function releases all the resources acquired by \textit{clLogInitialize()}. \textit{clLogFinalize} closes all the Log Streams 
that are open for this \textit{logHandle} and cancels all pending callbacks related to \textit{logHandle}. Some callbacks might still be processed as
the callback invocation is asynchronous.
\end{Desc}
\begin{Desc}
\item[Library File:] libLogClient\end{Desc}
\begin{Desc}
\item[Related Function(s):]\hyperlink{pagelog101}{clLogInitialize()}\end{Desc}
\newpage


\section{LogStream Lifecycle APIs}
\subsection{clLogStreamOpen}
\index{clLogStreamOpen@{clLogStreamOpen}}
\hypertarget{pagelog105}{}\paragraph{cl\-Log\-Stream\-Open}\label{pagelog101}
\begin{Desc}
\item[Synopsis:] Opens the specified Log Stream   \end{Desc}
\begin{Desc}
\item[Header File:] clLogApi.h \end{Desc}
\begin{Desc}
\item[Syntax:]
\footnotesize\begin{verbatim}        	
					ClRcT clLogStreamOpen(
           					CL_IN ClLogHandleT logHandle,
           					CL_IN ClNameT logStreamName,
           					CL_IN ClLogStreamScopeT logStreamScope
           					CL_IN ClLogStreamCreateAttributesT *logStreamCreateAttributes,
           					CL_IN ClLogStreamOpenFlags logStreamOpenFlags,
           					CL_IN ClTimeT timeout,
           					CL_OUT ClLogStreamHandleT *logStreamHandle
							     )
\end{verbatim}
\normalsize
\end{Desc}
\begin{Desc}
\item[Parameters:] \begin{description}
\item[{\em logHandle}]:(in) Handle obtained through a previous call to clLogInitialize.
\item[{\em logStreamName}]:(in) Name of the Log Stream to open.
\item[{\em logStreamScope}]:(in) Stream can be local to node or global in cluster.
\item[{\em logStreamCreateAttributes}]:(in) Various attributes of the Log Stream, required at the time of Log file creation.
\item[{\em logStreamOpenFlags}]:(in) Specify if the Log Service should create the Log Stream, if it does not exist.
\item[{\em timeout}]:(in) \textit{clLogStreamOpen()} is considered to have failed, if it does not complete by the specified time.
\item[{\em logStreamHandleT}]:(out) Pointer to the Log Stream handle, allocated in the address space of the invoking application. 
Identifies the stream in subsequent Log Service function calls.
\end{description}
\end{Desc}
\begin{Desc}
\item[Return values:]
\begin{description}
\item[{\em CL\_\-OK}]: Function completed successfully.
\item[{\em CL\_\-LOG\_\-ERR\_\-INVALID\_\-HANDLE}]: The handle is invalid.
\item[{\em CL\_\-LOG\_\-ERR\_\-TRY\_\-AGAIN}]: The service cannot be provided at this time. The process can try later.
\item[{\em CL\_\-LOG\_\-ERR\_\-INVALID\_\-PARAM}]: The parameter passed is not correct.
\item[{\em CL\_\-LOG\_\-ERR\_\-NO\_\-MEMORY}]: Out of memory. Unable to provide the service.
\item[{\em CL\_\-LOG\_\-ERR\_\-NO\_\-RESOURCE}]: Insufficient resources (other than memory) to provide the service.
\item[{\em CL\_\-LOG\_\-ERR\_\-TIMEOUT}]: Timeout occurred before the call could complete.

\item[{\em CL\_\-LOG\_\-ERR\_\-NULL\_\-POINTER}]: Expecting pointer to valid memory space.
\item[{\em CL\_\-LOG\_\-ERR\_\-NOT\_\-EXISTS}]: Requested resource exists.
\item[{\em CL\_\-LOG\_\-ERR\_\-ALREADY\_\-EXIST}]: Request entry does not exist.
\item[{\em CL\_\-LOG\_\-ERR\_\-BAD\_\-FLAG}]: The passed flag is invalid.

\end{description}
\end{Desc}
\begin{Desc}
\item[Description:] This function opens the specified Log Stream. If the stream does not exist, then, the behavior is governed by \textit{logStreamOpenFlags}.
If the intent is to open an existing Log Stream, then, \textit{logStreamCreateAttributes} can be specified as NULL and flag should not be set.
If the application wants to create and open a Log Stream that does not exist, it should fill the \textit{logStreamCreateAttributes} and set the flag.
If the application wants to open a (possibly) existing Log Stream, but still specify creation attribute values, then provided values must be 
identical to those values given at the time of creation and set the flag.
\par
\par
\textit{clLogStreamOpen()} is blocking and on successful completion a new Log Stream handle is returned. 
A Log Stream can be opened multiple times from the same application or by different applications.
A Log Stream can be opened multiple times with the same \textit{logHandle}. In this case, a new stream handle is returned every time.
\end{Desc}
\begin{Desc}
\item[Library File:] libLogClient\end{Desc}
\begin{Desc}
\item[Related Function(s):]\hyperlink{pagelog108}{clLogStreamClose()}, \hyperlink{pagelog107}{ClLogStreamOpenCallbackT()}\end{Desc}
\newpage



\subsection{clLogStreamOpenAsync}
\index{clLogStreamOpenAsync@{clLogStreamOpenAsync}}
\hypertarget{pagelog106}{}\paragraph{cl\-Log\-Stream\-Open\-Async}\label{pagelog101}
\begin{Desc}
\item[Synopsis:] Opens the specified Log Stream asynchronously.    \end{Desc}
\begin{Desc}
\item[Header File:] clLogApi.h \end{Desc}
\begin{Desc}
\item[Syntax:]
\footnotesize\begin{verbatim}        	
					ClRcT clLogStreamOpenAsync(
					           CL_IN ClLogHandleT logHandle,
           					   CL_IN ClNameT logStreamName,
					           CL_IN ClLogStreamScopeT logStreamScope
					           CL_IN ClLogStreamCreateAttributesT *logFileCreateAttributes,
					           CL_IN ClLogStreamOpenFlags logStreamOpenFlags,
					           CL_IN ClInvocationT invocation
								  )
\end{verbatim}
\normalsize
\end{Desc}
\begin{Desc}
\item[Parameters:] \begin{description}
\item[{\em logHandle}]:(in) Handle obtained through a previous call to clLogInitialize.
\item[{\em logStreamName}]:(in) Name of the Log Stream to open.
\item[{\em logStreamScope}]:(in) Stream can be local to node or global in cluster.
\item[{\em logStreamCreateAttributes}]:(in) Various attributes of the Log Stream, required at the time of Log File creation.
\item[{\em logStreamOpenFlags}]:(in) Specify if the Log Service should create the Log Stream if it does not exists.
\item[{\em timeout}]:(in) clLogStreamOpen() is considered to have failed if it does not complete by the time specified.
\item[{\em invocation}]:(in) Allows the invoking application to match this call with the corresponding \textit{(*ClLogStreamOpenCallbackT)().}
\item[{\em logStreamHandleT}]:(out) Pointer to the Log Stream handle, allocated in the address space of the invoking application. 
Identifies the stream in subsequent Log Service function calls.
\end{description}
\end{Desc}
\begin{Desc}
\item[Return values:]
\begin{description}
\item[{\em CL\_\-OK}]: Function completed successfully.
\item[{\em CL\_\-LOG\_\-ERR\_\-INVALID\_\-HANDLE}]: The handle is invalid.
\item[{\em CL\_\-LOG\_\-ERR\_\-TRY\_\-AGAIN}]: The service cannot be provided at this time. The process can try later.
\item[{\em CL\_\-LOG\_\-ERR\_\-INVALID\_\-PARAM}]: The parameter passed is not correct.
\item[{\em CL\_\-LOG\_\-ERR\_\-NO\_\-MEMORY}]: Out of memory. Unable to provide the service.
\item[{\em CL\_\-LOG\_\-ERR\_\-NO\_\-RESOURCE}]: Insufficient resources (other than memory) to provide the service.
\item[{\em CL\_\-LOG\_\-ERR\_\-TIMEOUT}]: Timeout occurred before the call could complete.
\item[{\em CL\_\-LOG\_\-ERR\_\-NULL\_\-POINTER}]: Expecting pointer to valid memory space.
\item[{\em CL\_\-LOG\_\-ERR\_\-NOT\_\-EXISTS}]: Requested resource exists.
\item[{\em CL\_\-LOG\_\-ERR\_\-ALREADY\_\-EXIST}]: Request entry does not exist.
\item[{\em CL\_\-LOG\_\-ERR\_\-BAD\_\-FLAG}]: The passed flag is invalid.

\end{description}
\end{Desc}
\begin{Desc}
\item[Description:] This function opens the specified Log Stream. If the stream does not exist then the behavior is governed by \textit{logStreamOpenFlags}.
If the intent is to open an existing Log Stream then \textit{logStreamCreateAttributes} can be specified as NULL and flag should not be set.
If the application wants to create and open a Log Stream that does not exits, it should fill the \textit{logStreamCreateAttributes} and set the flag.
If the application wants to open a (possibly) existing Log Stream, but still specify creation attribute values, then provided values must be 
identical to those values given at the time of creation and set the flag.
\par
\textit{clLogStreamOpenAsync()} is non-blocking and the corresponding callback function \textit{ClLogStreamOpenCallbackT()} should be passed 
with \textit{clLogInitialize()}. \textit{Invocation} parameter compares the callback with the invocation. If the callback is not 
specified during initialize, an error is returned.
\end{Desc}
\begin{Desc}
\item[Library File:] libLogClient\end{Desc}
\begin{Desc}
\item[Related Function(s):]\hyperlink{pagelog108}{clLogStreamClose()}, \hyperlink{pagelog107}{ClLogStreamOpenCallbackT()}\end{Desc}
\newpage





\subsection{ClLogStreamOpenCallbackT}
\index{ClLogStreamOpenCallbackT@{ClLogStreamOpenCallbackT}}
\hypertarget{pagelog107}{}\paragraph{cl\-Log\-Stream\-Open\-Callback}\label{pagelog101}
\begin{Desc}
\item[Synopsis:].\end{Desc}
\begin{Desc}
\item[Header File:] clLogApi.h \end{Desc}
\begin{Desc}
\item[Syntax:]
\footnotesize\begin{verbatim}        	
					typedef  void (*ClLogStreamOpenCallbackT) (
					           			CL_IN ClInvocation invocation,
					           			CL_IN ClLogStreamHandleT logStreamHandle,
					           			CL_IN ClRcT error
										  )
\end{verbatim}
\normalsize
\end{Desc}
\begin{Desc}
\item[Parameters:] \begin{description}
\item[{\em invocation}]:(in) Parameter given by the application in the corresponding call of \textit{clLogStreamOpenAsync()}.
\item[{\em logStreamHandle}]:(in) Handle of the Log Stream opened.
\item[{\em error}]:(in) Status of the \textit{clLogStreamOpenAsync()} call.
\end{description}
\end{Desc}
\begin{Desc}
\item[Return values:]
\begin{description}
\item[{\em CL\_\-OK}]: Function completed successfully.
\item[{\em CL\_\-LOG\_\-ERR\_\-INVALID\_\-HANDLE}]: The handle is invalid.
\item[{\em CL\_\-LOG\_\-ERR\_\-TRY\_\-AGAIN}]: The service cannot be provided at this time. The process can try later.
\item[{\em CL\_\-LOG\_\-ERR\_\-INVALID\_\-PARAM}]: The parameter passed is not correct.
\item[{\em CL\_\-LOG\_\-ERR\_\-NO\_\-MEMORY}]: Out of memory. Unable to provide the service.
\item[{\em CL\_\-LOG\_\-ERR\_\-NO\_\-RESOURCE}]: Insufficient resources (other than memory) to provide the service.
\item[{\em CL\_\-LOG\_\-ERR\_\-TIMEOUT}]: Timeout occurred before the call could complete.
\item[{\em CL\_\-LOG\_\-ERR\_\-NOT\_\-EXISTS}]: Requested resource exists.
\item[{\em CL\_\-LOG\_\-ERR\_\-ALREADY\_\-EXIST}]: Request entry does not exist.
\end{description}
\end{Desc}
\begin{Desc}
\item[Description:] Log Service invokes this function on completion of \textit{clLogStreamOpenAsync()}. This function is called in the context of 
the thread calling \textit{clLogDispatch()}.
\end{Desc}
\begin{Desc}
\item[Library File:] libLogClient\end{Desc}
\begin{Desc}
\item[Related Function(s):]\hyperlink{pagelog101}{clLogInitialize()}, \hyperlink{pagelog103}{clLogDispatch()}, 
\hyperlink{pagelog106}{clLogStreamOpenAsync()}\end{Desc}
\newpage



\subsection{clLogStreamClose}
\index{clLogStreamClose@{clLogStreamClose}}
\hypertarget{pagelog108}{}\paragraph{cl\-Log\-Stream\-Close}\label{pagelog101}
\begin{Desc}
\item[Synopsis:]Closes the Log Stream. \end{Desc}
\begin{Desc}
\item[Header File:] clLogApi.h \end{Desc}
\begin{Desc}
\item[Syntax:]
\footnotesize\begin{verbatim}        	
					ClRcT clLogStreamClose(
					           CL_IN ClLogStreamHandleT logStreamHandle
							      )
\end{verbatim}
\normalsize
\end{Desc}
\begin{Desc}
\item[Parameters:] \begin{description}
\item[{\em logStreamHandle}]:(in) Parameter given by application in the corresponding call of \textit{clLogStreamOpenAsync()}.
\end{description}
\end{Desc}
\begin{Desc}
\item[Return values:]
\begin{description}
\item[{\em CL\_\-OK}]: Function completed successfully.
\item[{\em CL\_\-LOG\_\-ERR\_\-INVALID\_\-HANDLE}]: The handle is invalid.
\item[{\em CL\_\-LOG\_\-ERR\_\-TRY\_\-AGAIN}]: The service cannot be provided at this time. The process can try later.
\item[{\em CL\_\-LOG\_\-ERR\_\-INVALID\_\-PARAM}]: The parameter passed is not correct.
\item[{\em CL\_\-LOG\_\-ERR\_\-NO\_\-MEMORY}]: Out of memory. Unable to provide the service.
\item[{\em CL\_\-LOG\_\-ERR\_\-NO\_\-RESOURCE}]: Insufficient resources (other than memory) to provide the service.
\item[{\em CL\_\-LOG\_\-ERR\_\-TIMEOUT}]: Timeout occurred before the call could complete.
\end{description}
\end{Desc}
\begin{Desc}
\item[Description:] This function closes the Log Stream identified by \textit{logStreamHandle}. 
If the application writing to the stream is the final application, the Log File will be flushed and all the resources associated with the Log Stream 
will be released. It cancels all callbacks that refer to the \textit{logStreamHandle}. Some callbacks might still be processed as the callback
invocation is asynchronous.
\end{Desc}
\begin{Desc}
\item[Library File:] libLogClient\end{Desc}
\begin{Desc}
\item[Related Function(s):]\hyperlink{pagelog105}{clLogStreamOpen()}, \hyperlink{pagelog106}{ClLogStreamOpenAsync()}\end{Desc}
\newpage


\section{Archiver Lifecycle API's}
\subsection{clLogArchiveInit}
\index{clLogArchiveInit@{clLogArchiveInit}}
\hypertarget{pagelog109}{}\paragraph{cl\-Log\-Archive\-Init}\label{pagelog101}
\begin{Desc}
\item[Synopsis:].\end{Desc}
\begin{Desc}
\item[Header File:] clLogApi.h \end{Desc}
\begin{Desc}
\item[Syntax:]
\footnotesize\begin{verbatim}        	
						ClRcT clLogArchiveInit(
						           CL_IN ClLogHandleT logHandle,
						           CL_IN ClNameT logFileName,
						           CL_IN ClNameT logFileURL,
						           CL_OUT ClLogRecNumT *lastArchivedRecNum,
						           CL_OUT ClLogRecNumT *currRecNum,
						           CL_OUT CLUint32T metaDataLen;
						           CL_OUT ClLog FileHandleT *logFileHandle        
								       )
\end{verbatim}
\normalsize
\end{Desc}
\begin{Desc}
\item[Parameters:] \begin{description}
\item[{\em logHandle}]:(in) Handle obtained through a previous call to clLogInitialize.
\item[{\em logFileName}]:(in) Identifies the Log File.
\item[{\em logFileURL}]:(in) Specifies the Log File location.
\item[{\em lastArchivedRecNum}]:(in) Record number where the last archive operation was performed.
\item[{\em currRecNum}]:(out) Log File is filled till record number.
\item[{\em metaDataLen}]:(out) Current length of the associated meta data file.
\item[{\em logFileHandle}]:(out) Captures association of application with the file.
\end{description}
\end{Desc}
\begin{Desc}
\item[Return values:]
\begin{description}
\item[{\em CL\_\-OK}]: Function completed successfully.
\item[{\em CL\_\-LOG\_\-ERR\_\-INVALID\_\-HANDLE}]: The handle is invalid.
\item[{\em CL\_\-LOG\_\-ERR\_\-TRY\_\-AGAIN}]: The service cannot be provided at this time. The process can try later.
\item[{\em CL\_\-LOG\_\-ERR\_\-INVALID\_\-PARAM}]: The parameter passed is not correct.
\item[{\em CL\_\-LOG\_\-ERR\_\-NO\_\-MEMORY}]: Out of memory. Unable to provide the service.
\item[{\em CL\_\-LOG\_\-ERR\_\-NO\_\-RESOURCE}]: Insufficient resources (other than memory) to provide the service.
\item[{\em CL\_\-LOG\_\-ERR\_\-TIMEOUT}]: Timeout occurred before the call could complete.
\item[{\em CL\_\-LOG\_\-ERR\_\-NULL\_\-POINTER}]: Expecting pointer to valid memory space.
\end{description}
\end{Desc}
\begin{Desc}
\item[Description:] This function returns the file handle for further use along with the record range that is 
not archived and the size of Metadata file.
\end{Desc}
\begin{Desc}
\item[Library File:] libLogClient\end{Desc}
\begin{Desc}
\item[Related Function(s):]\hyperlink{pagelog101}{clLogInitialize()}, \hyperlink{pagelog115}{clLogGetRecords()}, \hyperlink{pagelog110}
{clLogArchiveFinalize}
\end{Desc}
\newpage



\subsection{clLogArchiveFinalize}
\index{clLogArchiveFinalize@{clLogArchiveFinalize}}
\hypertarget{pagelog110}{}\paragraph{cl\-Log\-Archive\-Finalize}\label{pagelog101}
\begin{Desc}
\item[Synopsis:]\end{Desc}
\begin{Desc}
\item[Header File:] clLogApi.h \end{Desc}
\begin{Desc}
\item[Syntax:]
\footnotesize\begin{verbatim}        	
						ClRcT clLogArchiveFinalize(
           						ClLog FileHandleT logFileHandle,
           						ClRecNumT archiveRecNum        
									  )
\end{verbatim}
\normalsize
\end{Desc}
\begin{Desc}
\item[Parameters:] \begin{description}
\item[{\em logFileHandle}]:(in) Handle identifying Log File.
\item[{\em archiveRecNum}]:(in) Record number at which the archiving is complete.
\end{description}
\end{Desc}
\begin{Desc}
\item[Return values:]
\begin{description}
\item[{\em CL\_\-OK}]: Function completed successfully.
\item[{\em CL\_\-LOG\_\-ERR\_\-INVALID\_\-HANDLE}]: The handle is invalid.
\item[{\em CL\_\-LOG\_\-ERR\_\-TRY\_\-AGAIN}]: The service cannot be provided at this time. The process can try later.
\item[{\em CL\_\-LOG\_\-ERR\_\-INVALID\_\-PARAM}]: The parameter passed is not correct.
\item[{\em CL\_\-LOG\_\-ERR\_\-NO\_\-MEMORY}]: Out of memory. Unable to provide the service.
\item[{\em CL\_\-LOG\_\-ERR\_\-NO\_\-RESOURCE}]: Insufficient resources (other than memory) to provide the service.
\item[{\em CL\_\-LOG\_\-ERR\_\-TIMEOUT}]: Timeout occurred before the call could complete.
\end{description}
\end{Desc}
\begin{Desc}
\item[Description:] This function updates Log Service data related to the archived record mark.
Applications can then safely overwrite the records till the archived record number.
This function closes the association between the process and the Log File.
\end{Desc}
\begin{Desc}
\item[Library File:] libLogClient\end{Desc}
\begin{Desc}
\item[Related Function(s):]\hyperlink{pagelog109}{clLogArchiveInit()}, \hyperlink{pagelog115}{clLogGetRecords()}\end{Desc}
\newpage


\section{Functional APIs}
\subsection{clLogWriteRecord,clLogWriteRecordAsync}
\index{clLogWriteRecord,clLogWriteRecordAsync@{clLogWriteRecord,clLogWriteRecordAsync}}
\hypertarget{pagelog111}{}\paragraph{cl\-Log\-Write\-Record}\label{pagelog101}
\begin{Desc}
\item[Synopsis:] Writes the Log Record into the stream.\end{Desc}
\begin{Desc}
\item[Header File:] clLogApi.h \end{Desc}
\begin{Desc}
\item[Syntax:]
\footnotesize\begin{verbatim}        	
				ClRcT clLogWriteRecord(
				           CL_IN    ClLogStreamHandleT    logStreamHandle,
				           CL_IN    ClTimeT                         timeout,
				           CL_IN    ClUint32T                       serviceId,
				           CL_IN    ClUint32T                       msgId,
				           . . .
				)
				ClRcT clLogWriteRecordAsync(
				           CL_IN    ClLogStreamHandleT    logStreamHandle,
				           CL_IN    ClInvocationT                 invocation,
				           CL_IN    ClLogAckFlagsT            ackFlags,
				           CL_IN    ClUint32T                       serviceId,
				           CL_IN    ClUint32T                       msgId,
				           . . .
				)

\end{verbatim}
\normalsize
\end{Desc}
\begin{Desc}
\item[Parameters:] \begin{description}
\item[{\em logStreamHandle}]:(in) Handle identifying the corresponding Log Stream.
\item[{\em timeout}]:(in) \textit{clLogWriteRecord()} is considered to have failed if it does not complete by the time specified.
\item[{\em invocation}]:(in) Allows the invoking application to match this call with the corresponding \textit{(*ClLogWriteRecordCallbackT)()}.
\item[{\em ackFlags}]:(in) If the caller requires acknowledgement then it should be set to \textit{CL\_\-LOG\_\-RECORD\_\-WRITE\_\-ACK}.
\item[{\em serviceId}]:(in) Part of the application identifier, application module identifier.
\item[{\em msgId}]:(in) Identifies the message. Two msgIds are reserved:
\begin{itemize}
\item
\textit{0xFFFFFFFF:} Specifies that length preceded binary buffer follows.
\item
\textit{0xFFFFFFFE:} Specifies C printf style format string and the corresponding value parameter follows.
\end{itemize}

\end{description}
\end{Desc}
\begin{Desc}
\item[Return values:]
\begin{description}
\item[{\em CL\_\-OK}]: Function completed successfully.
\item[{\em CL\_\-LOG\_\-ERR\_\-INVALID\_\-HANDLE}]: The handle is invalid.
\item[{\em CL\_\-LOG\_\-ERR\_\-TRY\_\-AGAIN}]: The service cannot be provided at this time. The process can try later.
\item[{\em CL\_\-LOG\_\-ERR\_\-INVALID\_\-PARAM}]: The parameter passed is not correct.
\item[{\em CL\_\-LOG\_\-ERR\_\-NO\_\-MEMORY}]: Out of memory. Unable to provide the service.
\item[{\em CL\_\-LOG\_\-ERR\_\-NO\_\-RESOURCE}]: Insufficient resources (other than memory) to provide the service.
\item[{\em CL\_\-LOG\_\-ERR\_\-TIMEOUT}]: Timeout occurred before the call could complete.
\item[{\em CL\_\-LOG\_\-ERR\_\-NULL\_\-POINTER}]: Expecting pointer to valid memory space.
\item[{\em CL\_\-LOG\_\-ERR\_\-BAD\_\-FLAG}]: The passed flag is invalid.
\item[{\em CL\_\-LOG\_\-ERR\_\-FILE\_\-FULL}]: The Log File has reached its maximum capacity.
\end{description}
\end{Desc}
\begin{Desc}
\item[Description:] This function writes the Log Record into the stream specified by \textit{logStreamHandle}.
\textit{clLogWriteRecord()} is blocking and successful return denotes that the Log has been written to the stream. No assurance is made 
on the persistence of it. \textit{clLogWriteRecordAsync()} is non-blocking and the completion can be optionally signalled by 
invocation of \textit{ClLogWriteRecordCallbackT()} if the \textit{ackFlags} is appropriately set.
If \textit{ackFlag} is set and callback is not specified, there is an error.
\newline Three cases of message identifier exists:
\begin{itemize}
\item
\textit{msgId == 0xFFFFFFFF}
\begin{enumerate}
\item
\textit{msgId} is followed by length preceded binary buffer.
 \footnotesize\begin{verbatim}
     ....,
     ClUint32T  msgId = OxFFFFFFFF, 
     ClUint32T  bufLen, 
     ClPtrT     buf)
      \end{verbatim}
      \normalsize
where
\item
\textit{bufLen}(in) is the length of buffer.
\item
\textit{buf}(in) is the binary buffer.
\end{enumerate}

\item
\textit{msgId == 0xFFFFFFFE}
\begin{enumerate}
\item
\textit{msgId} is followed by C printf style format string and the data value.
       \footnotesize\begin{verbatim}
      ...,
      ClUint32T  msgId = OxFFFFFFFE, 
      ClCharT      *format, 
      . . .)
      \end{verbatim}
            \normalsize
where
\item
format(in) C printf style format specifier
\end{enumerate}

\item other messageIDs
\begin{itemize}
\item
\textit{msgId} is followed by data in TLV format, that is, we have list of TAG, LENGTH, VALUE.
\item List is terminated with special tag value of 0xFFFF
\footnotesize\begin{verbatim}
..., 
ClUint16T  tag,
ClUint16T  length,
CPtrT          value,
. . .
CUint16T  tag = OxFFFF)
      \end{verbatim}
            \normalsize
\item
Log Service will not interpret the data
\end{itemize}
\end{itemize}
\par
Log Service provides a set of C MACROS for the ease of use of the service by the applications when logging using TLV format. 
Macros defined are as follows:
\begin{itemize}
\item
\textit{\#define CL\_\-LOG\_\-TLV\_\-CHAR(var)                          }
             \newline \textit{     APP\_\-TAG\_\-CHAR, sizeof(char), \&(var)}
\item	\textit{\#define CL\_\-LOG\_\-TLV\_\-SHORT(var)}                       
                  \newline \textit{APP\_\-TAG\_\-SHORT, sizeof(short), \&(var)}
\item	\textit{\#define CL\_\-LOG\_\-TLV\_\-INT(var)                               }
                  \newline \textit{APP\_\-TAG\_\-INT, sizeof(int), \&(var)}
\item	\textit{\#define CL\_\-LOG\_\-TLV\_\-LONG(var)                                     }
                  \newline \textit{APP\_\-TAG\_\-LONG, sizeof(long), \&(var)}
\item	\textit{\#define CL\_\-LOG\_\-TLV\_\-LONGLONG(var)                           }
                 \newline \textit{APP\_\-TAG\_\-LONGLONG, sizeof(long long), \&(var) }
\item	\textit{\#define CL\_\-LOG\_\-TLV\_\-FLOAT(var)                                      }
                  \newline \textit{APP\_\-TAG\_\-FLOAT, sizeof(float), \&(var)}
\item	\textit{\#define CL\_\-LOG\_\-TLV\_\-DOUBLE(var)                                  }
                  \newline \textit{APP\_\-TAG\_\-DOUBLE, sizeof(double), \&(var)}
\item	\textit{\#define CL\_\-LOG\_\-TLV\_\-STRING(str)                                     }
                  \newline \textit{APP\_\-TAG\_\-STRING, strlen(str) + 1, str}
\item	\textit{\#define CL\_\-LOG\_\-TLV\_\-GENERIC(tag, var)                           }
                   \newline \textit{tag, sizeof((var)), \&(var)}
    \end{itemize}
 where, \textit{var} is any variable of type T, 
 \newline for which \textit{\&(var)} is a pointer to type T, 
 \newline \textit{sizeof(var) == sizeof(T)} is the number of bytes the value occupies, 
 \newline \textit{str} is a variable of type pointer to char (char*),
 \newline the buffer it points to is NULL terminated,
 \newline \textit{APP\_\-TAG\_\-<type>} is the application specific tag defined for that type.

\end{Desc}
\begin{Desc}
\item[Library File:] libLogClient\end{Desc}
\begin{Desc}
\item[Related Function(s):]\hyperlink{pagelog105}{clLogStreamOpen()}, \hyperlink{pagelog106}{clLogStreamOpenAsync()}, 
\hyperlink{pagelog112}{ClLogWriteRecordCallbackT}, \hyperlink{pagelog108}{clLogStreamClose()}\end{Desc}
\newpage



\subsection{ClLogWriteRecordCallbackT}
\index{ClLogWriteRecordCallbackT@{ClLogWriteRecordCallbackT}}
\hypertarget{pagelog112}{}\paragraph{cl\-Log\-Write\-Record\-CallbackT}\label{pagelog101}
\begin{Desc}
\item[Synopsis:].\end{Desc}
\begin{Desc}
\item[Header File:] clLogApi.h \end{Desc}
\begin{Desc}
\item[Syntax:]
\footnotesize\begin{verbatim}        	
			typedef  void (*ClLogWriteRecordCallbackT) (
			           CL_IN    ClInvocation    invocation,
			           CL_IN    ClRcT              error
				)
\end{verbatim}
\normalsize
\end{Desc}
\begin{Desc}
\item[Parameters:] \begin{description}
\item[{\em invocation}]:(in) Parameter given by application in the corresponding call of \textit{clLogWriteLog RecordAsync()}.
\item[{\em error}]:(in) Status of the \textit{clLogWriteLog RecordAsync()} call.
\end{description}
\end{Desc}
\begin{Desc}
\item[Return values:]
\begin{description}
\item[{\em CL\_\-OK}]: Function completed successfully.
\item[{\em CL\_\-LOG\_\-ERR\_\-INVALID\_\-HANDLE}]: The handle is invalid.
\item[{\em CL\_\-LOG\_\-ERR\_\-TRY\_\-AGAIN}]: The service cannot be provided at this time. The process can try later.
\item[{\em CL\_\-LOG\_\-ERR\_\-INVALID\_\-PARAM}]: The parameter passed is not correct.
\item[{\em CL\_\-LOG\_\-ERR\_\-NO\_\-MEMORY}]: Out of memory. Unable to provide the service.
\item[{\em CL\_\-LOG\_\-ERR\_\-NO\_\-RESOURCE}]: Insufficient resources (other than memory) to provide the service.
\item[{\em CL\_\-LOG\_\-ERR\_\-TIMEOUT}]: Timeout occurred before the call could complete.
\item[{\em CL\_\-LOG\_\-ERR\_\-NULL\_\-POINTER}]: Expecting pointer to valid memory space.
\item[{\em CL\_\-LOG\_\-ERR\_\-BAD\_\-FLAG}]: The passed flag is invalid.
\item[{\em CL\_\-LOG\_\-ERR\_\-FILE\_\-FULL}]: The Log File has reached its maximum capacity.
\end{description}
\end{Desc}
\begin{Desc}
\item[Description:] Log Service invokes this function on writing the Log Record to the Log Stream if
\textit{ackFlags} are set appropriately when \textit{ClLogWriteRecordAsync()} is called.\end{Desc}
\begin{Desc}
\item[Library File:] libLogClient\end{Desc}
\begin{Desc}
\item[Related Function(s):]\hyperlink{pagelog101}{clLogInitialize()}, \hyperlink{pagelog103}{clLogDispatch()},
\hyperlink{pagelog111}{clLogWriteRecordAsync()}\end{Desc}
\newpage



\subsection{clLogFilterSet}
\index{clLogFilterSet@{clLogFilterSet}}
\hypertarget{pagelog113}{}\paragraph{cl\-Log\-Filter\-Set}\label{pagelog101}
\begin{Desc}
\item[Synopsis:].\end{Desc}
\begin{Desc}
\item[Header File:] clLogApi.h \end{Desc}
\begin{Desc}
\item[Syntax:]
\footnotesize\begin{verbatim}        	
		ClRcT clLogFilterSet(
		           CL_IN    ClLogStreamHandleT    logStreamHandle,
		           CL_IN    ClLogFilterT                  logFilter
		)

\end{verbatim}
\normalsize
\end{Desc}
\begin{Desc}
\item[Parameters:] \begin{description}
\item[{\em logStreamHandle}]:(in) Identifies the stream for which the filter is specified.
\item[{\em logFilter}]:(in) Specifies the filter criteria. Currently we will filter on severity level, 
context Id and the message Id fields of the Log Record.
\end{description}
\end{Desc}
\begin{Desc}
\item[Return values:]
\begin{description}
\item[{\em CL\_\-OK}]: Function completed successfully.
\item[{\em CL\_\-LOG\_\-ERR\_\-INVALID\_\-HANDLE}]: The handle is invalid.
\item[{\em CL\_\-LOG\_\-ERR\_\-TRY\_\-AGAIN}]: The service cannot be provided at this time. The process can try later.
\item[{\em CL\_\-LOG\_\-ERR\_\-INVALID\_\-PARAM}]: The parameter passed is not correct.
\item[{\em CL\_\-LOG\_\-ERR\_\-NO\_\-MEMORY}]: Out of memory. Unable to provide the service.
\item[{\em CL\_\-LOG\_\-ERR\_\-NO\_\-RESOURCE}]: Insufficient resources (other than memory) to provide the service.
\item[{\em CL\_\-LOG\_\-ERR\_\-TIMEOUT}]: Timeout occurred before the call could complete.
\item[{\em CL\_\-LOG\_\-ERR\_\-NULL\_\-POINTER}]: Expecting pointer to valid memory space.
\item[{\em CL\_\-LOG\_\-ERR\_\-BAD\_\-FLAG}]: The passed flag is invalid.
\end{description}
\end{Desc}
\begin{Desc}
\item[Description:] This function may be invoked as a consequence of an administrative operation to set the filter for a particular Log Stream.
The Log Server is responsible for propagating this filter criteria to all the log clients currently logging to this stream.
The flags attribute of log Filter determines how the specified filtering criteria is used. It determines if the filtering criteria should be merged
with the existing filtering criteria to create a superset or should the existing filtering criteria be ignored. If a filter is set successfully, the 
log service will not write records to a Log Stream that satisfies the filter criteria.

\end{Desc}
\begin{Desc}
\item[Library File:] libLogClient\end{Desc}
\begin{Desc}
\item[Related Function(s):]\hyperlink{pagelog114}{clLogFilterSetCallbackT()}\end{Desc}
\newpage




\subsection{ClLogFilterSetCallbackT}
\index{ClLogFilterSetCallbackT@{ClLogFilterSetCallbackT}}
\hypertarget{pagelog114}{}\paragraph{cl\-Log\-Filter\-Set\-CallbackT}\label{pagelog101}
\begin{Desc}
\item[Synopsis:].\end{Desc}
\begin{Desc}
\item[Header File:] clLogApi.h \end{Desc}
\begin{Desc}
\item[Syntax:]
\footnotesize\begin{verbatim}        	
			typedef  void (*ClLogFilterSetCallbackT) (
			           CL_IN    ClLogStreamHandleT    logStreamHandle,
			           CL_IN    ClLogFilterT                  logFilter
			)
\end{verbatim}
\normalsize
\end{Desc}
\begin{Desc}
\item[Parameters:] \begin{description}
\item[{\em logStreamHandle}]:(in) Identifies the stream on which the filter is set.
\item[{\em logFilter}]:(in) new Filter set for the specified stream. 
\end{description}
\end{Desc}
\begin{Desc}
\item[Return values:]
None.
\end{Desc}
\begin{Desc}
\item[Description:] Log Service invokes this callback to update the application with the current filter criteria.
The application does not log the records that fail the filter criteria. However, if the application tries to log 
such a record, the Log Service performs the filtering operation and rejects the records that do not pass through the filter successfully.
This function may be invoked as a consequence of an administrative operation to set the filter for a particular Log Stream.

\end{Desc}
\begin{Desc}
\item[Library File:] libLogClient\end{Desc}
\begin{Desc}
\item[Related Function(s):]\hyperlink{pagelog101}{clLogInitialize()}, \hyperlink{pagelog113}{clLogFilterSet()}\end{Desc}
\newpage






\subsection{clLogGetRecords}
\index{clLogGetRecords@{clLogGetRecords}}
\hypertarget{pagelog115}{}\paragraph{cl\-Log\-Get\-Records}\label{pagelog101}
\begin{Desc}
\item[Synopsis:]Copies the Log Records.\end{Desc}
\begin{Desc}
\item[Header File:] clLogApi.h \end{Desc}
\begin{Desc}
\item[Syntax:]
\footnotesize\begin{verbatim}        	
			ClRcT clLogGetRecords(
			           CL_IN ClLog FileHandleT logFileHandle,
			           CL_IN ClLogRecNumT startRecNum,
			           CL_OUT ClLogRecNumT *recCount,
			           CL_IN ClUint32T bufLen,
			           CL_INOUT ClUint8T *buf
			)
\end{verbatim}
\normalsize
\end{Desc}
\begin{Desc}
\item[Parameters:] \begin{description}
\item[{\em logFileHandle}]:(in) Handle identifying the Log File.
\item[{\em startRecNum}]:(in) Records starting from this record number are required by the application.
\item[{\em recCount}]:(out) Number of records the service is returning.
\item[{\em bufLen}]:(in) Length of the buffer given by application to copy records.
\item[{\em buf}]:(in/out) Application supplied buffer to which the Log Service copies the records.
\end{description}
\end{Desc}
\begin{Desc}
\item[Return values:]
\begin{description}
\item[{\em CL\_\-OK}]: Function completed successfully.
\item[{\em CL\_\-LOG\_\-ERR\_\-INVALID\_\-HANDLE}]: The handle is invalid.
\item[{\em CL\_\-LOG\_\-ERR\_\-TRY\_\-AGAIN}]: The service cannot be provided at this time. The process can try later.
\item[{\em CL\_\-LOG\_\-ERR\_\-INVALID\_\-PARAM}]: The parameter passed is not correct.
\item[{\em CL\_\-LOG\_\-ERR\_\-NO\_\-MEMORY}]: Out of memory. Unable to provide the service.
\item[{\em CL\_\-LOG\_\-ERR\_\-NO\_\-RESOURCE}]: Insufficient resources (other than memory) to provide the service.
\item[{\em CL\_\-LOG\_\-ERR\_\-TIMEOUT}]: Timeout occurred before the call could complete.
\item[{\em CL\_\-LOG\_\-ERR\_\-NULL\_\-POINTER}]: Expecting pointer to valid memory space.
\end{description}
\end{Desc}
\begin{Desc}
\item[Description:] Function will copy the number of records that can fit into the buffer limited by the buffer length. \end{Desc}
\begin{Desc}
\item[Library File:] libLogClient\end{Desc}
\begin{Desc}
\item[Related Function(s):]\hyperlink{pagelog109}{clLogArchiveInit()}, \hyperlink{pagelog110}{clLogArchiveFinalize()}\end{Desc}
\newpage


\subsection{clLogGetMetadata}
\index{clLogGetMetadata@{clLogGetMetadata}}
\hypertarget{pagelog116}{}\paragraph{cl\-Log\-Get\-Metadata}\label{pagelog101}
\begin{Desc}
\item[Synopsis:] Copies the Meta data.\end{Desc}
\begin{Desc}
\item[Header File:] clLogApi.h \end{Desc}
\begin{Desc}
\item[Syntax:]
\footnotesize\begin{verbatim}        	
			ClRcT clLogGetMetadata(
			           CL_IN ClLog FileHandleT logFileHandle,
			           CL_IN ClUint32T bufLen,
			           CL_INOUT ClUint8T *buf
			)

\end{verbatim}
\normalsize
\end{Desc}
\begin{Desc}
\item[Parameters:] \begin{description}

\item[{\em logFileHandle}]:(in) Handle identifying the Log File whose meta data is required by the application.
\item[{\em startRecNum}]:(in) Length of the buffer given by application to copy the meta data.
\item[{\em recCount}]:(in/out) Application supplied buffer to which the Log Service copies the data.
\end{description}
\end{Desc}
\begin{Desc}
\item[Return values:]
\begin{description}
\item[{\em CL\_\-OK}]: Function completed successfully.
\item[{\em CL\_\-LOG\_\-ERR\_\-INVALID\_\-HANDLE}]: The handle is invalid.
\item[{\em CL\_\-LOG\_\-ERR\_\-TRY\_\-AGAIN}]: The service cannot be provided at this time. The process can try later.
\item[{\em CL\_\-LOG\_\-ERR\_\-INVALID\_\-PARAM}]: The parameter passed is not correct.
\item[{\em CL\_\-LOG\_\-ERR\_\-NO\_\-MEMORY}]: Out of memory. Unable to provide the service.
\item[{\em CL\_\-LOG\_\-ERR\_\-NO\_\-RESOURCE}]: Insufficient resources (other than memory) to provide the service.
\item[{\em CL\_\-LOG\_\-ERR\_\-TIMEOUT}]: Timeout occurred before the call could complete.
\item[{\em CL\_\-LOG\_\-ERR\_\-NULL\_\-POINTER}]: Expecting pointer to valid memory space.
\end{description}
\end{Desc}
\begin{Desc}
\item[Description:] Function will copy the Meta data that can fit in the buffer limited by buffer length. \end{Desc}
\begin{Desc}
\item[Library File:] libLogClient\end{Desc}
\begin{Desc}
\item[Related Function(s):]\hyperlink{pagelog109}{clLogArchiveInit()}, \hyperlink{pagelog110}{clLogArchiveFinalize()}\end{Desc}
\newpage



\chapter{Service Notifications}
\index{Service Notifications@{Service Notifications}}
\section{Alarms}
\subsection{Log Service Impaired}
This alarm is sent when the Log Service becomes unusable due to reasons including:
\begin{itemize}
\item	Unavailability of memory.
\item	Unavailability of other resources other than memory.
\item	Communication failures.
\end{itemize}

\subsection{Capacity HighWaterMark Crossed}
This is used to indicate if the Log File is reaching its capacity.
\begin{itemize}
\item	Sent when Log FileFullAction has come to a halt and the file is full.
\item	Configured HighWatermark for capacity is crossed.
\end{itemize}

\subsection{LogStream Impacted}
 This alarm is sent when the Log Stream is encountering problems such as:
\begin{itemize}
\item	Log Service element is not able to complete memory mapping operation.
\item	Log Records are lost.
\item	Failure occurs in some internal operations.
\end{itemize}



\section{Other Notifications}
\subsection{LogStreamCreate}
This notification is sent when a new stream is created.
\subsection{Log ServiceDelete}
This notification is sent when the last logger using the stream closes the stream.


\section{Logs}
The Log Streams in which logs are written. This is bootstrapping issue: where does the Log Service log. We rely on alarms to report error and other 
conditions requiring attention. Once the Log Services are operations we can make use of itself for logging purpose. 
 


\chapter*{Glossary}
\index{Glossary@{Glossary}}
\begin{Desc}
\item[Glossary of Log Service Terms:]
\par
\par
\begin{description}

\item[Archiver] Program/entity that extracts records from the Log File and archives them.

\item[Global Log Stream] A Log Stream visible in the cluster. Any component in the cluster can log into this stream.

\item[Local Log Stream] A Log Stream local to one node. This stream is not visible on any other node in the system. Components running on the same
node only can log into this stream.

\item[Log Client] Part of the Log Service linked to the Logger. This provides Log Service interface to the Logger.

\item[Log Service] ASP Service that provides logging facility for various applications and ASP components in the cluster.

\item[Logger] Program using Log Service to write the Log Records or events of intrest into the Log Streams.

\item[Log Record] Unit of information provided by the logger to be recorded in a particular Log Stream. One unit of information related to an event.
This is an ordered set of information. Log Record has two parts - header and user data. 
Header contains meta-information regarding the event and data part contains the actual information.

\item[Log Server] Log Service component maintaining Log Streams and Log Files.

\item[Log Stream] It is a conceptual flow of Log Records. Every Log Stream has a theme associated with it, based on which the viewer can decide 
whether the events of interest are available in this stream on not. Streams have names for addressing.

\item[Log File] A destination for Log Stream and persistent storage for Log Records.

\item[Log Configuration File] Persistent storage for the configuration information of a Log File. Configuration of all Log Streams moving 
into a specific Log File is identical. So, a single copy of such configuration per Log File needs to be maintained.
A collection of Log Streams can flow into one Log File for the ease of management of data.

\item[Local Log File] Log File that is collocated with the Logger on the same node.

\item[Remote Log File] Log File that may be on a different node than the Logger.

\item[Log Service Element] One process per ASP enabled node in the cluster that provides part of the functionality of the Log Service. All the Log 
Service Elements along with Log Clients form the Log Service. This element is owner for all the Local Log Streams on its node.

\item[Master Log Service Element] Log Service Element running on the System Controller node. This element is the owner of all the Global Log Streams 
in the cluster.

\item[Log Viewer] Application used for viewing the logged records. This is not part of Log Service.

\item[Archiever] Application used for siphoning off the logged records to another form of persistent storage. This is not part of Log Service.

\item[Log Reader] Application that programmatically consumes the logged records. Typical readers are applications looking for missing 
Notifications/Events.

\item[Filter] A pattern that controls the logging of records. Only the Log Records matching this pattern are actually logged into the Log Stream.

\item[TLV format] \textit{TAG LENGTH VALUE} format. One of the formats in which data can be specified. 
\begin{itemize}
\item
\textit{Tag} identifies the data type.
\item
\textit{Length} specifies the number of bytes in the value field.
\item
\textit{Value} indicates the data.
\end{itemize}

\item[MetaData File] File maintaining the data related to Log Records and Log Files.

\end{description}
\end{Desc}



\end{flushleft}