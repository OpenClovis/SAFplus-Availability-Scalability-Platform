\hypertarget{group__group6}{
\section{High Availibility}
\label{group__group6}\index{High Availibility@{High Availibility}}
}
