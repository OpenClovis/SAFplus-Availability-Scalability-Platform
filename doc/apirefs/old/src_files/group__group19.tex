\hypertarget{group__group19}{
\chapter{Functional Overview}
\label{group__group19}
}

\begin{flushleft}
The OpenOpenClovis Execution Object (EO) encapsulates each distinct OpenOpenClovis ASP aware software component and provides an execution environment for
the components. It provides a uniform interface between the software component and the rest of the system components. The interfaces fall into the
following two categories:
\begin{itemize}
\item Management Interface - This interface is used to control and configure the software components. 
\item Service Interface - This interface allows software components to expose component specific functionality. \par
 \par
 Both management and service interfaces are exposed using RMD APIs. EO provides threads for receiving RMD messages and worker threads to process them. It
 provides an execution environment, required by a software component, to the component user and component manager. \par
 \par
 The OpenClovis product suite provides a process of integrating a third party software component with OpenClovis ASP. This process is known as
 Componentization. Using Componentization, both management and service interfaces are exposed through RMD. \par
 \par
 Componentization provides the following functionality: 
 \begin{itemize}
 \item Component re-start
 \item Service Migration 
 \item Location Transparency 
 \item Easy debugging, statistics gathering, and profiling 
\end{itemize}
Componentization helps in features such as: 
\begin{itemize}
 \item Resource Management 
 \item Component start, stop, and restart 
 \item Debugging
 \end{itemize}
 \end{itemize}
EO communicates to other components using the OpenClovis Communication Core
components such as Event Manager (EM), Remote Method Dispatch (RMD), Intelligent Object Communication (IOC), and Name Service. 


\chapter{Service Model}
TBD

\chapter{Service APIs}

\section{Type Definitions}

\subsection{clEoExecutionObj}
\index{clEoExecutionObj@{clEoExecutionObj}}
\begin{tabbing}
xx\=xx\=xx\=xx\=xx\=xx\=xx\=xx\=xx\=\kill
\textit{typedef struct \{}\\
\>\>\>\\textit{ClEoAppCreateCallbackT clEoCreateCallout;}\\
\>\>\>\>\textit{ClEoAppDeleteCallbackT clEoDeleteCallout;}\\
\>\>\>\>\textit{ClEoAppHealthCheckCallbackT clEoHealthCheckCallout;}\\
\>\>\>\>\textit{ClEoAppStateChgCallbackT clEoStateChgCallout;}\\
\>\>\>\>\textit{ClIocCommPortHandleT commObj;}\\
\>\>\>\>\textit{ClEoIdT eoID;}\\
\>\>\>\>\textit{ClUint32T eoInitDone;}\\
\>\>\>\>\textit{ClOsalMutexIdT eoMutex;}\\
\>\>\>\>\textit{ClIocPortT eoPort;}\\
\>\>\>\>\textit{ClUint32T eoSetDoneCnt;}\\
\>\>\>\>\textit{ClCntHandleT eoTaskIdInfo;}\\
\>\>\>\>\textit{ClUint32T maxNoClients;}\\
\>\>\>\>\textit{ClCharT name \mbox{[}CL\_EO\_MAX\_NAME\_LEN\mbox{]};}\\
\>\>\>\>\textit{ClUint32T noOfThreads;}\\
\>\>\>\>\textit{ClEoClientObjT *pClient;}\\
\>\>\>\>\textit{ClCntHandleT pEOPrivDataHdl;}\\
\>\>\>\>\textit{ClOsalThreadPriorityT pri;}\\
\>\>\>\>\textit{ClUint32T refCnt;}\\
\>\>\>\>\textit{ClRmdObjHandleT rmdObj;}\\
\>\>\>\>\textit{ClEoStateT state;}\\
\>\>\>\>\textit{ClUint32T threadRunning;}\\
\textit{\} clEoExecutionObj;}\end{tabbing}
 The structure, {\tt{clEoExecutionObj}}, contains the properties of an EO execution object.
 These properties constitute the properties of running OS thread or process.
 \begin{itemize}
 \item
 \textit{appType} - Indicates if the application needs the main thread.
\item \textit{clEoCreateCallout} - Application function that is called from {\tt{main()}} during the initialization process.
\item \textit{clEoDeleteCallout} - Application function that is called when the EO is terminated.
\item \textit{clEoHealthCheckCallout} - Application function that is called when EO health check is
performed by CPM.
\item \textit{clEoStateChgCallout} - Application function that is called when the EO is moved into the suspended
state.
\item \textit{commObj} - EO communication object.
\item \textit{eoID} - Unique EOID of a blade.
\item \textit{eoInitDone} - This indicates if {\tt{EOInit()}} has been called.
\item \textit{eoMutex} - Mutex that is used to protect the Execution Object.
\item \textit{eoPort} - Requested IOC Communication Port.
\item \textit{eoSetDoneCnt} - Used to set state related flag and counter.
\item \textit{eoTaskIdInfo} - TaskID information of receive loop. It is used to delete the EO.
\item \textit{maxNoClients} - Maximum number of EO clients.
\item \textit{name\mbox{[}CL\_\-EO\_\-MAX\_\-NAME\_\-LEN\mbox{]}} - Execution object name.
\item \textit{noOfThreads} - Number of RMD threads spawned.
\item \textit{pClient} - Pointer to EO client functions.
\item \textit{pEOPrivDataHdl} - Handle of the container of EO specific data.
\item \textit{pri} - Priority of the EO threads where RMD is executed.
\item \textit{rmdObj} - RMD object associated with the state of the EO.
\item \textit{threadRunning} - State of the receive loop thread.
\end{itemize}



\subsection{ClEoProtoListT}
\index{ClEoProtoListT@{ClEoProtoListT}}
\begin{tabbing}
xx\=xx\=xx\=xx\=xx\=xx\=xx\=xx\=xx\=\kill
\textit{typedef struct \{}\\
\>\>\>\textit{ClUint8T            protoID;}\\
\>\>\>\textit{ClInt8T             name[20];}\\
\>\>\>\textit{ClEoProtoCallbackT   func;}\\
\textit{\}ClEoProtoListT;}
\end{tabbing}
The structure, {\tt{ClEoProtoListT}}, contains the list of protocols registered with EO. The attributes of this structure are:
\begin{itemize}
\item
\textit{protoID} - ID of the protocol being registered.
\item
\textit{name} - Name of the protocol being registered.
\item
\textit{func} - Receive function of the protocol.
\end{itemize}



\subsection{ClEoSchedFeedBackT}
\index{ClEoSchedFeedBackT@{ClEoSchedFeedBackT}}
\begin{tabbing}
xx\=xx\=xx\=xx\=xx\=xx\=xx\=xx\=xx\=\kill
\textit{typedef struct \{}\\
\>\>\>\textit{clEoPollingTypeT   freq;}\\
\>\>\>\textit{ClRcT              status;}\\
\textit{\}ClEoSchedFeedBackT;}
\end{tabbing}
The structure, {\tt{ClEoSchedFeedBackT}}, contains the feedback sent by the software component being polled in response to
heartbeat, {\tt{(is-Alive)}}. The attributes of this structure are:
\begin{itemize}
 \item
 \textit{freq} - Indicates the polling type {\tt{clEoPollingTypeT}}.
 \item
 \textit{status} - Indicates the health of the EO.
 \end{itemize}


\subsection{ClEoServiceObjT}
\index{ClEoServiceObjT@{ClEoServiceObjT}}
\begin{tabbing}
xx\=xx\=xx\=xx\=xx\=xx\=xx\=xx\=xx\=\kill
\textit{typedef struct clEoServiceObj \{}\\
\>\>\>\textit{void                    (*func)();}\\
\>\>\>\textit{struct clEoServiceObj   *pNextServObj;}\\
\textit{\}ClEoServiceObjT;}\end{tabbing}
The structure, {\tt{ClEoServiceObjT}}, contains the EO service object. The attributes of this structure are:
\begin{itemize}
 \item
 \textit{void (*func)()} - Pointer to the client service function.
 \item
 \textit{*pNextServObj} - Pointer to the next service on the same service ID.
\end{itemize}




\subsection{ClEoClientObjT}
\index{ClEoClientObjT@{ClEoClientObjT}}
\begin{tabbing}
xx\=xx\=xx\=xx\=xx\=xx\=xx\=xx\=xx\=\kill
\textit{typedef struct \{}\\
\>\>\>\textit{ClEoServiceObjT     funcs[CL\_EO\_MAX\_NO\_FUNC];}\\
\>\>\>\textit{ClEoDataT           data;}\\
\textit{\} ClEoClientObjT;}\end{tabbing}
This structure, {\tt{ClEoClientObjT}}, contains the pointer to the callback functions to be
registered with EO, and the data specific to the client. The attributes of this structure are:
\begin{itemize}
 \item
 \textit{funcs[CL\_\-EO\_\-MAX\_\-NO\_\-FUNC]} - Pointer to EO functions.
 \item
 \textit{data} - Data that is specific to the client
 \end{itemize}


\subsection{ClEoConfigT}
\index{ClEoConfigT@{ClEoConfigT}}
\begin{tabbing}
xx\=xx\=xx\=xx\=xx\=xx\=xx\=xx\=xx\=\kill
\textit{typedef struct \{}\\
\>\>\>\textit{ClCharT                EOname[CL\_EO\_MAX\_NAME\_LEN];}\\
\>\>\>\textit{ClOsalThreadPriorityT   pri;}\\
\>\>\>\textit{ClUint32T               noOfThreads;}\\
\>\>\>\textit{ClIocPortT              reqIocPort;}\\
\>\>\>\textit{ClUint32T               maxNoClients;}\\
\>\>\>\textit{ClEoApplicationTypeT    appType;}\\
\>\>\>\textit{ClEoAppCreateCallbackT  clEoCreateCallout;}\\
\>\>\>\textit{ClEoAppDeleteCallbackT  clEoDeleteCallout;}\\
\>\>\>\textit{ClEoAppStateChgCallbackT    clEoStateChgCallout;}\\
\>\>\>\textit{ClEoAppHealthCheckCallbackT clEoHealthCheckCallout;}\\
\>\>\>\textit{ClEoCustomActionT clEoCustomAction;}\\
\textit{\} ClEoConfigT;}\end{tabbing}
The structure, {\tt{ClEoConfigT}}, contains the configuration parameters related to the EO and is passed to the {\tt{clEoCreate}} function.
\begin{itemize}
 \item
 \textit{EOname[CL\_\-EO\_\-MAX\_\-NAME\_\-LEN]} - EO name.
 \item
 \textit{pri} - EO thread priority.
 \item
 \textit{noOfThreads} - Number of RMD threads.
 \item
 \textit{reqIocPort} - Requested IOC communication port.
 \item
 \textit{maxNoClients} - Maximum number of EO clients.
  \item
 \textit{appType} - Indicates if the application needs the main thread.
 \item
 \textit{clEoCreateCallout} - Application function that is called from {\tt{main()}} during the
 initialization process.
\item
\textit{clEoDeleteCallout} - Application function that is called when EO needs to be terminated.
\item
\textit{clEoStateChgCallout} - Application function that is called when EO enters suspended state.
\item
\textit{clEoHealthCheckCallout} - Application function that is called when EO health check is
performed by Component Manager.
\item
\textit{clEoCustomAction} - Application function that is called when a Water Mark is reached.
\end{itemize}


\subsection{ClEoStateT}
\index{ClEoStateT@{ClEoStateT}}
\begin{tabbing}
xx\=xx\=xx\=xx\=xx\=xx\=xx\=xx\=xx\=\kill
\textit{typedef enum \{}\\
\>\>\>\textit{CL\_EO\_STATE\_INIT        = 0x1,}\\
\>\>\>\textit{CL\_EO\_STATE\_ACTIVE      = 0x2,}\\
\>\>\>\textit{CL\_EO\_STATE\_STDBY       = 0x4,}\\
\>\>\>\textit{CL\_EO\_STATE\_SUSPEND     = 0x8,}\\
\>\>\>\textit{CL\_EO\_STATE\_STOP        = 0x10,}\\
\>\>\>\textit{CL\_EO\_STATE\_KILL        = 0x20,}\\
\>\>\>\textit{CL\_EO\_STATE\_RESUME      = 0x40,}\\
\>\>\>\textit{CL\_EO\_STATE\_FAILED      = 0x80}\\
\textit{\} ClEoStateT;}\end{tabbing}
The values of the enumeration, {\tt{ClEoStateT}}, contains the various states of EO.
\begin{itemize}
 \item
 \textit{CL\_\-EO\_\-STATE\_\-INIT} - Initial state of the EO.
 \item
 \textit{CL\_\-EO\_\-STATE\_\-ACTIVE} - EO is in the active state.
 \item
 \textit{CL\_\-EO\_\-STATE\_\-STDBY} - EO is in the standby state.
 \item
 \textit{CL\_\-EO\_\-STATE\_\-SUSPEND} - EO is the suspended state.
 \item
 \textit{CL\_\-EO\_\-STATE\_\-STOP} - EO is in the stopped state.
 \item
 \textit{CL\_\-EO\_\-STATE\_\-KILL} - EO is in the killed state.
 \item
 \textit{CL\_\-EO\_\-STATE\_\-RESUME} - EO is resumed from the standby state.
 \item
 \textit{CL\_\-EO\_\-STATE\_\-FAILED} - EO is in the failed state.
\end{itemize}
 
 

\subsection{ClEoApplicationTypeT}
\index{ClEoApplicationTypeT@{ClEoApplicationTypeT}}
\begin{tabbing}
xx\=xx\=xx\=xx\=xx\=xx\=xx\=xx\=xx\=\kill
\textit{typedef enum \{}\\
\>\>\>\\textit{CL\_EO\_USE\_THREAD\_FOR\_RECV   = CL\_TRUE,}\\
\>\>\>\\textit{CL\_EO\_USE\_THREAD\_FOR\_APP    = CL\_FALSE}\\
\textit{\}ClEoApplicationTypeT;}\end{tabbing}
\begin{itemize}
 \item
 \textit{CL\_\-EO\_\-USE\_\-THREAD\_\-FOR\_\-RECV} - If this is selected, the main thread is used to receive the RMD message. 
 The main thread is not blocked in {\tt{ClEoAppCreateCallbackT}} and returns immediately. 
 \item
 \textit{CL\_\-EO\_\-USE\_\-THREAD\_\-FOR\_\-APP} - The main thread is allotted to the user-application. The main
 thread is blocked in {\tt{ClEoAppCreateCallbackT}} or used by the
 application and returns only when the {\tt{ClEoAppDeleteCallbackT}} is called.
 \end{itemize}

 
\subsection{clEoPollingTypeT}
\index{clEoPollingTypeT@{clEoPollingTypeT}}
\begin{tabbing}
xx\=xx\=xx\=xx\=xx\=xx\=xx\=xx\=xx\=\kill
\textit{typedef enum \{}\\
\>\>\>\\textit{CL\_EO\_DONT\_POLL    = 0,}\\
\>\>\>\\textit{CL\_EO\_BUSY\_POLL    = 1,}\\
\>\>\>\\textit{CL\_EO\_DEFAULT\_POLL = 2}\\
\textit{\}clEoPollingTypeT;}\end{tabbing}
The enumeration, {\tt{clEoPollingTypeT}}, 
\begin{itemize}
 \item
 \textit{CL\_\-EO\_\-DONT\_\-POLL} - Component Manager stops the heartbeat of an EO
 if {\tt{CL\_\-CPM\_\-DONT\_\-POLL}} is received in response to the heartbeat.
\item
 \textit{CL\_\-EO\_\-BUSY\_\-POLL} - Component Manager increases the heartbeat timeout to the maximum polling timeout. You can configure the maximum
 timeout  while configuring the Component Manager.
 \item
 \textit{CL\_\-EO\_\-DEFAULT\_\-POLL} - Component Manager continues with the default heartbeat timeout. You can configure the default timeout 
 while configuring the Component Manager.
  \end{itemize}


\subsection{ClEOServiceInstallOrderT}
\index{ClEOServiceInstallOrderT@{ClEOServiceInstallOrderT}}
\begin{tabbing}
xx\=xx\=xx\=xx\=xx\=xx\=xx\=xx\=xx\=\kill
\textit{typedef enum \{}\\
\>\>\>\\textit{CL\_EO\_ADD\_TO\_FRONT  = 0,}\\
\>\>\>\\textit{CL\_EO\_ADD\_TO\_BACK   = 1}\\
\textit{\}ClEOServiceInstallOrderT;}\end{tabbing}
The enumeration, {\tt{ClEOServiceInstallOrderT}}, is used while installing a client function 
\begin{itemize}
 \item
 \textit{CL\_\-EO\_\-ADD\_\-TO\_\-FRONT} - Adds to the front of the list.
 \item
\textit{CL\_\-EO\_\-ADD\_\-TO\_\-BACK} - Adds to end of the list. This is used with the {\tt{clEoServiceValidate()}} function.
  \end{itemize}




\subsection{ClEoDataT}
\index{ClEoDataT@{ClEoDataT}}
\textit{typedef ClOsalTaskDataT ClEoDataT;}
  \newline
  \newline
The type of the EO data.

\subsection{ClIocPortT}
\index{ClIocPortT@{ClIocPortT}}
\textit{typedef ClUint32T ClIocPortT;}
  \newline
  \newline
The type of the identifier to the IOC communication port.


\subsection{ClEoIdT}
\index{ClEoIdT@{ClEoIdT}}
\textit{typedef ClUint16T ClEoIdT;}
  \newline
  \newline
The type of the EO ID, assigned to an EO as part of the registration with the Component Manager.




\subsection{ClEoPayloadWithReplyCallbackT}
\index{ClEoPayloadWithReplyCallbackT@{ClEoPayloadWithReplyCallbackT}}
\begin{tabbing}
xx\=xx\=xx\=xx\=xx\=xx\=xx\=xx\=xx\=\kill
\textit{typedef ClRcT (* ClEoPayloadWithReplyCallbackT) (}\\
\>\>\>\textit{CL\_IN   ClEoDataT data,}\\
\>\>\>\textit{CL\_IN   ClBufferHandleT  inMsgHandle,}\\
\>\>\>\textit{CL\_OUT  ClBufferHandleT  outMsgHandle);}
\end{tabbing}
RMD with payload (EO data) and pointer to the {\tt{reply}} function. This is the generic function
 prototype definition for all RMD functions, installed on the EO client object.
 \begin{itemize}
 \item
 \textit{data} - Data provided while invoking {\tt{clEoClientInstall()}}.
 \item
 \textit{inMsgHandle} - Received message over RMD.
 \item
 \textit{outMsgHandle} - Reply message if any.
 \end{itemize}
 
 
 \subsection{ClEoPayloadWithReplyCallbackT}
 \index{ClEoPayloadWithReplyCallbackT@{ClEoPayloadWithReplyCallbackT}}
 \begin{tabbing}
 xx\=xx\=xx\=xx\=xx\=xx\=xx\=xx\=xx\=\kill
\textit{ typedef ClRcT (*ClEoCallFuncCallbackT) (}\\
\>\>\>\textit{CL\_IN ClEoPayloadWithReplyCallbackT func,}\\
\>\>\>\textit{CL\_IN   ClEoDataT                   eoArg,}\\
\>\>\>\textit{CL\_IN   ClBufferHandleT      inMsgHdl,}\\
\>\>\>\textit{CL\_OUT  ClBufferHandleT      outMsgHdl);}\end{tabbing}
 Function callout definition required for the {\tt{clEoWalk()}} function.
\begin{itemize}
\item
\textit{func} - Function that needs to be invoked.
\item
\textit{eoArg} - Arguments that need to be passed.
\item
\textit{inMsgHdl} - Request packet received including the protocol header.
\item
\textit{outMsgHdl} - Data portion of the response to a protocol (PDU).
\end{itemize}

 
  \newpage


\section{Library Life Cycle APIs}
\subsection{clEoLibInitialize}
\index{clEoLibInitialize@{clEoLibInitialize}}
\hypertarget{pageeo201}{}\paragraph{cl\-Eo\-Lib\-Initialize}\label{pageeo201}
\begin{Desc}
\item[Synopsis:]Initializes the EO library.\end{Desc}
\begin{Desc}
\item[Header File:]clEoConfigApi.h\end{Desc}
\begin{Desc}
\item[Library Files:]lib\-Cl\-Eo\end{Desc}
\begin{Desc}
\item[Syntax:]

\footnotesize\begin{verbatim}     ClRcT clEoLibInitialize();
\end{verbatim}
  \normalsize
\end{Desc}
\begin{Desc}
\item[Parameters:]None\end{Desc}
\begin{Desc}
\item[Return values:]
\begin{description}
\item[{\em CL\_\-OK:}]The function executed successfully.\end{description}
\end{Desc}
\begin{Desc}
\item[Description:]This function is used to initialize the EO library. It creates a list that contains the mapping of EO port to EO objects.
\end{Desc}
\begin{Desc}
\item[Related APIs:]\hyperlink{pageeo202}{clEoLibFinalize}\end{Desc}




  \newpage
\subsection{clEoLibFinalize}
\index{clEoLibFinalize@{clEoLibFinalize}}
\hypertarget{pageeo202}{}\paragraph{cl\-Eo\-Lib\-Finalize}\label{pageeo202}
\begin{Desc}
\item[Synopsis:]Frees the resources of the EO component acquired during initialization.\end{Desc}
\begin{Desc}
\item[Header File:]clEoConfigApi.h\end{Desc}
\begin{Desc}
\item[Library Files:]lib\-Cl\-Eo\end{Desc}
\begin{Desc}
\item[Syntax:]

\footnotesize\begin{verbatim}     ClRcT clEoLibFinalize();
\end{verbatim}
  normalsize
\end{Desc}
\begin{Desc}
\item[Parameters:]None\end{Desc}
\begin{Desc}
\item[Return values:]
\begin{description}
\item[{\em CL\_\-OK:}]The function executed successfully.\end{description}
\end{Desc}
\begin{Desc}
\item[Description:]Frees the resources of the EO component acquired during initialization of the EO library.\end{Desc}
\begin{Desc}
\item[Related APIs:]\hyperlink{pageeo201}{clEoLibFinalize}\end{Desc}

\newpage

\section{Functional APIs}
\subsection{clEoWalk}
\index{clEoWalk@{clEoWalk}}
\hypertarget{pageeo103}{}\paragraph{cl\-Eo\-Walk}\label{pageeo103}
\begin{Desc}
\item[Synopsis:]Performs a walk operation on the EO component.\end{Desc}
\begin{Desc}
\item[Header File:]clEoApi.h\end{Desc}
\begin{Desc}
\item[Library Files:]Cl\-Eo\end{Desc}
\begin{Desc}
\item[Syntax:]

\footnotesize\begin{verbatim}     extern ClRcT clEoWalk(
                 			CL_IN ClEoExecutionObjT          *pThis,
                 			CL_IN ClUint32T                  func,
                 			CL_IN ClEoCallFuncCallbackT      pFuncCallout,
                 			CL_IN ClBufferHandleT     inMsgHdl,
                 			CL_OUT ClBufferHandleT    outMsgHdl);
\end{verbatim}
  normalsize
\end{Desc}
\begin{Desc}
\item[Parameters:]
\begin{description}
\item[{\em p\-This:}]Handle of the EO. 
\item[{\em func:}]Function number of the function to be executed. 
\item[{\em p\-Func\-Callout:}]Function that performs the execution. 
\item[{\em in\-Msg\-Hdl:}]Request message received including protocol header. 
\item[{\em out\-Msg\-Hdl:}](out) Data part of the response to a protocol (PDU).\end{description}
\end{Desc}
\begin{Desc}
\item[Return values:]
\begin{description}
\item[{\em CL\_\-OK:}]The function executed successfully. 
\item[{\em CL\_\-ERR\_\-NULL\_\-POINTER:}]{\tt{pThis}} contains a NULL pointer. 
\item[{\em CL\_\-EO\_\-ERR\_\-FUNC\_\-NOT\_\-REGISTERED:}]This function is not registered. 
\item[{\em CL\_\-EO\_\-ERR\_\-EO\_\-SUSPENDED:}]EO is in the suspended state.\end{description}
\end{Desc}
\begin{Desc}
\item[Description:]This function is used to perform a walk operation through the EO for a given RMD function number. It calls {\tt{rmdInvoke}} for every 
callback function registered with an EO for that RMD function number.\end{Desc}
\begin{Desc}
\item[Related APIs:]\hyperlink{pageeo104}{cl\-Eo\-Service\-Validate} \end{Desc}


  \newpage
\subsection{clEoServiceValidate}
\index{clEoServiceValidate@{clEoServiceValidate}}
\hypertarget{pageeo104}{}\paragraph{cl\-Eo\-Service\-Validate}\label{pageeo104}
\begin{Desc}
\item[Synopsis:]Validates the function registration.\end{Desc}
\begin{Desc}
\item[Header File:]clEoApi.h\end{Desc}
\begin{Desc}
\item[Library Files:]Cl\-Eo\end{Desc}
\begin{Desc}
\item[Syntax:]

\footnotesize\begin{verbatim}     extern ClRcT clEoServiceValidate(
                 			CL_IN    ClEoExecutionObjT   *pThis,
                 			CL_IN    ClUint32T           func);
\end{verbatim}
  normalsize
\end{Desc}
\begin{Desc}
\item[Parameters:]
\begin{description}
\item[{\em p\-This:}]Handle of the EO. 
\item[{\em func:}]Function to be invoked.\end{description}
\end{Desc}
\begin{Desc}
\item[Return values:]
\begin{description}
\item[{\em CL\_\-OK:}]The function executed successfully. 
\item[{\em CL\_\-EO\_\-ERR\_\-FUNC\_\-NOT\_\-REGISTERED:}]The function is not registered. 
\item[{\em CL\_\-EO\_\-ERR\_\-EO\_\-SUSPENDED:}]The EO is in the suspended state. 
\item[{\em CL\_\-ERR\_\-NULL\_\-POINTER:}]{\tt{pThis}} contains a NULL pointer.\end{description}
\end{Desc}
\begin{Desc}
\item[Description:]This function is used to validate if the function for which the request is made is registered. It can be used to check 
if the services provided by an EO is available before calling the {\tt{cl\-Eo\-Walk()}} function.\end{Desc}
\begin{Desc}
\item[Related APIs:]\hyperlink{pageeo103}{cl\-Eo\-Walk}\end{Desc}


  \newpage
\subsection{clEoClientInstall}
\index{clEoClientInstall@{clEoClientInstall}}
\hypertarget{pageeo106}{}\paragraph{cl\-Eo\-Client\-Install}\label{pageeo106}
\begin{Desc}
\item[Synopsis:]Installs the function table for a client.\end{Desc}
\begin{Desc}
\item[Header File:]clEoApi.h\end{Desc}
\begin{Desc}
\item[Library Files:]Cl\-Eo\end{Desc}
\begin{Desc}
\item[Syntax:]

\footnotesize\begin{verbatim}     extern ClRcT clEoClientInstall(
                 			CL_IN ClEoExecutionObjT      *pThis,
                 			CL_IN ClUint32T              clientID,
                 			CL_IN ClEoPayloadWithReplyCallbackT *pFuncs,
                 			CL_IN ClEoDataT              data,
                 			CL_IN ClUint32T              nFuncs);
\end{verbatim}
  normalsize
\end{Desc}
\begin{Desc}
\item[Parameters:]
\begin{description}
\item[{\em p\-This:}]Handle of the EO. 
\item[{\em client\-Id:}]ID of the client.   
\item[{\em p\-Funcs:}]Pointer to the function table. 
\item[{\em data:}]Data specific to the client.
\item[{\em n\-Funcs:}]Number of functions that are being installed.\end{description}
\end{Desc}
\begin{Desc}
\item[Return values:]
\begin{description}
\item[{\em CL\_\-OK:}]The function executed successfully. 
\item[{\em CL\_\-ERR\_\-NULL\_\-POINTER:}]{\tt{pThis}} or {\tt{pFuncs}} contains a NULL pointer. 
\item[{\em CL\_\-EO\_\-NO\_\-MEMORY:}]Memory allocation failure. 
\item[{\em CL\_\-EO\_\-CL\_\-INVALID\_\-CLIENTID:}]The client ID is invalid. 
\item[{\em CL\_\-EO\_\-CL\_\-INVALID\_\-SERVICEID:}]The service ID is invalid.\end{description}
\end{Desc}
\begin{Desc}
\item[Description:]This function is called by the client application to install its function table with the EO. The client exports
the APIs that it provides to the users using this function. This function can be invoked through RMD calls.
\end{Desc}
\begin{Desc}
\item[Related APIs:]\hyperlink{pageeo107}{cl\-Eo\-Client\-Uninstall} \end{Desc}


  \newpage
\subsection{clEoClientUninstall}
\index{clEoClientUninstall@{clEoClientUninstall}}
\hypertarget{pageeo107}{}\paragraph{cl\-Eo\-Client\-Uninstall}\label{pageeo107}
\begin{Desc}
\item[Synopsis:]Uninstalls the function table for the client.\end{Desc}
\begin{Desc}
\item[Header File:]clEoApi.h\end{Desc}
\begin{Desc}
\item[Library Files:]Cl\-Eo\end{Desc}
\begin{Desc}
\item[Syntax:]

\footnotesize\begin{verbatim}     extern ClRcT clEoClientUninstall(
                 			CL_IN ClEoExecutionObjT* pThis,
                 			CL_IN ClUint32T clientId);
\end{verbatim}
  normalsize
\end{Desc}
\begin{Desc}
\item[Parameters:]
\begin{description}
\item[{\em p\-This:}]Handle of the EO. 
\item[{\em client\-Id:}]ID of the client.\end{description}
\end{Desc}
\begin{Desc}
\item[Return values:]
\begin{description}
\item[{\em CL\_\-OK:}]The function executed successfully. 
\item[{\em CL\_\-ERR\_\-NULL\_\-POINTER:}]{\tt{pThis}} contains a NULL pointer. 
\item[{\em CL\_\-EO\_\-ERR\_\-INVALID\_\-CLIENTID:}]The client\-ID is invalid.\end{description}
\end{Desc}
\begin{Desc}
\item[Description:]This function is called by the client to uninstall its function table with the EO. After this function is successfully executed,
the functions previously exported by this client using {\tt{cl\-Eo\-Client\-Install()}}, cannot be invoked as RMD calls.
\end{Desc}
\begin{Desc}
\item[Related APIs:]\hyperlink{pageeo106}{cl\-Eo\-Client\-Install} \end{Desc}




  \newpage
\subsection{clEoClientDataSet}
\index{clEoClientDataSet@{clEoClientDataSet}}
\hypertarget{pageeo108}{}\paragraph{cl\-Eo\-Client\-Data\-Set}\label{pageeo108}
\begin{Desc}
\item[Synopsis:]Stores the data specific to the client.\end{Desc}
\begin{Desc}
\item[Header File:]clEoApi.h\end{Desc}
\begin{Desc}
\item[Library Files:]Cl\-Eo\end{Desc}
\begin{Desc}
\item[Syntax:]

\footnotesize\begin{verbatim}     extern ClRcT clEoClientDataSet(
                 			CL_IN ClEoExecutionObjT  *pThis,
                 			CL_IN ClUint32T          clientId,
                 			CL_IN ClEoDataT          data);
\end{verbatim}
  normalsize
\end{Desc}
\begin{Desc}
\item[Parameters:]
\begin{description}
\item[{\em p\-This:}]Handle of the EO. 
\item[{\em client\-Id:}]ID of the client.   
\item[{\em data:}]Data specific to the client.\end{description}
\end{Desc}
\begin{Desc}
\item[Return values:]
\begin{description}
\item[{\em CL\_\-OK:}]The function executed successfully. 
\item[{\em CL\_\-ERR\_\-NULL\_\-POINTER:}]{\tt{pThis}} contains a NULL pointer. 
\item[{\em CL\_\-EO\_\-ERR\_\-INVALID\_\-CLIENTID:}]The client ID is invalid.\end{description}
\end{Desc}
\begin{Desc}
\item[Description:]This function is used to store the data specific to the client application.\end{Desc}
\begin{Desc}
\item[Related APIs:]\hyperlink{pageeo109}{cl\-Eo\-Client\-Data\-Get} \end{Desc}




  \newpage
\subsection{clEoClientDataGet}
\index{clEoClientDataGet@{clEoClientDataGet}}
\hypertarget{pageeo109}{}\paragraph{cl\-Eo\-Client\-Data\-Get}\label{pageeo109}
\begin{Desc}
\item[Synopsis:]Returns the client specific data.\end{Desc}
\begin{Desc}
\item[Header File:]clEoApi.h\end{Desc}
\begin{Desc}
\item[Library Files:]Cl\-Eo\end{Desc}
\begin{Desc}
\item[Syntax:]

\footnotesize\begin{verbatim}     extern ClRcT clEoClientDataGet(
                 			CL_IN ClEoExecutionObjT *pThis,
                 			CL_IN   ClUint32T   clientId,
                 			CL_OUT  ClEoDataT   *pData);
\end{verbatim}
  normalsize
\end{Desc}
\begin{Desc}
\item[Parameters:]
\begin{description}
\item[{\em p\-This:}]Handle of the EO. 
\item[{\em client\-Id:}]ID of the client. 
\item[{\em p\-Data:}](out) Data specific to the client.\end{description}
\end{Desc}
\begin{Desc}
\item[Return values:]
\begin{description}
\item[{\em CL\_\-OK:}]The function executed successfully. 
\item[{\em CL\_\-ERR\_\-NULL\_\-POINTER:}]{\tt{pThis}} or {\tt{pData}} contains a NULL pointer. 
\item[{\em CL\_\-EO\_\-ERR\_\-INVALID\_\-CLIENTID:}]The client ID is invalid.\end{description}
\end{Desc}
\begin{Desc}
\item[Description:]This function is used to retrieve the data specific to the client  .\end{Desc}
\begin{Desc}
\item[Related APIs:]\hyperlink{pageeo108}{cl\-Eo\-Client\-Data\-Set} \end{Desc}





  \newpage
\subsection{clEoServiceInstall}
\index{clEoServiceInstall@{clEoServiceInstall}}
\hypertarget{pageeo110}{}\paragraph{cl\-Eo\-Service\-Install}\label{pageeo110}
\begin{Desc}
\item[Synopsis:]Installs a particular client function.\end{Desc}
\begin{Desc}
\item[Header File:]clEoApi.h\end{Desc}
\begin{Desc}
\item[Library Files:]Cl\-Eo\end{Desc}
\begin{Desc}
\item[Syntax:]

\footnotesize\begin{verbatim}     extern ClRcT clEoServiceInstall(
                 			CL_IN ClEoExecutionObjT              *pThis,
                 			CL_IN ClEoPayloadWithReplyCallbackT  pFunction,
                 			CL_IN ClUint32T                      iFuncNum,
                 			CL_IN ClUint32T                      order);
\end{verbatim}
  normalsize
\end{Desc}
\begin{Desc}
\item[Parameters:]
\begin{description}
\item[{\em p\-This:}]Handle of the EO. 
\item[{\em p\-Function:}]Function pointer to be installed. 
\item[{\em i\-Func\-Num:}]Function number. 
\item[{\em order:}]Specifies if the service that is to be installed, should be added to the front or the end of the table.\end{description}
\end{Desc}
\begin{Desc}
\item[Return values:]
\begin{description}
\item[{\em CL\_\-OK:}]The function executed successfully. 
\item[{\em CL\_\-ERR\_\-NULL\_\-POINTER:}]{\tt{pThis}} contains a NULL pointer. 
\item[{\em CL\_\-ERR\_\-NO\_\-MEMORY:}]Memory allocation failure. 
\item[{\em CL\_\-EO\_\-CL\_\-INVALID\_\-SERVICEID:}]The service ID is invalid.
\item[{\em CL\_\-ERR\_\-INVALID\_\-PARAMETER:}]A parameter is not set correctly.\end{description}
\end{Desc}
\begin{Desc}
\item[Description:]This function is used to install a particular client function, identified by {\tt{i\-Func\-Num}} in the EO function table. Using this function,
the application can register the service it provides to other components. This function can install the new service either to the front or back of 
the table by specifying the {\tt{order}}.\end{Desc}
\begin{Desc}
\item[Related APIs:]\hyperlink{pageeo111}{cl\-Eo\-Service\-Uninstall} \end{Desc}



  \newpage
\subsection{clEoServiceUninstall}
\index{clEoServiceUninstall@{clEoServiceUninstall}}
\hypertarget{pageeo111}{}\paragraph{cl\-Eo\-Service\-Uninstall}\label{pageeo111}
\begin{Desc}
\item[Synopsis:]Uninstalls a particular client function.\end{Desc}
\begin{Desc}
\item[Header File:]clEoApi.h\end{Desc}
\begin{Desc}
\item[Library Files:]Cl\-Eo\end{Desc}
\begin{Desc}
\item[Syntax:]

\footnotesize\begin{verbatim}     extern ClRcT clEoServiceUninstall(
                 			CL_IN ClEoExecutionObjT* pThis,
                 			CL_IN ClEoPayloadWithReplyCallbackT pFunction,
                 			CL_IN ClUint32T iFuncNum);
\end{verbatim}
  normalsize
\end{Desc}
\begin{Desc}
\item[Parameters:]
\begin{description}
\item[{\em p\-This:}]Handle of the EO. 
\item[{\em p\-Function:}]Function pointer to be uninstalled. 
\item[{\em i\-Func\-Num:}]Function number.\end{description}
\end{Desc}
\begin{Desc}
\item[Return values:]
\begin{description}
\item[{\em CL\_\-OK:}]The function executed successfully. 
\item[{\em CL\_\-ERR\_\-NULL\_\-POINTER:}]{\tt{pThis}} contains a NULL pointer. 
\item[{\em CL\_\-EO\_\-FUNC\_\-NOT\_\-REGISTERED:}]Cannot de-register a function that is not registered.
\item[{\em CL\_\-EO\_\-CL\_\-INVALID\_\-SERVICEID:}]The service ID is invalid.
\item[{\em CL\_\-ERR\_\-INVALID\_\-PARAMETER:}]A parameter is not set correctly.\end{description}
\end{Desc}
\begin{Desc}
\item[Description:]This function is used to uninstall a particular client function from the EO function table. After this function is executed, 
the service, {\em p\-Function\/}, becomes unavailable to be invoked as an RMD call.\end{Desc}
\begin{Desc}
\item[Related APIs:]\hyperlink{pageeo110}{cl\-Eo\-Service\-Install} \end{Desc}




  \newpage
\subsection{clEoPrivateDataSet}
\index{clEoPrivateDataSet@{clEoPrivateDataSet}}
\hypertarget{pageeo112}{}\paragraph{cl\-Eo\-Private\-Data\-Set}\label{pageeo112}
\begin{Desc}
\item[Synopsis:]Stores data in the area specific to the EO.\end{Desc}
\begin{Desc}
\item[Header File:]clEoApi.h\end{Desc}
\begin{Desc}
\item[Library Files:]Cl\-Eo\end{Desc}
\begin{Desc}
\item[Syntax:]

\footnotesize\begin{verbatim}     extern ClRcT clEoPrivateDataSet(
                 			CL_IN ClEoExecutionObjT* pThis,
                 			CL_IN ClUint32T type,
                 			CL_IN void *pData);
\end{verbatim}
  normalsize
\end{Desc}
\begin{Desc}
\item[Parameters:]
\begin{description}
\item[{\em p\-This:}]Handle of the EO. 
\item[{\em type:}]User specified key. 
\item[{\em p\-Data:}]EO specific data.
\end{description}
\end{Desc}

\begin{Desc}
\item[Return Values:]
\begin{description}
\item[{\em CL\_\-ERR\_\-NULL\_\-POINTER:}] {\tt{pThis}} or {\tt{pDdata}} contains a NULL pointer.
 \par
 This function also returns the return values of the {\tt{cl\-Cnt\-Node\-Add()}} function.
 \end{description}
\end{Desc}
\begin{Desc}
\item[Description:]This function is used to store data in a data area specific to EO. For a unique key, there can be only one node.\end{Desc}
\begin{Desc}
\item[Related APIs:]\hyperlink{pageeo109}{cl\-Eo\-Private\-Data\-Get} \end{Desc}




  \newpage
\subsection{clEoPrivateDataGet}
\index{clEoPrivateDataGet@{clEoPrivateDataGet}}
\hypertarget{pageeo113}{}\paragraph{cl\-Eo\-Private\-Data\-Get}\label{pageeo113}
\begin{Desc}
\item[Synopsis:]Retrieves data from the area specific to the EO.\end{Desc}
\begin{Desc}
\item[Header File:]clEoApi.h\end{Desc}
\begin{Desc}
\item[Library Files:]Cl\-Eo\end{Desc}
\begin{Desc}
\item[Syntax:]

\footnotesize\begin{verbatim}     extern ClRcT clEoPrivateDataGet(
                 			CL_IN ClEoExecutionObjT* pThis,
                 			CL_IN ClUint32T type,
                 			CL_OUT void **data);
\end{verbatim}
  normalsize
\end{Desc}
\begin{Desc}
\item[Parameters:]
\begin{description}
\item[{\em p\-This:}]Handle of the EO. 
\item[{\em type:}]User specified key. 
\item[{\em data:}](out) Data specific to the EO.\end{description}
\end{Desc}
\begin{Desc}
\item[Return values:]
\begin{description}

\item[{\em CL\_\-ERR\_\-NULL\_\-POINTER:}] {\tt{pThis}} or {\tt{data}} contains a NULL pointer.\par
 This function also returns the result of {\tt{cl\-Cnt\-Node\-User\-Data\-Get}}.
 \end{description}
 \end{Desc}
\begin{Desc}
\item[Description:]This function is used to retrieve the data private to the EO.\end{Desc}
\begin{Desc}
\item[Related APIs:]\hyperlink{pageeo116}{cl\-Eo\-My\-Eo\-Ioc\-Port\-Set} \end{Desc}



  \newpage
\subsection{clEoMyEoIocPortSet}
\index{clEoMyEoIocPortSet@{clEoMyEoIocPortSet}}
\hypertarget{pageeo116}{}\paragraph{cl\-Eo\-My\-Eo\-Ioc\-Port\-Set}\label{pageeo116}
\begin{Desc}
\item[Synopsis:]Sets the EO thread {\tt{iocPort}}.\end{Desc}
\begin{Desc}
\item[Header File:]clEoApi.h\end{Desc}
\begin{Desc}
\item[Library Files:]Cl\-Eo\end{Desc}
\begin{Desc}
\item[Syntax:]

\footnotesize\begin{verbatim}     extern ClRcT clEoMyEoIocPortSet(
                 			CL_IN ClIocPortT iocPort);
\end{verbatim}
  normalsize
\end{Desc}
\begin{Desc}
\item[Parameters:]
\begin{description}
\item[{\em ioc\-Port:}]Contains the IOC port information that needs to be set.\end{description}
\end{Desc}
\begin{Desc}
\item[Return values:]
\begin{description}
\item[{\em CL\_\-OK:}]The function executed successfully.\end{description}
\end{Desc}
\begin{Desc}
\item[Description:]This function is used to set the EO ID.\end{Desc}
\begin{Desc}
\item[Related APIs:]\hyperlink{pageeo117}{cl\-Eo\-My\-Eo\-Ioc\-Port\-Get} \end{Desc}




  \newpage
\subsection{clEoMyEoIocPortGet}
\index{clEoMyEoIocPortGet@{clEoMyEoIocPortGet}}
\hypertarget{pageeo117}{}\paragraph{cl\-Eo\-My\-Eo\-Ioc\-Port\-Get}\label{pageeo117}
\begin{Desc}
\item[Synopsis:]Retrieves the {\tt{Ioc\-Port}} information of the EO.\end{Desc}
\begin{Desc}
\item[Header File:]clEoApi.h\end{Desc}
\begin{Desc}
\item[Library Files:]Cl\-Eo\end{Desc}
\begin{Desc}
\item[Syntax:]

\footnotesize\begin{verbatim}     extern ClRcT clEoMyEoIocPortGet(
                 			CL_OUT ClIocPortT *pIocPort);
\end{verbatim}
  normalsize
\end{Desc}
\begin{Desc}
\item[Parameters:]
\begin{description}
\item[{\em p\-Ioc\-Port:}](out) This is populated with the retrieved IOC port.\end{description}
\end{Desc}
\begin{Desc}
\item[Return values:]
\begin{description}
\item[{\em CL\_\-OK:}]The function executed successfully. 
\item[{\em CL\_\-ERR\_\-INVALID\_\-STATE:}]The state of the EO is invalid. 
\item[{\em CL\_\-ERR\_\-NULL\_\-POINTER:}]{\tt{pIocPort}} contains a NULL pointer.\end{description}
\end{Desc}
\begin{Desc}
\item[Description:]This function is used to retrieve the EO {\tt{Ioc\-Port}}.\end{Desc}
\begin{Desc}
\item[Related APIs:]\hyperlink{pageeo116}{cl\-Eo\-My\-Eo\-Ioc\-Port\-Set} \end{Desc}




  \newpage
\subsection{clEoMyEoObjectSet}
\index{clEoMyEoObjectSet@{clEoMyEoObjectSet}}
\hypertarget{pageeo118}{}\paragraph{cl\-Eo\-My\-Eo\-Object\-Set}\label{pageeo118}
\begin{Desc}
\item[Synopsis:]Stores the EO Object.\end{Desc}
\begin{Desc}
\item[Header File:]clEoApi.h\end{Desc}
\begin{Desc}
\item[Library Files:]Cl\-Eo\end{Desc}
\begin{Desc}
\item[Syntax:]

\footnotesize\begin{verbatim}     extern ClRcT clEoMyEoObjectSet(
                 			CL_IN ClEoExecutionObjT* eoObj);
\end{verbatim}
  normalsize
\end{Desc}
\begin{Desc}
\item[Parameters:]
\begin{description}
\item[{\em p\-Eo\-Obj:}]Contains the {\tt{Cl\-Eo\-Execution\-Obj\-T}} information that is to be stored.\end{description}
\end{Desc}
\begin{Desc}
\item[Return values:]
\begin{description}
\item[{\em CL\_\-OK:}]The function executed successfully.\end{description}
\end{Desc}
\begin{Desc}
\item[Description:]This function is used to store the EO Object.\end{Desc}
\begin{Desc}
\item[Related APIs:]\hyperlink{pageeo119}{cl\-Eo\-My\-Eo\-Object\-Get}. \end{Desc}




  \newpage
\subsection{clEoMyEoObjectGet}
\index{clEoMyEoObjectGet@{clEoMyEoObjectGet}}
\hypertarget{pageeo119}{}\paragraph{cl\-Eo\-My\-Eo\-Object\-Get}\label{pageeo119}
\begin{Desc}
\item[Synopsis:]Retrieves the EO Object.\end{Desc}
\begin{Desc}
\item[Header File:]clEoApi.h\end{Desc}
\begin{Desc}
\item[Library Files:]Cl\-Eo\end{Desc}
\begin{Desc}
\item[Syntax:]

\footnotesize\begin{verbatim}     extern ClRcT clEoMyEoObjectGet(
                 			CL_OUT ClEoExecutionObjT** pEOObj);
\end{verbatim}
  normalsize
\end{Desc}
\begin{Desc}
\item[Parameters:]
\begin{description}
\item[{\em p\-EOObj:}](out) This is populated with the EO object.\end{description}
\end{Desc}
\begin{Desc}
\item[Return values:]
\begin{description}
\item[{\em CL\_\-OK:}]The function executed successfully. 
\item[{\em CL\_\-ERR\_\-INVALID\_\-STATE:}]The state of the EO is invalid.\end{description}
\end{Desc}
\begin{Desc}
\item[Description:]This function is used to retrieve the EO Object.\end{Desc}
\begin{Desc}
\item[Related APIs:]\hyperlink{pageeo118}{cl\-Eo\-My\-Eo\-Object\-Set}. \end{Desc}







\chapter{Service Management Information Model}
TBD

\chapter{Service Notifications}
TBD

\chapter{Configuration}
TBD

\chapter{Debug CLI}
TBD
\end{flushleft}