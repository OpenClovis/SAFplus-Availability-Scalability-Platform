
\hypertarget{group__group22}{
\chapter{Functional Overview}
\index{Functional Overview@{Functional Overview}}
\label{group__group22}
}

\begin{flushleft}

The \begin{bf}OpenClovis Group Membership Service (GMS)\end{bf} is a
high availability infrastructure component that allows a set of
nodes to form a group and provide track notifications to registered
applications. GMS is implemented in compliance with the Cluster
Membership Service (CLM) as defined by SA Forum. It is the
generalized form of CLM, where each member of a group is a process
in a cluster interested in becoming a member of that group.
\par
Apart from the services defined by SA Forum CLM specification, GMS
also provides following services:
\begin{itemize}
\item Leader election among a group of nodes.
\item Formation of process groups and their management along with
track notification.
\end{itemize}
In this document, the Process Group functionality of GMS is
indicated as "Process Group Service", and the SAF CLM implementation
is indicated as "CLM Service". The following sections describe the
functionality, usage scenarios, and the various service
interfaces provided for each of these services.

\chapter{Service Model}
\section{Usage Model}
TBD
\section{Functional Description}
\begin{enumerate}
\item \begin{bf}CLM Service \end{bf}
\newline
CLM Service manages the membership of the cluster with specified
cluster configuration, for the nodes that are administratively
configured to be part of it. It provides a consistent view of the
cluster across all nodes of the cluster.
\newline
\par
A node must be member of the cluster to host high availability
applications on it. CLM Service allows a node to be a member of a
cluster only if it the node is healthy and is well-connected to the
cluster.
\newline
\par
When ASP is started, Component Manager (CPM) sends a request to GMS
for the membership of the node for the cluster. GMS processes the
request and decides upon the membership of the node. It also elects
a leader and a deputy node for the cluster based on the leadership
credentials and the boot timestamp of the member nodes. If the
process is executed successfully, GMS provides the current view of
the cluster along with the leader and deputy information to CPM.
\newline
\par
An application can register for track notifications with GMS using
the cluster track APIs. After registration, GMS notifies the
application for any changes in the cluster membership such as when a
node joins, leaves the cluster or a node is reconfigured. The
application continues to receive such notifications until it invokes
the track stop API.
\newline
\par
GMS provides the following functions for the clustering service:
\begin{itemize}
\item Allows a node to join a cluster using {\tt{clGmsClusterJoin()}} function .
item Allows a member node to leave the cluster using
clGmsClusterLeave function .
\item Manages cluster membership of the nodes based on their health and communication with the cluster.
\item Allows you to track a cluster by providing consistent view of
the cluster across nodes and notifies if any changes are made using
{\tt{clGmsClusterTrack()}} function .
\item Allows you to stop receiving track notifications using {\tt{clGmsClusterTrackStop()}} function.
\item Elects a leader and a deputy node automatically for a
cluster whenever the cluster configuration is modified.
\item Provides information about a member of a cluster using
{\tt{clGmsClusterMemberGet()}} and {\tt{clGmsClusterMemberGetAsync()}} functions.
\item Initiates an explicit leader election process on the cluster
nodes using {\tt{clGmsClusterLeaderElect()}} function .
\item Removes a member from the cluster using {\tt{clGmsClusterMemberEject()}}.
\end{itemize}
\begin{Desc}
\item[Note:]In the ASP framework, joining or leaving a cluster is generally
managed by CPM. The applications do not use these APIs as it can
affect the other services of ASP.\end{Desc}

\item \begin{bf}Process Group Service \end{bf}
\newline
GMS generalizes the CLM Service by allowing any set of applications
or processes in the cluster to form a group and manage such groups.
This service, referred to as the Process Group Service, also allows
you to track the group by providing a consistent view of the group
and notifies about the changes in the membership of the group.
\newline
\par
The Process Group Service provides the following functionality:
\begin{itemize}
\item Creates a group using {\tt{clGmsGroupCreate()}} function .
\item Deletes or destroys a group using {\tt{clGmsGroupDestroy()}} function.
\item Allows a process in the cluster to join a group using {\tt{clGmsGroupJoin()}} function .
\item Allows a member to leave the group using {\tt{clGmsGroupLeave()}} function .
\item Manages membership of the group based on the health of the member processes and the containing node.
\item Allows you to track the membership by providing a consistent view of the group and notifies if any changes are made using {\tt{clGmsGroupTrack()}} function.
\item Allows you to stop receiving track notifications for a given group using {\tt{clGmsGroupTrackStop()}} function.
\item Provides a list of meta information on all the groups at any given time using {\tt{clGmsGroupInfoListGet()}} function.
\item Provides the meta information on any given group using {\tt{clGmsGetGroupInfo()}} function.
\end{itemize}
\end{enumerate}


\begin{Desc}
\item[Usage Scenario:]
\begin{enumerate}
\item \begin{bf}CLM Service \end{bf}
\newline
The main objective of CLM service is to provide high availability of
services in co-ordination with AMF and CPM. These ASP services
register with GMS for track notification on the cluster. So at any
given time, AMF running on each node is aware of the leader (or
master System Controller node) in the cluster.
\newline
\par
If the master System Controller node is terminated, GMS instance
informs all the nodes and applications through the track
notifications. Based on this information, AMF makes the standby
System Controller as Active. The other applications like
Checkpointing Server can take necessary actions to move their
Checkpoint Master Server on the newly active System Controller node.

\item \begin{bf}Process Group Service \end{bf}
\newline
The main objective of GMS is to allow a set of processes,
implementing a distributed component, service, or application to
form a group in order to share data or to co-ordinate access to
shared resources. Generally these processes are multiple instances
of the same code running on different nodes.
\newline
\par
For example, a Name Service (NS) implementation will depend on a
local NS daemon running on every node in a system. An NS entry
created by an application on a node is registered with the local NS
daemon. The other nodes can view this entry by allowing the NS
daemons to form a group and using group communication to disseminate
information among them.
\newline
\par
This can be achieved in two possible ways. Both the solutions
require that the current group view is available to all members.
\begin{itemize}
\item By using a leader that is a dedicated member and holds the primary repository of the NS database, or
\item In a distributed fashion where each member owns its local data and multicasts the changes in the data to other members of the group.
\end{itemize}
\end{enumerate}
\end{Desc}

\begin{Desc}
\item[Interaction with other components:]
GMS depends on CPM for component management and uses RMD
infrastructure between the GMS client and the server instances. Also
it uses the event service to find out if any of the components that
are member of the group goes down. This component death event is
given by CMP service, and GMS would remove this member from all the
groups of which it was member.
\end{Desc}

\begin{Desc}
\item[Configuration:]
GMS provides a set of the configurable parameters. You can configure
the values of these parameters through the {\tt{clGmsConfig.xml}}  file
located in the {\tt{\$ASP\_\-CONFIG}} directory.
\begin{enumerate}
\item \begin{bf}\textit{Clustername}: \end{bf} Name of the cluster being formed.
\item \begin{bf}\textit{linkname}: \end{bf} Name of the IP network interface of the machine where GMS instance will bind.
\item \begin{bf}\textit{MulticastAddress}: \end{bf} Multicast IP address where GMS Server binds and exchanges its information (such as handshake messages 
with other GMS instances in the cluster).
\item \begin{bf}\textit{MulticastPort}: \end{bf} Port number to be used to bind for the multicast socket. This socket is used along with the Multicast
 Address to exchange internal information with other GMS instances in the cluster.
\item \begin{bf}\textit{MaxNoOfGroups}: \end{bf} Maximum number of process groups allowed to be formed in a cluster.
\item \begin{bf}\textit{consolelog}: \end{bf} Specifies whether the log messages from GMS are printed on the console or not. Its values can be either 
{\tt{on}} or {\tt{off}}.
\item \begin{bf}\textit{logLevel}: \end{bf} Controls the log messages printed in the log file. It can be debug, info, error, and so on.
\end{enumerate}
\end{Desc}


\begin{Desc}
\item[Note:]The values of all the parameters must be identical across all nodes in the cluster, saving
linkname. \end{Desc}









\chapter{Service APIs}

\section{Type Definitions}
\subsection{ClGmsClusterManageCallbacksT}
\index{ClGmsClusterManageCallbacksT@{ClGmsClusterManageCallbacksT}}
\begin{tabbing}
xx\=xx\=xx\=xx\=xx\=xx\=xx\=xx\=xx\=\kill
\textit{typedef struct \{}\\
\>\>\>\>\textit{ClGmsClusterMemberEjectCallbackT clGmsMemberEjectCallback;}\\
\textit{\} ClGmsClusterManageCallbacksT;}\end{tabbing}

The structure, {\tt{ClGmsClusterManageCallbacksT}}, contains the cluster managing callbacks
provided at the joining time by the member. The structure contains the ejection callback which is
called when the member is ejected from the cluster. The callback is invoked after the member is
ejected and the reason for ejection is passed as an argument to the callback. The attributes of this structure are:
\par
 \textit{clGmsMemberEjectCallback} - Pointer to the eject callback function.



\subsection{ClGmsGroupMemberT}
\index{ClGmsGroupMemberT@{ClGmsGroupMemberT}}
\begin{tabbing}
xx\=xx\=xx\=xx\=xx\=xx\=xx\=xx\=xx\=\kill
\textit{typedef struct \{}\\
\>\>\>\>\textit{ClGmsMemberIdT              memberId ;}\\
\>\>\>\>\textit{ClIocAddressT               memberAddress;}\\
\>\>\>\>\textit{ClGmsMemberNameT            memberName ;}\\
\>\>\>\>\textit{ClBoolT                     memberActive ;}\\
\>\>\>\>\textit{ClTimeT                     joinTimestamp ;}\\
\>\>\>\>\textit{ClUint64T                   initialViewNumber ;}\\
\>\>\>\>\textit{ClGmsLeadershipCredentialsT credential ;}\\
\textit{\} ClGmsGroupMemberT;}\end{tabbing}
The structure, {\tt{ClGmsGroupMemberT}} contains the member component descriptor. The attributes of this structure are:
\begin{itemize}
\item
\textit{memberId} - Unique ID of a member of a given group.
\item
\textit{memberAddress} - IOC physical address of the member.
\item
\textit{memberName} - Textual name of the member.
\item
\textit{memberActive} - True if the node is a member of group.
\item
\textit{joinTimestamp} - The instant at which the member joined the group.
\item
\textit{initialViewNumber} - The view of the cluster at the time the member joined.
\item
\textit{credential} - Credentials for becoming the leader. The higher the credential,
 larger is the possibility of the node being elected as leader.
\end{itemize}




\subsection{ClGmsLeadershipCredentialsT}
\index{ClGmsLeadershipCredentialsT@{ClGmsLeadershipCredentialsT}}
\textit{typedef ClUint32T   ClGmsLeadershipCredentialsT;}
\newline
\newline
The type of an identifier for the credentials of leader election. Only members with non-zero value
 in the group are considered as candidates for leadership.


\subsection{ClGmsClusterMemberEjectCallbackT}
\index{ClGmsClusterMemberEjectCallbackT@{ClGmsClusterMemberEjectCallbackT}}
\textit{typedef void (*ClGmsClusterMemberEjectCallbackT) (}
\newline \textit{CL\_\-IN ClGmsMemberEjectReasonT   reasonCode);}
\newline
\newline
The type of the callback function to indicate that a member has been expelled from the cluster. This functions takes
\textit{reasonCode} as the parameter and returns void. This type definition is later used to define member eject callback
  structure parameters.
\begin{itemize}
\item
\textit{reasonCode} -  It can have the following two values:
 \begin{enumerate}
 \item
 {\tt{CL\_\-GMS\_\-MEMBER\_\-EJECT\_\-REASON\_\-UNKNOWN = 0}}
 \item
 {\tt{CL\_\-GMS\_\-MEMBER\_\-EJECT\_\-REASON\_\-API\_\-REQUEST =1}}
 \end{enumerate}
\end{itemize}



\subsection{ClGmsClusterManageCallbacksT}
\index{ClGmsClusterManageCallbacksT@{ClGmsClusterManageCallbacksT}}
\begin{tabbing}
xx\=xx\=xx\=xx\=xx\=xx\=xx\=xx\=xx\=\kill
\textit{typedef struct ClLogFileFullActionT\{}\\
\>\>\>\>\textit{ClGmsClusterMemberEjectCallbackT clGmsMemberEjectCallback;}\\
\textit{\} ClGmsClusterManageCallbacksT;}\end{tabbing}

 The structure, {\tt{ClGmsClusterManageCallbacksT}}, contains the cluster managing callbacks provided by the member when it joins the cluster.
 The structure contains the ejection callback function
 which is invoked when the member is ejected from the cluster. The reason for ejection is
 passed as argument for the callback.
\begin{itemize}
\item
\textit{clGmsMemberEjectCallback} - Pointer to the Eject Callback function.
\end{itemize}



\subsection{ClGmsMemberIdT}
\index{ClGmsMemberIdT@{ClGmsMemberIdT}}
\textit{typedef ClUint32T ClGmsMemberIdT;}
\newline
\newline
The type of an identifier for the group-unique ID of a member.




\subsection{ClGmsGroupIdT}
\index{ClGmsGroupIdT@{ClGmsGroupIdT}}
\textit{typedef ClUint32T ClGmsGroupIdT;}
\newline
\newline
{\tt{ClGmsGroupIdT}} is a system-wide unique ID of the group. It is generated by GMS and you can use this ID for performing further operations on
the group.


\subsection{ClGmsClusterMemberT}
\index{ClGmsClusterMemberT@{ClGmsClusterMemberT}}
\begin{tabbing}
xx\=xx\=xx\=xx\=xx\=xx\=xx\=xx\=xx\=\kill
\textit{typedef struct \{}\\
\>\>\>\>\textit{ClGmsNodeIdT                nodeId ;}\\
\>\>\>\>\textit{ClIocAddressT               nodeAddress ;}\\
\>\>\>\>\textit{ClGmsNodeAddressT nodeIpAddress ;}\\
\>\>\>\>\textit{ClNameT                     nodeName ;}\\
\>\>\>\>\textit{ClBoolT                     memberActive ;}\\
\>\>\>\>\textit{ClTimeT                     bootTimestamp ;}\\
\>\>\>\>\textit{ClUint64T                   initialViewNumber ;}\\
\>\>\>\>\textit{ClGmsLeadershipCredentialsT credential ;}\\
\>\>\>\>\textit{ClVersionT gmsVersion;}\\
\textit{\} ClGmsClusterMemberT;}\end{tabbing}
This structure, {\tt{ClGmsClusterMemberT}}, describes a member (or node) of the cluster.
The attributes of this structure are:
\begin{itemize}
\item
\textit{nodeId} - Unique ID of node.
\item
\textit{nodeAddress} - Physical address of node.
\item
\textit{nodeIpAddress} - Node IP Address.
\item
\textit{nodeName} - Textual name of node.
\item
\textit{memberActive} -  This is {\tt{TRUE}} if the node is a member of the cluster
 For tracking nodes it is not set.
\item
\textit{bootTimestamp} - The time at which GMS was started on the node.
\item
\textit{initialViewNumber} - The view number when the node joined.
\item
\textit{credential} -  This is an integer value specifying the
 leadership credibility of the node. Larger the value higher is the
 possibility of the node becoming a leader. Member with credentials
 {\tt{CL\_\-GMS\_\-INELIGIBLE\_\-CREDENTIALS}} cannot participate in the leader election
\item
\textit{gmsVersion} -  Version information of the GMS software running on the node, information is
 sent to the other peers in the cluster while joining the cluster. If
 there is a version mismatch the node is not allowed to join the Cluster.
\end{itemize}





\subsection{ClGmsGroupInfo}
\index{ClGmsGroupInfo@{ClGmsGroupInfo}}
\begin{tabbing}
xx\=xx\=xx\=xx\=xx\=xx\=xx\=xx\=xx\=\kill
\textit{typedef struct ClGmsGroupInfo \{}\\
\>\>\>\>\textit{ClGmsGroupNameT     groupName;}\\
\>\>\>\>\textit{ClGmsGroupIdT       groupId;}\\
\>\>\>\>\textit{ClGmsGroupParamsT   groupParams;}\\
\>\>\>\>\textit{ClUint32T           noOfMembers;}\\
\>\>\>\>\textit{ClBoolT             setForDelete;}\\
\>\>\>\>\textit{ClIocMulticastAddressT iocMulticastAddr;}\\
\>\>\>\>\textit{ClTimeT                 creationTimestamp;}\\
\>\>\>\>\textit{ClTimeT                 lastChangeTimestamp;}\\
\textit{\} ClGmsGroupInfoT;}\end{tabbing}
The structure, {\tt{ClGmsGroupInfoT}}, contains the values of a group. The attributes of this structure are:

\begin{itemize}
\item
\textit{groupName} - Name of the group.
\item
\textit{groupId} - Group ID.
\item
\textit{groupParams} - Requested group parameters.
\item
\textit{noOfMembers} - Number of members in the group.
\item
\textit{setForDelete} - No more joins are allowed.
\item
\textit{iocMulticastAddr} - IOC multicast address created by GMS.
\item
\textit{creationTimestamp} - Time at which group was created.
\item
\textit{lastChangeTimestamp} - Time at which the last view changed.
\end{itemize}





\subsection{clGmsGroupInfoList}
\index{clGmsGroupInfoList@{clGmsGroupInfoList}}
\begin{tabbing}
xx\=xx\=xx\=xx\=xx\=xx\=xx\=xx\=xx\=\kill
\textit{typedef struct clGmsGroupInfoList \{}\\
\>\>\>\>\textit{ClUint32T           noOfGroups;}\\
\>\>\>\>\textit{ClGmsGroupInfoT     *groupInfoList;}\\
\textit{\} ClGmsGroupInfoListT;}\end{tabbing}
The structure, {\tt{ClGmsGroupInfoT}}, contains the information about the existing groups. The attributes of this structure are:
\begin{itemize}
\item
\textit{noOfGroups} - Number of groups.
\item
\textit{groupInfoList} - Array of {\tt{ClGmsGroupT}} data.
\end{itemize}




\subsection{ClGmsHandleT}
\index{ClGmsHandleT@{ClGmsHandleT}}
\textit{typedef ClHandleT ClGmsHandleT;}
\newline
\newline
 The type of the handle for the GMS API. This handle is assigned during the initialization
 of the Group Membership Service.
 It must be passed as first parameter for all operations pertaining to the GMS library.



\subsection{ClGmsNodeIdT}
\index{ClGmsNodeIdT@{ClGmsNodeIdT}}
\textit{typedef ClUint32T ClGmsNodeIdT;}
\newline
\newline
The type of a unique and consistent identifier for a Node - Node ID.



\subsection{ClGmsCallbacksT}
\index{ClGmsCallbacksT@{ClGmsCallbacksT}}
\begin{tabbing}
xx\=xx\=xx\=xx\=xx\=xx\=xx\=xx\=xx\=\kill
\textit{typedef struct \{}\\
\>\>\>\>\textit{ClGmsClusterMemberGetCallbackT clGmsClusterMemberGetCallback;}\\
\>\>\>\>\textit{ClGmsClusterTrackCallbackT     clGmsClusterTrackCallback;}\\
\>\>\>\>\textit{ClGmsGroupTrackCallbackT       clGmsGroupTrackCallback;}\\
\>\>\>\>\textit{ClGmsGroupMemberGetCallbackT   clGmsGroupMemberGetCallback;}\\
\textit{\} ClGmsCallbacksT;}\end{tabbing}
The type of the callback structure provided to the GMS library during initialization. Its attributes are:
\begin{itemize}
\item
\textit{clGmsClusterMemberGetCallback} - This callback is called when the response is received from the server side for
 the asynchronous request made by the {\tt{clGmsClusterMemberGetAsync()}}. The callback
 is invoked with the invocation ID and the requested information of the member.
\item
\textit{clGmsClusterTrackCallback} - This callback is used to register for receiving any change notifications in the
 cluster. This registration with the server is performed
 by the {\tt{clGmsClusterTrack()}} function. The {\tt{trackFlags}} should be set to
 {\tt{CL\_\-GMS\_\-TRACK\_\-CHANGES}} or {\tt{CL\_\-GMS\_\-TRACK\_\-CHANGES\_\-ONLY}}. The callback is
 invoked when there is a change in the cluster and with the
 notification buffer containing the information of all members in the
 cluster. The callback is also called when {\tt{clGmsClusterTrack()}} function is
 called with {\tt{CL\_\-TRACK\_\-CURRENT}} flag and {\tt{notificationBuffer}} parameter
 is NULL.
\item
\textit{clGmsGroupTrackCallback} - This callback is used to register for receiving any change notifications in the
  This registration with the server is performed
 by the {\tt{clGmsGroupTrack()}} function. The {\tt{trackFlags}} should be
 {\tt{CL\_\-GMS\_\-TRACK\_\-CHANGES}} or {\tt{CL\_\-GMS\_\-TRACK\_\-CHANGES\_\-ONLY}}. The callback is
 invoked when there is a change in the group and with the
 notification buffer containing the information of all members in the
 group. The callback is also called when {\tt{clGmsClusterTrack()}} function is
 called with {\tt{CL\_\-TRACK\_\-CURRENT}} flag and {\tt{notificationBuffer}} parameter
 is NULL.

\item
\textit{clGmsGroupMemberGetCallback} - This callback is invoked when the response is received from the server side for
  the asynchronous request made by {\tt{clGmsGroupMemberGetAsync()}}. The callback
 is invoked with the invocation ID and the requested information of the member.
\end{itemize}


\subsection{ClGmsTrackFlagsT}
\textit{typedef enum ClGmsTrackFlags \{}
\newline\textit{CL\_\-GMS\_\-TRACK\_\-CURRENT        = 0x01, }
\newline\textit{CL\_\-GMS\_\-TRACK\_\-CHANGES        = 0x02, }
\newline\textit{CL\_\-GMS\_\-TRACK\_\-CHANGES\_\-ONLY   = 0x04  }
\newline\textit{\} ClGmsTrackFlagsT;}
\newline
The enumeration, {\tt{ClGmsTrackFlagsT}}, contains flags for tracking request flag.
\begin{itemize}
\item
\textit{CL\_\-GMS\_\-TRACK\_\-CURRENT} - Returns current view.
\item
\textit{CL\_\-GMS\_\-TRACK\_\-CHANGES} - To subscribe for complete view notifications.
\item
\textit{CL\_\-GMS\_\-TRACK\_\-CHANGES\_\-ONLY} - To subscribe for delta notifications.
\end{itemize}


\subsection{ClGmsClusterNotificationBufferT}
  \index{ClGmsClusterNotificationBufferT@{ClGmsClusterNotificationBufferT}}
\begin{tabbing}
xx\=xx\=xx\=xx\=xx\=xx\=xx\=xx\=xx\=\kill
\textit{typedef struct \{}\\
\>\>\>\>\textit{ClUint64T                   viewNumber;}\\
\>\>\>\>\textit{ClUint32T                   numberOfItems;}\\
\>\>\>\>\textit{ClGmsClusterNotificationT  *notification;}\\
\>\>\>\>\textit{ClGmsNodeIdT                leader;}\\
\>\>\>\>\textit{ClGmsNodeIdT                deputy;}\\
\>\>\>\>\textit{ClBoolT                     leadershipChanged;}\\
\textit{\} ClGmsClusterNotificationBufferT;}\end{tabbing} 
The
structure, {\tt{ClGmsClusterNotificationBufferT}}, contains a buffer
to communicate the view. The view is the list of nodes and their
status. deputy Node marked as deputy. The attributes of this structure are:
\begin{itemize}
\item \textit{viewNumber} - Current view number.
\item \textit{numberOfItems} - Length of the notification array.
\item \textit{notification} - Array of nodes.
\item \textit{leader} - Node ID of current leader.
\item \textit{deputy} - Node marked as deputy.
\item \textit{leadershipChanged} - Check if the leader has changed since the last view.
\end{itemize}


\subsection{ClGmsGroupInfoT}
  \index{ClGmsGroupInfoT@{ClGmsGroupInfoT}}
\begin{tabbing}
xx\=xx\=xx\=xx\=xx\=xx\=xx\=xx\=xx\=\kill
\textit{typedef struct ClGmsGroupInfo\{}\\
\>\>\>\>\textit{ClGmsGroupNameT     groupName;}\\
\>\>\>\>\textit{ClGmsGroupIdT       groupId;}\\
\>\>\>\>\textit{ClGmsGroupParamsT   groupParams;}\\
\>\>\>\>\textit{ClUint32T           noOfMembers;}\\
\>\>\>\>\textit{ClBoolT             setForDelete;}\\
\>\>\>\>\textit{ClIocMulticastAddressT iocMulticastAddr;}\\
\>\>\>\>\textit{ClTimeT                 creationTimestamp;}\\
\>\>\>\>\textit{ClTimeT                 lastChangeTimestamp;}\\
\textit{\} ClGmsGroupInfoT;}\end{tabbing} 
The structure, {\tt{ClGmsGroupInfoT}}, contains the values of a group. Its
attributes have the following interpretation:
\begin{itemize}
\item \textit{groupName} - Name of the group.
\item \textit{groupId} - Id of the group.
\item \textit{groupParams} - Parameters of the group.
\item \textit{noOfMembers} - Number of members in the group.
\item \textit{setForDelete} - NO more joins are allowed.
\item \textit{iocMulticastAddr} - IOC multicast address created by GMS.
\item \textit{creationTimestamp} - Time when the group was created.
\item \textit{lastChangeTimestamp} - Time when the last view changed.
\end{itemize}



\subsection{ClGmsGroupInfoListT}
  \index{ClGmsGroupInfoListT@{ClGmsGroupInfoListT}}
\begin{tabbing}
xx\=xx\=xx\=xx\=xx\=xx\=xx\=xx\=xx\=\kill
\textit{typedef struct clGmsGroupInfoList\{}\\
\>\>\>\>\textit{ClUint32T           noOfGroups;}\\
\>\>\>\>\textit{ClGmsGroupInfoT     *groupInfoList;}\\
\textit{\} ClGmsGroupInfoListT;}\end{tabbing} 
The structure, {\tt{ClGmsGroupInfoListT}} is used to return the information on
all the existing groups. The attributes of this structure are:
interpretation:
\begin{itemize}
\item \textit{noOfGroups} - Number of existing groups.
\item \textit{groupInfoList} - Array of ClGmsGroupT data.
\end{itemize}

\newpage

\section{Library Life Cycle APIs}
\subsection{clGmsInitialize}
\index{clGmsInitialize@{clGmsInitialize}}
\hypertarget{pagegms101}{}\paragraph{cl\-Gms\-Initialize}\label{pagegms101}
\begin{Desc}
\item[Synopsis:]Initializes the GMS library and registers the callback functions.\end{Desc}
\begin{Desc}
\item[Header File:]clGmsViewApi.h\end{Desc}
\begin{Desc}
\item[Syntax:]

\footnotesize\begin{verbatim}         ClRcT clGmsInitialize(
                                		CL_OUT   ClGmsHandleT          *gmsHandle,
                                		CL_IN    const ClGmsCallbacksT *gmsCallbacks,
                                		CL_INOUT ClVersionT            *version);
\end{verbatim}
\normalsize
\end{Desc}
\begin{Desc}
\item[Parameters:]
\begin{description}
\item[{\em gms\-Handle:}](in/out) GMS service handle created by the library. This is used in subsequent use of the library in this session.
\item[{\em gms\-Callbacks:}](in) This is an optional parameter. This is a pointer to the array of callback functions you can provide.
If {\tt{gms\-Callbacks}} is NULL, the callback is not registered.
If {\tt{gms\-Callbacks}} is not NULL, it acts a pointer to
{\tt{cl\-Gms\-Callbacks\-T}} structure, containing the callback functions
of the process that the Group Membership Service invokes. Only
non-NULL callback functions in this structure are registered. If any
callback is NULL, the corresponding asynchronous operation returns
error.

\item[{\em version:}](in/out) It can have the following values: \begin{itemize}
\item On input, this is the version required by you. \item On return, the library returns the version it supports. \end{itemize}
\end{description}
\end{Desc}
\begin{Desc}
\item[Return values:]
\begin{description}
\item[{\em CL\_\-OK:}]The function executed successfully.
\item[{\em CL\_\-ERR\_\-NOT\_\-INITIALIZED:}]The library was not initialized.
\item[{\em CL\_\-ERR\_\-NULL\_\-POINTER:}]{\tt{gmsHandle}}, {\tt{gmsCallbacks}}, or {\tt{version}} contains a NULL pointer.
\item[{\em CL\_\-ERR\_\-VERSION\_\-MISMATCH:}]The requested version is not compatible with the library version supported by OpenClovis ASP.
\item[{\em CL\_\-ERR\_\-NO\_\-RESOURCE:}]An instance or a new handle cannot be created.\end{description}
\end{Desc}
\begin{Desc}
\item[Description:]This function initializes the Group Membership Service (GMS) for the invoking process, registers the various callback functions
and negotiates the version of GMS library. This function must be invoked prior to the invocation of any other Group Membership Service
functionality. The handle {\tt{gms\-Handle}} is returned as the reference to this association between the process and the Group Membership Service.
The process uses this handle in subsequent communication with the GMS.\end{Desc}
\begin{Desc}
\item[Library File:]Cl\-Gms\end{Desc}
\begin{Desc}
\item[Related Function(s):]\hyperlink{pagegms102}{cl\-Gms\-Finalize} \end{Desc}
\newpage


\subsection{clGmsFinalize}
\index{clGmsFinalize@{clGmsFinalize}}
\hypertarget{pagegms102}{}\paragraph{cl\-Gms\-Finalize}\label{pagegms102}
\begin{Desc}
\item[Synopsis:]Finalizes the handle associated with a prior initialization of the GMS client library.\end{Desc}
\begin{Desc}
\item[Header File:]clGmsViewApi.h\end{Desc}
\begin{Desc}
\item[Syntax:]

\footnotesize\begin{verbatim}         ClRcT clGmsFinalize(
                                		CL_IN ClGmsHandleT gmsHandle);
\end{verbatim}
\normalsize
\end{Desc}
\begin{Desc}
\item[Parameters:]
\begin{description}
\item[{\em gms\-Handle:}](in) The handle, obtained through the \textit{clGmsInitialize()} function, designating this
particular initialization of the Group Membership Service.\end{description}
\end{Desc}
\begin{Desc}
\item[Return values:]
\begin{description}
\item[{\em CL\_\-OK:}]The function executed successfully. 
\item[{\em CL\_\-ERR\_\-INVALID\_\-HANDLE:}]{\tt{gmsHandle}} is an invalid handle.\end{description}
\end{Desc}
\begin{Desc}
\item[Description:]The {\tt{clGmsFinalize()}} function closes the association, represented by the {\tt{gms\-Handle}} parameter,
between the invoking process and the Group Membership Service. A process must invoke this function once for each handle it acquires by invoking
{\tt{clGmsInitialize()}}. \par
 \par
 If the {\tt{clGmsFinalize()}} function executes successfully, it releases all the resources acquired when {\tt{clGmsInitialize()}} was invoked.
 It stops any tracking associated with
 the handle and cancels all pending callbacks related to the handle. As the callback invocation is asynchronous, some callbacks
 are processed after this call returns successfully. After {\tt{clGmsFinalize()}} is invoked, the selection
 object is no longer valid.\end{Desc}
\begin{Desc}
\item[Note:]On successful execution of this function, it releases all the resources allocated during the initialization of the library.\end{Desc}
\begin{Desc}
\item[Library File:]Cl\-Gms\end{Desc}
\begin{Desc}
\item[Related Function(s):]\hyperlink{pagegms101}{cl\-Gms\-Initialize} \end{Desc}
\newpage


\section{Functional APIs}
\subsection{clGmsClusterJoin}
\index{clGmsClusterJoin@{clGmsClusterJoin}}
\hypertarget{pagegms201}{}\paragraph{cl\-Gms\-Cluster\-Join}\label{pagegms201}
\begin{Desc}
\item[Synopsis:]Allows a node to join a cluster as a member.\end{Desc}
\begin{Desc}
\item[Header File:]clGmsClusterManageApi.h\end{Desc}
\begin{Desc}
\item[Syntax:]

\footnotesize\begin{verbatim}        ClRcT clGmsClusterJoin(
                                		CL_IN ClGmsHandleT                        gmsHandle,
                                		CL_IN const ClGmsClusterManageCallbacksT *clusterManageCallbacks,
                                		CL_IN ClGmsLeadershipCredentialsT         credentials,
                                		CL_IN ClTimeT                             timeout,
                                		CL_IN ClGmsNodeIdT                        nodeId,
                                		CL_IN ClNameT                            *nodeName);
\end{verbatim}
\normalsize
\end{Desc}
\begin{Desc}
\item[Parameters:]
\begin{description}
\item[{\em gms\-Handle:}](in) The handle, obtained through the {\tt{clGmsInitialize()}} function, designating this particular initialization of the
Group Membership Service. \item[{\em cluster\-Manage\-Callbacks:}](in) Callbacks for managing the cluster.
\item[{\em credentials:}](in) This is an integer
value specifying the leadership credibility of the node. Larger the value, higher is the possibility of the node becoming a leader. Member with
credentials {\tt{CL\_\-GMS\_\-INELIGIBLE\_\-CREDENTIALS}} cannot participate in the leader election.
\item[{\em timeout:}](in) If the cluster join is not
completed within this time, then the join request is timed out. \item[{\em node\-Id:}](in) Node ID of the member that will join the cluster.
\item[{\em node\-Name:}](in) Name of the node that will join the cluster.\end{description}
\end{Desc}
\begin{Desc}
\item[Return values:]
\begin{description}
\item[{\em CL\_\-OK:}]The function executed successfully.
\item[{\em CL\_\-ERR\_\-INVALID\_\-HANDLE:}]{\tt{gmsHandle}} is an invalid handle. The handle must be obtained from the
{\tt{clGmsInitialize()}} function.
\item[{\em CL\_\-ERR\_\-TIMEOUT:}]The join request timed out.
\item[{\em CL\_\-ERR\_\-ALREADY\_\-EXIST:}]The node is already part of the cluster.
\item[{\em CL\_\-ERR\_\-INVALID\_\-PARAMETER:}]An invalid parameter has been passed to the function. A parameter is not set correctly.
\item[{\em CL\_\-ERR\_\-NULL\_\-POINTER:}]{\tt{cluster\-Manage\-Callbacks}} or {\tt{node\-Name}} contains a NULL pointer.
\item[{\em CL\_\-ERR\_\-TRY\_\-AGAIN:}] GMS server is not ready to process the request.
\end{description}
\end{Desc}
\begin{Desc}
\item[Description:]This function is used to include a node to the cluster as a member of the cluster. Success or failure is reported through the
return value. Members who have registered for tracking, get notified by the tracking callback function.\end{Desc}
\begin{Desc}
\item[Library File:]Cl\-Gms\end{Desc}
\begin{Desc}
\item[Note:] This function is used internally by the CPM service. The user is not required to use this function.\end{Desc}
\begin{Desc}
\item[Related Function(s):]\hyperlink{pagegms203}{cl\-Gms\-Cluster\-Leave} \end{Desc}
\newpage


\subsection{clGmsClusterLeave}
\index{clGmsClusterLeave@{clGmsClusterLeave}}
\hypertarget{pagegms203}{}\paragraph{cl\-Gms\-Cluster\-Leave}\label{pagegms203}
\begin{Desc}
\item[Synopsis:]Allows a node to leave a cluster.\end{Desc}
\begin{Desc}
\item[Header File:]clGmsClusterManageApi.h\end{Desc}
\begin{Desc}
\item[Syntax:]

\footnotesize\begin{verbatim}        ClRcT clGmsClusterLeave(
                                		CL_IN ClGmsHandleT                        gmsHandle,
                                		CL_IN ClTimeT                             timeout,
                                		CL_IN ClGmsNodeIdT                        nodeId);
\end{verbatim}
\normalsize
\end{Desc}
\begin{Desc}
\item[Parameters:]
\begin{description}
\item[{\em gms\-Handle:}](in) The handle, obtained through the {\tt{clGmsInitialize()}} function, designating this particular initialization of the
Group Membership Service \item[{\em timeout:}](in) If the cluster leave operation is not completed within this time, then the leave request is timed out.
\item[{\em node\-Id:}](in) Node ID of the member that is leaving the cluster.\end{description}
\end{Desc}
\begin{Desc}
\item[Return values:]
\begin{description}
\item[{\em CL\_\-OK:}]The function executed successfully.
\item[{\em CL\_\-ERR\_\-INVALID\_\-HANDLE:}]{\tt{gmsHandle}} is an invalid handle. The handle must be obtained from the
{\tt{clGmsInitialize()}} function.
\item[{\em CL\_\-ERR\_\-INVALID\_\-PARAMETER:}]An invalid parameter has been passed to the function. A parameter is not set correctly.\end{description}
\end{Desc}
\begin{Desc}
\item[Description:]This function can be used by a node to leave a cluster. After the node leaves the cluster and the groups/components are expelled,
a reason for expulsion is returned through their callback functions.\end{Desc}
\begin{Desc}
\item[Library File:]Cl\-Gms\end{Desc}
\begin{Desc}
\item[Note:] This function is used internally by the CPM service. The user is not required to use this function.\end{Desc}
\begin{Desc}
\item[Related Function(s):]\hyperlink{pagegms201}{cl\-Gms\-Cluster\-Join} \end{Desc}
\newpage




\subsection{clGmsClusterLeaderElect}
\index{clGmsClusterLeaderElect@{clGmsClusterLeaderElect}}
\hypertarget{pagegms205}{}\paragraph{cl\-Gms\-Cluster\-Leader\-Elect}\label{pagegms205}
\begin{Desc}
\item[Synopsis:]Initiates leader election synchronously.\end{Desc}
\begin{Desc}
\item[Header File:]clGmsClusterManageApi.h\end{Desc}
\begin{Desc}
\item[Syntax:]

\footnotesize\begin{verbatim}        ClRcT clGmsClusterLeaderElect(
                                		CL_IN ClGmsHandleT     gmsHandle);
\end{verbatim}
\normalsize
\end{Desc}
\begin{Desc}
\item[Parameters:]
\begin{description}
\item[{\em gms\-Handle:}](in) The handle, obtained through the {\tt{clGmsInitialize()}} function, designating this particular initialization of the Group
Membership Service\end{description}
\end{Desc}
\begin{Desc}
\item[Return values:]
\begin{description}
\item[{\em CL\_\-OK:}]The function executed successfully.
\item[{\em CL\_\-ERR\_\-INVALID\_\-HANDLE:}]{\tt{gmsHandle}} is an invalid handle. The handle must be obtained from the
{\tt{clGmsInitialize()}} function.\end{description}
\end{Desc}
\begin{Desc}
\item[Description:]This function is used to initiate leader election synchronously. The elected leader is announced through the tracking callback.
This function is invoked when a node leaves or joins, or on any event which would alter the leadership of the cluster. The algorithm is then run
by the GMS server engine, and a leader and deputy leader are elected.\end{Desc}
\begin{Desc}
\item[Library File:]Cl\-Gms\end{Desc}
\begin{Desc}
\item[Related Function(s):]None \end{Desc}
\newpage


\subsection{clGmsClusterMemberEject}
\index{clGmsClusterMemberEject@{clGmsClusterMemberEject}}
\hypertarget{pagegms206}{}\paragraph{cl\-Gms\-Cluster\-Member\-Eject}\label{pagegms206}
\begin{Desc}
\item[Synopsis:]Removes a member from the cluster.\end{Desc}
\begin{Desc}
\item[Header File:]clGmsClusterManageApi.h\end{Desc}
\begin{Desc}
\item[Syntax:]

\footnotesize\begin{verbatim}        ClRcT clGmsClusterMemberEject(
                                		CL_IN ClGmsHandleT                      gmsHandle,
                                		CL_IN ClGmsNodeIdT                      nodeId,
                                		CL_IN ClGmsMemberEjectReasonT           reason);
\end{verbatim}
\normalsize
\end{Desc}
\begin{Desc}
\item[Parameters:]
\begin{description}
\item[{\em gms\-Handle:}](in) The handle, obtained through the {\tt{clGmsInitialize()}} function, designating this particular initialization of the
Group Membership Service \item[{\em node\-Id:}](in) Node ID of the member to be ejected out. \item[{\em reason:}](in) Reason for ejecting the member out 
of the cluster. A member can be ejected from the cluster either upon request, or for an unknown reason. {\tt{reason\-Code}} can have two values:
\begin{itemize}
\item 
{\tt{CL\_\-GMS\_\-MEMBER\_\-EJECT\_\-REASON\_\-UNKNOWN = 0}}
\item 
{\tt{CL\_\-GMS\_\-MEMBER\_\-EJECT\_\-REASON\_\-API\_\-REQUEST = 1}}
\end{itemize}
\end{description}
\end{Desc}
\begin{Desc}
\item[Return values:]
\begin{description}
\item[{\em CL\_\-OK:}]The function executed successfully.
\item[{\em CL\_\-ERR\_\-INVALID\_\-HANDLE:}]{\tt{gmsHandle}} is an invalid handle. The handle must be obtained from the
{\tt{clGmsInitialize()}} function.
\item[{\em CL\_\-ERR\_\-INVALID\_\-PARAMETER:}]{\tt{nodeId}} or {\tt{reason}} is invalid.\end{description}
\end{Desc}
\begin{Desc}
\item[Description:]This function is used to remove a member forcibly from the cluster. A reason is given when a member is removed.
The tracking members of the cluster are notified
through the tracking callback.\end{Desc}
\begin{Desc}
\item[Library File:]Cl\-Gms\end{Desc}
\begin{Desc}
\item[Related Function(s):]\hyperlink{pagegms205}{cl\-Gms\-Cluster\-Leader\-Elect} \end{Desc}
\newpage

\subsection{clGmsClusterMemberGet}
\index{clGmsClusterMemberGet@{clGmsClusterMemberGet}}
\hypertarget{pagegms105}{}\paragraph{cl\-Gms\-Cluster\-Member\-Get}\label{pagegms105}
\begin{Desc}
\item[Synopsis:]Returns cluster member information.\end{Desc}
\begin{Desc}
\item[Header File:]clGmsViewApi.h\end{Desc}
\begin{Desc}
\item[Syntax:]

\footnotesize\begin{verbatim}        ClRcT  clGmsClusterMemberGet(
                                		CL_IN  ClGmsHandleT             gmsHandle,
                                		CL_IN  ClGmsNodeIdT             nodeId,
                                		CL_IN  ClTimeT                  timeout,
                                		CL_OUT ClGmsClusterMemberT     *clusterMember);
\end{verbatim}
\normalsize
\end{Desc}
\begin{Desc}
\item[Parameters:]
\begin{description}
\item[{\em gms\-Handle:}](in) The handle, obtained through the {\tt{clGmsInitialize()}} function, designating this particular initialization of the Group
Membership Service. 
\item[{\em node\-Id:}](in) The identifier of the cluster node for which the {\tt{cluster\-Node}} information structure is to be
retrieved. 
\item[{\em timeout:}](in) The {\tt{clGmsClusterMemberGet()}} invocation is considered to have failed if it
does not get complete during the time specified through this parameter. 
\item[{\em cluster\-Member:}](out) A pointer to a cluster node
structure that contains information about a cluster node. The invoking process provides space for this structure, and the Group Membership Service fills
in the fields of this structure.\end{description}
\end{Desc}
\begin{Desc}
\item[Return values:]
\begin{description}
\item[{\em CL\_\-OK:}]The function executed successfully.
\item[{\em CL\_\-ERR\_\-INVALID\_\-HANDLE:}]{\tt{gmsHandle}} is an invalid handle. The handle must be obtained from the
{\tt{clGmsInitialize()}} function.
\item[{\em CL\_\-ERR\_\-NULL\_\-POINTER:}]The parameter {\tt{cluster\-Member}} contains a NULL pointer.
\item[{\em CL\_\-ERR\_\-INVALID\_\-PARAM:}]The requested node does not exist.
\item[{\em CL\_\-ERR\_\-TIMEOUT:}]Communication request timed out.\end{description}
\end{Desc}
\begin{Desc}
\item[Description:]This function is used to retrieve the information about a given cluster member and check if the node is a member of the cluster.
You should allocate the space for the node. The information about a cluster member is retrieved synchronously. The node is identified by
{\tt{node\-Id}} parameter. The cluster node information is returned in the {\tt{cluster\-Node}} parameter. \par
 \par
 By invoking this function, a process can obtain the cluster node information for the node, designated by {\tt{node\-Id}}, and can then check the
 member field to determine whether this node is a member of the cluster. If the constant {\tt{CL\_\-GMS\_\-LOCAL\_\-NODE\_\-ID}} is used as 
 {\tt{node\-Id}},  the function returns information about the cluster node that hosts the invoking process.\end{Desc}
\begin{Desc}
\item[Library File:]Cl\-Gms\end{Desc}
\begin{Desc}
\item[Related Function(s):]\hyperlink{pagegms108}{cl\-Gms\-Cluster\-Track},\hyperlink{pagegms109}{cl\-Gms\-Cluster\-Track\-Stop},
\hyperlink{pagegms111}{cl\-Gms\-Cluster\-Member\-Get\-Async} \end{Desc}
\newpage

\subsection{clGmsClusterMemberGetAsync}
\index{clGmsClusterMemberGetAsync@{clGmsClusterMemberGetAsync}}
\hypertarget{pagegms106}{}\paragraph{cl\-Gms\-Cluster\-Member\-Get\-Async}\label{pagegms106}
\begin{Desc}
\item[Synopsis:]Returns information on the cluster node asynchronously through {\tt{clGmsClusterMemberGetCallback()}}.\end{Desc}
\begin{Desc}
\item[Header File:]clGmsViewApi.h\end{Desc}
\begin{Desc}
\item[Syntax:]

\footnotesize\begin{verbatim}        ClRcT clGmsClusterMemberGetAsync(
                                		CL_IN ClGmsHandleT              gmsHandle,
                                		CL_IN ClInvocationT             invocation,
                                		CL_IN ClGmsNodeIdT              nodeId);
\end{verbatim}
\normalsize
\end{Desc}
\begin{Desc}
\item[Parameters:]
\begin{description}
\item[{\em gms\-Handle:}](in) The handle, obtained through the {\tt{clGmsInitialize()}} function, designating this particular initialization of the 
Group Membership Service. 
\item[{\em invocation:}] (in) Correlates the invocation with the corresponding callback. This parameter allows the invoking process to
match this invocation of {\tt{clGmsClusterMemberGetAsync()}} with the corresponding
{\tt{cl\-Gms\-Cluster\-Member\-Get\-Callback()}}. 
\item[{\em node\-Id:}](in) The identifier of the cluster node for which the information is to be retrieved.\end{description}
\end{Desc}
\begin{Desc}
\item[Return values:]
\begin{description}
\item[{\em CL\_\-OK:}]The function executed successfully. \item[{\em CL\_\-ERR\_\-INVALID\_\-HANDLE:}]{\tt{gmsHandle}} is an invalid handle. The handle 
must be obtained from the {\tt{clGmsInitialize()}} function.
\item[{\em CL\_\-ERR\_\-INVALID\_\-PARAM:}]The requested node does not exist.\end{description}
\end{Desc}
\begin{Desc}
\item[Description:]This function is used to query the information on a given cluster node. This function requests information about
a particular cluster node, identified by the {\tt{node\-Id}} parameter. The information about a cluster is provided asynchronously.
\par
If {\tt{CL\_\-GMS\_\-LOCAL\_\-NODE\_\-ID}} is used as {\tt{node\-Id}},
the function returns information about the cluster node that hosts the invoking process. \par
 \par
 The process matches the corresponding callback, {\tt{cl\-Gms\-Cluster\-Member\-Get\-Callback()}},
 with this particular invocation. The {\tt{cl\-Gms\-Cluster\-Member\-Get\-Callback()}} callback function is provided when
 the process invokes the {\tt{clGmsInitialize()}}.\end{Desc}
\begin{Desc}
\item[Library File:]Cl\-Gms\end{Desc}
\begin{Desc}
\item[Related Function(s):]\hyperlink{pagegms103}{cl\-Gms\-Cluster\-Track}, \hyperlink{pagegms104}{cl\-Gms\-Cluster\-Track\-Stop},
\hyperlink{pagegms110}{cl\-Gms\-Cluster\-Member\-Get} \end{Desc}

\newpage
\subsection{clGmsClusterTrack}
\index{clGmsClusterTrack@{clGmsClusterTrack}}
\hypertarget{pagegms103}{}\paragraph{cl\-Gms\-Cluster\-Track}\label{pagegms103}
\begin{Desc}
\item[Synopsis:]Configures the cluster tracking mode.\end{Desc}
\begin{Desc}
\item[Header File:]clGmsViewApi.h\end{Desc}
\begin{Desc}
\item[Syntax:]

\footnotesize\begin{verbatim}        ClRcT clGmsClusterTrack(
                                		CL_IN    ClGmsHandleT           gmsHandle,
                                		CL_IN    ClUint8T               trackFlags,
                                		CL_INOUT ClGmsClusterNotificationBufferT *notificationBuffer);
\end{verbatim}
\normalsize
\end{Desc}
\begin{Desc}
\item[Parameters:]
\begin{description}
\item[{\em gms\-Handle:}](in) The handle, obtained through the {\tt{clGmsInitialize()}} function, designating this particular initialization of the
Group Membership Service. 
\item[{\em track\-Flags:}](in) Requested tracking mode.
\item[{\em notification\-Buffer:}](in/out) This is an optional parameter and is used when {\tt{trackFlags}} is set to {\tt{CL\_\-TRACK\_\-CURRENT}}.
It provides a buffer that contains the current cluster information when the function returns successfully. If it is NULL,
the information is returned through an invocation to {\tt{clGmsClusterTrackCallback()}}.\end{description}
\end{Desc}
\begin{Desc}
\item[Return values:]
\begin{description}
\item[{\em CL\_\-OK:}]The function executed successfully. 
\item[{\em CL\_\-ERR\_\-INVALID\_\-HANDLE:}]{\tt{gmsHandle}} is an invalid handle. The handle must be obtained from the
{\tt{clGmsInitialize()}} function. 
\item[{\em CL\_\-ERR\_\-INVALID\_\-PARAMETER:}]{\tt{CL\_\-GMS\_\-TRACK\_\-CURRENT}} flag is set and notification buffer is provided, but size of 
allocated array is not set (0). 
\item[{\em CL\_\-GMS\_\-ERR\_\-INVALID\_\-TRACKFLAGS:}]Both {\tt{CHANGES}} and {\tt{CHANGES\_\-ONLY}} flags are set. 
\item[{\em CL\_\-ERR\_\-NO\_\-CALLBACK:}]The request was asynchronous, but no callback was registered. 
\item[{\em CL\_\-ERR\_\-TRY\_\-AGAIN:}]Communication error, try again. 
\item[{\em CL\_\-ERR\_\-TIMEOUT:}]Communication request timed out.\end{description}
\end{Desc}
\begin{Desc}
\item[Description:]This function is used to configure the cluster tracking mode for the caller. It can be called subsequently to modify the requested 
tracking mode. This function is used to obtain the current cluster membership and request notification of changes in the cluster membership or of changes
in an attribute of a cluster node, depending on the value of the {\tt{track\-Flags}} parameter. \par
 \par
 These changes are notified through invocation of the {\tt{cl\-Gms\-Cluster\-Track\-Callback()}} callback function, which must have been supplied when the
 process invoked the {\tt{clGmsInitialize()}} call. An application may call {\tt{clGmsClusterTrack()}} repeatedly
 for the same values of {\tt{gms\-Handle}}, regardless of whether the call initiates a one-time status request or a series of callback notifications.
 \end{Desc}
\begin{Desc}
\item[Library File:]Cl\-Gms\end{Desc}
\begin{Desc}
\item[Related Function(s):]\hyperlink{pagegms104}{cl\-Gms\-Cluster\-Track\-Stop} , \hyperlink{pagegms110}{cl\-Gms\-Cluster\-Member\-Get},
\hyperlink{pagegms111}{cl\-Gms\-Cluster\-Member\-Get\-Async} \end{Desc}
\newpage


\subsection{clGmsClusterTrackStop}
\index{clGmsClusterTrackStop@{clGmsClusterTrackStop}}
\hypertarget{pagegms104}{}\paragraph{cl\-Gms\-Cluster\-Track\-Stop}\label{pagegms104}
\begin{Desc}
\item[Synopsis:]Stops all the clusters tracking.\end{Desc}
\begin{Desc}
\item[Header File:]clGmsViewApi.h\end{Desc}
\begin{Desc}
\item[Syntax:]

\footnotesize\begin{verbatim}        ClRcT clGmsClusterTrackStop(
                                		CL_IN ClGmsHandleT              gmsHandle);
\end{verbatim}
\normalsize
\end{Desc}
\begin{Desc}
\item[Parameters:]
\begin{description}
\item[{\em gms\-Handle:}](in) The handle, obtained through the {\tt{clGmsInitialize()}} function, designating this particular initialization of the
Group Membership Service.\end{description}
\end{Desc}
\begin{Desc}
\item[Return values:]
\begin{description}
\item[{\em CL\_\-OK:}]The function executed successfully. \item[{\em CL\_\-ERR\_\-INVALID\_\-HANDLE:}]{\tt{gmsHandle}} is an invalid handle. The handle
must be obtained from the
{\tt{clGmsInitialize()}} function. \end{description}
\end{Desc}
\begin{Desc}
\item[Description:]This function is used to immediately stop the tracking of all the clusters for a given client. This function stops any further 
notifications through the handle {\tt{gms\-Handle}}. Pending callbacks are removed. This is usually invoked during shut down of the application.\end{Desc}
\begin{Desc}
\item[Library File:]Cl\-Gms\end{Desc}
\begin{Desc}
\item[Related Function(s):]\hyperlink{pagegms103}{cl\-Gms\-Cluster\-Track},\hyperlink{pagegms110}{cl\-Gms\-Cluster\-Member\-Get},
\hyperlink{pagegms111}{cl\-Gms\-Cluster\-Member\-Get\-Async} \end{Desc}
\newpage


\subsection{clGmsGroupCreate}
\index{clGmsGroupCreate@{clGmsGroupCreate}}
\hypertarget{pagegms110}{}\paragraph{cl\-Gms\-Group\-Create}\label{pagegms110}

\begin{Desc}

\item[Synopsis:]

 Creates a group.  \end{Desc}

\begin{Desc}
  \item[Header File:]
  clGmsGroupManageApi.h\end{Desc}

  \begin{Desc}
  \item[Syntax:]
  \footnotesize\begin{verbatim}        ClRcT clGmsGroupCreate(
                            			CL_IN    ClGmsHandleT                        gmsHandle,
                            			CL_IN    ClGmsGroupNameT                    *groupName,
                            			CL_OUT   ClGmsGroupIdT                      *groupId,
                            			CL_IN    const ClGmsGroupManageCallbacksT   *groupManageCallbacks,
                            			CL_INOUT ClGmsGroupParamsT                  *groupParams);
 \end{verbatim}
  \normalsize
  \end{Desc}

 \begin{Desc}
\item[Parameters:]
\begin{description}
  \item[{\em gmsHandle:}] (in) gmsHandle provided during {\tt{clGmsInitialize()}}.
 \item[{\em groupName:}] (in) Name of the group. Specify the value and length.
 \item[{\em groupId:}] (out) Pointer to the memory to store groupId generated by GMS.
 \item[{\em groupManageCallbacks:}](in) It can be NULL as currently no functionality is provided.
 \item[{\em groupParams:}] (in/out)Includes parameters such as isIocGroup etc. By default all
              the groups are IOC Groups.
 \end{description}
 \end{Desc}

 \begin{Desc}
 \item[Return values:]
 \begin{description}
 \item[{\em CL\_\-OK:}]The function executed successfully.
\item[{\em CL\_\-ERR\_\-TRY\_\-AGAIN:}] Server is not ready to serve group functionality.
      \item[{\em CL\_\-ERR\_\-VERSION\_\-MISMATCH:}] Client version is not supported.
      \item[{\em CL\_\-ERR\_\-INVALID\_\-HANDLE:}] {\tt{gmsHandle}} is an invalid handle. The handle must be obtained from the
{\tt{clGmsInitialize()}} function. 
      \item[{\em CL\_\-ERR\_\-NULL\_\-POINTER:}] {\tt{groupName}} or {\tt{groupId}} params are NULL.
      \item[{\em CL\_\-ERR\_\-NO\_\-MEMORY:}] Could not allocate memory to create groups.
      \item[{\em CL\_\-ERR\_\-NO\_\-RESOURCE:}] Could not allocate resource to create groups.
      \item[{\em CL\_\-ERR\_\-ALREADY\_\-EXIST:}] Group is already created. At this point the created
                                groupId is returned in groupId parameter.
      \item[{\em CL\_\-ERR\_\-TIMEOUT:}] Group Creation timeout.
     \item[{\em CL\_\-ERR\_\-UNSPECIFIED:}] Group creation failed with unknown error.

  \end{description}
   \end{Desc}

  \begin{Desc}
  \item[Description:]

User needs to pass groupName and groupId pointer. GMS will
 generate a {\tt{groupId}} which will be unique across the cluster, and it will be
 returned through {\tt{groupId}} pointer. \end{Desc}
 \begin{Desc}
\item[Library File:]Cl\-Gms\end{Desc}
\begin{Desc}
\item[Related Function(s):] \hyperlink{pagegms111}{clGmsGroupDestroy}.
  \end{Desc}




  \newpage
  \subsection{clGmsGroupDestroy}
  \index{clGmsGroupDestroy@{clGmsGroupDestroy}}
  \hypertarget{pagegms111}{}\paragraph{cl\-Gms\-Group\-Destroy}\label{pagegms111}
  \begin{Desc}

  \item[Synopsis:]

  Destroys a group.  \end{Desc}

  \begin{Desc}
    \item[Header File:]
    clGmsGroupManageApi.h\end{Desc}

    \begin{Desc}
    \item[Syntax:]
    \footnotesize\begin{verbatim}       ClRcT clGmsGroupDestroy(
                            			CL_IN   ClGmsHandleT                    gmsHandle,
                            			CL_IN   ClGmsGroupIdT                   groupId);

   \end{verbatim}
    \normalsize
    \end{Desc}

   \begin{Desc}
  \item[Parameters:]
  \begin{description}
    \item[{\em gmsHandle:}] (in) {\tt{gmsHandle}} provided during {\tt{clGmsInitialize()}}.
   \item[{\em groupId:}] (in) {\tt{groupId}} provided during {\tt{GroupCreate()}}
   \end{description}
   \end{Desc}

   \begin{Desc}
   \item[Return values:]
   \begin{description}
   \item[{\em CL\_\-OK:}]The function executed successfully.
  \item[{\em CL\_\-ERR\_\-TRY\_\-AGAIN:}] Server is not ready to serve group functionality.
      \item[{\em CL\_\-ERR\_\-VERSION\_\-MISMATCH:}] Client version is not supported.
     \item[{\em CL\_\-ERR\_\-INVALID\_\-HANDLE:}] {\tt{gmsHandle}} is an invalid handle. The handle must be obtained from the
{\tt{clGmsInitialize()}} function. 
     \item[{\em CL\_\-ERR\_\-DOESNT\_\-EXIST:}] Requested group with groupId doesn't exist.
     \item[{\em CL\_\-ERR\_\-INUSE:}] Group is not empty. However, the group will be set {\tt{inActive}} and 
                                no further joins can occur.

    \end{description}
     \end{Desc}

    \begin{Desc}
    \item[Description:]

The group will be destroyed on all nodes
 across the cluster. \end{Desc}
   \begin{Desc}
  \item[Library File:]Cl\-Gms\end{Desc}
  \begin{Desc}
  \item[Related Function(s):] \hyperlink{pagegms110}{clGmsGroupCreate}
  \end{Desc}




   \newpage
    \subsection{clGmsGroupJoin}
    \index{clGmsGroupJoin@{clGmsGroupJoin}}
  \hypertarget{pagegms112}{}\paragraph{cl\-Gms\-Group\-Join}\label{pagegms112}
    \begin{Desc}

    \item[Synopsis:]
Allows a member to join a group.  \end{Desc}

    \begin{Desc}
      \item[Header File:]
      clGmsGroupManageApi.h\end{Desc}

      \begin{Desc}
      \item[Syntax:]
      \footnotesize\begin{verbatim}
       ClRcT clGmsGroupJoin(
          		CL_IN ClGmsHandleT                      gmsHandle,
          		CL_IN ClGmsGroupIdT                     groupId,
          		CL_IN ClGmsMemberIdT                    memberId,
          		CL_IN ClGmsMemberNameT                 *memberName,
          		CL_IN ClGmsLeadershipCredentialsT       credentials,
          		CL_IN ClTimeT                           timeout);

     \end{verbatim}
      \normalsize
      \end{Desc}

     \begin{Desc}
    \item[Parameters:]
    \begin{description}
      \item[{\em gmsHandle:}] (in) {\tt{gmsHandle}} provided during {\tt{clGmsInitialize()}}.
     \item[{\em groupId:}] (in) {\tt{groupId}} provided during {\tt{GroupCreate()}}.
    \item[{\em memberId:}] (in) ID of the member joining the group.
    \item[{\em memberName:}] (in) Name of the member joining the group. This is optional.
    \item[{\em credentials:}] Leadership credentials of the group member. This parameter is currently not used by GMS. It is meant for future 
    enhancements.
    \item[{\em timeout:}] (in) Join timeout.

   \end{description}
     \end{Desc}

     \begin{Desc}
     \item[Return values:]
     \begin{description}
     \item[{\em CL\_\-OK:}]The function executed successfully.
    \item[{\em CL\_\-ERR\_\-TRY\_\-AGAIN:}] Server is not ready to serve group functionality.
        \item[{\em CL\_\-ERR\_\-VERSION\_\-MISMATCH:}] Client version is not supported.
       \item[{\em CL\_\-ERR\_\-INVALID\_\-HANDLE:}] {\tt{gmsHandle}} is an invalid handle. The handle must be obtained from the
{\tt{clGmsInitialize()}} function. 
       \item[{\em CL\_\-ERR\_\-DOESNT\_\-EXIST:}] Requested group with groupId doesn't exist.
                  \item[{\em CL\_\-ERR\_\-TIMEOUT:}] {\tt{Groupjoin()}} timeout.
                  \item[{\em CL\_\-ERR\_\-ALREADY\_\-EXIST:}] The application is an existing member of the group.
          \item[{\em CL\_\-ERR\_\-INVALID\_\-OPERATION:}] Join is denied as the group is marked to be destroyed.
      \end{description}
       \end{Desc}

      \begin{Desc}
      \item[Description:]
    This function can be used by any application to join an existing group as a member. Applications which have registered for track notifications will
    be notified by invoking {\tt{clGmsGroupTrackCallback()}}.
 \end{Desc}
     \begin{Desc}
    \item[Library File:]Cl\-Gms\end{Desc}
    \begin{Desc}
  \item[Related Function(s):] \hyperlink{pagegms113}{clGmsGroupLeave}
  \end{Desc}



  \newpage
      \subsection{clGmsGroupLeave}
      \index{clGmsGroupLeave@{clGmsGroupLeave}}
  \hypertarget{pagegms113}{}\paragraph{cl\-Gms\-Group\-Leave}\label{pagegms113}
      \begin{Desc}

      \item[Synopsis:]

  Allows a member to leave a group.  \end{Desc}

      \begin{Desc}
        \item[Header File:]
        clGmsGroupManageApi.h\end{Desc}

        \begin{Desc}
        \item[Syntax:]
        \footnotesize\begin{verbatim}
    ClRcT clGmsGroupLeave(
            CL_IN ClGmsHandleT                      gmsHandle,
            CL_IN ClGmsGroupIdT                     groupId,
            CL_IN ClGmsMemberIdT                    memberId,
            CL_IN ClTimeT                           timeout);

       \end{verbatim}
        \normalsize
        \end{Desc}

       \begin{Desc}
      \item[Parameters:]
      \begin{description}
        \item[{\em gmsHandle:}] (in) {\tt{gmsHandle}} provided during {\tt{clGmsInitialize()}}.
       \item[{\em groupId:}] (in) {\tt{groupId}} provided during {\tt{GroupCreate()}}.
      \item[{\em memberId:}] (in) ID of the member joining the group.
      \item[{\em timeout:}] (in) Join timeout.

     \end{description}
       \end{Desc}

       \begin{Desc}
       \item[Return values:]
       \begin{description}
       \item[{\em CL\_\-OK:}]The function executed successfully.
      \item[{\em CL\_\-ERR\_\-TRY\_\-AGAIN:}] Server is not ready to serve group functionality.
          \item[{\em CL\_\-ERR\_\-VERSION\_\-MISMATCH:}] Client version is not supported.
         \item[{\em CL\_\-ERR\_\-INVALID\_\-HANDLE:}] {\tt{gmsHandle}} is an invalid handle. The handle must be obtained from the
{\tt{clGmsInitialize()}} function. 
         \item[{\em CL\_\-ERR\_\-DOESNT\_\-EXIST:}] The member with given {\tt{memberId}} does not exist in the given group with {\tt{groupId}}.
            \item[{\em CL\_\-GMS\_\-ERR\_\-GROUP\_\-DOESNT\_\-EXIST:}] The requested group does not exist.
             \item[{\em CL\_\-ERR\_\-TIMEOUT:}] {\tt{Groupjoin}} timeout.
        \end{description}
         \end{Desc}

        \begin{Desc}
        \item[Description:]

This function can be used by a member of the group to leave the group. Applications registered for track notifications on the group,
will get notified through {\tt{clGmsGroupTrackCallback()}}.  \end{Desc}
       \begin{Desc}
      \item[Library File:]Cl\-Gms\end{Desc}
      \begin{Desc}
  \item[Related Function(s):] \hyperlink{pagegms112}{clGmsGroupJoin}
  \end{Desc}


\newpage
\subsection{clGmsGroupTrack}
\index{clGmsGroupTrack@{clGmsGroupTrack}}
\hypertarget{pagegms107}{}\paragraph{cl\-Gms\-Group\-Track}\label{pagegms107}

\begin{Desc}

\item[Synopsis:]Configures the group tracking mode.\end{Desc}
\begin{Desc}
  \item[Header File:]
  clGmsViewApi.h\end{Desc}
\begin{Desc}
  \item[Syntax:]
  \footnotesize\begin{verbatim}      ClRcT clGmsGroupTrack(
                                		CL_IN    ClGmsHandleT           gmsHandle,
                                		CL_IN    ClGmsGroupIdT          groupId,
                                		CL_IN    ClUint8T               trackFlags,
                                		CL_INOUT ClGmsGroupNotificationBufferT *notificationBuffer);
  \end{verbatim}
  \normalsize
  \end{Desc}

\begin{Desc}
\item[Parameters:]
\begin{description}
  \item[{\em gms\-Handle:}] (in) Handle of the GMS service session.
 \item[{\em groupId:}] (in) Id of the group.
  \item[{\em trackFlags:}] (in) Requested tracking mode.
  \item[{\em notificationBuffer:}] (in/out) his is an optional parameter and is used when {\tt{trackFlags}} is set to {\tt{CL\_\-TRACK\_\-CURRENT}}.
It provides a buffer that contains the current cluster information
when the function returns successfully. If it is NULL, the information is
returned through an invocation to {\tt{clGmsGroupTrackCallback()}}
\end{description}
\end{Desc}

\begin{Desc}
\item[Return values:]
\begin{description}
\item[{\em CL\_\-OK:}]The function executed successfully.
\item[{\em CL\_\-ERR\_\-INVALID\_\-HANDLE:}] {\tt{gmsHandle}} is an invalid handle. The handle must be obtained from the
{\tt{clGmsInitialize()}} function. 
\item[{\em CL\_\-GMS\_\-ERR\_\-GROUP\_\-DOES\_\-NOT\_\-EXIST:}] The requested group does not exist.
 \item[{\em CL\_\-ERR\_\-INVALID\_\-PARAMETER:}] An invalid parameter has been passed to the function. A parameter is not set correctly.
  \item[{\em CL\_\-ERR\_\-NULL\_\-POINTER:}] {\tt{notificationBuffer}} contains a NULL pointer.
 \end{description}
  \end{Desc}

\begin{Desc}
\item[Description:]

  This function is used to configure the group tracking mode for the caller. It can be called subsequently to modify the requested tracking mode.
  This function is used to obtain the current group membership and request notification of changes in the cluster membership or of changes in an
  attribute of a cluster node, depending on the value of the {\tt{track\-Flags}} parameter. \par
   \par
   These changes are notified through invocation of the {\tt{cl\-Gms\-Cluster\-Track\-Callback()}} callback function, which must have been supplied when the
   process invoked the {\tt{clGmsInitialize()}} call. An application may call {\tt{clGmsClusterTrack()}} repeatedly
   for the same values of {\tt{gms\-Handle}}, regardless of whether the call initiates a one-time status request or a series of callback notifications.
\end{Desc}
 \begin{Desc}
\item[Library File:]Cl\-Gms\end{Desc}
\begin{Desc}
  \item[Related Function(s):] \hyperlink{pagegms108}{clGmsGroupTrackStop}
  \end{Desc}

\newpage
\subsection{clGmsGroupTrackStop}
\index{clGmsGroupTrackStop@{clGmsGroupTrackStop}}
\hypertarget{pagegms108}{}\paragraph{cl\-Gms\-Group\-Track\-Stop}\label{pagegms108}

\begin{Desc}

\item[Synopsis:]
  Stops all the group tracking.\end{Desc}

\begin{Desc}
  \item[Header File:]
  clGmsViewApi.h\end{Desc}

 \begin{Desc}
  \item[Syntax:]
  \footnotesize\begin{verbatim}       ClRcT clGmsGroupTrackStop(
                                		CL_IN ClGmsHandleT              gmsHandle,
                                		CL_IN ClGmsGroupIdT             groupId);
  \end{verbatim}
  \normalsize
  \end{Desc}

\begin{Desc}
\item[Parameters:]
\begin{description}
  \item[{\em gmsHandle:}]
 (in) Handle of the GMS service session.
  \item[{\em  groupId: }](in) ID of the group.
\end{description}
\end{Desc}

\begin{Desc}
\item[Return values:]
\begin{description}
\item[{\em CL\_\-OK:}]The function executed successfully.
\item[{\em CL\_\-ERR\_\-INVALID\_\-HANDLE:}] {\tt{gmsHandle}} is an invalid handle. The handle must be obtained from the
{\tt{clGmsInitialize()}} function. 
 \item[{\em CL\_\-GMS\_\-ERR\_\-GROUP\_\-DOESNT\_\-EXIST:}] The requested group does not exist.
 \end{description}
  \end{Desc}


\begin{Desc}
\item[Description:]

  This function is used to immediately stop all the group tracking for a group.
\end{Desc}


 \begin{Desc}
\item[Library File:]Cl\-Gms\end{Desc}
\begin{Desc}
  \item[Related Function(s):] \hyperlink{pagegms107}{clGmsGroupTrack}
  \end{Desc}



\newpage
\subsection{clGmsGetGroupInfo}
\index{clGmsGetGroupInfo@{clGmsGetGroupInfo}}
\hypertarget{pagegms109}{}\paragraph{cl\-Gms\-Get\-Group\-Info}\label{pagegms109}


\begin{Desc}

\item[Synopsis:]
  Returns the information of a group specified by the {\tt{groupName}}.\end{Desc}

\begin{Desc}
  \item[Header File:]
  clGmsViewApi.h\end{Desc}

  \begin{Desc}
  \item[Syntax:]
  \footnotesize\begin{verbatim}       ClRcT clGmsGetGroupInfo(
                           		CL_IN     ClGmsHandleT             gmsHandle,
                           		CL_IN     ClGmsGroupNameT         *groupName,
                           		CL_IN     ClTimeT                  timeout,
                           		CL_INOUT  ClGmsGroupInfoT         *groupInfo);
 \end{verbatim}
  \normalsize
  \end{Desc}

 \begin{Desc}
\item[Parameters:]
\begin{description}
  \item[{\em gmsHandle:}] (in) Handle of the GMS service session.
  \item[{\em groupName:}] (in) Pointer to {\tt{ClGmsGroupNameT}} structure holding the value and
                    length for the name of the group for which info is requested.
  \item[{\em timeout:}] (in) Max timeout value for the function.
  \item[{\em groupInfo:}] (in/out) Pointer to {\tt{ClGmsGroupInfoT}} structure. User has to allocate
                    memory for the {\tt{ClGmsGroupInfoT}} structure and pass the
                    address of the memory location through groupInfo pointer.
                    GMS will fill-in the values in the memory pointed by groupInfo
\end{description}
\end{Desc}

\begin{Desc}
\item[Return values:]
\begin{description}
\item[{\em CL\_\-OK:}]The function executed successfully.
\item[{\em CL\_\-ERR\_\-INVALID\_\-HANDLE:}] {\tt{gmsHandle}} is an invalid handle. The handle must be obtained from the
{\tt{clGmsInitialize()}} function. 
 \item[{\em CL\_\-ERR\_\-NULL\_\-POINTER:}] {\tt{groupName}} or {\tt{groupInfo}} pointers are NULL.
  \item[{\em CL\_\-ERR\_\-TIMEOUT:}] Operation timed out.
 \item[{\em CL\_\-ERR\_\-DOESNT\_\-EXIST:}] Group indicated by the {\tt{groupName}} does not exist.
 \item[{\em CL\_\-ERR\_\-TRY\_\-AGAIN:}] Server is not ready to serve the request.
 \end{description}
  \end{Desc}

 \begin{Desc}
 \item[Description:]

  This function is used to retrieve the information on a given group member.
  The space for the member node must be allocated by you.
 \end{Desc}
 \begin{Desc}
\item[Library File:]Cl\-Gms\end{Desc}
\begin{Desc}
  \item[Related Function(s):] \hyperlink{pagegms114}{clGmsGroupsInfoListGet}
  \end{Desc}





\newpage
\subsection{clGmsGroupsInfoListGet}
\index{clGmsGroupsInfoListGet@{clGmsGroupsInfoListGet}}
\hypertarget{pagegms114}{}\paragraph{cl\-Gms\-Groups\-Info\-List\-Get}\label{pagegms114}


\begin{Desc}

\item[Synopsis:]

 Returns the information of all the groups.  \end{Desc}

\begin{Desc}
  \item[Header File:]
  clGmsViewApi.h\end{Desc}

  \begin{Desc}
  \item[Syntax:]
  \footnotesize\begin{verbatim}        ClRcT clGmsGroupsInfoListGet(
                                     		CL_IN     ClGmsHandleT             gmsHandle,
                                     		CL_IN     ClTimeT                  timeout,
                                     		CL_INOUT  clGmsGroupInfoListT          *groups);
 \end{verbatim}
  \normalsize
  \end{Desc}

 \begin{Desc}
\item[Parameters:]
\begin{description}
  \item[{\em gmsHandle:}] (in) Handle of the GMS service session.
  \item[{\em timeout:}] (in) If the operation is not
completed within this time, then the request is timed out.
  \item[{\em groups:}] (in/out) Pointer to the structure holding {\tt{noOfGroups}} param and a pointer.
  The user should specify the pointer to the structure of type {\tt{clGmsGroupInfoT}},
  in which GMS will allocate the memory for groupInfo pointer and fills the {\tt{noOfGroups}}
  value. The user should de-allocate the memory given with {\tt{groupInfo}} pointer.
 \end{description}
 \end{Desc}

 \begin{Desc}
 \item[Return values:]
 \begin{description}
 \item[{\em CL\_\-OK:}]The function executed successfully.
\item[{\em CL\_\-ERR\_\-INVALID\_\-HANDLE:}] {\tt{gmsHandle}} is an invalid handle. The handle must be obtained from the
{\tt{clGmsInitialize()}} function. 
  \item[{\em CL\_\-ERR\_\-INVALID\_\-PARAMETER:}] An invalid parameter has been passed to the function. A parameter is not set correctly.
  \item[{\em CL\_\-ERR\_\-NULL\_\-POINTER:}] {\tt{groups}} is a NULL pointer.

  \end{description}
   \end{Desc}

  \begin{Desc}
  \item[Description:]

  This function is used to retrieve the information on all the groups existing on the node.
  The user should pass the pointer to the {\tt{ClGmsGroupInfoListT}} data structure. GMS
  allocates the memory for all the groups.
 \end{Desc}
 \begin{Desc}
\item[Library File:]Cl\-Gms\end{Desc}
\begin{Desc}
  \item[Related Function(s):] \hyperlink{pagegms109}{clGmsGetGroupInfo}
  \end{Desc}
\end{flushleft}


\chapter{Service Management Information Model}
TBD


\chapter{Service Notifications}
TBD


\chapter{Debug CLI}
TBD
