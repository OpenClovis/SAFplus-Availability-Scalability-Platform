\hypertarget{group__group13}{
\chapter{Introduction}
\label{group__group13}
}

\begin{flushleft}

COR is a repository for ASP Managed Objects (MOs) that represent network element resources. This document provides the functional details of the COR service 
that enables \textit{Read} operation on Managed Objects. This COR service provides APIs that can be used to access the values of run-time attributes 
of an object. An object is identified by its MoID. An application can discover the object hierarchy and can use this service to read the values of the 
run-time attributes.
\par
In a distributed communication system there are many managed objects that are not related to a blade or a node. This is either
because the location of the owner application is irrelevant to the north-bound, or there are multiple applications across multiple 
nodes which have equal stake in the MO. COR provides services that enables modeling and managing of such managed objects.



\chapter{Service Model}
\section{Usage Model}
A management application provide inputs on the desired sets of objects via configuration attributes.
A management application would create/delete/set objects to provide the configuration values. COR realizes this through
object create/delete/set services.
A component registers with COR the object it implements. This component then becomes the Object implementer for these objects. COR invokes the OI
whenever any modification in the associated objects occurs.
An OI (ASP component and application components) implementing the above object would require the read configuration from COR during its 
initialization. This would be performed via COR�s  object get functionality.
An OI would be required to implement any changes in the object supplied during the life time of the object as supplied by Object manager. 
This would be again achieved via the OI registration process. 
An OI would reflect the current status of the object they implement through CORs run time attributes. COR invokes the OI when a read on a
runtime attribute read is initiated by management applicaton.
COR�s notification services provides a mechanism to send notification when specified objects are either are created/deleted/set. 

\section{Functional Description}
\subsection{COR MO-Class Attribute Type}
COR MO classes contains attributes that have one of the following types:
\begin{Desc}
\item
[Configuration]
\end{Desc}
Configuration attributes are the means by which system management applications provide input on the desired sets of objects and their handling 
that an Object Implementer should implement. The set of configuration objects
and attributes constitute the prescriptive part of the information model.
\par
Configuration attributes are typically under the control of system
Management applications. They are of a persistent nature and must survive a
full cluster power-off.
\par
Configuration attributes are read-write attributes from an Object Management
perspective but read-only from an Object Implementer perspective.
Configuration attributes are further qualified by Writable and Initialized parameter. 
\begin{itemize}
\item
Writable: Setting this flag for a configuration attribute indicates that the attribute can be modified. If the flag is not present, the configuration. 
Attribute can only be set when the object is created and cannot be modified or deleted later on.
\item
Initialized: Setting this flag for a configuration attribute indicates that a value must be specified for this attribute when the object is created. 
This flag may not be set in the definition of a configuration attribute, which includes a default value for the attribute.
\end{itemize}

\begin{Desc}
\item
[Runtime]
\end{Desc}

Runtime objects and attributes are the means by which Object Implementers
reflect in the information model the current state of the objects they implement.
The set of runtime objects and attributes constitute the descriptive part of the
information model. Runtime objects and attributes are typically under the control
of Object Implementers.
\par
Runtime objects, which contain persistent runtime attributes are persistent and
must survive a full cluster power-off. Non-persistent runtime attributes do not
survive a full cluster power-off.
\par
Runtime attributes are read-only attributes from an Object Management perspective
but read-write from an Object Implementer perspective
Runtime attributes are further qualified by the following parameters:
\begin{itemize}
\item
Persistent: Setting this flag for runtime attributes indicates that the attribute must be stored in a persistent manner by the COR Service. 
\item
Cached: Setting this flag for a runtime attribute indicates that
the value of the attribute must be cached by the IMM Service.
\end{itemize}
COR Mo Classes and its attributes are defined during modeling phase through IDE.

\begin{Desc}
\item
[COR bundle capability]
\end{Desc}

The COR bundle capability provides a mechanism to execute groups of job in an efficient manner by minimizing the number of RMD calls between 
COR client, COR server and Object Implementer.
\par
A bundle execution can either be transactional or non transactional. COR services ensure that all the jobs in transactional bundle either are executed 
successful or all are failure.  For non transactional bundle, no such restriction exists. In such a bundle their could be jobs that are not successfully
executed by COR.
\par

The bundle semantics of executing jobs is done in four phases by the application:
\begin{itemize}
\item
Bundle Creation.
\item
Read Job Population.
\item
Bundle Apply.
\item
Bundle Finalize.
\end{itemize}


On creation a bundle is empty with no jobs associated. A bundle handle is returned that identifies this bundle.
\par
In the next phase, jobs are added to this bundle.  The added jobs are en-queued at the COR client. Associated with the job are its status and a buffer 
descriptor.  The buffer descriptor contains the value of the attribute to be set or get (based on the operation) The success or failure of jobs execution
is reflected in the status.
\par
Application, upon completion of Job population phase, performs the apply operation. The apply operation can be performed synchronously or 
asynchronously.
\par
An asynchronous apply, queues the bundle (and its job) in the IOC layer and immediately returns. In such a case, after the jobs in the bundle has 
been executed by COR server, the application is informed via execution of a callback.  
\par
 A synchronous apply, sends the bundle request to COR server, waits for the COR server to respond back to this bundle request (for a pre-determined 
 time period). Upon receiving the response the bundle applies is deemed complete.
\par
COR server executes the bundle.. For Get-Job containing configuration attributes or runtime-cached attributes, values are read from COR database. 
Other values of runtime attributes are read from the OI. COR merges the value of all these attributes corresponding to different read-job of a particular
bundle, from different sources and responds back to client.  
\par
For the case of asynchronous bundle apply operation, the application callback is invoked when a response is obtained from the server for the bundle. 
For the case of synchronous bundle operation, the bundle apply API returns after the  response is received from COR server.

\begin{Desc}
\item
[OI Registration Functionality]
\end{Desc}

A managed object that has multiple OIs  is termed as Shared Managed Object. A shared MO class is modeled as hanging from the root of MO class tree. 
IDE enables specification of hierarchically related shared MO classes hanging from root of MO class tree.
\par
OIs  are associated with its respective MO during modeling phase.  Multiple component can act as an OI for a shared MO. For the case of MOs with 
runtime attribute, COR needs to contact exactly one OI to obtain the attribute values. This OI is termed as Primary-OI. During modeling phase, primaryOI
for such MOs needs to be specified. It is upto the application to decide who the primaryOI amongst a given set of OI is. The IDE generates an XML file 
that captures this information. PROV client running in the context of component-OI, parses the XML file and performs the OI-MO association during its 
startup. 
\par
Associated with Each OI and a MO, are the callback functions that implement MO create/set/get/delete operation. IDE generates template functions for 
each OI-MO pair, that the application needs to fill. The IDE also generates tables  to map these callback function with MoID. ASP PROV client running 
in the context of OI parses this table and associates OI with its respective callback.
\par
A Component can de-register and re-register as the OI and as the  primaryOI via COR specified APIs anytime during the life cycle of the component. On a 
death of an OI component COR automatically de-registers the OI (and it being primary). In here again, it is up to the application to decide who the 
primaryOI amongst a given set of OI.


\chapter{Service APIs}

\section{Type Definitions}


\subsection{ClProvTxnDataT}
\index{ClProvTxnDataT@{ClProvTxnDataT}}
\begin{tabbing}
xx\=xx\=xx\=xx\=xx\=xx\=xx\=xx\=xx\=\kill
\textit{typedef struct ClProvTxnData \{}\\
\>\>\>\>\textit{ClUint32T provCmd;}\\
\>\>\>\>\textit{void* pMoId;}\\
\>\>\>\>\textit{ClCorAttrPathPtrT attrPath;}\\
\>\>\>\>\textit{ClCorAttrTypeT    attrType;}\\
\>\>\>\>\textit{ClCorAttrIdT      attrId;}\\
\>\>\>\>\textit{void*             *pProvData;}\\
\>\>\>\>\textit{ClUint32T         size;}\\
\>\>\>\>\textit{ClUint32T         index;}\\
\textit{\} ClProvTxnDataT;}\end{tabbing}
This structure is used to specify attribute properties on which a set or a get has occurred. This structure is passed to the OI callback function when a creation/deletion/set/get operation occurs on an MO.
The OI needs to fill * pProvData in case of a get operation, and apply the value in *pProvData in case of a set operation.
All other members are filled by COR Infrastructure.
\begin{itemize}
\item
\textit{provCmd} - It takes one of the following  values. It specifies the kind of operation.on the MO.
\begin{itemize}
\item
CL\_\-COR\_\-OP\_\-CREATE = 0x1,
\item
CL\_\-COR\_\-OP\_\-SET =    0x2,
\item
CL\_\-COR\_\-OP\_\-DELETE = 0x4,
\item
CL\_\-COR\_\-OP\_\-GET = 0x8
\end{itemize}
\item
\textit{attrPath} - Relevant for contained attributes only. This will be NULL for PROV MSO.
\item
\textit{attrType} - Type of the attribute.
\item
\textit{attrId} - Numerical Identifier of the attribute.
\item
\textit{*pProvData} - On a CL\_\-COR\_\-OP\_\-SET operation, this value needs to be applied to the attribute. On a CL\_\-COR\_\-OP\_\-GET operation this value
needs to supplied by OI. The COR infrastructure allocates the memory both in case of set and get operation.
\item
\textit{size} - This is the size of  *pProvData.
\item
\textit{index} - This is valid for an array attribute. This specifies the index of array attribute.
\end{itemize}


\subsection{clProvOiReadCallbackPtrT}
\index{clProvOiReadCallbackPtrT@{clProvOiReadCallbackPtrT}}
\begin{tabbing}
xx\=xx\=xx\=xx\=xx\=xx\=xx\=xx\=xx\=\kill
\textit{typedef ClRcT (*clProvOiReadCallbackPtrT) (}\\
\>\>\>\>\textit{ClHandleT corBundleHandle,}\\
\textit{ClProvTxnDataT *pProvTxnData);}\end{tabbing}
This is OI callback that provides data for RunTime attribute. 
In the modeling phase a component is associated with multiple MO classes as OI. For each component-MO-class pair a MO read  callback function is generated.
The COR infrastructure invokes this callback when a Get operation occurs on the MO.
This callback will be invoked for each each MO-AttributeId pair present as a part  Get-Job.
The OI needs to fill the value corresponding to the attribute   (pProvTxnData->pProvData). In case of succesfull read OI must return CL_OK
In case the attribute read fails, OI should  return the status as  CL_ERR_NOT_EXISTS.
\begin{itemize}
\item
\textit{corReadSessionHandle} - A unique handle that is common for all jobs in a COR session.
\item
\textit{pThis} - OM handle.
\item
\textit{pProvTxnData} - Pointer to the attribute definition structure.
\end{itemize}




\subsection{clCorBundleCallbackPtrT}
\index{clCorBundleCallbackPtrT@{clCorBundleCallbackPtrT}}
\begin{tabbing}
xx\=xx\=xx\=xx\=xx\=xx\=xx\=xx\=xx\=\kill
\textit{typedef ClRcT (*clCorBundleCallbackPtrT) (}\\
\>\>\>\>\textit{ClCorBundleHandleT BundleHandle,}\\
\textit{ClPtrT userArg )}\end{tabbing}
This call back function is registered for an asynchronous  execution of a bundle, through the clCorBundleApplyAsync API. 
This function is called when the bundle operation is complete. 
\begin{itemize}
\item
\textit{bundleHandle} - Unique handle as returned by clCorBundleInitialize API.
\item
\textit{userArg} - This is the user argument that is supplied while doing clCorBundleApplyAsync.
\end{itemize}



ClCorBundleOperationType 


\subsection{clCorReadSessionCallBackPtrT}
\index{clCorReadSessionCallBackPtrT@{clCorReadSessionCallBackPtrT}}
\begin{tabbing}
xx\=xx\=xx\=xx\=xx\=xx\=xx\=xx\=xx\=\kill
\textit{typedef ClRcT (*clCorReadSessionCallbackPtrT) (}\\
\>\>\>\>\textit{ClCorReadSessionHandleT sessionHandle,}\\
\textit{ClPtrT userArg )}\end{tabbing}
This call back function is registered for an asynchronous read session, through the clCorReadSessionApply API. A value of NULL
indicates that the session is syncrhonous.
This function is called when the session completes. Applicaiton can access status of each MO-attribute read and its value
from this callback function.
\begin{itemize}
\item
\textit{sessionHandle} - Unique session handle as returned by clCorReadSessionInitialize API. 
\item
\textit{userArg} - This is the user argument that is supplied along clCorReadSessionInitialize.
\end{itemize}


\subsection{ClCorReadJobT}
\index{ClCorReadJobT@{ClCorReadJobT}}
\begin{tabbing}
xx\=xx\=xx\=xx\=xx\=xx\=xx\=xx\=xx\=\kill
\textit{struct clCorReadJob \{}\\
\>\>\>\>\textit{ClCorObjHandleT    		objHandle,}\\
\>\>\>\>\textit{ClCorAttrPathPtrT  			attrPath,}\\
\>\>\>\>\textit{ClCorAttrIdT  	attrId,}\\
\textit{\} ;}\end{tabbing}
\textit{typedef  struct ClCorReadJob ClCorReadJobT;}
\newline
\newline
\textit{typedef  ClCorReadJobT  *ClCorReadJobPtrT;}
\newline
\newline
The structure represents read job. This structure represents MO-attrid that needs to be read.
\begin{itemize}
\item
\textit{objHandle} - is the handle towards MO  on which read needs to occur
\item
\textit{attrPath} - is the containment path of the attribute.  This path is NULL for PROV MSOs.
\item
\textit{attriD} - is the  attribute ID.
\end{itemize}


\subsection{ClCorBufferDescriptorT}
\index{ClCorBufferDescriptorT@{ClCorBufferDescriptorT}}
\begin{tabbing}
xx\=xx\=xx\=xx\=xx\=xx\=xx\=xx\=xx\=\kill
\textit{struct ClCorBufferDescriptor \{}
\>\>\>\>\textit{ClPtrT         bufferPtr,}
\>\>\>\>\textit{ClInt32T    buferSize,}
\textit{\};}\end{tabbing}
\textit{typedef  struct  ClCorBufferDescriptor   ClCorBufferDescriptorT; }
\newline
\newline
\textit{typedef   ClCorBufferDescriptorT     *ClCorBufferDescriptorPtrT;}
\newline
\newline
The structure is used to describe a buffer.
\begin{itemize}
\item
bufferPtr  is  the pointer towards data.. 
\item
bufferSize  is the size of the data  (in  bytes) pointed by bufferPtr.. 
\item
This structure is used to  describe the size and data requirement of an attribute.
\end{itemize}





\subsection{ClCorTxnSessionIdT}
\index{ClCorTxnSessionIdT@{ClCorTxnSessionIdT}}
\textit{typedef ClHandleT ClCorTxnSessionIdT;}
\newline
\newline
The type of the handle of a COR transaction session.


\subsection{ClCorMOIdPtrT}
\index{ClCorMOIdPtrT@{ClCorMOIdPtrT}}
\textit{typedef ClCorMOIdT* ClCorMOIdPtrT;}
\newline
\newline
A pointer type to \textit{ClCorMOIdT}.


\subsection{ClCorOpsT}
\index{ClCorOpsT@{ClCorOpsT}}
\begin{tabbing}
xx\=xx\=xx\=xx\=xx\=xx\=xx\=xx\=xx\=\kill
\textit{typedef enum \{}\\
\>\>\>\>\textit{CL\_COR\_OP\_RESERVED = 0,}\\
\>\>\>\>\textit{CL\_COR\_OP\_CREATE = 0x1,}\\
\>\>\>\>\textit{CL\_COR\_OP\_SET = 0x2,}\\
\>\>\>\>\textit{CL\_COR\_OP\_DELETE = 0x4,}\\
\>\>\>\>\textit{CL\_COR\_OP\_ALL = }\\
\>\>\>\>\>\>\>\textit{(CL\_COR\_OP\_CREATE }\\
\>\>\>\>\>\>\>\textit{CL\_COR\_OP\_SET}\\
\>\>\>\>\>\>\>\textit{CL\_COR\_OP\_DELETE)}\\
\textit{\} ClCorOpsT;}\end{tabbing}
The \textit{ClCorOpsT} enumeration contains the various operations that can be performed in COR. A combination of these operations can be used.



\subsection{ClCorCommInfo}
\index{ClCorCommInfo@{ClCorCommInfo}}
\begin{tabbing}
xx\=xx\=xx\=xx\=xx\=xx\=xx\=xx\=xx\=\kill
\textit{typedef struct ClCorCommInfo\{}\\
\>\>\>\>\textit{ClCorAddrT         addr;}\\
\>\>\>\>\textit{ClUint32T        timeout;}\\
\>\>\>\>\textit{ClUint16T        maxRetries;}\\
\>\>\>\>\textit{ClUint16T        maxSessions;}\\
\textit{\};}\end{tabbing}
\textit{typedef struct ClCorCommInfo ClCorCommInfoT;}
\newline\textit{ typedef ClCorCommInfoT*     ClCorCommInfoPtrT;}
 \newline
 \newline
 The type of the pointer for COR communication configuration information.
\newline
\newline
This structure \textit{ClCorCommInfo} contains the communication information for COR. Used for COR communication configuration.Its attributes are:
\begin{itemize}
\item
\textit{addr} - Address to communicate.
\item
\textit{timeout} - Time-out value is in milliseconds.
\item
\textit{maxRetries} - Maximum number of retries.
\item
\textit{maxSessions} - Maximum number of sessions.
\end{itemize}



\subsection{ClCorAddrT}
\index{ClCorAddrT@{ClCorAddrT}}
\textit{typedef ClIocPhysicalAddressT ClCorAddrT;}
\newline
\newline
The type of an identifier for the physical address of the COR object.



\subsection{ClCorMOId}
\index{ClCorMOId@{ClCorMOId}}
\begin{tabbing}
xx\=xx\=xx\=xx\=xx\=xx\=xx\=xx\=xx\=\kill
\textit{typedef struct ClCorMOId\{}\\
\>\>\>\>\textit{ClCorMOHandleT        node[CL\_COR\_HANDLE\_MAX\_DEPTH];}\\
\>\>\>\>\textit{ClCorMOServiceIdT     svcId;}\\
\>\>\>\>\textit{ClUint16T        depth;}\\
\>\>\>\>\textit{ClCorMoPathQualifierT qualifier;}\\
\textit{\};}\end{tabbing}
 The structure \textit{ClCorMOId} contains the MOID of the object which is the address of the COR object. It provides a unique
 representation for the MO object. The type of Service Id is assumed to be 16-bits. The attributes of this structure are:
\begin{itemize}
\item
\textit{node} - MO Handle address.
\item
\textit{svcId} - Service ID.
\item
\textit{depth} - Depth of \em MoId.
\item
\textit{qualifier} - Handle qualifier.
\end{itemize}




\subsection{ClCorObjectHandle}
\index{ClCorObjectHandle@{ClCorObjectHandle}}
\begin{tabbing}
xx\=xx\=xx\=xx\=xx\=xx\=xx\=xx\=xx\=\kill
\textit{typedef struct \{}\\
\>\>\>\>\textit{ClUint8T tree \mbox{[}8\mbox{]};}\\
\textit{\}ClCorObjectHandle;}\end{tabbing}
The structure \textit{ClCorObjectHandle} contains the COR Object Handle. The handle identifies the hierarchy in
 the object tree.
 \textit{objHandle} hierarchy is compressed and indicates the indexes.
\newline
\textit{tree} represents the objTree handle.


\subsection{ClCorObjectHandleT}
\index{ClCorObjectHandleT@{ClCorObjectHandleT}}
\textit{typedef struct ClCorObjectHandle ClCorObjectHandleT;}
\newline
\newline
The type of the handle for the COR object.



\subsection{ClCorAttrPathT}
\index{ClCorAttrPathT@{ClCorAttrPathT}}
\begin{tabbing}
xx\=xx\=xx\=xx\=xx\=xx\=xx\=xx\=xx\=\kill
\textit{typedef struct \{}\\
\>\>\>\>\textit{ClUint16T depth;}\\
\>\>\>\>\textit{ClCorAttrIdIdxPairT node \mbox{[}CL\_COR\_CONT\_ATTR\_MAX\_DEPTH\mbox{]};}\\
\>\>\>\>\textit{ClUint16T tmp;}\\
\textit{\}ClCorAttrPathT;}\end{tabbing}
The structure \textit{ClCorAttrPathT} contains the path-list of the attribute.
\begin{itemize}
\item
\textit{depth} - represents the depth of the path.
\item
\textit{node} - represents the attribute id and index pair.
\item
\textit{tmp} - is for padding.
\end{itemize}



\subsection{ClCorAttrPathPtrT}
\index{ClCorAttrPathPtrT@{ClCorAttrPathPtrT}}
\textit{typedef ClCorAttrPathT* ClCorAttrPathPtrT;}
\newline
\newline
The type of the pointer to \textit{ClCorAttrPathT}.



\subsection{ClCorAttrIdT}
\index{ClCorAttrIdT@{ClCorAttrIdT}}
\textit{typedef ClInt32T ClCorAttrIdT;}
\newline
\newline
The type of an identifier for a COR attribute.

\subsection{ClCorObjectWalkFunT}
\index{ClCorObjectWalkFunT@{ClCorObjectWalkFunT}}
\textit{typedef ClRcT(*ClCorObjectWalkFunT)(void *data, void *cookie);}
\newline
\newline
The prototype for the COR Object Walk Function.


\subsection{ClCorObjAttrWalkFilter}
\index{ClCorObjAttrWalkFilter@{ClCorObjAttrWalkFilter}}
\begin{tabbing}
xx\=xx\=xx\=xx\=xx\=xx\=xx\=xx\=xx\=\kill
\textit{typedef struct ClCorObjAttrWalkFilter\{}\\
\>\>\>\>\textit{ClUint8T              baseAttrWalk;}\\
\>\>\>\>\textit{ClCorAttrPathT       *pAttrPath;}\\
\>\>\>\>\textit{ClCorAttrIdT          attrId;}\\
\>\>\>\>\textit{ClInt32T              index;}\\
\>\>\>\>\textit{ClCorAttrCmpFlagT     cmpFlag;}\\
\>\>\>\>\textit{ClCorAttrWalkOpT      attrWalkOption;}\\
\>\>\>\>\textit{ClUint32T             size;}\\
\>\>\>\>\textit{void                 *value;}\\
\textit{\};}\end{tabbing}
\textit{typedef struct ClCorObjAttrWalkFilter ClCorObjAttrWalkFilterT;}
\newline
\newline
The structure \textit{ClCorObjAttrWalkFilter} is used to filter the attributes of a COR object during the walk through. The attributes of the structure are:
\begin{itemize}
\item
\textit{baseAttrWalk} - Base object attribute walk. It can be either \textit{ CL\_\-TRUE or CL\_\-FALSE.}
\item
\textit{pAttrPath} - Here, NULL means walk through all the contained objects.
\item
\textit{attrId} - It can be \textit{CL\_\-COR\_\-INVALID\_\-ATTR\_\-ID} which means no attribute value comparison.  The following structure-members do-not make
            any sense if attrId \textit{CL\_\-COR\_\-INVALID\_\-ATTR\_\-ID.}
\item
\textit{index} - For a SIMPLE attribute, index id \textit{CL\_\-COR\_\-INVALID\_\-ATTR\_\-IDX}.
\item
\textit{cmpFlag} - Comparison flags.
\item
\textit{attrWalkOption} - If comparison condition comes true, then \textit{attrWalkOption} can either
 be \textit{CL\_\-COR\_\-ATTR\_\-WALK\_\-ALL\_\-ATTR or \c CL\_\-COR\_\-ATTR\_\-WALK\_\-ONLY\_\-MATCHED\_\-ATTR}.
\item
\textit{size} - Value size.
\item
\textit{value} - Value pointer.
\end{itemize}


\subsection{ClCorObjAttrWalkFilterT}
\index{ClCorObjAttrWalkFilterT@{ClCorObjAttrWalkFilterT}}
\textit{typedef struct ClCorObjAttrWalkFilter ClCorObjAttrWalkFilterT;}
\newline
\newline
 The type of the callback function that is invoked for every attribute during the walk through within a COR object.
 The information of the attribute is passed as the parameter for the function.



\subsection{ClCorClassTypeT}
\index{ClCorClassTypeT@{ClCorClassTypeT}}
\textit{typedef ClInt32T ClCorClassTypeT;}
\newline
\newline
The type of an identifier for the COR class.


\subsection{ClCorInstanceIdT}
\index{ClCorInstanceIdT@{ClCorInstanceIdT}}
\textit{typedef ClInt32T ClCorInstanceIdT;}
\newline
\newline
The type of an identifier for a COR instance.


\subsection{ClCorMoIdClassGetFlagsT}
\index{ClCorMoIdClassGetFlagsT@{ClCorMoIdClassGetFlagsT}}
\begin{tabbing}
xx\=xx\=xx\=xx\=xx\=xx\=xx\=xx\=xx\=\kill
\textit{typedef enum \{}\\
\>\>\>\>\textit{CL\_COR\_MO\_CLASS\_GET,}\\
\>\>\>\>\textit{CL\_COR\_MSO\_CLASS\_GET}\\
\textit{\} ClCorMoIdClassGetFlagsT;}\end{tabbing}
The values of the \textit{ClCorMoIdClassGetFlagsT} enumeration type refers to the MoId Class Get Flags.



\subsection{ClCorObjAttrWalkFuncT}
\index{ClCorObjAttrWalkFuncT@{ClCorObjAttrWalkFuncT}}
\begin{tabbing}
xx\=xx\=xx\=xx\=xx\=xx\=xx\=xx\=xx\=\kill
\textit{typedef ClRcT (ClCorObjAttrWalkFuncT) ( \{}\\
\>\>\>\>\textit{ClCorAttrPathPtrT pAttrPath,}\\
\>\>\>\>\textit{ClCorAttrIdT attrId,}\\
\>\>\>\>\textit{ClCorAttrTypeT attrType,}\\
\>\>\>\>\textit{ClCorTypeT attrDataType,}\\
\>\>\>\>\textit{void *value,}\\
\>\>\>\>\textit{ClUint32T size,}\\
\>\>\>\>\textit{void *cookie);}\\
\end{tabbing}
 The type of the callback function that is invoked for every attribute during the walk through within a COR object.
 The information of the attribute is passed as the parameter for the function.
 \begin{itemize}
 \item
 \textit{pAttrPath} - Attribute path of contained object, whose attribute is being walked.
 pAttrPath is NULL, if attribute are of the base object (containing object).
 \item
 \textit{attrId} - Attribute ID.
 \item
 \textit{attrType} - Attribute type. It can be any of the following values:
 \begin{itemize}
 \item
 CL\_\-COR\_\-SIMPLE\_\-ATTR
 \item
 CL\_\-COR\_\-ARRAY\_\-ATTR
 \item
 CL\_\-COR\_\-ASSOCIATION\_\-ATTR
\end{itemize}
 \item
 \textit{attrDataType} - Data type of an attribute. E.g.: CL\_\-COR\_\-UINT32.
 For attrType \textit{CL\_\-COR\_\-ASSOCIATION\_\-ATTR} attrDataType is \textit{CL\_\-COR\_\-INVALID\_\-DATA\_\-TYPE}.
 \item
 \textit{value} - Pointer to actual value of attrId.
 \item
 \textit{size} - Size of value.
 \item
 \textit{cookie} - Cookie passed by you.
 \end{itemize}



\subsection{ClCorMOServiceIdT}
\index{ClCorMOServiceIdT@{ClCorMOServiceIdT}}
\textit{typedef ClInt16T ClCorMOServiceIdT;}
\newline
\newline
The type of the COR managed object service ID.




\subsection{ClCorTxnFuncT}
\index{ClCorTxnFuncT@{ClCorTxnFuncT}}
\textit{typedef ClRcT(*ClCorTxnFuncT)(ClCorTxnIdT trans, ClCorTxnJobIdT jobId, void *cookie);}
\newline
\newline
\textit{ClCorTxnFuncT} is the function type called by \textit{walk} function.


\subsection{ClTxnJobDefnHandleT}
\index{ClTxnJobDefnHandleT@{ClTxnJobDefnHandleT}}
\textit{typedef ClHandleT ClTxnJobDefnHandleT;}
\newline
\newline
The type of the handle for a user-defined job definition. This is passed only during creation
of the job.




\subsection{ClCorType}
\index{ClCorType@{ClCorType}}
\begin{tabbing}
xx\=xx\=xx\=xx\=xx\=xx\=xx\=xx\=xx\=\kill
\textit{typedef enum ClCorType\{}\\
\>\>\>\>\textit{CL\_COR\_INVALID\_DATA\_TYPE = -1,}\\
\>\>\>\>\textit{CL\_COR\_VOID,}\\
\>\>\>\>\textit{CL\_COR\_INT8,}\\
\>\>\>\>\textit{CL\_COR\_UINT8,}\\
\>\>\>\>\textit{CL\_COR\_INT16,}\\
\>\>\>\>\textit{CL\_COR\_UINT16,}\\
\>\>\>\>\textit{CL\_COR\_INT32,}\\
\>\>\>\>\textit{CL\_COR\_UINT32,}\\
\>\>\>\>\textit{CL\_COR\_INT64,}\\
\>\>\>\>\textit{CL\_COR\_UINT64,}\\
\>\>\>\>\textit{CL\_COR\_FLOAT,}\\
\>\>\>\>\textit{CL\_COR\_DOUBLE,}\\
\>\>\>\>\textit{CL\_COR\_COUNTER32,}\\
\>\>\>\>\textit{CL\_COR\_COUNTER64,}\\
\>\>\>\>\textit{CL\_COR\_SEQUENCE32,}\\
\textit{\} ClCorTypeT;}\end{tabbing}
The \textit{ClCorType} enumeration contains the various COR types. Its values have the following interpretation:
\begin{itemize}
\item
\textit{CL\_\-COR\_\-VOID} - Void data type.
\item
\textit{ CL\_\-COR\_\-INT8} - Character data type.
\item
\textit{  CL\_\-COR\_\-UINT8} - Unsigned character.
\item
\textit{ CL\_\-COR\_\-INT16} - Short data type.
\item
\textit{  CL\_\-COR\_\-UINT16} - Unsigned short.
\item
\textit{  CL\_\-COR\_\-INT32} - Integer data type.
\item
\textit{  CL\_\-COR\_\-UINT32} - Unsigned integer data type.
\item
\textit{  CL\_\-COR\_\-INT64} - Long data type.
\item
\textit{  CL\_\-COR\_\-UINT64} - Unsigned long data type.
\item
\textit{  CL\_\-COR\_\-FLOAT} - Float data type. This data type will be supported in future releases, if used will default to \c CL\_\-COR\_\-UINT32.
\item
\textit{  CL\_\-COR\_\-DOUBLE} - Double data type. This data type will be supported in future releases, if used will default to \c CL\_\-COR\_\-UINT32.
\item
\textit{    CL\_\-COR\_\-COUNTER32} - Counter data type. This data type will be supported in future releases, if used will default to \c CL\_\-COR\_\-UINT32.
\item
\textit{  CL\_\-COR\_\-COUNTER64} -  Counter a 64-bits data type. This data type will be supported in future releases, if used will default to \c CL\_\-COR\_\-UINT32.
\item
\textit{    CL\_\-COR\_\-SEQUENCE32} -  Sequence number data type. This data type will be supported in future releases, if used will default to \c CL\_\-COR\_\-UINT32.
\end{itemize}



\subsection{ClCorTypeT}
\index{ClCorTypeT@{ClCorTypeT}}
\textit{typedef enum ClCorType ClCorTypeT;}
\newline
\newline
\textit{ClCorTypeT} represents a COR data type.


\subsection{ClCorAttrTypeT}
\index{ClCorAttrTypeT@{ClCorAttrTypeT}}
\begin{tabbing}
xx\=xx\=xx\=xx\=xx\=xx\=xx\=xx\=xx\=\kill
\textit{typedef enum \{}\\
\>\>\>\>\textit{CL\_COR\_MAX\_TYPE,}\\
\>\>\>\>\textit{CL\_COR\_SIMPLE\_ATTR,}\\
\>\>\>\>\textit{CL\_COR\_ARRAY\_ATTR,}\\
\>\>\>\>\textit{CL\_COR\_CONTAINMENT\_ATTR,}\\
\>\>\>\>\textit{CL\_COR\_ASSOCIATION\_ATTR,}\\
\>\>\>\>\textit{CL\_COR\_VIRTUAL\_ATTR}\\
\textit{\} ClCorAttrTypeT;}\end{tabbing}
The values of the \textit{ClCorAttrTypeT} enumeration type refer to the COR Attribute types.



\subsection{ClCorTxnJobIdT}
\index{ClCorTxnJobIdT@{ClCorTxnJobIdT}}
\textit{typedef ClUint32T ClCorTxnJobIdT;}
\newline
\newline
The type of the handle of a COR Transaction-Job.



\subsection{ClCorTxnIdT}
\index{ClCorTxnIdT@{ClCorTxnIdT}}
\textit{typedef ClHandleT ClCorTxnIdT;}
\newline
\newline
The type fo the handle of a COR Transaction.


\subsection{ClCorOpsT}
\index{ClCorOpsT@{ClCorOpsT}}
\begin{tabbing}
xx\=xx\=xx\=xx\=xx\=xx\=xx\=xx\=xx\=\kill
\textit{typedef enum \{}\\
\>\>\>\>\textit{CL\_COR\_OP\_RESERVED,}\\
\>\>\>\>\textit{CL\_COR\_OP\_CREATE,}\\
\>\>\>\>\textit{CL\_COR\_OP\_SET,}\\
\>\>\>\>\textit{CL\_COR\_OP\_DELETE,}\\
\>\>\>\>\textit{CL\_COR\_OP\_ALL}\\
\textit{\} ClCorOpsT;}\end{tabbing}
The values of the \textit{ClCorOpsT} enumeration type refer to the Operation IDs. A combination can be used.



\subsection{ClCorObjWalkFlagsT}
\index{ClCorObjWalkFlagsT@{ClCorObjWalkFlagsT}}
\begin{tabbing}
xx\=xx\=xx\=xx\=xx\=xx\=xx\=xx\=xx\=\kill
\textit{typedef struct ClCorMOClassPath\{}\\
\>\>\>\>\textit{ClCorClassTypeT    node[CL\_COR\_HANDLE\_MAX\_DEPTH];}\\
\>\>\>\>\textit{ClUint32T        depth;}\\
\>\>\>\>\textit{ClCorMoPathQualifierT qualifier;}\\
\textit{\};}\end{tabbing}
\textit{typedef struct ClCorMOClassPath  ClCorMOClassPathT;}
\newline \textit{typedef ClCorMOClassPathT*      ClCorMOClassPathPtrT;}
The type of the pointer to ClCorMOClassPathT.
\newline
\newline
 The structure ClCorMOClassPath is used to identify the COR managed object instance with its hierarchy information.
 It contains the COR object handle. The attributes of the structure are:
 \begin{itemize}
 \item
 \textit{node} - COR address.
 \item
 \textit{depth} - Depth of COR address.
 \item
 \textit{qualifier} - Handle qualifier.
\end{itemize}





\subsection{ClCorObjWalkFlagsT}
\index{ClCorObjWalkFlagsT@{ClCorObjWalkFlagsT}}
\begin{tabbing}
xx\=xx\=xx\=xx\=xx\=xx\=xx\=xx\=xx\=\kill
\textit{typedef enum \{}\\
\>\>\>\>\textit{CL\_COR\_MOTREE\_WALK,}\\
\>\>\>\>\textit{CL\_COR\_MO\_WALK,}\\
\>\>\>\>\textit{CL\_COR\_MSO\_WALK,}\\
\>\>\>\>\textit{CL\_COR\_MO\_SUBTREE\_WALK,}\\
\>\>\>\>\textit{CL\_COR\_MSO\_SUBTREE\_WALK,}\\
\>\>\>\>\textit{CL\_COR\_MO\_WALK\_UP,}\\
\>\>\>\>\textit{CL\_COR\_MSO\_WALK\_UP}\\
\textit{\} ClCorObjWalkFlagsT;}\end{tabbing}
 The \textit{ClCorObjWalkFlagsT} enumeration type contains walk related definitions. It is used for walking through an MO tree.
 The following are currently supported:
\begin{itemize}
\item
CL\_\-COR\_\-MO\_\-WALK
\item
CL\_\-COR\_\-MSO\_\-WALK
\item
CL\_\-COR\_\-MO\_\-SUBTREE\_\-WALK
\end{itemize}




\subsection{ClCorObjTypesT}
\index{ClCorObjTypesT@{ClCorObjTypesT}}
\begin{tabbing}
xx\=xx\=xx\=xx\=xx\=xx\=xx\=xx\=xx\=\kill
\textit{typedef enum \{}\\
\>\>\>\>\textit{CL\_COR\_OBJ\_TYPE\_SIMPLE,}\\
\>\>\>\>\textit{CL\_COR\_OBJ\_TYPE\_MO,}\\
\>\>\>\>\textit{CL\_COR\_OBJ\_TYPE\_MSO}\\
\textit{\} ClCorObjTypesT;}\end{tabbing}
The values of the \textit{ClCorObjWalkFlagsT} enumeration type contains the type of the COR object. For example, an MO or an MSO object.



\subsection{ClCorServiceIdT}
\index{ClCorServiceIdT@{ClCorServiceIdT}}
\begin{tabbing}
xx\=xx\=xx\=xx\=xx\=xx\=xx\=xx\=xx\=\kill
\textit{typedef enum \{}\\
\>\>\>\>\textit{CL\_COR\_INVALID\_SRVC\_ID,}\\
\>\>\>\>\textit{CL\_COR\_SVC\_ID\_FAULT\_MANAGEMENT,}\\
\>\>\>\>\textit{CL\_COR\_SVC\_ID\_ALARM\_MANAGEMENT,}\\
\>\>\>\>\textit{CL\_COR\_SVC\_ID\_PROVISIONING\_MANAGEMENT,}\\
\>\>\>\>\textit{CL\_COR\_SVC\_ID\_DUMMY\_MANAGEMENT,}\\
\>\>\>\>\textit{CL\_COR\_SVC\_ID\_MAX,}\\
\>\>\>\>\textit{CL\_COR\_SVC\_ID\_FORCED}\\               
\textit{\} ClCorServiceIdT;}\end{tabbing}
The values of the \textit{ClCorServiceIdT} enumeration type contains the service ids(fixed) for all the MSPs.
\begin{itemize}
\item
\textit{CL\_\-COR\_\-SVC\_\-ID\_\-FAULT\_\-MANAGEMENT} - represents the OpenClovis Fault Manager.
\item
\textit{CL\_\-COR\_\-SVC\_\-ID\_\-ALARM\_\-MANAGEMENT} - represents the OpenClovis Alarm Agent.
\item
\textit{CL\_\-COR\_\-SVC\_\-ID\_\-PROVISIONING\_\-MANAGEMENT} - represents the OpenClovis Provisioning Manager.
\item
\textit{CL\_\-COR\_\-SVC\_\-ID\_\-DUMMY\_\-MANAGEMENT} - represents the OpenClovis Dummy MSP.
\item
\textit{CL\_\-COR\_\-SVC\_\-ID\_\-MAX} - represents the end of OpenClovis Service IDs.
\end{itemize}



\newpage
\section{Library Life Cycle APIs}






\subsection{clCorReadSessionInitialize}
\index{clCorReadSessionInitialize@{clCorReadSessionInitialize}}
\hypertarget{pagecor100}{}\paragraph{cl\-Cor\-Read\-Session\_Initialize}\label{pagecor100}
\begin{Desc}
\item[Synopsis:]             \end{Desc}
\begin{Desc}
\item[Header File:]		\end{Desc}
\begin{Desc}
\item[Syntax:]

\footnotesize\begin{verbatim} 		ClRcT clCorReadSessionInitialize ( 
						CL_OUT ClCorReadSessionHandleT *sessionHandle);

\end{verbatim}
\normalsize
\end{Desc}
\begin{Desc}
\item[Parameters:]
\begin{description}
\item[{\em session\-Handle:}](out) This is the session handle that identifies this session.
\end{description}
\end{Desc}
\begin{Desc}
\item[Return values:]
\begin{description}
\item[{\em CL\_\-OK:}]The session completion was successful.
\item[{\em CL\_\-COR\_\-ERR\_\-NO\_\-MEM:}]Session creation failed due to insufficient memory.
\item[{\em CL\_\-COR\_\-ERR\_\-SESSION\_\-INIT\_\-FAILURE:}]Generic session creation failure.
\item[{\em CL\_\-COR\_\-ERR\_\-NULL\_\-PTR:}]SessionHandle is NULL.
\end{description}
\end{Desc}
\begin{Desc}
\item[Description:]This API creates a session. This API returns a unique handle that identifies this session. Subsequent clCorReadSessionJobEnqueue,
clCorReadSessionApply API will use this handle to uniquely identify the session. 
\end{Desc}
\begin{Desc}
\item[Library File:]		\end{Desc}
\begin{Desc}
\item[Related Function(s):]\hyperlink{pagecor127}{clCorReadSessionJobEnqueue} , \hyperlink{pagecor126}{clCorReadSessionApply} , 
\hyperlink{pagecor130}{clCorReadSessionFinalize} \end{Desc}
\newpage



\subsection{clCorReadSessionFinalize}
\index{clCorReadSessionFinalize@{clCorReadSessionFinalize}}
\hypertarget{pagecor101}{}\paragraph{cl\-Cor\-Read\-Session\_Finalize}\label{pagecor101}
\begin{Desc}
\item[Synopsis:]             \end{Desc}
\begin{Desc}
\item[Header File:]		\end{Desc}
\begin{Desc}
\item[Syntax:]

\footnotesize\begin{verbatim} 		ClRcT clCorReadSessionFinalize(
    						CL_INOUT ClCorReadSessionHandleT *sessionHandle);


\end{verbatim}
\normalsize
\end{Desc}
\begin{Desc}
\item[Parameters:]
\begin{description}
\item[{\em session\-Handle:}](in/out)  The parameter which identifies the session.
\end{description}
\end{Desc}
\begin{Desc}
\item[Return values:]
\begin{description}
\item[{\em CL\_\-OK:}]The session completion was successful.
\item[{\em CL\_\-COR\_\-ERR\_\-SESSION\_\-FINALIZE:}]Failure while deleting the resources.
\item[{\em CL\_\-COR\_\-ERR\_\-INVALID\_\-HANDLE:}]Invalid session handle. 
\end{description}
\end{Desc}
\begin{Desc}
\item[Description:]This API should be used to finalize the session. It will free all the resources that were associated with the session. 
Additionally the session handle is invalidated.
\end{Desc}
\begin{Desc}
\item[Library File:]		\end{Desc}
\begin{Desc}
\item[Related Function(s):]\hyperlink{pagecor127}{clCorReadSessionInitalize} \end{Desc}
\newpage



\newpage
\section{Functional APIs}
\subsection{clCorReadSessionJobEnqueue}
\index{clCorReadSessionJobEnqueue@{clCorReadSessionJobEnqueue}}
\hypertarget{pagecor102}{}\paragraph{cl\-Cor\-Read\-Session\-Job\-Enqueue}\label{pagecor102}
\begin{Desc}
\item[Synopsis:]             \end{Desc}
\begin{Desc}
\item[Header File:]		\end{Desc}
\begin{Desc}
\item[Syntax:]

\footnotesize\begin{verbatim} 		ClRcT  clCorReadSessionJobEnqueue ( 
						CL_IN ClCorReadSessionHandleT sessionHandle,	                         
						CL_IN    const ClCorReadJobPtrT            readJob,
						CL_INOUT  ClCorBufferDescriptorPtrT  readJobBuffer,
						CL_OUT     	ClCorJobStatusT 	*readJobstatus,
					);


\end{verbatim}
\normalsize
\end{Desc}
\begin{Desc}
\item[Parameters:]
\begin{description}
\item[{\em session\-Handle:}](in) This parameter identifies the session.
\item[{\em read\-Job:}](in)  This parameter contains the description of the job that needs to be read. A job is described as the MO and the 
attribute that needs to be read.
\item[{\em read\-Job\-Buffer:}](in/out) This parameter describes the data buffer. Required for specifying the attribute value. Note that 
the type of data pointed by  readJobBuffer->buffer,  should be same type as that of  readJob->attrId attribute..  For e.g. if   
readJob->attriID corresponds to ClUint32T, then readJobBuffer->buffer must point to an  ClUint32T data type. An incorrect specification of 
type of this data, would lead to application getting incorrect values for the attribute.
\item{\em read\-Job\-status:}](out)  This parameter indicates the status of the read on this job. Following are the values it can take.
\item{\em CL\_\-OK:} - The attribute was read successfully.
\item{\em CL\_\-COR\_\-ERR\_\-OI\_\-NOT\_\-REGISTERED:} - OI is not  registered for the MO and the attribute is qualified as runtime-non-cached 
or operational.
\item{\em CL\_\-COR\_\-ERR\_\-OI\_\-DATA\_\-NOT\_\-FOUND:} - OI did not find data corresponding to the attribute. 
\item{\em CL\_\-COR\_\-ERR\_\-OBJ\_\-ATTR\_\-NOT\_\-PRESENT:} - Invalid  attribute ID.
\item{\em CL\_\-COR\_\-ERR\_\-INVALID\_\-SIZE:} - Buffer size passed is not exactly equal to the size of the attribute.
\item{\em CL\_\-COR\_\-ERR\_\-INVALID\_\-HANDLE:} - if the Object Handle passed is invalid.
This parameter can be read only after session completes.
\end{description}
\end{Desc}
\begin{Desc}
\item[Return values:]
\begin{description}
\item[{\em CL\_\-OK:}]API was executed successfully. The Job was successfully queued in the session.
\item[{\em CL\_\-COR\_\-ERR\_\-NO\_\-MEM:}]Session unsuccessful due to insufficient memory . This job is not added into the session.
or operational.
\item[{\em CL\_\-COR\_\-ERR\_\-NULL\_\-PTR:}]dataPtr or status is NULL. This job is not added into the session.
\item[{\em CL\_\-COR\_\-ERR\_\-INVALID\_\-PARAM:}]Object Handle or the size parameter is 0. This job is not added into the session.
\item[{\em CL\_\-COR\_\-ERR\_\-DUPLICATE\_\-ATTR\_\-ID:}]This MO-attrID being added is already present in the session. This job is not added 
into the session.
\item[{\em CL\_\-COR\_\-ERR\_\-INVALID\_\-HANDLE:}]Invalid session handle. This job is not added into the session.
\end{description}
\end{Desc}
\begin{Desc}
\item[Description:]This API populates a session with jobs that needs to be read.  The API can be called repeatedly used to queue all the required
jobs as part of a session. The API returns after queuing the jobs in the session. 
The value (readJobBuffer) and status (readJobStatus) is updated only after the session completes. Application can access this value only
after session completes and the status  indicates a success (i.e readJobStatus takes value of CL\_\-OK).

\end{Desc}
\begin{Desc}
\item[Library File:]		\end{Desc}
\begin{Desc}
\item[Related Function(s):]\hyperlink{pagecor127}{clCorReadSessionInitialize} , \hyperlink{pagecor126}{clCorReadSessionApply} , 
\hyperlink{pagecor130}{clCorReadSessionFinalize} \end{Desc}
\newpage




\subsection{clCorReadSessionApply}
\index{clCorReadSessionApply@{clCorReadSessionApply}}
\hypertarget{pagecor101}{}\paragraph{cl\-Cor\-Read\-Session\-Apply}\label{pagecor101}
\begin{Desc}
\item[Synopsis:]             \end{Desc}
\begin{Desc}
\item[Header File:]		\end{Desc}
\begin{Desc}
\item[Syntax:]
\footnotesize\begin{verbatim} 		ClRcT clCorReadSessionApply( 
  						CL_IN  ClCorReadSessionHandleT sessionHandle,
  						CL_IN  const ClCorReadSessionCallBackPtrT   sessionCallBack,
  						CL_IN  const ClPtrT sessionCallBackCookie.
					);
\end{verbatim}
\normalsize
\end{Desc}
\begin{Desc}
\item[Parameters:]
\begin{description}
\item[{\em session\-Handle:}](in) This parameter identifies the session. 
\item[{\em session\-Call\-Back:}](in) Identifies the callback that will be executed when the session completes. 
\item[{\em session\-Call\-Back\-Cookie:}](in) This is cookie associated with this callback for this session.
\end{description}
\end{Desc}
\begin{Desc}
\item[Return values:]
\begin{description}
\item[{\em CL\_\-OK:}]If the call was successful. For an asynchronous call this implies that the session has been queued in IOC. 
For synchronous call this implies that the session has completed.
\item[{\em CL\_\-COR\_\-ERR\_\-ZERO\_\-JOBS\_\-SESSION:}]No MO-attrID are present in the session.
\item[{\em CL\_\-COR\_\-ERR\_\-INVALID\_\-HANDLE:}]Invalid session handle. 
\item[{\em CL\_\-COR\_\-ERR\_\-NO\_\-MEM:}]Memory allocation failure. 
\item[{\em CL\_\-COR\_\-ERR\_\-SESSION\_\-APPLY\_\-FAILURE:}]Generic failure. 
\item[{\em CL\_\-COR\_\-ERR\_\-SESSION\_\-TIMED\_\-OUT:}]For a synchronous session ,the server failed to send the response within the time limit.
\end{description}
\end{Desc}
\begin{Desc}
\item[Description:]This API operates asynchronously when a sessionCallBack  is specified. On completion of the session, the sessionCallBack is executed.  
The API operates synchronously when sessionCallBack is NULL . In such a case the API will return only after completion of the session. 
Note that a session will not be initiated if this API returns any errors specified below. 

\end{Desc}
\begin{Desc}
\item[Library File:]		\end{Desc}
\begin{Desc}
\item[Related Function(s):]\hyperlink{pagecor127}{clCorReadSessionInitalize}, \hyperlink{pagecor127}{clCorReadSessionJobEnqueue} \end{Desc}
\newpage





\subsection{clCorReadOIRegister}
\index{clCorReadOIRegister@{clCorReadOIRegister}}
\hypertarget{pagecor102}{}\paragraph{cl\-Cor\-Read\-OI\-Register}\label{pagecor102}
\begin{Desc}
\item[Synopsis:]             \end{Desc}
\begin{Desc}
\item[Header File:]		\end{Desc}
\begin{Desc}
\item[Syntax:]

\footnotesize\begin{verbatim} 		ClRcT clCorReadOIRegister ( 
						CL_IN 	const ClCorMOIdPtrT pMoId,
						CL_IN 	const ClCorAddrPtrT  pCompAddr);




\end{verbatim}
\normalsize
\end{Desc}
\begin{Desc}
\item[Parameters:]
\begin{description}
\item[{\em p\-Moid:}](in) This is pointer to  MoID of the MO.
\item[{\em p\-Comp\-Addr:}](in) This is the IOC address of OI.
\end{description}
\end{Desc}
\begin{Desc}
\item[Return values:]
\begin{description}
\item[{\em CL\_\-OK:}]The OI was successfully registered
\item[{\em CL\_\-COR\_\-ERR\_\-NO\_\-MEM:}]OI registration failed due to in sufficient memory.
\item[{\em CL\_\-COR\_\-ERR\_\-OI\_\-ALREADY\_\-REGISTERED:}]A OI for this MO is already registered.
\item[{\em CL\_\-COR\_\-ERR\_\-TIMED\_\-OUT:}]The server failed to send the response within the time limit.  
\item[{\em CL\_\-COR\_\-ERR\_\-NULL\_\-PTR:}]Either pM oid or pCompAddr is NULL.
\end{description}
\end{Desc}
\begin{Desc}
\item[Description:]This API is synchronous in nature.
This API registers the component specified by compAddr to be the ReadOI, for the MO pointed  by *pMoID. 
The MoID can be a fully qualified MoId or a wild card MoID.
The OI specified by compAddr, will be contacted by COR to read any run-time-non-cached and operation attribute.
 

\end{Desc}
\begin{Desc}
\item[Library File:]		\end{Desc}
\begin{Desc}
\item[Related Function(s):]\hyperlink{pagecor127}{clCorReadOIUnregister}, \hyperlink{pagecor127}{clCorReadOIGet} \end{Desc}
\newpage  


\subsection{clCorReadOIUnregister}
\index{clCorReadOIUnregister@{clCorReadOIUnregister}}
\hypertarget{pagecor102}{}\paragraph{cl\-Cor\-Read\-OI\-Unregister}\label{pagecor102}
\begin{Desc}
\item[Synopsis:]             \end{Desc}
\begin{Desc}
\item[Header File:]		\end{Desc}
\begin{Desc}
\item[Syntax:]

\footnotesize\begin{verbatim} 		ClRcT clCorReadOIUnRegister (
						 CL_IN ClCorMOIdPtrT pMoid,
 						 CL_IN ClCorAddrPtrT pCompAddr);

\end{verbatim}
\normalsize
\end{Desc}
\begin{Desc}
\item[Parameters:]
\begin{description}
\item[{\em p\-Moid:}](in) This is pointer to MoID of the MO.
\item[{\em p\-Comp\-Addr:}](in) The component which needs to be un-registered as OI.
\end{description}
\end{Desc}
\begin{Desc}
\item[Return values:]
\begin{description}
\item[{\em CL\_\-OK:}]On successful execution completion of the API un-registers the OI.
\item[{\em CL\_\-COR\_\-ERR\_\-OI\_\-NOT\_\-REGISTERED:}]component was  not registered as a read OI for the MO.
\item[{\em CL\_\-COR\_\-ERR\_\-TIMED\_\-OUT:}] The server failed to send the response within the time limit.
\item[{\em CL\_\-COR\_\-ERR\_\-NULL\_\-PTR:}]NULL pointer is passed for pMoid or pCompAddr.
\end{description}
\end{Desc}
\begin{Desc}
\item[Description:]On successful execution completion of the API un-registers the OI.
 

\end{Desc}
\begin{Desc}
\item[Library File:]		\end{Desc}
\begin{Desc}
\item[Related Function(s):]\hyperlink{pagecor127}{clCorReadOIRegister}, \hyperlink{pagecor127}{clCorReadOIGet} \end{Desc}
\newpage  




\subsection{clCorReadOIGet}
\index{clCorReadOIGet@{clCorReadOIGet}}
\hypertarget{pagecor102}{}\paragraph{cl\-Cor\-Read\-OI\-Get}\label{pagecor102}
\begin{Desc}
\item[Synopsis:]             \end{Desc}
\begin{Desc}
\item[Header File:]		\end{Desc}
\begin{Desc}
\item[Syntax:]

\footnotesize\begin{verbatim} 		ClRcT clCorReadOIGet(
						CL_IN const ClCorMOIdPtrT pMoid, 				
						CL_OUT ClCorAddrPtrT pCompAddr);

\end{verbatim}
\normalsize
\end{Desc}
\begin{Desc}
\item[Parameters:]
\begin{description}
\item[{\em p\-Moid:}](in) This is pointer to  MoID of the MO.
\item[{\em p\-Comp\-Addr:}](out) Pointer to the component address structure.
\end{description}
\end{Desc}
\begin{Desc}
\item[Return values:]
\begin{description}
\item[{\em CL\_\-OK:}] If the call is successful the component's address will be populated in the second parameter.
\item[{\em CL\_\-COR\_\-ERR\_\-OI\_\-NOT\_\-REGISTERED:}] If there is  no read OI registered for the MO.
\item[{\em CL\_\-COR\_\-ERR\_\-TIMED\_\-OUT:}] The server failed to send the response within the time limit.
\item[{\em CL\_\-COR\_\-ERR\_\-NULL\_\-PTR:}] NULL pointer is passed for pMoid or pCompAddr.
\end{description}
\end{Desc}
\begin{Desc}
\item[Description:] On successful execution, this function obtains the ReadOI for the MO.  
\end{Desc}
\begin{Desc}
\item[Library File:]		\end{Desc}
\begin{Desc}
\item[Related Function(s):]\hyperlink{pagecor127}{clCorReadOIRegister}, \hyperlink{pagecor127}{clCorReadOIUnregister} \end{Desc}
\newpage  




\subsection{clCorObjectAttributeSet}
\index{clCorObjectAttributeSet@{clCorObjectAttributeSet}}
\hypertarget{pagecor102}{}\paragraph{cl\-Cor\-Object\-Attribute\-Set}\label{pagecor102}
\begin{Desc}
\item[Synopsis:]             \end{Desc}
\begin{Desc}
\item[Header File:]		\end{Desc}
\begin{Desc}
\item[Syntax:]

\footnotesize\begin{verbatim} 		ClRcT clCorObjectAttributeSet(
                                  			CL_INOUT ClCorTxnSessionIdT *writeSessionId
                                  			CL_IN    ClCorObjectHandleT objectHandle,
                                  			CL_IN    ClCorAttrPathPtrT pAttrPath,
                                  			CL_IN    ClCorAttrIdT attrId,
                                  			CL_IN    ClUint32T index,
                                  			CL_IN    ClPtrT pValue,
                                  			CL_IN    ClUint32T size);


\end{verbatim}
\normalsize
\end{Desc}
\begin{Desc}
\item[Parameters:]
\begin{description}
\item[{\em write\-Session\-Id:}](in/out) This is unique handle for a session which can modify (set/create/delete) MOs in a single request to the server. 
Using this handle all the jobs(MO-AttrId pair) can be queued in one session. This an IN/OUT parameter. The first call to this function should pass 
the initialized value of this parameter (equal to zero) and all the subsequent calls to this function and the call to clCorTxnSessionCommit should 
use this handle to send the request to server.
\item[{\em object\-Handle:}](in) Handle of the object whose attribute is being set.
\item[{\em p\-Attr\-Path:}](in) Attribute path for the contained MO. This will be NULL for non-alarm attributes.
\item[{\em attr\-Id:}](in) Identifier of the attribute.
\item[{\em index:}](in) Attribute index. (CL\_\-COR\_\-INVALID\_\-ATTR\_\-IDX for simple attribute).
\item[{\em p\-Value:}](in) Pointer to the value to be set.
\item[{\em size:}](in) Size of the value.
\end{description}
\end{Desc}
\begin{Desc}
\item[Return values:]
\begin{description}
\item[{\em CL\_\-OK:}] The function executed successfully.
\item[{\em CL\_\-COR\_\-ERR\_\-RUNTIME\_\-CACHED\_\-SET:}] A set was performed on a run time cached attribute by a Non ReadOI.
\item[{\em CL\_\-COR\_\-ERR\_\-NULL\_\-PTR:}] On passing a NULL pointer.
\item[{\em CL\_\-COR\_\-ERR\_\-INVALID\_\-PARAM:}] On passing an invalid parameter.
\item[{\em CL\_\-COR\_\-TXN\_\-ERR\_\-INVALID\_\-JOB\_\-ID:}] On passing an invalid job ID.
\item[{\em CL\_\-COR\_\-ERR\_\-CLASS\_\-ATTR\_\-INVALID\_\-INDEX:}] When index is used in case of simple attribute.
\item[{\em CL\_\-COR\_\-ERR\_\-CLASS\_\-ATTR\_\-NOT\_\-PRESENT:}] The attribute ID is not present.
\item[{\em CL\_\-COR\_\-ERR\_\-INVALID\_\-SIZE:}] For simple attributes, the parameter \textit{size} is less than the size associated with the attribute attrId. For array attributes, this error is returned in one of the following cases.
Size is greater than the size of associated array elements that must be updated.
The parameter \textit{size} is less than  size of an individual array element.
The parameter size is not an integer multiple of the individual array element.
\item[{\em CL\_\-COR\_\-ERR\_\-OBJ\_\-ATTR\_\-INVALID\_\-SET:}] This error code is returned in case of simple attributes, when the value is not 
within the range specified by \textit{min} and \textit{max} values associated with the attribute.
\item[{\em CL\_\-COR\_\-ERR\_\-CLASS\_\-ATTR\_\-INVALID\_\-INDEX:}] Invalid index in case of complex attribute.
\item[{\em CL\_\-COR\_\-ERR\_\-CLASS\_\-ATTR\_\-INVALID\_\-RELATION:}] Size of the attribute is invalid.
\item[{\em CL\_\-COR\_\-ERR\_\-VERSION\_\-UNSUPPORTED:}] version is not supported.

\end{description}
\end{Desc}
\begin{Desc}
\item[Description:] The first parameter WriteSessionId, uniquely identifies the session. 
This function should be used to enqueue all the required jobs on which set needc to be performed.
After en-queuing all the jobs, the function clCorTxnSessionCommit should be called for the committing the session. This  will send the session 
request to COR server for processing. 
For each MO in this session COR performs a two phase apply process.  (Details to be filled in later).  
A set operation on a  job whose attribute is of type runtime-cached attribute , can be initiated only by its ReadOI.  In such a case no
OI callback will be executed. COR will ensure that only ReadOI can perform a set on a runtime-cached attribute. 

\end{Desc}
\begin{Desc}
\item[Library File:]		\end{Desc}
\begin{Desc}
\item[Related Function(s):]\hyperlink{pagecor127}{clCorObjectCreate}, \hyperlink{pagecor127}{clCorObjectAttributeGet}, 
\hyperlink{pagecor127}{clCorObjectDelete}\end{Desc}
\newpage  














\subsection{clCorObjectCreate}
\index{clCntLlistCreate@{clCntLlistCreate}}
\hypertarget{pagecor129}{}\paragraph{cl\-Cor\-Object\-Create}\label{pagecor129}
\begin{Desc}
\item[Synopsis:]Creates a COR object.\end{Desc}
\begin{Desc}
\item[Header File:]clCorApi.h\end{Desc}
\begin{Desc}
\item[Syntax:]

\footnotesize\begin{verbatim} ClRcT clCorObjectCreate(
                           CL_INOUT ClCorTxnSessionIdT *txnSessionId,
                           CL_IN  ClCorMOIdPtrT moId,
                           CL_OUT ClCorObjectHandleT *handle);
\end{verbatim}
\normalsize
\end{Desc}
\begin{Desc}
\item[Parameters:]
\begin{description}
\item[{\em txn\-Session\-Id:}](in) (in/out) Transaction Session ID. For a simple transaction it is {\tt CL\_\-COR\_\-SIMPLE\_\-TXN}. For a complex 
transaction, it is initialized to zero and passed to all the calls of the object-create function and finally passed to the 
\textit{clCorTxnSessionCommit()} function. 
\item[{\em mo\-Id:}](in) ID of the object to be created. \item[{\em handle:}](in) (out) Pointer to the object handle.\end{description}
\end{Desc}
\begin{Desc}
\item[Return values:]
\begin{description}
\item[{\em CL\_\-OK:}]The function executed successfully. \item[{\em CL\_\-COR\_\-ERR\_\-NULL\_\-PTR:}]On passing a NULL pointer. 
\item[{\em CL\_\-COR\_\-ERR\_\-NO\_\-MEM:}]On memory allocation failure. \item[{\em CL\_\-COR\_\-INST\_\-ERR\_\-INVALID\_\-MOID:}]{\em Mo\-Id\/} passed 
is invalid \item[{\em CL\_\-COR\_\-ERR\_\-CLASS\_\-NOT\_\-PRESENT:}]Class is not present. \item[{\em CL\_\-COR\_\-INST\_\-ERR\_\-NODE\_\-NOT\_\-FOUND:}]
Parent class is not present in the instance tree. \item[{\em CL\_\-COR\_\-MO\_\-TREE\_\-ERR\_\-NODE\_\-NOT\_\-FOUND:}]{\em mo\-Tree\/} node is not 
present for the class. \item[{\em CL\_\-COR\_\-INST\_\-ERR\_\-MAX\_\-INSTANCE:}]Maximum Instance count for this object is reached.
\item[{\em CL\_\-COR\_\-INST\_\-ERR\_\-MO\_\-ALREADY\_\-PRESENT:}]MO is already present in the instance tree.
\item[{\em CL\_\-COR\_\-ERR\_\-VERSION\_\-UNSUPPORTED:}]Version is not supported. 
\item[{\em CL\_\-COR\_\-INST\_\-ERR\_\-MSO\_\-ALREADY\_\-PRESENT:}]MSO is already present in the Object Tree. 
\item[{\em CL\_\-COR\_\-MO\_\-TREE\_\-ERR\_\-CLASS\_\-NO\_\-PRESENT:}]MSO Class is not present in the MO Tree.\end{description}
\end{Desc}
\begin{Desc}
\item[Description:]This function is used to create a COR object. The {\em mo\-Id\/} passed to this function must be valid (the MO class tree must be present for that {\em MOId\/}). While this function may be used for creation of both MO and MSO, you must be careful while creating an MSO object. The service ID of an MSO object is not invalid in the {\em mo\-Id\/}, whereas for the MO object, the service ID is invalid in the {\em mo\-Id\/}. Also in the case of an object-create involving multiple {\em mo\-Ids\/} in one transaction, the {\em object\/} handle will be valid only after the transaction is committed without any errors. \par
 \par
 In the current implementation, an incorrect {\em MOID\/} passed to this function results in incorrect completion of the transaction. All operations that are queued before the failing operation will be completed; however, operations enqueued after the failing operation will remain incomplete. \par
 For a simple transaction, the transaction ID {\tt txn\-Session\-Id} is {\tt CL\_\-COR\_\-SIMPLE\_\-TXN}. A transaction that involves setting of multiple 
 attributes for multiple MOs is termed as Complex Transaction. Through a transaction, an application can atomically create/delete MOs, and set
 attributes. To initiate a complex transaction, you must initialize the {\em $\ast$txn\-Session\-Id\/} variable to zero and pass {\em txn\-Session\-ID\/} 
 as the first argument to the \textit{clCorObjectCreate()} function. The function in turn, returns a transaction ID through {\em $\ast$txn\-Session\/}.
 This ID must be passed as the first parameter for subsequent invocation of cl\-Cor\-Object\-Attribute\-Set/cl\-Cor\-Object\-Create/cl\-Cor\-Object\-Delete functions. A Complex Transaction is ended via the call \textit{clCorTxnSessionCommit()}.\end{Desc}
\begin{Desc}
\item[Library File:]Cl\-Cor\-Client\end{Desc}
\begin{Desc}
\item[Related Function(s):]\hyperlink{pagecor127}{cl\-Cor\-Object\-Attribute\-Get} , \hyperlink{pagecor126}{cl\-Cor\-Object\-Attribute\-Set} , 
\hyperlink{pagecor130}{cl\-Cor\-Object\-Delete} \end{Desc}
\newpage


\subsection{clCorObjectAttributeSet}
\index{clCorObjectAttributeSet@{clCorObjectAttributeSet}}
\hypertarget{pagecor126}{}\paragraph{cl\-Cor\-Object\-Attribute\-Set}\label{pagecor126}
\begin{Desc}
\item[Synopsis:]Sets the attribute of a COR object\end{Desc}
\begin{Desc}
\item[Header File:]clCorApi.h\end{Desc}
\begin{Desc}
\item[Syntax:]

\footnotesize\begin{verbatim}  ClRcT clCorObjectAttributeSet(
                                  CL_INOUT ClCorTxnSessionIdT *txnSessionId
                                  CL_IN    ClCorObjectHandleT pHandle,
                                  CL_IN    ClCorAttrPathPtrT contAttrPath,
                                  CL_IN    ClCorAttrIdT attrId,
                                  CL_IN    ClUint32T index,
                                  CL_IN    void *value,
                                  CL_IN    ClUint32T size);
\end{verbatim}
\normalsize
\end{Desc}
\begin{Desc}
\item[Parameters:]
\begin{description}
\item[{\em txn\-Session\-Id:}](in) (in/out) Transaction Session ID. 
\item[{\em p\-Handle:}](in) Handle of the object whose attribute is being set. 
\item[{\em cont\-Attr\-Path:}](in) Containment hierarchy. 
\item[{\em attr\-Id:}](in) ID of the attribute. 
\item[{\em index:}](in) Attribute index. ({\tt CL\_\-COR\_\-INVALID\_\-ATTR\_\-IDX} for simple attribute). 
\item[{\em value:}](in) Pointer to the value to be set. 
\item[{\em size:}](in) Size of the value.\end{description}
\end{Desc}
\begin{Desc}
\item[Return values:]
\begin{description}
\item[{\em CL\_\-OK:}]The function executed successfully. \item[{\em CL\_\-COR\_\-ERR\_\-NULL\_\-PTR:}]On passing a NULL pointer. \item[{\em CL\_\-COR\_\-ERR\_\-INVALID\_\-PARAM:}]On passing an invalid parameter. \item[{\em CL\_\-COR\_\-TXN\_\-ERR\_\-INVALID\_\-JOB\_\-ID:}]On passing an invalid job ID. \item[{\em CL\_\-COR\_\-ERR\_\-CLASS\_\-ATTR\_\-INVALID\_\-INDEX:}]When index is used in case of simple attribute. \item[{\em CL\_\-COR\_\-ERR\_\-CLASS\_\-ATTR\_\-NOT\_\-PRESENT:}]The attribute ID is not present.\item[{\em CL\_\-COR\_\-ERR\_\-INVALID\_\-SIZE:}]For simple attributes, the parameter {\em size\/} is less than the {\em size\/} associated with the attribute {\em attr\-Id\/}. For array attributes, this error is returned in one of the following cases.\begin{enumerate}
\item {\em size\/} is greater than the size of associated array elements that must be updated.\item the parameter {\em size\/} is less than size of an individual array element.\item The parameter size is not an integer multiple of the individual array element. \end{enumerate}
\item[{\em CL\_\-COR\_\-ERR\_\-OBJ\_\-ATTR\_\-INVALID\_\-SET:}]This error code is returned in case of simple attributes, when the value is not within the range specified by {\em min\/} and {\em max\/} values associated with the attribute.\item[{\em CL\_\-COR\_\-ERR\_\-CLASS\_\-ATTR\_\-INVALID\_\-INDEX:}]Invalid index in case of complex attribute. \item[{\em CL\_\-COR\_\-ERR\_\-CLASS\_\-ATTR\_\-INVALID\_\-RELATION:}]Size of the attribute is invalid. \item[{\em CL\_\-COR\_\-ERR\_\-VERSION\_\-UNSUPPORTED:}]version is not supported.\end{description}
\end{Desc}
\begin{Desc}
\item[Description:]This is a generic function which is used to set all kinds of attributes. The \textit{clCorObjectAttributeSet()} function uses transaction agent to set the object attribute. The first argument is the transaction ID. For a simple transaction, the transaction ID is {\tt CL\_\-COR\_\-SIMPLE\_\-TXN}. A transaction that involves setting of multiple attributes for multiple MOs is termed as Complex Transaction. Via a transaction, an application can atomically create/delete MOs, and set attributes. \par
 To initiate a complex transaction, you must initialize the {\em $\ast$txn\-Session\-Id\/} variable to zero and pass {\em txn\-Session\-ID\/} as first argument to the \textit{clCorObjectAttributeSet()} function. The function in turn returns a transaction ID in {\em txn\-Session\/}. This ID must be passed as the first parameter for subsequent invocation of cl\-Cor\-Object\-Attribute\-Set/cl\-Cor\-Object\-Create/cl\-Cor\-Object\-Delete APIs. A Complex Transaction is ended via the call \textit{clCorTxnSessionCommit()}. \par
 A simple attribute is defined by the enum {\em Cl\-Cor\-Type\-T\/}. For simple attributes (such as Cl\-Uint8T, Cl\-Int32T) the index parameter is -1. The index parameter is used only in case $\ast$ of an array attribute. It specifies the element number of the array, that must be operated upon. For simple attribute, {\em attrpath\/} is NULL.\end{Desc}
\begin{Desc}
\item[Note:]The {\em size\/} of the value to be set must not be less than its actual size. In the current implementation incorrect {\em MOID\/} being passed to this function will result in incorrect completion of the $\ast$ transaction, i.e. all operations (queued before the failing operation) will be completed; however, operations enqueued $\ast$ $\ast$ after the failing operation will remain incomplete.\end{Desc}
\begin{Desc}
\item[Library File:]Cl\-Cor\-Client\end{Desc}
\begin{Desc}
\item[Related Function(s):]\hyperlink{pagecor129}{cl\-Cor\-Object\-Create} , \hyperlink{pagecor127}{cl\-Cor\-Object\-Attribute\-Get} ,
\hyperlink{pagecor130}{cl\-Cor\-Object\-Delete} \end{Desc}
\newpage


\subsection{clCorObjectDelete}
\index{clCorObjectDelete@{clCorObjectDelete}}
\hypertarget{pagecor130}{}\paragraph{cl\-Cor\-Object\-Delete}\label{pagecor130}
\begin{Desc}
\item[Synopsis:]Deletes a COR object.\end{Desc}
\begin{Desc}
\item[Header File:]clCorApi.h\end{Desc}
\begin{Desc}
\item[Syntax:]

\footnotesize\begin{verbatim}   ClRcT clCorObjectDelete(
                         CL_INOUT  ClCorTxnSessionIdT *txnSessionId,
                         CL_IN     ClCorObjectHandleT handle);
\end{verbatim}
\normalsize
\end{Desc}
\begin{Desc}
\item[Parameters:]
\begin{description}
\item[{\em txn\-Session\-Id:}](in) (in/out) Transaction Session ID. \item[{\em handle:}](in) Handle of the object to be deleted.\end{description}
\end{Desc}
\begin{Desc}
\item[Return values:]
\begin{description}
\item[{\em CL\_\-OK:}]The function executed successfully. \item[{\em CL\_\-COR\_\-ERR\_\-NULL\_\-PTR:}]On passing a NULL pointer. \item[{\em CL\_\-COR\_\-INVALID\_\-SRVC\_\-ID:}]Service ID is incorrect. \item[{\em CL\_\-COR\_\-INST\_\-ERR\_\-INVALID\_\-MOID:}]{\em mo\-Id\/} is invalid. \item[{\em CL\_\-COR\_\-INST\_\-ERR\_\-CHILD\_\-MO\_\-EXIST:}]Child MO exists for the MO object node. \item[{\em CL\_\-COR\_\-INST\_\-ERR\_\-MSO\_\-EXIST:}]MSO exists for the MO object node. \item[{\em CL\_\-COR\_\-INST\_\-ERR\_\-NODE\_\-NOT\_\-FOUND:}]Node not found in the object tree. \item[{\em CL\_\-COR\_\-ERR\_\-VERSION\_\-UNSUPPORTED:}]Version is not supported. \item[{\em CL\_\-COR\_\-MO\_\-TREE\_\-ERR\_\-NODE\_\-NOT\_\-FOUND:}]MO tree node not found.\end{description}
\end{Desc}
\begin{Desc}
\item[Description:]This function is used to delete a COR object. The Object handle passed as the second parameter can be obtained using the function \textit{clCorObjectHandleGet()}. All the MO/MSO children corresponding to the handle must be deleted prior to deleting the parent. Also, to delete an MSO, appropriate service ID must be used in the {\em MOId\/}. To delete an MO, service ID of the {\em MOId\/} must be invalid. A deletion of MO with children will not be effected.\par
 \par
 In the current implementation, an incorrect {\em MOID\/} being passed to this function will result in incorrect completion of the transaction, i.e. all operations (queued before the failing operation) will be completed; however, operations enqueued after the failing operation will be incomplete. \par
 The first argument is the transaction ID. For a simple transaction, the transaction ID is {\tt CL\_\-COR\_\-SIMPLE\_\-TXN}. A transaction that involves setting of multiple attributes for multiple MOs is termed as Complex Transaction. Via a transaction, applications can atomically create/delete MOs and set attributes. To initiate a complex transaction, you must initialize the {\em $\ast$txn\-Session\-Id\/} variable to zero and pass {\em txn\-Session\-ID\/} as the first argument to the cl\-Cor\-Object\-Delete function. The function in turn, returns a transaction ID through {\em txn\-Session\/}.\par
 \par
 This ID must be passed as the first parameter for subsequent invocation of cl\-Cor\-Object\-Attribute\-Set or cl\-Cor\-Object\-Create or cl\-Cor\-Object\-Delete functions. A complex transaction is ended via the call \textit{clCorTxnSessionCommit()}.\end{Desc}
\begin{Desc}
\item[Library File:]Cl\-Cor\-Client\end{Desc}
\begin{Desc}
\item[Related Function(s):]\hyperlink{pagecor129}{cl\-Cor\-Object\-Create} \end{Desc}
\newpage

\subsection{clCorObjectAttributeGet}
\index{clCorObjectAttributeGet@{clCorObjectAttributeGet}}
\hypertarget{pagecor127}{}\paragraph{cl\-Cor\-Object\-Attribute\-Get}\label{pagecor127}
\begin{Desc}
\item[Synopsis:]Retrieves the value of an attribute.\end{Desc}
\begin{Desc}
\item[Header File:]clCorApi.h\end{Desc}
\begin{Desc}
\item[Syntax:]

\footnotesize\begin{verbatim} ClRcT clCorObjectAttributeGet(
                                  CL_IN ClCorObjectHandleT pHandle,
                                  CL_IN ClCorAttrPathPtrT contAttrPath,
                                  CL_IN ClCorAttrIdT attrId,
                                  CL_IN ClInt32T index,
                                  CL_OUT void *value,
                                  CL_INOUT ClUint32T *size);
\end{verbatim}
\normalsize
\end{Desc}
\begin{Desc}
\item[Parameters:]
\begin{description}
\item[{\em p\-Handle:}](in) Handle of the object whose attribute is being read. \item[{\em cont\-Attr\-Path:}](in) Containment hierarchy. 
\item[{\em attr\-Id:}](in) ID of the attribute. \item[{\em index:}](in) Attribute index. ({\tt CL\_\-COR\_\-INVALID\_\-ATTR\_\-IDX} for simple attribute). \item[{\em value:}](in) (out) Pointer to the value. Attribute value is copied into {\em value\/}. \item[{\em size:}](in) (in/out) Size of the value.\end{description}
\end{Desc}
\begin{Desc}
\item[Return values:]
\begin{description}
\item[{\em CL\_\-OK:}]The function executed successfully. \item[{\em CL\_\-COR\_\-ERR\_\-NULL\_\-PTR:}]On passing a NULL pointer. \item[{\em CL\_\-COR\_\-ERR\_\-NO\_\-MEM:}]On memory allocation failure. \item[{\em CL\_\-COR\_\-ERR\_\-CLASS\_\-ATTR\_\-NOT\_\-PRESENT:}]Attribute ID passed is not present. \item[{\em CL\_\-COR\_\-ERR\_\-CLASS\_\-ATTR\_\-INVALID\_\-RELATION:}]Size of the attribute is invalid.\item[{\em CL\_\-COR\_\-ERR\_\-INVALID\_\-SIZE:}]For simple attributes, the parameter {\em size\/} is less than the size associated with the attribute {\em attr\-Id\/}. For array attributes, this error will be returned in one of the following cases.\begin{enumerate}
\item {\em size\/} is greater than the size of associated array elements that must be updated.\item the parameter {\em size\/} is less than size of an individual array element.\item The parameter size is not an integer multiple of the individual array element.\end{enumerate}
\item[{\em CL\_\-COR\_\-ERR\_\-CLASS\_\-ATTR\_\-INVALID\_\-INDEX:}]When {\em index\/} is used in case of simple attribute. \item[{\em CL\_\-COR\_\-ERR\_\-VERSION\_\-UNSUPPORTED:}]version is not supported.\end{description}
\end{Desc}
\begin{Desc}
\item[Description:]This is a generic function which is used to retrieve all kinds of attributes. The {\em p\-Handle\/} is the handle of the Object for which the attribute value needs to be set. This object handle is obtained when the object is created. The object handle can also be obtained by passing {\em mo\-Id\/} to the function \textit{clCorObjectHandleGet()}.\par
 \par
 \begin{itemize}
\item If the size of the value as passed through the {\em size\/} parameter is less than the actual size of the value of the attribute, this function returns an error. \item If the size supplied is greater than the actual size, the actual size is filled in the {\tt }\mbox{[}CL\_\-INOUT\mbox{]}size. For simple attribute the {\em attrpath\/} is NULL, but in case of containment attributes the {\em attrpath\/} is a valid containment path. \par
 \end{itemize}
\end{Desc}
\begin{Desc}
\item[Library File:]Cl\-Cor\-Client\end{Desc}
\begin{Desc}
\item[Note:]Before invoking the \textit{clCorObjectAttributeGet()} function, you must create the object. Also, the attribute will be created with default values of the attribute given at the time of attribute creation.\end{Desc}
\begin{Desc}
\item[Related Function(s):]\hyperlink{pagecor129}{cl\-Cor\-Object\-Create} , \hyperlink{pagecor126}{cl\-Cor\-Object\-Attribute\-Set} \end{Desc}
\newpage


\subsection{clCorObjectHandleGet}
\index{clCorObjectHandleGet@{clCorObjectHandleGet}}
\hypertarget{pagecor128}{}\paragraph{cl\-Cor\-Object\-Handle\-Get}\label{pagecor128}
\begin{Desc}
\item[Synopsis:]Returns the Object handle.\end{Desc}
\begin{Desc}
\item[Header File:]clCorApi.h\end{Desc}
\begin{Desc}
\item[Syntax:]

\footnotesize\begin{verbatim}   ClRcT clCorObjectHandleGet(
                            CL_IN ClCorMOIdPtrT pMoId,
                            CL_OUT ClCorObjectHandleT *objHandle);
\end{verbatim}
\normalsize
\end{Desc}
\begin{Desc}
\item[Parameters:]
\begin{description}
\item[{\em p\-Mo\-Id:}](in) Pointer to the MOId. \item[{\em obj\-Handle:}](out) Pointer to object handle.\end{description}
\end{Desc}
\begin{Desc}
\item[Return values:]
\begin{description}
\item[{\em CL\_\-OK:}]The function executed successfully. \item[{\em CL\_\-COR\_\-ERR\_\-NULL\_\-PTR:}]On passing a NULL pointer. \item[{\em CL\_\-COR\_\-INST\_\-ERR\_\-INVALID\_\-MOID:}]If the {\em mo\-Id\/} passed in invalid. \item[{\em CL\_\-COR\_\-ERR\_\-INVALID\_\-DEPTH:}]If the configured maximum depth is less than {\em mo\-Id\/} depth. \item[{\em CL\_\-COR\_\-UTILS\_\-ERR\_\-INVALID\_\-KEY:}]On failure to locate the node in the MO tree. \item[{\em CL\_\-COR\_\-ERR\_\-INVALID\_\-PARAM:}]On passing an invalid parameter. \item[{\em CL\_\-COR\_\-ERR\_\-VERSION\_\-UNSUPPORTED:}]version is not supported.\end{description}
\end{Desc}
\begin{Desc}
\item[Description:]This function is used to return the handle to object of a given {\em Mo\-Id\/}. This is the compressed value of {\em mo\-Id\/} which is represented through object handle.\end{Desc}
\begin{Desc}
\item[Library File:]Cl\-Cor\-Client\end{Desc}
\begin{Desc}
\item[Related Function(s):]\hyperlink{pagecor133}{cl\-Cor\-Object\-Handle\-To\-Mo\-Id\-Get} , 
\hyperlink{pagecor132}{cl\-Cor\-Object\-Handle\-To\-Type\-Get} , \hyperlink{pagecor127}{cl\-Cor\-Object\-Attribute\-Get} \end{Desc}
\newpage


\subsection{clCorObjectWalk}
\index{clCorObjectWalk@{clCorObjectWalk}}
\hypertarget{pagecor131}{}\paragraph{cl\-Cor\-Object\-Walk}\label{pagecor131}
\begin{Desc}
\item[Synopsis:]Walks through the object tree.\end{Desc}
\begin{Desc}
\item[Header File:]clCorApi.h\end{Desc}
\begin{Desc}
\item[Syntax:]

\footnotesize\begin{verbatim}   ClRcT clCorObjectWalk(
                           CL_IN ClCorMOIdPtrT moIdRoot,
                           CL_IN ClCorMOIdPtrT moIdFilter,
                           CL_IN ClCorObjectWalkFunT fp,
                           CL_IN ClCorObjWalkFlagsT flags,
                           CL_IN void * cookie);
\end{verbatim}
\normalsize
\end{Desc}
\begin{Desc}
\item[Parameters:]
\begin{description}
\item[{\em mo\-Id\-Root:}](in) {\em Mo\-Id\/} from where the walk is started. 
\item[{\em mo\-Id\-Filter:}](in) {\em Mo\-Id\/} with wild card. For example, if the {\em mo\-Id\-Root\/} is {\tt /chassis}:0/blade:1/ then 
{\em mo\-Id\-Filter\/} can have a value such as {\tt /} chassis:0 / blade:1 / port:$\ast$ / channel:$\ast$ / 
\item[{\em fp:}](in) Function to be called. It expects two parameters \item[{\em data:}]Void pointer which is taking object handle. 
\item[{\em cookie}]:(in)  Void pointer to the cookie, the pointer passed by the user. 
\item[{\em flags:}](in) Flags. This parameter accepts the following values: 
\begin{itemize}
\item {\tt CL\_\-COR\_\-MO\_\-WALK} \item {\tt CL\_\-COR\_\-MSO\_\-WALK} \item {\tt CL\_\-COR\_\-MO\_\-SUBTREE\_\-WALK} \end{itemize}
\item[{\em cookie:}](in) Pointer to user-defined data. With the help of this you can pass any information through the object walk.\end{description}
\end{Desc}
\begin{Desc}
\item[Return values:]
\begin{description}
\item[{\em CL\_\-OK:}]The function executed successfully. \item[{\em CL\_\-COR\_\-ERR\_\-NULL\_\-PTR:}]On passing a NULL pointer. \item[{\em CL\_\-COR\_\-SVC\_\-ERR\_\-INVALID\_\-ID:}]Service ID is invalid. \item[{\em CL\_\-COR\_\-ERR\_\-VERSION\_\-UNSUPPORTED:}]version is not supported. \item[{\em CL\_\-COR\_\-ERR\_\-NO\_\-MEM:}]On memory allocation failure.\end{description}
\end{Desc}
\begin{Desc}
\item[Description:]This function is used to perform a walk through the COR objects present in the object tree. The first parameter is the {\em mo\-Id\/} of the root, from where the object walk is started. If entire tree is involved then this parameter can be NULL. Second parameter is {\em mo\-Id\-Filter\/}, which is the filter for the search. This parameter can be a wild card entry or a pure {\em mo\-Id\/}. This parameter can be NULL if you don't need any filter for the object walk. The third parameter {\em fp\/} is a callback function which has two parameters, the first of which, {\em data\/} is supplied with the object handle. The second parameter {\em cookie\/} will be getting the cookie which is passed in the object walk call. The fourth parameter {\em flags\/} indicates the type of object Walk that is desired. Last parameter is the user argument cookie.\end{Desc}
\begin{Desc}
\item[Library File:]Cl\-Cor\-Client\end{Desc}
\begin{Desc}
\item[Note:]The last parameter can be used to pass the value which can be used in callback function. If both first and second parameters are NULL then whole of the MO/MSO objects are walked whatever is specified in the flag.\end{Desc}
\begin{Desc}
\item[Related Function(s):]\hyperlink{pagecor146}{cl\-Cor\-Object\-Attribute\-Walk} , \hyperlink{group__group13}{cl\-Cor\-Mo\-Id\-Append} \end{Desc}
\newpage


\subsection{clCorObjectHandleToTypeGet}
\index{clCorObjectHandleToTypeGet@{clCorObjectHandleToTypeGet}}
\hypertarget{pagecor132}{}\paragraph{cl\-Cor\-Object\-Handle\-To\-Type\-Get}\label{pagecor132}
\begin{Desc}
\item[Synopsis:]Returns Type of an object given its Object Handle.\end{Desc}
\begin{Desc}
\item[Header File:]clCorApi.h\end{Desc}
\begin{Desc}
\item[Syntax:]

\footnotesize\begin{verbatim}  ClRcT clCorObjectHandleToTypeGet(
                           CL_IN   ClCorObjectHandleT pHandle,
                           CL_OUT  ClCorObjTypesT* type);
\end{verbatim}
\normalsize
\end{Desc}
\begin{Desc}
\item[Parameters:]
\begin{description}
\item[{\em CL\_\-IN}]p\-Handle: (in) Object handle. \item[{\em CL\_\-OUT}]type: (out) Type.\end{description}
\end{Desc}
\begin{Desc}
\item[Return values:]
\begin{description}
\item[{\em CL\_\-OK:}]The function is executed successfully. \item[{\em CL\_\-COR\_\-ERR\_\-NULL\_\-PTR:}]On passing a NULL pointer. \item[{\em CL\_\-COR\_\-UTILS\_\-ERR\_\-INVALID\_\-KEY:}]Invalid value of object Handle. \item[{\em CL\_\-COR\_\-ERR\_\-INVALID\_\-PARAM:}]On passing an invalid parameter. \item[{\em CL\_\-COR\_\-ERR\_\-INVALID\_\-DEPTH:}]Invalid depth of the {\em mo\-Id\/}. \item[{\em CL\_\-COR\_\-ERR\_\-CLASS\_\-INVALID\_\-PATH:}]Qualifiers are invalid. \item[{\em CL\_\-COR\_\-SVC\_\-ERR\_\-INVALID\_\-ID:}]Service ID of the {\em Mo\-Id\/} is invalid. \item[{\em CL\_\-COR\_\-ERR\_\-VERSION\_\-UNSUPPORTED:}]version is not supported.\end{description}
\end{Desc}
\begin{Desc}
\item[Description:]This function is used to retrieve the Object Type given an object handle. Type of an object signifies whether it is an MO or an MSO.\end{Desc}
\begin{Desc}
\item[Library File:]Cl\-Cor\-Client\end{Desc}
\begin{Desc}
\item[Related Function(s):]\hyperlink{pagecor128}{cl\-Cor\-Object\-Handle\-Get} , 
\hyperlink{pagecor133}{cl\-Cor\-Object\-Handle\-To\-Mo\-Id\-Get} \end{Desc}
\newpage


\subsection{clCorObjectHandleToMoIdGet}
\index{clCorObjectHandleToMoIdGet@{clCorObjectHandleToMoIdGet}}
\hypertarget{pagecor133}{}\paragraph{cl\-Cor\-Object\-Handle\-To\-Mo\-Id\-Get}\label{pagecor133}
\begin{Desc}
\item[Synopsis:]Returns the MOId.\end{Desc}
\begin{Desc}
\item[Header File:]clCorApi.h\end{Desc}
\begin{Desc}
\item[Syntax:]

\footnotesize\begin{verbatim}       ClRcT clCorObjectHandleToMoIdGet(
                           CL_IN ClCorObjectHandleT objHandle,
                           CL_OUT ClCorMOIdPtrT moId,
                           CL_OUT ClCorServiceIdT *srvcId);
\end{verbatim}
\normalsize
\end{Desc}
\begin{Desc}
\item[Parameters:]
\begin{description}
\item[{\em CL\_\-IN}]obj\-Handle:(in)  Object handle. \item[{\em CL\_\-OUT}]mo\-Id: (out) Mo\-Id. \item[{\em CL\_\-OUT}]srvc\-Id: (out) ID of the service.\end{description}
\end{Desc}
\begin{Desc}
\item[Return values:]
\begin{description}
\item[{\em CL\_\-OK:}]The function executed successfully. \item[{\em CL\_\-COR\_\-ERR\_\-NULL\_\-PTR:}]On passing a NULL pointer. \item[{\em CL\_\-COR\_\-UTILS\_\-ERR\_\-INVALID\_\-KEY:}]Invalid value of object handle passed. \item[{\em CL\_\-COR\_\-ERR\_\-INVALID\_\-PARAM:}]On passing an invalid parameter. \item[{\em CL\_\-COR\_\-ERR\_\-VERSION\_\-UNSUPPORTED:}]version is not supported.\end{description}
\end{Desc}
\begin{Desc}
\item[Description:]This function returns the {\em MOId\/} of a given object handle. The service ID is also filled along with the {\em mo\-Id\/} which is a valid value for an MSO.\end{Desc}
\begin{Desc}
\item[Library File:]Cl\-Cor\-Client\end{Desc}
\begin{Desc}
\item[Related Function(s):]\hyperlink{pagecor128}{cl\-Cor\-Object\-Handle\-Get} \end{Desc}
\newpage





\subsection{clCorObjectAttributeWalk}
\index{clCorObjectAttributeWalk@{clCorObjectAttributeWalk}}
\hypertarget{pagecor146}{}\paragraph{cl\-Cor\-Object\-Attribute\-Walk}\label{pagecor146}
\begin{Desc}
\item[Synopsis:]Walks though the object attributes.\end{Desc}
\begin{Desc}
\item[Header File:]clCorApi.h\end{Desc}
\begin{Desc}
\item[Syntax:]

\footnotesize\begin{verbatim}       ClRcT clCorObjectAttributeWalk(
                              CL_IN  ClCorObjectHandleT objH,
                              CL_IN  ClCorObjAttrWalkFilterT *pFilter,
                              CL_IN  ClCorObjAttrWalkFuncT fp,
                              CL_IN  void * cookie);
\end{verbatim}
\normalsize
\end{Desc}
\begin{Desc}
\item[Parameters:]
\begin{description}
\item[{\em CL\_\-IN}]obj\-H:(in)  Handle of the object 
\item[{\em CL\_\-IN}]p\-Filter:(in)  Pointer to the attribute filter. You must fill the structure. 
\item[{\em CL\_\-IN}]fp:(in)  User callout. 
\item[{\em CL\_\-IN}]cookie:(in)  Cookie passed to the user callout.\end{description}
\end{Desc}
\begin{Desc}
\item[Return values:]
\begin{description}
\item[{\em CL\_\-OK:}]if the function executes successfully. \item[{\em CL\_\-COR\_\-ERR\_\-INVALID\_\-PARAM:}]Invalid parameter passed. \item[{\em CL\_\-COR\_\-ERR\_\-NULL\_\-PTR:}]Null pointer is passed. \item[{\em CL\_\-COR\_\-ERR\_\-NO\_\-MEM:}]Failed to allocate memory. \item[{\em CL\_\-COR\_\-ERR\_\-NOT\_\-SUPPORTED:}]If cmp flag is not invalid. \item[{\em CL\_\-COR\_\-ERR\_\-VERSION\_\-UNSUPPORTED:}]Client version not supported. \item[{\em CL\_\-COR\_\-UTILS\_\-ERR\_\-INVALID\_\-KEY:}]Invalid parameter passed. \item[{\em CL\_\-COR\_\-INST\_\-ERR\_\-INVALID\_\-MOID:}]Invalid Mo\-Id passed.\end{description}
\end{Desc}
\begin{Desc}
\item[Filter Usage]The parameter {\em p\-Filter\/} can be given different values to apply filters to the walk criteria. \par
 \par
\begin{enumerate}
\item {\tt  p\-Filter = NULL } \par
 NO filter is applied and it walks through all the attributes of object. \par
 \par
\item {\tt  p\-Filter-$>$base\-Attr\-Walk = CL\_\-TRUE } \par
 Walks through the attributes of base object (native attributes of object pointed by {\em obj\-H\/}). \par
 \par
\item {\tt  p\-Filter-$>$base\-Attr\-Walk = CL\_\-FALSE } \par
 Does not walk through the native attributes. \par
 \par
\item {\tt  p\-Filter-$>$p\-Attr\-Path = NULL } \par
 NO filter is applied on the attribute path. It walks all possible attribute paths.The attribute path can be specified with wildcards either in {\em attr\-Id\/} and {\em index\/} or in both. For example,{\em  $\backslash$$\ast$attr\-A$\ast$$\backslash$$\ast$attr\-B$\backslash$$\ast$$\ast$\/}. Specifying the {\em attr\-Path\/} refines the scope of search. \par
 \par
\item {\tt  p\-Filter-$>$cmp\-Flag} \par
 This is comparison flag for {\em attr\-Id\/} with value specified. As of now it should be CL\_\-COR\_\-ATTR\_\-CMP\_\-FLAG\_\-INVALID. \par
 \par
\item {\tt  p\-Filter-$>$attr\-Walk\-Option} \par
 This can have one of the following two values: \begin{itemize}
\item {\tt CL\_\-COR\_\-ATTR\_\-WALK\_\-ALL\_\-ATTR:} All the attributes of contained (and/or base object, depending on {\em base\-Attr\-Walk\/}) object are considered for walk, provided the {\tt p\-Filter-$>$cmp\-Flag } condition goes true.\end{itemize}
\end{enumerate}
\end{Desc}
\begin{Desc}
\item[Description:]This function is used to perform a walk through the Object attributes (including the ones contained) based on the filter. The attribute walk takes object handle as the first parameter. This can be obtained from the MOID by using the function \textit{clCorObjectHandleGet()}. Second parameter is the {\em Filter\/} for the attribute walk. This filter is NULL when there is no need for the filter. But if the search is based on filter, all the elements of the filter are properly initialized. One of the elements of the filter is {\em base\-Attr\-Walk\/}, which must be initialized to {\tt CL\_\-TRUE} if it is desired that the base class attributes are also walked in addition to the children Object. Otherwise this can be {\tt CL\_\-FALSE}. The {\em p\-Attr\-Path\/} is the pointer to the containment path, if we want to start our search from the containment path. The third element is the {\em attr\-Walk\-Option\/} which is {\tt CL\_\-COR\_\-ATTR\_\-WALK\_\-ALL\_\-ATTR}, this is given for walking all the attributes. The third parameter to the function is the callback which is called for all the attributes walked. The comparison of the attributes and its value should be done in this callback only. The fourth parameter is the cookie. This is a user-passed parameter which is used in the callback. Usually the cookie passed is the {\em p\-Mo\-Id\/} from which the object handle is derived.\end{Desc}
\begin{Desc}
\item[Library File:]Cl\-Cor\-Client\end{Desc}
\begin{Desc}
\item[Note:]The attribute walk doesn't support comparison flags (should be CL\_\-COR\_\-ATTR\_\-CMP\_\-FLAG\_\-INVALID). In case you desire any comparison on attribute value, such as equal to, less than, greater than, this comparison must be done in the callback specified in the third parameter.\end{Desc}
\begin{Desc}
\item[Related Function(s):]\hyperlink{pagecor128}{cl\-Cor\-Object\-Handle\-Get} , \hyperlink{pagecor245}{cl\-Cor\-Mo\-Id\-Compare} \end{Desc}
\newpage


\subsection{clCorObjectAttributeWalk}
\index{clCorObjectAttributeWalk@{clCorObjectAttributeWalk}}
\hypertarget{pagecor147}{}\paragraph{cl\-Cor\-Version\-Check}\label{pagecor147}
\begin{Desc}
\item[Synopsis:]Verifies the version supported by COR.\end{Desc}
\begin{Desc}
\item[Header File:]clCorApi.h\end{Desc}
\begin{Desc}
\item[Syntax:]

\footnotesize\begin{verbatim}        ClRcT clCorVersionCheck(
                      CL_INOUT ClVersionT *version  );
\end{verbatim}
\normalsize
\end{Desc}
\begin{Desc}
\item[Parameters:]
\begin{description}
\item[{\em CL\_\-INOUT}]version: (in/out) Version supported.\end{description}
\end{Desc}
\begin{Desc}
\item[Return values:]
\begin{description}
\item[{\em CL\_\-OK:}]The function executed successfully. \item[{\em CL\_\-COR\_\-ERR\_\-VERSION\_\-UNSUPPORTED:}]If version is not supported.\end{description}
\end{Desc}
\begin{Desc}
\item[Description:]This function is used to verify if the version is supported or not. If the version is not supported then the error is returned with the value of supported version filled in the out parameter.\end{Desc}
\begin{Desc}
\item[Library File:]Cl\-Cor\-Client\end{Desc}
\begin{Desc}
\item[Related Function(s):]None. \end{Desc}
\newpage



\subsection{clCorMoIdToNodeNameGet}
\index{clCorMoIdToNodeNameGet@{clCorMoIdToNodeNameGet}}
\hypertarget{pagecor148}{}\paragraph{cl\-Cor\-Mo\-Id\-To\-Node\-Name\-Get}\label{pagecor148}
\begin{Desc}
\item[Synopsis:]This function can be used to get node name corresponding to a mo\-Id which is specified in the ASP configuration file.\end{Desc}
\begin{Desc}
\item[Description:]The function can be used to get the node name corresponding to the moid, supplied in the config file of the ASP (amf\-Config.xml).\end{Desc}
\begin{Desc}
\item[Syntax:]

\footnotesize\begin{verbatim}        ClRcT clCorMoIdToNodeNameGet(
                      CL_IN  ClCorMOIdPtrT pMoId,
                      CL_OUT ClNameT* nodeName);
\end{verbatim}
\normalsize
\end{Desc}
\begin{Desc}
\item[Parameters:]
\begin{description}
\item[{\em CL\_\-IN}]p\-Mo\-Id:(in)  Mo\-Id for which the node name is required. \item[{\em CL\_\-OUT}]node\-Name: (out) Node Name corresponding to the {\em mo\-Id\/} supplied.\end{description}
\end{Desc}
\begin{Desc}
\item[Return values:]
\begin{description}
\item[{\em CL\_\-OK:}]The function executed successfully. \item[{\em CL\_\-COR\_\-ERR\_\-NOT\_\-EXIST:}]If node name for the mo\-Id is not present. \item[{\em CL\_\-COR\_\-ERR\_\-VERSION\_\-UNSUPPORTED:}]version is not supported.\end{description}
\end{Desc}
\begin{Desc}
\item[Related Function(s):]\hyperlink{pagecor149}{cl\-Cor\-Node\-Name\-To\-Mo\-Id\-Get} \end{Desc}
\newpage


\subsection{clCorNodeNameToMoIdGet}
\index{clCorNodeNameToMoIdGet@{clCorNodeNameToMoIdGet}}
\hypertarget{pagecor149}{}\paragraph{cl\-Cor\-Node\-Name\-To\-Mo\-Id\-Get}\label{pagecor149}
\begin{Desc}
\item[Synopsis:]This function can be used to give the Mo\-Id corresponding to the node name supplied.\end{Desc}
\begin{Desc}
\item[Description:]This function is used to get the mo\-Id corresponding to the node name which is there in the config file of the ASP(amf\-Config.xml).\end{Desc}
\begin{Desc}
\item[Syntax:]

\footnotesize\begin{verbatim}        ClRcT clCorNodeNameToMoIdGet(
                      CL_IN  ClNameT nodeName,
                      CL_OUT ClCorMOIdPtrT pMoId);
\end{verbatim}
\normalsize
\end{Desc}
\begin{Desc}
\item[Parameters:]
\begin{description}
\item[{\em CL\_\-IN}]node\-Name: (in) Node name for which the {\em mo\-Id\/} is desired. \item[{\em CL\_\-OUT}]p\-Mo\-Id: (out) {\em mo\-Id\/} is supplied by the user.\end{description}
\end{Desc}
\begin{Desc}
\item[Return values:]
\begin{description}
\item[{\em CL\_\-OK:}]The function executed successfully. \item[{\em CL\_\-COR\_\-ERR\_\-NOT\_\-EXIST:}]If {\em mo\-Id\/} for the node name is not present. \item[{\em CL\_\-COR\_\-ERR\_\-VERSION\_\-UNSUPPORTED:}]version is not supported.\end{description}
\end{Desc}
\begin{Desc}
\item[Related Function(s):]\hyperlink{pagecor149}{cl\-Cor\-Mo\-Id\-To\-Node\-Name\-Get} \end{Desc}
\newpage


\subsection{clCorMoIdInitialize}
\index{clCorMoIdInitialize@{clCorMoIdInitialize}}
\hypertarget{pagecor228}{}\paragraph{cl\-Cor\-Mo\-Id\-Initialize}\label{pagecor228}
\begin{Desc}
\item[Synopsis:]Initializes an empty managed object ID (Mo\-Id) or resets the content of an existing non-empty MOID.\end{Desc}
\begin{Desc}
\item[Header File:]clCorUtilityApi.h\end{Desc}
\begin{Desc}
\item[Syntax:]

\footnotesize\begin{verbatim}   ClRcT clCorMoIdInitialize(
             CL_INOUT ClCorMOIdPtrT pMoId);
\end{verbatim}
\normalsize
\end{Desc}
\begin{Desc}
\item[Parameters:]
\begin{description}
\item[{\em p\-Mo\-Id:}](in/out) A valid pointer to an existing {\em Mo\-Id\/} structure.\end{description}
\end{Desc}
\begin{Desc}
\item[Return values:]
\begin{description}
\item[{\em CL\_\-OK:}]The function executed successfully. \item[{\em CL\_\-COR\_\-ERR\_\-NULL\_\-PTR:}]On passing a NULL pointer.\end{description}
\end{Desc}
\begin{Desc}
\item[Description:]This function is used to (re-)initialize a {\em Cl\-Cor\-Mo\-Id\/} structure. It resets the path information, if present, and re-initializes it as an empty path. It can be applied on {\em Mo\-Ids\/} that have not been initialized and used before, and also on {\em Mo\-Ids\/} with actual path content. The empty {\em Mo\-Ids\/} then can be manipulated with the \textit{clCorMoIdSet()} and \textit{clCorMoIdAppend()} functions This function is used to initialize {\em cl\-Cor\-Mo\-Id\/}. It resets the path information, if present, and re-initializes it as an empty path. This function must be called each time you need to reset the {\em Mo\-Id\/} to begin a fresh operation. \end{Desc}
\begin{Desc}
\item[Note:]This function need not be called after invoking the cl\-Cor\-Mo\-Id\-Alloc function, since the latter initializes the {\em mo\-Id\/}.\end{Desc}
\begin{Desc}
\item[Library File:]Cl\-Cor\-Client\end{Desc}
\begin{Desc}
\item[Related Function(s):]\hyperlink{pagecor229}{cl\-Cor\-Mo\-Id\-Alloc}, \hyperlink{pagecor230}{cl\-Cor\-Mo\-Id\-Free},
\hyperlink{pagecor231}{cl\-Cor\-Mo\-Id\-Truncate}, \hyperlink{pagecor232}{cl\-Cor\-Mo\-Id\-Set}, 
\hyperlink{pagecor233}{cl\-Cor\-Mo\-Id\-Append}, \hyperlink{pagecor235}{cl\-Cor\-Mo\-Id\-Depth\-Get}, 
\hyperlink{pagecor236}{cl\-Cor\-Mo\-Id\-Show}, \hyperlink{pagecor237}{cl\-Cor\-Mo\-Id\-To\-Mo\-Class\-Get}, 
\hyperlink{pagecor273}{cl\-Cor\-Mo\-Id\-Name\-To\-Mo\-Id\-Get}, \hyperlink{pagecor274}{cl\-Cor\-Mo\-Id\-To\-Mo\-Id\-Name\-Get}, 
\hyperlink{pagecor238}{cl\-Cor\-Mo\-Id\-First\-Instance\-Get}, \hyperlink{pagecor249}{cl\-Cor\-Mo\-Id\-Next\-Sibling\-Get}, 
\hyperlink{pagecor240}{cl\-Cor\-Mo\-Id\-Validate}, \hyperlink{pagecor242}{cl\-Cor\-Mo\-Id\-To\-Instance\-Get}, 
\hyperlink{pagecor243}{cl\-Cor\-Mo\-Id\-To\-Mo\-Class\-Path\-Get}, \hyperlink{pagecor244}{cl\-Cor\-Mo\-Id\-Clone}, 
\hyperlink{pagecor245}{cl\-Cor\-Mo\-Id\-Compare} \end{Desc}
\newpage


\subsection{clCorMoIdAlloc}
\index{clCorMoIdAlloc@{clCorMoIdAlloc}}
\hypertarget{pagecor229}{}\paragraph{cl\-Cor\-Mo\-Id\-Alloc}\label{pagecor229}
\begin{Desc}
\item[Synopsis:]Creates an Mo\-Id.\end{Desc}
\begin{Desc}
\item[Header File:]clCorUtilityApi.h\end{Desc}
\begin{Desc}
\item[Syntax:]

\footnotesize\begin{verbatim}  ClRcT clCorMoIdAlloc(
             CL_INOUT ClCorMOIdPtrT *pMoId);
\end{verbatim}
\normalsize
\end{Desc}
\begin{Desc}
\item[Parameters:]
\begin{description}
\item[{\em p\-Mo\-Id:}](in/out) Handle of new {\em MOId\/}.\end{description}
\end{Desc}
\begin{Desc}
\item[Return values:]
\begin{description}
\item[{\em CL\_\-OK:}]The function executed successfully. \item[{\em CL\_\-COR\_\-ERR\_\-NO\_\-MEM:}]On memory allocation failure. \item[{\em CL\_\-COR\_\-ERR\_\-NULL\_\-PTR:}]On passing a NULL pointer.\end{description}
\end{Desc}
\begin{Desc}
\item[Description:]This function is used as a constructor for \textit{ClCorMOId} as it creates an {\em Mo\-Id\/}. It initializes the memory and returns an empty \textit{ClCorMOId}. \par
 \par
 By default, the value for both instance ID and class ID is {\em -1\/}. The default depth for the {\em Mo\-Id\/} is 20. This value is incremented dynamically whenever a new entry is added. This function allocates memory and initializes the {\em mo\-Id\/} structure.\end{Desc}
\begin{Desc}
\item[Library File:]Cl\-Cor\-Client\end{Desc}
\begin{Desc}
\item[Note:]The memory allocated by this function must be freed after usage, to avoid memory leaks. This memory can be freed by calling the cl\-Cor\-Mo\-Id\-Free function.\end{Desc}
\begin{Desc}
\item[Related Function(s):]\hyperlink{pagecor228}{cl\-Cor\-Mo\-Id\-Initialize} , \hyperlink{pagecor229}{cl\-Cor\-Mo\-Id\-Alloc}, 
\hyperlink{pagecor230}{cl\-Cor\-Mo\-Id\-Free},
\hyperlink{pagecor231}{cl\-Cor\-Mo\-Id\-Truncate}, 
\hyperlink{pagecor232}{cl\-Cor\-Mo\-Id\-Set}, 
\hyperlink{pagecor233}{cl\-Cor\-Mo\-Id\-Append}, 
\hyperlink{pagecor235}{cl\-Cor\-Mo\-Id\-Depth\-Get}, 
\hyperlink{pagecor236}{cl\-Cor\-Mo\-Id\-Show}, 
\hyperlink{pagecor237}{cl\-Cor\-Mo\-Id\-To\-Mo\-Class\-Get}, 
\hyperlink{pagecor273}{cl\-Cor\-Mo\-Id\-Name\-To\-Mo\-Id\-Get}, 
\hyperlink{pagecor274}{cl\-Cor\-Mo\-Id\-To\-Mo\-Id\-Name\-Get}, 
\hyperlink{pagecor238}{cl\-Cor\-Mo\-Id\-First\-Instance\-Get},
\hyperlink{pagecor249}{cl\-Cor\-Mo\-Id\-Next\-Sibling\-Get}, 
\hyperlink{pagecor240}{cl\-Cor\-Mo\-Id\-Validate}, 
\hyperlink{pagecor242}{cl\-Cor\-Mo\-Id\-To\-Instance\-Get}, 
\hyperlink{pagecor243}{cl\-Cor\-Mo\-Id\-To\-Mo\-Class\-Path\-Get}, 
\hyperlink{pagecor244}{cl\-Cor\-Mo\-Id\-Clone}, 
\hyperlink{pagecor245}{cl\-Cor\-Mo\-Id\-Compare}\end{Desc}
\newpage


\subsection{clCorMoIdFree}
\index{clCorMoIdFree@{clCorMoIdFree}}
\hypertarget{pagecor230}{}\paragraph{cl\-Cor\-Mo\-Id\-Free}\label{pagecor230}
\begin{Desc}
\item[Synopsis:]Deletes the \textit{ClCorMOId} handle.\end{Desc}
\begin{Desc}
\item[Header File:]clCorUtilityApi.h\end{Desc}
\begin{Desc}
\item[Syntax:]

\footnotesize\begin{verbatim}  ClRcT clCorMoIdFree(
             CL_INOUT ClCorMOIdPtrT pMoId);
\end{verbatim}
\normalsize
\end{Desc}
\begin{Desc}
\item[Parameters:]
\begin{description}
\item[{\em p\-Mo\-Id:}](in/out) Handle of \textit{ClCorMOId}.\end{description}
\end{Desc}
\begin{Desc}
\item[Return values:]
\begin{description}
\item[{\em CL\_\-OK:}]The function executed successfully. \item[{\em CL\_\-COR\_\-ERR\_\-NULL\_\-PTR:}]On passing a NULL pointer.\end{description}
\end{Desc}
\begin{Desc}
\item[Description:]This function is used as destructor of \textit{ClCorMOId} as it deletes the handle of {\em Mo\-Id\/}. It removes the handle and frees the memory associated with it. To avoid memory leaks, each time you invoke the \textit{clCorMoIdAlloc()} function, you must eventually free the memory by invoking the cl\-Cor\-Mo\-Id\-Free function.\end{Desc}
\begin{Desc}
\item[Library File:]Cl\-Cor\-Client\end{Desc}
\begin{Desc}
\item[Related Function(s):]\hyperlink{pagecor228}{cl\-Cor\-Mo\-Id\-Initialize} ,
\hyperlink{pagecor229}{cl\-Cor\-Mo\-Id\-Alloc},  \hyperlink{pagecor231}{cl\-Cor\-Mo\-Id\-Truncate} , 
\hyperlink{pagecor232}{cl\-Cor\-Mo\-Id\-Set} , \hyperlink{pagecor233}{cl\-Cor\-Mo\-Id\-Append} , 
\hyperlink{pagecor233}{cl\-Cor\-Mo\-Id\-Append}, 
\hyperlink{pagecor235}{cl\-Cor\-Mo\-Id\-Depth\-Get}, 
\hyperlink{pagecor236}{cl\-Cor\-Mo\-Id\-Show}, 
\hyperlink{pagecor237}{cl\-Cor\-Mo\-Id\-To\-Mo\-Class\-Get}, 
\hyperlink{pagecor273}{cl\-Cor\-Mo\-Id\-Name\-To\-Mo\-Id\-Get}, 
\hyperlink{pagecor274}{cl\-Cor\-Mo\-Id\-To\-Mo\-Id\-Name\-Get}, 
\hyperlink{pagecor238}{cl\-Cor\-Mo\-Id\-First\-Instance\-Get},
\hyperlink{pagecor249}{cl\-Cor\-Mo\-Id\-Next\-Sibling\-Get}, 
\hyperlink{pagecor240}{cl\-Cor\-Mo\-Id\-Validate}, 
\hyperlink{pagecor242}{cl\-Cor\-Mo\-Id\-To\-Instance\-Get}, 
\hyperlink{pagecor243}{cl\-Cor\-Mo\-Id\-To\-Mo\-Class\-Path\-Get}, 
\hyperlink{pagecor244}{cl\-Cor\-Mo\-Id\-Clone}, 
\hyperlink{pagecor245}{cl\-Cor\-Mo\-Id\-Compare}\end{Desc}
\newpage


\subsection{clCorMoIdTruncate}
\index{clCorMoIdTruncate@{clCorMoIdTruncate}}
\hypertarget{pagecor231}{}\paragraph{cl\-Cor\-Mo\-Id\-Truncate}\label{pagecor231}
\begin{Desc}
\item[Synopsis:]Removes the node after specified level.\end{Desc}
\begin{Desc}
\item[Header File:]clCorUtilityApi.h\end{Desc}
\begin{Desc}
\item[Syntax:]

\footnotesize\begin{verbatim}  ClRcT clCorMoIdTruncate(
             CL_INOUT ClCorMOIdPtrT pMoId,
             CL_IN ClInt16T level);
\end{verbatim}
\normalsize
\end{Desc}
\begin{Desc}
\item[Parameters:]
\begin{description}
\item[{\em p\-Mo\-Id:}](in/out) Handle of the {\em Mo\-Id\/}. \item[{\em level:}](in) Level to which the {\em Mo\-Id\/} needs to be truncated.\end{description}
\end{Desc}
\begin{Desc}
\item[Return values:]
\begin{description}
\item[{\em CL\_\-OK:}]The function executed successfully. \item[{\em CL\_\-COR\_\-ERR\_\-INVALID\_\-DEPTH:}]If the level specified is invalid. \item[{\em CL\_\-COR\_\-ERR\_\-NULL\_\-PTR:}]On passing a NULL pointer.\end{description}
\end{Desc}
\begin{Desc}
\item[Description:]This function is used to remove all the nodes and reset the {\em Mo\-Id\/} until a specified level is reached. The level is specified based on the depth to which you wish to return so as to continue on the other sibling tree node.\end{Desc}
\begin{Desc}
\item[Library File:]Cl\-Cor\-Client\end{Desc}
\begin{Desc}
\item[Related Function(s):]\hyperlink{pagecor101}{cl\-Cor\-Mo\-Id\-Initialize} , \hyperlink{pagecor229}{cl\-Cor\-Mo\-Id\-Alloc}, 
\hyperlink{pagecor230}{cl\-Cor\-Mo\-Id\-Free},
\hyperlink{pagecor231}{cl\-Cor\-Mo\-Id\-Truncate}, 
\hyperlink{pagecor232}{cl\-Cor\-Mo\-Id\-Set}, 
\hyperlink{pagecor233}{cl\-Cor\-Mo\-Id\-Append}, 
\hyperlink{pagecor235}{cl\-Cor\-Mo\-Id\-Depth\-Get}, 
\hyperlink{pagecor236}{cl\-Cor\-Mo\-Id\-Show}, 
\hyperlink{pagecor237}{cl\-Cor\-Mo\-Id\-To\-Mo\-Class\-Get}, 
\hyperlink{pagecor273}{cl\-Cor\-Mo\-Id\-Name\-To\-Mo\-Id\-Get}, 
\hyperlink{pagecor274}{cl\-Cor\-Mo\-Id\-To\-Mo\-Id\-Name\-Get}, 
\hyperlink{pagecor238}{cl\-Cor\-Mo\-Id\-First\-Instance\-Get},
\hyperlink{pagecor249}{cl\-Cor\-Mo\-Id\-Next\-Sibling\-Get}, 
\hyperlink{pagecor240}{cl\-Cor\-Mo\-Id\-Validate}, 
\hyperlink{pagecor242}{cl\-Cor\-Mo\-Id\-To\-Instance\-Get}, 
\hyperlink{pagecor243}{cl\-Cor\-Mo\-Id\-To\-Mo\-Class\-Path\-Get}, 
\hyperlink{pagecor244}{cl\-Cor\-Mo\-Id\-Clone}, 
\hyperlink{pagecor245}{cl\-Cor\-Mo\-Id\-Compare}\end{Desc}
\newpage



\subsection{clCorMoIdSet}
\index{clCorMoIdSet@{clCorMoIdSet}}
\hypertarget{pagecor232}{}\paragraph{cl\-Cor\-Mo\-Id\-Set}\label{pagecor232}
\begin{Desc}
\item[Synopsis:]Sets the class type and {\em instance\-Id\/} at a given node or level.\end{Desc}
\begin{Desc}
\item[Header File:]clCorUtilityApi.h\end{Desc}
\begin{Desc}
\item[Syntax:]

\footnotesize\begin{verbatim}  ClRcT clCorMoIdSet(
             CL_INOUT ClCorMOIdPtrT pMoId,
             CL_IN ClUint16T level,
             CL_IN ClCorClassTypeT type,
             CL_IN ClCorInstanceIdT instance);
\end{verbatim}
\normalsize
\end{Desc}
\begin{Desc}
\item[Parameters:]
\begin{description}
\item[{\em p\-Mo\-Id:}](in/out) Handle of {\em MOId\/} Path. 
\item[{\em level:}](in) Level of the node 
\item[{\em type:}](in) Class type to be set. 
\item[{\em instance:}](in) {\em Instance\-Id\/} to be set.\end{description}
\end{Desc}
\begin{Desc}
\item[Return values:]
\begin{description}
\item[{\em CL\_\-OK:}]The function executed successfully. \item[{\em CL\_\-COR\_\-ERR\_\-INVALID\_\-DEPTH:}]If the level specified is invalid. \item[{\em CL\_\-COR\_\-ERR\_\-NULL\_\-PTR:}]On passing a NULL pointer. \item[{\em CL\_\-COR\_\-ERR\_\-INVALID\_\-CLASS:}]Invalid class type to set.\end{description}
\end{Desc}
\begin{Desc}
\item[Description:]This function is used to set the class type and the {\em instance\-Id\/} properties at a given node or level. This level should be less than the current depth of the mo\-Id. This feature can be used in order to set the Mo\-Id for a level when it is required to operate on that part of the instance tree.\end{Desc}
\begin{Desc}
\item[Library File:]Cl\-Cor\-Client\end{Desc}
\begin{Desc}
\item[Note:]The second parameter level should be always less than the current depth of the mo\-Id.\end{Desc}
\begin{Desc}
\item[Related Function(s):]\hyperlink{pagecor101}{cl\-Cor\-Mo\-Id\-Initialize} , \hyperlink{pagecor229}{cl\-Cor\-Mo\-Id\-Alloc}, 
\hyperlink{pagecor230}{cl\-Cor\-Mo\-Id\-Free},
\hyperlink{pagecor231}{cl\-Cor\-Mo\-Id\-Truncate}, 
\hyperlink{pagecor232}{cl\-Cor\-Mo\-Id\-Set}, 
\hyperlink{pagecor233}{cl\-Cor\-Mo\-Id\-Append}, 
\hyperlink{pagecor235}{cl\-Cor\-Mo\-Id\-Depth\-Get}, 
\hyperlink{pagecor236}{cl\-Cor\-Mo\-Id\-Show}, 
\hyperlink{pagecor237}{cl\-Cor\-Mo\-Id\-To\-Mo\-Class\-Get}, 
\hyperlink{pagecor273}{cl\-Cor\-Mo\-Id\-Name\-To\-Mo\-Id\-Get}, 
\hyperlink{pagecor274}{cl\-Cor\-Mo\-Id\-To\-Mo\-Id\-Name\-Get}, 
\hyperlink{pagecor238}{cl\-Cor\-Mo\-Id\-First\-Instance\-Get},
\hyperlink{pagecor249}{cl\-Cor\-Mo\-Id\-Next\-Sibling\-Get}, 
\hyperlink{pagecor240}{cl\-Cor\-Mo\-Id\-Validate}, 
\hyperlink{pagecor242}{cl\-Cor\-Mo\-Id\-To\-Instance\-Get}, 
\hyperlink{pagecor243}{cl\-Cor\-Mo\-Id\-To\-Mo\-Class\-Path\-Get}, 
\hyperlink{pagecor244}{cl\-Cor\-Mo\-Id\-Clone}, 
\hyperlink{pagecor245}{cl\-Cor\-Mo\-Id\-Compare}\end{Desc}
\newpage


\subsection{clCorMoIdAppend}
\index{clCorMoIdAppend@{clCorMoIdAppend}}
\hypertarget{pagecor233}{}\paragraph{cl\-Cor\-Mo\-Id\-Append}\label{pagecor233}
\begin{Desc}
\item[Synopsis:]Adds an entry to the Mo\-Id.\end{Desc}
\begin{Desc}
\item[Header File:]clCorUtilityApi.h\end{Desc}
\begin{Desc}
\item[Syntax:]

\footnotesize\begin{verbatim}   ClRcT clCorMoIdAppend(
              CL_INOUT ClCorMOIdPtrT pMoId,
              CL_IN ClCorClassTypeT type,
              CL_IN ClCorInstanceIdT instance);
\end{verbatim}
\normalsize
\end{Desc}
\begin{Desc}
\item[Parameters:]
\begin{description}
\item[{\em p\-Mo\-Id:}](in/out) Handle of the {\em MOId\/} \item[{\em type:}](in) Node type. \item[{\em instance:}](in) ID of the node instance.\end{description}
\end{Desc}
\begin{Desc}
\item[Return values:]
\begin{description}
\item[{\em CL\_\-OK:}]The function executed successfully. \item[{\em CL\_\-COR\_\-ERR\_\-MAX\_\-DEPTH:}]If depth exceeded the maximum limit. \item[{\em CL\_\-COR\_\-ERR\_\-INVALID\_\-CLASS:}]Invalid class type for append. \item[{\em CL\_\-COR\_\-ERR\_\-INVALID\_\-PARAM:}]Invalid parameter is passed.\end{description}
\end{Desc}
\begin{Desc}
\item[Description:]This function is used to add an entry to \textit{ClCorMOId}. You can explicitly specify the type and the instance for the entry. The class\-Id and instance Id are appended at the end of the current mo\-Id and the depth of the mo\-Id is incremented.\end{Desc}
\begin{Desc}
\item[Library File:]Cl\-Cor\-Client\end{Desc}
\begin{Desc}
\item[Related Function(s):]\hyperlink{pagecor101}{cl\-Cor\-Mo\-Id\-Initialize} , \hyperlink{pagecor229}{cl\-Cor\-Mo\-Id\-Alloc}, 
\hyperlink{pagecor230}{cl\-Cor\-Mo\-Id\-Free},
\hyperlink{pagecor231}{cl\-Cor\-Mo\-Id\-Truncate}, 
\hyperlink{pagecor232}{cl\-Cor\-Mo\-Id\-Set}, 
\hyperlink{pagecor233}{cl\-Cor\-Mo\-Id\-Append}, 
\hyperlink{pagecor235}{cl\-Cor\-Mo\-Id\-Depth\-Get}, 
\hyperlink{pagecor236}{cl\-Cor\-Mo\-Id\-Show}, 
\hyperlink{pagecor237}{cl\-Cor\-Mo\-Id\-To\-Mo\-Class\-Get}, 
\hyperlink{pagecor273}{cl\-Cor\-Mo\-Id\-Name\-To\-Mo\-Id\-Get}, 
\hyperlink{pagecor274}{cl\-Cor\-Mo\-Id\-To\-Mo\-Id\-Name\-Get}, 
\hyperlink{pagecor238}{cl\-Cor\-Mo\-Id\-First\-Instance\-Get},
\hyperlink{pagecor249}{cl\-Cor\-Mo\-Id\-Next\-Sibling\-Get}, 
\hyperlink{pagecor240}{cl\-Cor\-Mo\-Id\-Validate}, 
\hyperlink{pagecor242}{cl\-Cor\-Mo\-Id\-To\-Instance\-Get}, 
\hyperlink{pagecor243}{cl\-Cor\-Mo\-Id\-To\-Mo\-Class\-Path\-Get}, 
\hyperlink{pagecor244}{cl\-Cor\-Mo\-Id\-Clone}, 
\hyperlink{pagecor245}{cl\-Cor\-Mo\-Id\-Compare}\end{Desc}
\newpage


\subsection{clCorMoIdDepthGet}
\index{clCorMoIdDepthGet@{clCorMoIdDepthGet}}
\hypertarget{pagecor235}{}\paragraph{cl\-Cor\-Mo\-Id\-Depth\-Get}\label{pagecor235}
\begin{Desc}
\item[Synopsis:]Returns node depth of the COR {\em Mo\-Id\/} \end{Desc}
\begin{Desc}
\item[Header File:]clCorUtilityApi.h\end{Desc}
\begin{Desc}
\item[Syntax:]

\footnotesize\begin{verbatim}   ClInt16T clCorMoIdDepthGet(
              CL_IN ClCorMOIdPtrT pMoId);
\end{verbatim}
\normalsize
\end{Desc}
\begin{Desc}
\item[Parameters:]
\begin{description}
\item[{\em p\-Mo\-Id:}](in) Handle of the {\em MOId\/}.\end{description}
\end{Desc}
\begin{Desc}
\item[Return values:]{\em Cl\-Int16T\/}, the number of elements.\end{Desc}
\begin{Desc}
\item[Description:]This function is used to return the number of nodes in the hierarchy within the COR {\em Mo\-Id\/}.\end{Desc}
\begin{Desc}
\item[Library File:]Cl\-Cor\-Client\end{Desc}
\begin{Desc}
\item[Related Function(s):]\hyperlink{group__group13}{cl\-Cor\-Mo\-Id\-Initialize}, \hyperlink{pagecor229}{cl\-Cor\-Mo\-Id\-Alloc}, \hyperlink{pagecor230}{cl\-Cor\-Mo\-Id\-Free}, \hyperlink{pagecor231}{cl\-Cor\-Mo\-Id\-Truncate}, \hyperlink{group__group13}{cl\-Cor\-Mo\-Id\-Set}, \hyperlink{group__group13}{cl\-Cor\-Mo\-Id\-Append}, \hyperlink{group__group13}{cl\-Cor\-Mo\-Id\-Show}, \hyperlink{group__group13}{cl\-Cor\-Mo\-Id\-To\-Mo\-Class\-Get}, \hyperlink{group__group13}{cl\-Cor\-Mo\-Id\-Name\-To\-Mo\-Id\-Get}, \hyperlink{group__group13}{cl\-Cor\-Mo\-Id\-To\-Mo\-Id\-Name\-Get}, \hyperlink{group__group13}{cl\-Cor\-Mo\-Id\-First\-Instance\-Get}, \hyperlink{group__group13}{cl\-Cor\-Mo\-Id\-Next\-Sibling\-Get}, \hyperlink{group__group13}{cl\-Cor\-Mo\-Id\-Validate}, \hyperlink{group__group13}{cl\-Cor\-Mo\-Id\-To\-Instance\-Get}, \hyperlink{group__group13}{cl\-Cor\-Mo\-Id\-To\-Mo\-Class\-Path\-Get}, \hyperlink{group__group13}{cl\-Cor\-Mo\-Id\-Clone}, \hyperlink{pagecor245}{cl\-Cor\-Mo\-Id\-Compare} \end{Desc}
\newpage


\subsection{clCorMoIdShow}
\index{clCorMoIdShow@{clCorMoIdShow}}
\hypertarget{pagecor236}{}\paragraph{cl\-Cor\-Mo\-Id\-Show}\label{pagecor236}
\begin{Desc}
\item[Synopsis:]Displays the \textit{ClCorMOId} Handle in debug mode only.\end{Desc}
\begin{Desc}
\item[Syntax:]

\footnotesize\begin{verbatim}   void clCorMoIdShow(
              CL_IN ClCorMOIdPtrT pMoId);
\end{verbatim}
\normalsize
\end{Desc}
\begin{Desc}
\item[Parameters:]
\begin{description}
\item[{\em p\-Mo\-Id:}](in) Handle of the {\em MOId\/}.\end{description}
\end{Desc}
\begin{Desc}
\item[Return values:]{\em void\/}, no value\end{Desc}
\begin{Desc}
\item[Description:]This function is used to display all the entries within the COR Mo\-Id. The Mo\-Id show will show all the current active nodes in the mo\-Id and its service id.\end{Desc}
\begin{Desc}
\item[Related Function(s):]\hyperlink{pagecor101}{cl\-Cor\-Mo\-Id\-Initialize} , \hyperlink{pagecor229}{cl\-Cor\-Mo\-Id\-Alloc}, 
\hyperlink{pagecor230}{cl\-Cor\-Mo\-Id\-Free},
\hyperlink{pagecor231}{cl\-Cor\-Mo\-Id\-Truncate}, 
\hyperlink{pagecor232}{cl\-Cor\-Mo\-Id\-Set}, 
\hyperlink{pagecor233}{cl\-Cor\-Mo\-Id\-Append}, 
\hyperlink{pagecor235}{cl\-Cor\-Mo\-Id\-Depth\-Get}, 
\hyperlink{pagecor236}{cl\-Cor\-Mo\-Id\-Show}, 
\hyperlink{pagecor237}{cl\-Cor\-Mo\-Id\-To\-Mo\-Class\-Get}, 
\hyperlink{pagecor273}{cl\-Cor\-Mo\-Id\-Name\-To\-Mo\-Id\-Get}, 
\hyperlink{pagecor274}{cl\-Cor\-Mo\-Id\-To\-Mo\-Id\-Name\-Get}, 
\hyperlink{pagecor238}{cl\-Cor\-Mo\-Id\-First\-Instance\-Get},
\hyperlink{pagecor249}{cl\-Cor\-Mo\-Id\-Next\-Sibling\-Get}, 
\hyperlink{pagecor240}{cl\-Cor\-Mo\-Id\-Validate}, 
\hyperlink{pagecor242}{cl\-Cor\-Mo\-Id\-To\-Instance\-Get}, 
\hyperlink{pagecor243}{cl\-Cor\-Mo\-Id\-To\-Mo\-Class\-Path\-Get}, 
\hyperlink{pagecor244}{cl\-Cor\-Mo\-Id\-Clone}, 
\hyperlink{pagecor245}{cl\-Cor\-Mo\-Id\-Compare}\end{Desc}
\newpage


\subsection{clCorMoIdToMoClassGet}
\index{clCorMoIdToMoClassGet@{clCorMoIdToMoClassGet}}
\hypertarget{pagecor237}{}\paragraph{cl\-Cor\-Mo\-Id\-To\-Mo\-Class\-Get}\label{pagecor237}
\begin{Desc}
\item[Synopsis:]Returns the class type.\end{Desc}
\begin{Desc}
\item[Header File:]clCorUtilityApi.h\end{Desc}
\begin{Desc}
\item[Syntax:]

\footnotesize\begin{verbatim}   ClRcT clCorMoIdToClassGet(
                         CL_IN ClCorMOIdPtrT pMoId,
                         CL_IN ClCorMoIdClassGetFlagsT flag,
                         CL_OUT ClCorClassTypeT *pClassId);
\end{verbatim}
\normalsize
\end{Desc}
\begin{Desc}
\item[Parameters:]
\begin{description}
\item[{\em p\-Mo\-Id:}](in) Handle of the {\em MOId\/}. \item[{\em flag:}](in) Specifies type of MO class to be returned. Its value can be either 
{\tt CL\_\-COR\_\-MO\_\-CLASS\_\-GET} or {\tt CL\_\-COR\_\-MSO\_\-CLASS\_\-GET}. \item[{\em p\-Class\-Id:}](out) The {\em class\-Id\/} of the class.\end{description}
\end{Desc}
\begin{Desc}
\item[Return values:]
\begin{description}
\item[{\em CL\_\-OK:}]The function executed successfully.\end{description}
\end{Desc}
\begin{Desc}
\item[Description:]This function is used to return the class type within the COR {\em Mo\-Id\/}. It refers to the class type at the bottom of the hierarchy.\end{Desc}
\begin{Desc}
\item[Library File:]Cl\-Cor\-Client\end{Desc}
\begin{Desc}
\item[Related Function(s):]\hyperlink{pagecor101}{cl\-Cor\-Mo\-Id\-Initialize} , \hyperlink{pagecor229}{cl\-Cor\-Mo\-Id\-Alloc}, 
\hyperlink{pagecor230}{cl\-Cor\-Mo\-Id\-Free},
\hyperlink{pagecor231}{cl\-Cor\-Mo\-Id\-Truncate}, 
\hyperlink{pagecor232}{cl\-Cor\-Mo\-Id\-Set}, 
\hyperlink{pagecor233}{cl\-Cor\-Mo\-Id\-Append}, 
\hyperlink{pagecor235}{cl\-Cor\-Mo\-Id\-Depth\-Get}, 
\hyperlink{pagecor236}{cl\-Cor\-Mo\-Id\-Show}, 
\hyperlink{pagecor237}{cl\-Cor\-Mo\-Id\-To\-Mo\-Class\-Get}, 
\hyperlink{pagecor273}{cl\-Cor\-Mo\-Id\-Name\-To\-Mo\-Id\-Get}, 
\hyperlink{pagecor274}{cl\-Cor\-Mo\-Id\-To\-Mo\-Id\-Name\-Get}, 
\hyperlink{pagecor238}{cl\-Cor\-Mo\-Id\-First\-Instance\-Get},
\hyperlink{pagecor249}{cl\-Cor\-Mo\-Id\-Next\-Sibling\-Get}, 
\hyperlink{pagecor240}{cl\-Cor\-Mo\-Id\-Validate}, 
\hyperlink{pagecor242}{cl\-Cor\-Mo\-Id\-To\-Instance\-Get}, 
\hyperlink{pagecor243}{cl\-Cor\-Mo\-Id\-To\-Mo\-Class\-Path\-Get}, 
\hyperlink{pagecor244}{cl\-Cor\-Mo\-Id\-Clone}, 
\hyperlink{pagecor245}{cl\-Cor\-Mo\-Id\-Compare}\end{Desc}
\newpage


\subsection{clCorMoIdNameToMoIdGet}
\index{clCorMoIdNameToMoIdGet@{clCorMoIdNameToMoIdGet}}
\hypertarget{pagecor273}{}\paragraph{cl\-Cor\-Mo\-Id\-Name\-To\-Mo\-Id\-Get}\label{pagecor273}
\begin{Desc}
\item[Synopsis:]Retrieves {\em mo\-Id\/} in Cl\-Cor\-MOId\-T format, when {\em mo\-Id\/} is given in \textit{ClNameT} format.\end{Desc}
\begin{Desc}
\item[Syntax:]

\footnotesize\begin{verbatim}  ClRcT clCorMoIdNameToMoIdGet
                         CL_IN ClNameT *moIdName,
                         CL_OUT ClCorMOIdT *moId);
\end{verbatim}
\normalsize
\end{Desc}
\begin{Desc}
\item[Parameters:]
\begin{description}
\item[{\em mo\-Id\-Name:}](in) {\em mo\-Id\/} in String format. \item[{\em mo\-Id:}](out) {\em mo\-Id\/} returned. It has to be allocated by user.\end{description}
\end{Desc}
\begin{Desc}
\item[Return values:]
\begin{description}
\item[{\em CL\_\-OK:}]The function executed successfully.\end{description}
\end{Desc}
\begin{Desc}
\item[Description:]This function is used to retrieve {\em mo\-Id\/} in Cl\-Cor\-MOId\-T format, when {\em mo\-Id\/} is given in \textit{ClNameT} format.\end{Desc}
\begin{Desc}
\item[Related Function(s):]\hyperlink{pagecor101}{cl\-Cor\-Mo\-Id\-Initialize} , \hyperlink{pagecor229}{cl\-Cor\-Mo\-Id\-Alloc}, 
\hyperlink{pagecor230}{cl\-Cor\-Mo\-Id\-Free},
\hyperlink{pagecor231}{cl\-Cor\-Mo\-Id\-Truncate}, 
\hyperlink{pagecor232}{cl\-Cor\-Mo\-Id\-Set}, 
\hyperlink{pagecor233}{cl\-Cor\-Mo\-Id\-Append}, 
\hyperlink{pagecor235}{cl\-Cor\-Mo\-Id\-Depth\-Get}, 
\hyperlink{pagecor236}{cl\-Cor\-Mo\-Id\-Show}, 
\hyperlink{pagecor237}{cl\-Cor\-Mo\-Id\-To\-Mo\-Class\-Get}, 
\hyperlink{pagecor273}{cl\-Cor\-Mo\-Id\-Name\-To\-Mo\-Id\-Get}, 
\hyperlink{pagecor274}{cl\-Cor\-Mo\-Id\-To\-Mo\-Id\-Name\-Get}, 
\hyperlink{pagecor238}{cl\-Cor\-Mo\-Id\-First\-Instance\-Get},
\hyperlink{pagecor249}{cl\-Cor\-Mo\-Id\-Next\-Sibling\-Get}, 
\hyperlink{pagecor240}{cl\-Cor\-Mo\-Id\-Validate}, 
\hyperlink{pagecor242}{cl\-Cor\-Mo\-Id\-To\-Instance\-Get}, 
\hyperlink{pagecor243}{cl\-Cor\-Mo\-Id\-To\-Mo\-Class\-Path\-Get}, 
\hyperlink{pagecor244}{cl\-Cor\-Mo\-Id\-Clone}, 
\hyperlink{pagecor245}{cl\-Cor\-Mo\-Id\-Compare}\end{Desc}
\newpage


\subsection{clCorMoIdToMoIdNameGet}
\index{clCorMoIdToMoIdNameGet@{clCorMoIdToMoIdNameGet}}
\hypertarget{pagecor274}{}\paragraph{cl\-Cor\-Mo\-Id\-To\-Mo\-Id\-Name\-Get}\label{pagecor274}
\begin{Desc}
\item[Synopsis:]Function to retrieve {\em mo\-Id\/} in \textit{ClNameT} format, when given {\em mo\-Id\/} is in Cl\-Cor\-MOId\-T format.\end{Desc}
\begin{Desc}
\item[Header File:]clCorUtilityApi.h\end{Desc}
\begin{Desc}
\item[Syntax:]

\footnotesize\begin{verbatim}  ClRcT clCorMoIdToMoIdNameGet(
                         CL_IN ClCorMOIdT *moId,
                         CL_OUT ClNameT *moIdName);
\end{verbatim}
\normalsize
\end{Desc}
\begin{Desc}
\item[Parameters:]
\begin{description}
\item[{\em mo\-Id:}](in) {\em mo\-Id\/} structure. \item[{\em mo\-Id\-Name:}](out) {\em mo\-Id\/} in String format.\end{description}
\end{Desc}
\begin{Desc}
\item[Return values:]
\begin{description}
\item[{\em CL\_\-OK:}]The function executed successfully.\end{description}
\end{Desc}
\begin{Desc}
\item[Description:]This function is used to retrieve {\em mo\-Id\/} in \textit{ClNameT} format, when given {\em mo\-Id\/} is in Cl\-Cor\-MOId\-T format.\end{Desc}
\begin{Desc}
\item[Library File:]Cl\-Cor\-Client\end{Desc}
\begin{Desc}
\item[Related Function(s):]\hyperlink{pagecor101}{cl\-Cor\-Mo\-Id\-Initialize} , \hyperlink{pagecor229}{cl\-Cor\-Mo\-Id\-Alloc}, 
\hyperlink{pagecor230}{cl\-Cor\-Mo\-Id\-Free},
\hyperlink{pagecor231}{cl\-Cor\-Mo\-Id\-Truncate}, 
\hyperlink{pagecor232}{cl\-Cor\-Mo\-Id\-Set}, 
\hyperlink{pagecor233}{cl\-Cor\-Mo\-Id\-Append}, 
\hyperlink{pagecor235}{cl\-Cor\-Mo\-Id\-Depth\-Get}, 
\hyperlink{pagecor236}{cl\-Cor\-Mo\-Id\-Show}, 
\hyperlink{pagecor237}{cl\-Cor\-Mo\-Id\-To\-Mo\-Class\-Get}, 
\hyperlink{pagecor273}{cl\-Cor\-Mo\-Id\-Name\-To\-Mo\-Id\-Get}, 
\hyperlink{pagecor274}{cl\-Cor\-Mo\-Id\-To\-Mo\-Id\-Name\-Get}, 
\hyperlink{pagecor238}{cl\-Cor\-Mo\-Id\-First\-Instance\-Get},
\hyperlink{pagecor249}{cl\-Cor\-Mo\-Id\-Next\-Sibling\-Get}, 
\hyperlink{pagecor240}{cl\-Cor\-Mo\-Id\-Validate}, 
\hyperlink{pagecor242}{cl\-Cor\-Mo\-Id\-To\-Instance\-Get}, 
\hyperlink{pagecor243}{cl\-Cor\-Mo\-Id\-To\-Mo\-Class\-Path\-Get}, 
\hyperlink{pagecor244}{cl\-Cor\-Mo\-Id\-Clone}, 
\hyperlink{pagecor245}{cl\-Cor\-Mo\-Id\-Compare}
\end{Desc}
\newpage


\subsection{clCorMoIdFirstInstanceGet}
\index{clCorMoIdFirstInstanceGet@{clCorMoIdFirstInstanceGet}}
\hypertarget{pagecor238}{}\paragraph{cl\-Cor\-Mo\-Id\-First\-Instance\-Get}\label{pagecor238}
\begin{Desc}
\item[Synopsis:]Returns the first child.\end{Desc}
\begin{Desc}
\item[Header File:]clCorUtilityApi.h\end{Desc}
\begin{Desc}
\item[Syntax:]

\footnotesize\begin{verbatim}   ClRcT clCorMoIdFirstInstanceGet(
                         CL_INOUT ClCorMOIdPtrT pMoId);
\end{verbatim}
\normalsize
\end{Desc}
\begin{Desc}
\item[Parameters:]
\begin{description}
\item[{\em p\-Mo\-Id:}](in/out) Updated \textit{ClCorMOId} is returned. Memory allocation for the {\em Mo\-Id\/} must be done by you.\end{description}
\end{Desc}
\begin{Desc}
\item[Return values:]
\begin{description}
\item[{\em CL\_\-OK:}]The function executed successfully.\end{description}
\end{Desc}
\begin{Desc}
\item[Description:]This function is used to retrieve the first child instance from the {\em Mo\-Id\/} tree. On successful execution, it updates the COR {\em Mo\-Id\/}.\end{Desc}
\begin{Desc}
\item[Library File:]Cl\-Cor\-Client\end{Desc}
\begin{Desc}
\item[Note:]The last node class tag must be completed and {\em instance\-Id\/} must be left at zero which will be updated by this function.\end{Desc}
\begin{Desc}
\item[Related Function(s):]\hyperlink{pagecor101}{cl\-Cor\-Mo\-Id\-Initialize} , \hyperlink{pagecor229}{cl\-Cor\-Mo\-Id\-Alloc}, 
\hyperlink{pagecor230}{cl\-Cor\-Mo\-Id\-Free},
\hyperlink{pagecor231}{cl\-Cor\-Mo\-Id\-Truncate}, 
\hyperlink{pagecor232}{cl\-Cor\-Mo\-Id\-Set}, 
\hyperlink{pagecor233}{cl\-Cor\-Mo\-Id\-Append}, 
\hyperlink{pagecor235}{cl\-Cor\-Mo\-Id\-Depth\-Get}, 
\hyperlink{pagecor236}{cl\-Cor\-Mo\-Id\-Show}, 
\hyperlink{pagecor237}{cl\-Cor\-Mo\-Id\-To\-Mo\-Class\-Get}, 
\hyperlink{pagecor273}{cl\-Cor\-Mo\-Id\-Name\-To\-Mo\-Id\-Get}, 
\hyperlink{pagecor274}{cl\-Cor\-Mo\-Id\-To\-Mo\-Id\-Name\-Get}, 
\hyperlink{pagecor238}{cl\-Cor\-Mo\-Id\-First\-Instance\-Get},
\hyperlink{pagecor249}{cl\-Cor\-Mo\-Id\-Next\-Sibling\-Get}, 
\hyperlink{pagecor240}{cl\-Cor\-Mo\-Id\-Validate}, 
\hyperlink{pagecor242}{cl\-Cor\-Mo\-Id\-To\-Instance\-Get}, 
\hyperlink{pagecor243}{cl\-Cor\-Mo\-Id\-To\-Mo\-Class\-Path\-Get}, 
\hyperlink{pagecor244}{cl\-Cor\-Mo\-Id\-Clone}, 
\hyperlink{pagecor245}{cl\-Cor\-Mo\-Id\-Compare}\end{Desc}
\newpage


\subsection{clCorMoIdNextSiblingGet}
\index{clCorMoIdNextSiblingGet@{clCorMoIdNextSiblingGet}}
\hypertarget{pagecor239}{}\paragraph{cl\-Cor\-Mo\-Id\-Next\-Sibling\-Get}\label{pagecor239}
\begin{Desc}
\item[Synopsis:]Returns the next sibling.\end{Desc}
\begin{Desc}
\item[Header File:]clCorUtilityApi.h\end{Desc}
\begin{Desc}
\item[Syntax:]

\footnotesize\begin{verbatim}   ClRcT clCorMoIdNextSiblingGet(
                         CL_INOUT ClCorMOIdPtrT pMoId);
\end{verbatim}
\normalsize
\end{Desc}
\begin{Desc}
\item[Parameters:]
\begin{description}
\item[{\em p\-Mo\-Id:}](in/out) Updated \textit{ClCorMOId} is returned. You must allocate the memory for the {\em Mo\-Id\/}.\end{description}
\end{Desc}
\begin{Desc}
\item[Return values:]
\begin{description}
\item[{\em CL\_\-OK:}]The function executed successfully.\end{description}
\end{Desc}
\begin{Desc}
\item[Description:]This function is used to return the next sibling from the {\em MOId\/} tree.\end{Desc}
\begin{Desc}
\item[Library File:]Cl\-Cor\-Client\end{Desc}
\begin{Desc}
\item[Related Function(s):]\hyperlink{pagecor101}{cl\-Cor\-Mo\-Id\-Initialize} , \hyperlink{pagecor229}{cl\-Cor\-Mo\-Id\-Alloc}, 
\hyperlink{pagecor230}{cl\-Cor\-Mo\-Id\-Free},
\hyperlink{pagecor231}{cl\-Cor\-Mo\-Id\-Truncate}, 
\hyperlink{pagecor232}{cl\-Cor\-Mo\-Id\-Set}, 
\hyperlink{pagecor233}{cl\-Cor\-Mo\-Id\-Append}, 
\hyperlink{pagecor235}{cl\-Cor\-Mo\-Id\-Depth\-Get}, 
\hyperlink{pagecor236}{cl\-Cor\-Mo\-Id\-Show}, 
\hyperlink{pagecor237}{cl\-Cor\-Mo\-Id\-To\-Mo\-Class\-Get}, 
\hyperlink{pagecor273}{cl\-Cor\-Mo\-Id\-Name\-To\-Mo\-Id\-Get}, 
\hyperlink{pagecor274}{cl\-Cor\-Mo\-Id\-To\-Mo\-Id\-Name\-Get}, 
\hyperlink{pagecor238}{cl\-Cor\-Mo\-Id\-First\-Instance\-Get},
\hyperlink{pagecor249}{cl\-Cor\-Mo\-Id\-Next\-Sibling\-Get}, 
\hyperlink{pagecor240}{cl\-Cor\-Mo\-Id\-Validate}, 
\hyperlink{pagecor242}{cl\-Cor\-Mo\-Id\-To\-Instance\-Get}, 
\hyperlink{pagecor243}{cl\-Cor\-Mo\-Id\-To\-Mo\-Class\-Path\-Get}, 
\hyperlink{pagecor244}{cl\-Cor\-Mo\-Id\-Clone}, 
\hyperlink{pagecor245}{cl\-Cor\-Mo\-Id\-Compare}\end{Desc}
\newpage


\subsection{clCorMoIdValidate}
\index{clCorMoIdValidate@{clCorMoIdValidate}}
\hypertarget{pagecor240}{}\paragraph{cl\-Cor\-Mo\-Id\-Validate}\label{pagecor240}
\begin{Desc}
\item[Synopsis:]Validates {\em Mo\-Id\/} in the input argument.\end{Desc}
\begin{Desc}
\item[Header File:]clCorUtilityApi.h\end{Desc}
\begin{Desc}
\item[Syntax:]

\footnotesize\begin{verbatim}   ClRcT clCorMoIdValidate(
                         CL_IN ClCorMOIdPtrT pMoId);
\end{verbatim}
\normalsize
\end{Desc}
\begin{Desc}
\item[Parameters:]
\begin{description}
\item[{\em p\-Mo\-Id:}](in) Handle of the {\em MOId\/}.\end{description}
\end{Desc}
\begin{Desc}
\item[Return values:]
\begin{description}
\item[{\em CL\_\-OK:}]The function executed successfully.\end{description}
\end{Desc}
\begin{Desc}
\item[Description:]This function is used to validate the {\em Mo\-Id\/} passed in the input argument.\end{Desc}
\begin{Desc}
\item[Library File:]Cl\-Cor\-Client\end{Desc}
\begin{Desc}
\item[Related Function(s):]\hyperlink{pagecor101}{cl\-Cor\-Mo\-Id\-Initialize} , \hyperlink{pagecor229}{cl\-Cor\-Mo\-Id\-Alloc}, 
\hyperlink{pagecor230}{cl\-Cor\-Mo\-Id\-Free},
\hyperlink{pagecor231}{cl\-Cor\-Mo\-Id\-Truncate}, 
\hyperlink{pagecor232}{cl\-Cor\-Mo\-Id\-Set}, 
\hyperlink{pagecor233}{cl\-Cor\-Mo\-Id\-Append}, 
\hyperlink{pagecor235}{cl\-Cor\-Mo\-Id\-Depth\-Get}, 
\hyperlink{pagecor236}{cl\-Cor\-Mo\-Id\-Show}, 
\hyperlink{pagecor237}{cl\-Cor\-Mo\-Id\-To\-Mo\-Class\-Get}, 
\hyperlink{pagecor273}{cl\-Cor\-Mo\-Id\-Name\-To\-Mo\-Id\-Get}, 
\hyperlink{pagecor274}{cl\-Cor\-Mo\-Id\-To\-Mo\-Id\-Name\-Get}, 
\hyperlink{pagecor238}{cl\-Cor\-Mo\-Id\-First\-Instance\-Get},
\hyperlink{pagecor249}{cl\-Cor\-Mo\-Id\-Next\-Sibling\-Get}, 
\hyperlink{pagecor240}{cl\-Cor\-Mo\-Id\-Validate}, 
\hyperlink{pagecor242}{cl\-Cor\-Mo\-Id\-To\-Instance\-Get}, 
\hyperlink{pagecor243}{cl\-Cor\-Mo\-Id\-To\-Mo\-Class\-Path\-Get}, 
\hyperlink{pagecor244}{cl\-Cor\-Mo\-Id\-Clone}, 
\hyperlink{pagecor245}{cl\-Cor\-Mo\-Id\-Compare} \end{Desc}
\newpage


\subsection{clCorMoIdToInstanceGet}
\index{clCorMoIdToInstanceGet@{clCorMoIdToInstanceGet}}
\hypertarget{pagecor242}{}\paragraph{cl\-Cor\-Mo\-Id\-To\-Instance\-Get}\label{pagecor242}
\begin{Desc}
\item[Synopsis:]Returns the instance.\end{Desc}
\begin{Desc}
\item[Header File:]clCorUtilityApi.h\end{Desc}
\begin{Desc}
\item[Syntax:]

\footnotesize\begin{verbatim}    ClCorInstanceIdT clCorMoIdToInstanceGet(
                          CL_IN ClCorMOIdPtrT pMoId);
\end{verbatim}
\normalsize
\end{Desc}
\begin{Desc}
\item[Parameters:]
\begin{description}
\item[{\em p\-Mo\-Id:}](in) Handle of the Mo\-Id.\end{description}
\end{Desc}
\begin{Desc}
\item[Return values:]{\em Cl\-Cor\-Instance\-Id\-T\/}, associated Instance ID.\end{Desc}
\begin{Desc}
\item[Description:]This function is used to return the instance that is being queried. It refers to the class type and instance ID at the bottom of the hierarchy.\end{Desc}
\begin{Desc}
\item[Library File:]Cl\-Cor\-Client\end{Desc}
\begin{Desc}
\item[Related Function(s):]\hyperlink{pagecor101}{cl\-Cor\-Mo\-Id\-Initialize} , \hyperlink{pagecor229}{cl\-Cor\-Mo\-Id\-Alloc}, 
\hyperlink{pagecor230}{cl\-Cor\-Mo\-Id\-Free},
\hyperlink{pagecor231}{cl\-Cor\-Mo\-Id\-Truncate}, 
\hyperlink{pagecor232}{cl\-Cor\-Mo\-Id\-Set}, 
\hyperlink{pagecor233}{cl\-Cor\-Mo\-Id\-Append}, 
\hyperlink{pagecor235}{cl\-Cor\-Mo\-Id\-Depth\-Get}, 
\hyperlink{pagecor236}{cl\-Cor\-Mo\-Id\-Show}, 
\hyperlink{pagecor237}{cl\-Cor\-Mo\-Id\-To\-Mo\-Class\-Get}, 
\hyperlink{pagecor273}{cl\-Cor\-Mo\-Id\-Name\-To\-Mo\-Id\-Get}, 
\hyperlink{pagecor274}{cl\-Cor\-Mo\-Id\-To\-Mo\-Id\-Name\-Get}, 
\hyperlink{pagecor238}{cl\-Cor\-Mo\-Id\-First\-Instance\-Get},
\hyperlink{pagecor249}{cl\-Cor\-Mo\-Id\-Next\-Sibling\-Get}, 
\hyperlink{pagecor240}{cl\-Cor\-Mo\-Id\-Validate}, 
\hyperlink{pagecor242}{cl\-Cor\-Mo\-Id\-To\-Instance\-Get}, 
\hyperlink{pagecor243}{cl\-Cor\-Mo\-Id\-To\-Mo\-Class\-Path\-Get}, 
\hyperlink{pagecor244}{cl\-Cor\-Mo\-Id\-Clone}, 
\hyperlink{pagecor245}{cl\-Cor\-Mo\-Id\-Compare}\end{Desc}
\newpage


\subsection{clCorMoIdToMoClassPathGet}
\index{clCorMoIdToMoClassPathGet@{clCorMoIdToMoClassPathGet}}
\hypertarget{pagecor243}{}\paragraph{cl\-Cor\-Mo\-Id\-To\-Mo\-Class\-Path\-Get}\label{pagecor243}
\begin{Desc}
\item[Synopsis:]Derives the COR path from given a {\em Mo\-Id\/}.\end{Desc}
\begin{Desc}
\item[Header File:]clCorUtilityApi.h\end{Desc}
\begin{Desc}
\item[Syntax:]

\footnotesize\begin{verbatim}   ClRcT clCorMoIdToMoClassPathGet(
                 CL_IN ClCorMOIdPtrT moIdh,
                 CL_OUT ClCorMOClassPathPtrT corIdh);
\end{verbatim}
\normalsize
\end{Desc}
\begin{Desc}
\item[Parameters:]
\begin{description}
\item[{\em mo\-Idh:}](in) Handle of the {\em Mo\-Id\/}. \item[{\em cor\-Idh:}](out) Handle of COR path which is updated.\end{description}
\end{Desc}
\begin{Desc}
\item[Return values:]
\begin{description}
\item[{\em CL\_\-OK:}]The function executed successfully.\end{description}
\end{Desc}
\begin{Desc}
\item[Description:]This function is used to update the {\em COR\/} path based on the {\em Mo\-Id\/}. Both {\em Mo\-Id\/} and COR path must be allocated by you.\end{Desc}
\begin{Desc}
\item[Library File:]Cl\-Cor\-Client\end{Desc}
\begin{Desc}
\item[Related Function(s):]\hyperlink{pagecor101}{cl\-Cor\-Mo\-Id\-Initialize} , \hyperlink{pagecor229}{cl\-Cor\-Mo\-Id\-Alloc}, 
\hyperlink{pagecor230}{cl\-Cor\-Mo\-Id\-Free},
\hyperlink{pagecor231}{cl\-Cor\-Mo\-Id\-Truncate}, 
\hyperlink{pagecor232}{cl\-Cor\-Mo\-Id\-Set}, 
\hyperlink{pagecor233}{cl\-Cor\-Mo\-Id\-Append}, 
\hyperlink{pagecor235}{cl\-Cor\-Mo\-Id\-Depth\-Get}, 
\hyperlink{pagecor236}{cl\-Cor\-Mo\-Id\-Show}, 
\hyperlink{pagecor237}{cl\-Cor\-Mo\-Id\-To\-Mo\-Class\-Get}, 
\hyperlink{pagecor273}{cl\-Cor\-Mo\-Id\-Name\-To\-Mo\-Id\-Get}, 
\hyperlink{pagecor274}{cl\-Cor\-Mo\-Id\-To\-Mo\-Id\-Name\-Get}, 
\hyperlink{pagecor238}{cl\-Cor\-Mo\-Id\-First\-Instance\-Get},
\hyperlink{pagecor249}{cl\-Cor\-Mo\-Id\-Next\-Sibling\-Get}, 
\hyperlink{pagecor240}{cl\-Cor\-Mo\-Id\-Validate}, 
\hyperlink{pagecor242}{cl\-Cor\-Mo\-Id\-To\-Instance\-Get}, 
\hyperlink{pagecor243}{cl\-Cor\-Mo\-Id\-To\-Mo\-Class\-Path\-Get}, 
\hyperlink{pagecor244}{cl\-Cor\-Mo\-Id\-Clone}, 
\hyperlink{pagecor245}{cl\-Cor\-Mo\-Id\-Compare}\end{Desc}
\newpage


\subsection{clCorMoIdClone}
\index{clCorMoIdClone@{clCorMoIdClone}}
\hypertarget{pagecor244}{}\paragraph{cl\-Cor\-Mo\-Id\-Clone}\label{pagecor244}
\begin{Desc}
\item[Synopsis:]Clones a particular {\em MOId\/}.\end{Desc}
\begin{Desc}
\item[Header File:]clCorUtilityApi.h\end{Desc}
\begin{Desc}
\item[Syntax:]

\footnotesize\begin{verbatim}   ClRcT clCorMoIdClone(
                         CL_IN ClCorMOIdPtrT pMoId,
                         CL_OUT ClCorMOIdPtrT* newH);
\end{verbatim}
\normalsize
\end{Desc}
\begin{Desc}
\item[Parameters:]
\begin{description}
\item[{\em p\-Mo\-Id:}](in) Handle of the {\em Mo\-Id\/}. \item[{\em new\-H:}](out) Handle of the new clone.\end{description}
\end{Desc}
\begin{Desc}
\item[Return values:]
\begin{description}
\item[{\em CL\_\-OK:}]The function executed successfully. \item[{\em CL\_\-COR\_\-ERR\_\-NO\_\-MEM:}]On memory allocation failure.\end{description}
\end{Desc}
\begin{Desc}
\item[Description:]This function is used to clone a particular {\em Mo\-Id\/}. It allocates and makes a copy of the contents of the given {\em Mo\-Id\/} to a new {\em Mo\-Id\/}.\end{Desc}
\begin{Desc}
\item[Library File:]Cl\-Cor\-Client\end{Desc}
\begin{Desc}
\item[Related Function(s):]\hyperlink{pagecor101}{cl\-Cor\-Mo\-Id\-Initialize} , \hyperlink{pagecor229}{cl\-Cor\-Mo\-Id\-Alloc}, 
\hyperlink{pagecor230}{cl\-Cor\-Mo\-Id\-Free},
\hyperlink{pagecor231}{cl\-Cor\-Mo\-Id\-Truncate}, 
\hyperlink{pagecor232}{cl\-Cor\-Mo\-Id\-Set}, 
\hyperlink{pagecor233}{cl\-Cor\-Mo\-Id\-Append}, 
\hyperlink{pagecor235}{cl\-Cor\-Mo\-Id\-Depth\-Get}, 
\hyperlink{pagecor236}{cl\-Cor\-Mo\-Id\-Show}, 
\hyperlink{pagecor237}{cl\-Cor\-Mo\-Id\-To\-Mo\-Class\-Get}, 
\hyperlink{pagecor273}{cl\-Cor\-Mo\-Id\-Name\-To\-Mo\-Id\-Get}, 
\hyperlink{pagecor274}{cl\-Cor\-Mo\-Id\-To\-Mo\-Id\-Name\-Get}, 
\hyperlink{pagecor238}{cl\-Cor\-Mo\-Id\-First\-Instance\-Get},
\hyperlink{pagecor249}{cl\-Cor\-Mo\-Id\-Next\-Sibling\-Get}, 
\hyperlink{pagecor240}{cl\-Cor\-Mo\-Id\-Validate}, 
\hyperlink{pagecor242}{cl\-Cor\-Mo\-Id\-To\-Instance\-Get}, 
\hyperlink{pagecor243}{cl\-Cor\-Mo\-Id\-To\-Mo\-Class\-Path\-Get}, 
\hyperlink{pagecor244}{cl\-Cor\-Mo\-Id\-Clone}, 
\hyperlink{pagecor245}{cl\-Cor\-Mo\-Id\-Compare} \end{Desc}
\newpage



\subsection{clCorMoIdCompare}
\index{clCorMoIdCompare@{clCorMoIdCompare}}
\hypertarget{pagecor245}{}\paragraph{cl\-Cor\-Mo\-Id\-Compare}\label{pagecor245}
\begin{Desc}
\item[Synopsis:]Compares two {\em Mo\-Ids\/} and verifies whether they are equal.\end{Desc}
\begin{Desc}
\item[Header File:]clCorUtilityApi.h\end{Desc}
\begin{Desc}
\item[Syntax:]

\footnotesize\begin{verbatim}   ClInt32T clCorMoIdCompare(
                       CL_IN ClCorMOIdPtrT pMoId,
                       CL_IN ClCorMOIdPtrT cmp);
\end{verbatim}
\normalsize
\end{Desc}
\begin{Desc}
\item[Parameters:]
\begin{description}
\item[{\em p\-Mo\-Id:}](in) Handle of the first {\em \hyperlink{struct_cl_cor_m_o_id}{Cl\-Cor\-MOId}\/}. \item[{\em cmp:}](in) Handle of the second \textit{ClCorMOId}.\end{description}
\end{Desc}
\begin{Desc}
\item[Return values:]
\begin{description}
\item[{\em 0:}]If both COR Mo\-Ids are same, even if they have the same instances, like wildcards. \item[{\em -1:}]If the {\em Mo\-Ids\/} mismatch. \item[{\em 1:}]If wildcards are same.\end{description}
\end{Desc}
\begin{Desc}
\item[Description:]This function is used to make the comparison between two \textit{ClCorMOId} and verifies whether they are same. This comparison goes through till internals of both the {\em Mo\-Ids\/}.\end{Desc}
\begin{Desc}
\item[Library File:]Cl\-Cor\-Client\end{Desc}
\begin{Desc}
\item[Related Function(s):]\hyperlink{pagecor101}{cl\-Cor\-Mo\-Id\-Initialize} , \hyperlink{pagecor229}{cl\-Cor\-Mo\-Id\-Alloc}, 
\hyperlink{pagecor230}{cl\-Cor\-Mo\-Id\-Free},
\hyperlink{pagecor231}{cl\-Cor\-Mo\-Id\-Truncate}, 
\hyperlink{pagecor232}{cl\-Cor\-Mo\-Id\-Set}, 
\hyperlink{pagecor233}{cl\-Cor\-Mo\-Id\-Append}, 
\hyperlink{pagecor235}{cl\-Cor\-Mo\-Id\-Depth\-Get}, 
\hyperlink{pagecor236}{cl\-Cor\-Mo\-Id\-Show}, 
\hyperlink{pagecor237}{cl\-Cor\-Mo\-Id\-To\-Mo\-Class\-Get}, 
\hyperlink{pagecor273}{cl\-Cor\-Mo\-Id\-Name\-To\-Mo\-Id\-Get}, 
\hyperlink{pagecor274}{cl\-Cor\-Mo\-Id\-To\-Mo\-Id\-Name\-Get}, 
\hyperlink{pagecor238}{cl\-Cor\-Mo\-Id\-First\-Instance\-Get},
\hyperlink{pagecor249}{cl\-Cor\-Mo\-Id\-Next\-Sibling\-Get}, 
\hyperlink{pagecor240}{cl\-Cor\-Mo\-Id\-Validate}, 
\hyperlink{pagecor242}{cl\-Cor\-Mo\-Id\-To\-Instance\-Get}, 
\hyperlink{pagecor243}{cl\-Cor\-Mo\-Id\-To\-Mo\-Class\-Path\-Get}, 
\hyperlink{pagecor244}{cl\-Cor\-Mo\-Id\-Clone}, 
\hyperlink{pagecor245}{cl\-Cor\-Mo\-Id\-Compare} \end{Desc}
\newpage

\subsection{clCorMoIdServiceGet}
\index{clCorMoIdServiceGet@{clCorMoIdServiceGet}}
\hypertarget{pagecor246}{}\paragraph{cl\-Cor\-Mo\-Id\-Service\-Get}\label{pagecor246}
\begin{Desc}
\item[Synopsis:]Returns the service ID.\end{Desc}
\begin{Desc}
\item[Header File:]clCorUtilityApi.h\end{Desc}
\begin{Desc}
\item[Syntax:]

\footnotesize\begin{verbatim}   ClCorMOServiceIdT clCorMoIdServiceGet(
                   CL_IN ClCorMOIdPtrT pMoId);
\end{verbatim}
\normalsize
\end{Desc}
\begin{Desc}
\item[Parameters:]
\begin{description}
\item[{\em p\-Mo\-Id:}](in) Handle of the {\em Mo\-Id\/}.\end{description}
\end{Desc}
\begin{Desc}
\item[Return Values:]It returns the service ID through {\em Cl\-Cor\-MOService\-Id\-T\/}.\end{Desc}
\begin{Desc}
\item[Description:]This function is used to return the service ID associated with the COR {\em Mo\-Id\/}.\end{Desc}
\begin{Desc}
\item[Library File:]Cl\-Cor\-Client\end{Desc}
\begin{Desc}
\item[Related Function(s):]\hyperlink{pagecor246}{cl\-Cor\-Mo\-Id\-Service\-Set}, \hyperlink{pagecor248}{cl\-Cor\-Service\-Id\-Validate} \end{Desc}
\newpage


\subsection{clCorMoIdServiceSet}
\index{clCorMoIdServiceSet@{clCorMoIdServiceSet}}
\hypertarget{pagecor247}{}\paragraph{cl\-Cor\-Mo\-Id\-Service\-Set}\label{pagecor247}
\begin{Desc}
\item[Synopsis:]Sets the service ID.\end{Desc}
\begin{Desc}
\item[Header File:]clCorUtilityApi.h\end{Desc}
\begin{Desc}
\item[Syntax:]

\footnotesize\begin{verbatim}   ClRcT clCorMoIdServiceSet(
                         CL_INOUT ClCorMOIdPtrT pMoId,
                         CL_IN ClCorMOServiceIdT svc);
\end{verbatim}
\normalsize
\end{Desc}
\begin{Desc}
\item[Parameters:]
\begin{description}
\item[{\em p\-Mo\-Id:}](in/out) Handle of the Mo\-Id. \item[{\em svc:}](in) Service ID.\end{description}
\end{Desc}
\begin{Desc}
\item[Return values:]
\begin{description}
\item[{\em CL\_\-OK:}]The function executed successfully.\end{description}
\end{Desc}
\begin{Desc}
\item[Description:]This function is used to set the service ID for a particular COR {\em Mo\-Id\/}.\end{Desc}
\begin{Desc}
\item[Library File:]Cl\-Cor\-Client\end{Desc}
\begin{Desc}
\item[Related Function(s):]\hyperlink{pagecor246}{cl\-Cor\-Mo\-Id\-Service\-Get}, \hyperlink{pagecor248}{cl\-Cor\-Service\-Id\-Validate} \end{Desc}
\newpage


\subsection{clCorMoIdInstanceSet}
\index{clCorMoIdInstanceSet@{clCorMoIdInstanceSet}}
\hypertarget{pagecor249}{}\paragraph{cl\-Cor\-Mo\-Id\-Instance\-Set}\label{pagecor249}
\begin{Desc}
\item[Synopsis:]Sets the Instance.\end{Desc}
\begin{Desc}
\item[Header File:]clCorUtilityApi.h\end{Desc}
\begin{Desc}
\item[Syntax:]

\footnotesize\begin{verbatim}   ClRcT clCorMoIdInstanceSet(
                         CL_INOUT ClCorMOIdPtrT pMoId,
                         CL_IN ClUint16T ndepth,
                         CL_IN ClCorInstanceIdT newInstance);
\end{verbatim}
\normalsize
\end{Desc}
\begin{Desc}
\item[Parameters:]
\begin{description}
\item[{\em p\-Mo\-Id:}](in/out) Handle of the {\em MOId\/}. \item[{\em ndepth:}](in) Depth at which the instance is to be set. 
\item[{\em new\-Instance:}](in) The instance to be set.\end{description}
\end{Desc}
\begin{Desc}
\item[Return values:]
\begin{description}
\item[{\em CL\_\-OK:}]The function executed successfully.\end{description}
\end{Desc}
\begin{Desc}
\item[Description:]This function is used to set the instance field at a specified level of {\em Mo\-Id\/}.\end{Desc}
\begin{Desc}
\item[Library File:]Cl\-Cor\-Client\end{Desc}
\begin{Desc}
\item[Related Function(s):]\hyperlink{pagecor238}{cl\-Cor\-Mo\-Id\-First\-Instance\-Get} \end{Desc}
\newpage


\subsection{clCorMoIdConcatenate}
\index{clCorMoIdConcatenate@{clCorMoIdConcatenate}}
\hypertarget{pagecor252}{}\paragraph{cl\-Cor\-Mo\-Id\-Concatenate}\label{pagecor252}
\begin{Desc}
\item[Synopsis:]Concatenates an {\em Mo\-Id\/} to another.\end{Desc}
\begin{Desc}
\item[Header File:]clCorUtilityApi.h\end{Desc}
\begin{Desc}
\item[Syntax:]

\footnotesize\begin{verbatim}   ClRcT clCorMoIdConcatenate(
                         CL_INOUT ClCorMOIdPtrT part1,
                         CL_INOUT ClCorMOIdPtrT part2,
                         CL_IN ClInt32T copyWhere);
\end{verbatim}
\normalsize
\end{Desc}
\begin{Desc}
\item[Parameters:]
\begin{description}
\item[{\em part1:}](in/out) Upper part of the {\em Mo\-Id\/}. \item[{\em part2:}]( in/out)Lower part of the {\em Mo\-Id\/}. \item[{\em copy\-Where:}](in) Indicates where concatenated {\em Mo\-Id\/} to be copied. \begin{itemize}
\item If 0, {\em part1\/} is modified by suffixing {\em part2\/} to it. \item If 1, {\em part2\/} is modified by pre-pending {\em part1\/} to it.\end{itemize}
\end{description}
\end{Desc}
\begin{Desc}
\item[Return values:]
\begin{description}
\item[{\em CL\_\-OK:}]The function executed successfully. \item[{\em CL\_\-COR\_\-ERR\_\-MAX\_\-DEPTH:}]If depth exceeded the maximum limit.\end{description}
\end{Desc}
\begin{Desc}
\item[Description:]This function is used to concatenate two {\em Mo\-Ids\/}.\end{Desc}
\begin{Desc}
\item[Library File:]Cl\-Cor\-Client\end{Desc}
\begin{Desc}
\item[Related Function(s):]\hyperlink{pagecor233}{cl\-Cor\-Mo\-Id\-Append}, \hyperlink{pagecor231}{cl\-Cor\-Mo\-Id\-Truncate} \end{Desc}
\newpage




\subsection{clCorEventSubscribe}
\index{clCorEventSubscribe@{clCorEventSubscribe}}
\hypertarget{pagecor301}{}\paragraph{cl\-Cor\-Event\-Subscribe}\label{pagecor301}
\begin{Desc}
\item[Synopsis:]Subscribes for the attribute change notification.\end{Desc}
\begin{Desc}
\item[Header File:]clCorNotifyApi.h\end{Desc}
\begin{Desc}
\item[Syntax:]

\footnotesize\begin{verbatim}        ClRcT clCorEventSubscribe(
                                 CL_IN ClEvtChannelHandleT channelHandle,
                                 CL_IN ClCorMOIdPtr changedObj,
                                 CL_IN ClCorAttrPathPtrT  pAttrPath,
                                 CL_IN ClCorAttrListPtr attrList,
                                 CL_IN ClCorOpsT    flags,
                                 CL_IN void * cookie,
                                 CL_IN ClEvtSubscriptionIdT subscriptionId);
\end{verbatim}
\normalsize
\end{Desc}
\begin{Desc}
\item[Parameters:]
\begin{description}
\item[{\em channel\-Handle:}](in) Handle of the COR channel, allocated by you.\item[{\em changed\-Obj:}](in) Subscriber (Caller) is interested in changes 
to this.
object (refer to as publisher). This is full path to the object. Wildcards can be used to specify a 'class' or 'subtree' of objects. Refer to
{\em MOId\/} documentation for further details.\item[{\em p\-Attr\-Path:}](in) In order to subscribe for contained attributes of contained objects an 
{\em attr\-Path\/} which identifies a contained object must be specified. If this path is specified as NULL, it means that the attributes specified in 
attr\-List belong to the object specified by changed\-Obj parameter. The {\em attr\-Path\/} can have instances as wildcards.\item[{\em attr\-List:}](in)
If the subscriber is interested in notification only if certain attribute(s) of the object change, list of these attribute Ids can be specified here. 
NULL can be passed to indicate an attribute Id wildcard, in which case the subscriber gets notified on change to any attribute. \par
 Following are the critical usage restrictions for this parameter:\begin{itemize}
\item This parameter is interpreted only if \#ops contains one or more {\tt \_\-SET\_\-} operations. Refer to \textit{ClCorOpsT} for all the possible operations types.\item For attribute Ids in {\em attr\-List\/} to make sense, {\em changed\-Obj\/} and {\em svc\-Id\/} combination have to resolve in exactly one object class. This means if you are interested at attribute level change:\begin{enumerate}
\item You can not use wildcard for the service ID.\item You can not use wildcard for any class value in the {\em changed\-Obj\/}.\item It is ok to use 
wildcard for instance value in {\em changed\-Obj\/}.\end{enumerate}
\end{itemize}
\item[{\em flags:}](in) The operation for which you can subscribe. The operation can be:\begin{enumerate}
\item {\tt CREATE} \item {\tt DELETE} \item {\tt SET} \end{enumerate}
\item[{\em cookie:}](in) Subscriber can store an arbitrary cookie value. It is returned as it is to the change notification callback.
\item[{\em subscription\-Id:}](in) Subscription Id is provided by caller of API. Same subscription Id has to be used while unsubscribing.\end{description}
\end{Desc}
\begin{Desc}
\item[Return values:]\end{Desc}
\begin{Desc}
\item[Return values:]
\begin{description}
\item[{\em CL\_\-OK:}]The API is successfully executed. \par
 Value returned by {\em cl\-Event\-Subscribe\/}, on failure. \item[{\em CL\_\-COR\_\-ERR\_\-NULL\_\-PTR:}]If the {\em changed\-Obj\/} passed is null. \item[{\em CL\_\-COR\_\-NOTIFY\_\-ERR\_\-INVALID\_\-OP:}]If the value passed is not one of these values: {\em CL\_\-COR\_\-OP\_\-CREATE\/}, {\em CL\_\-COR\_\-OP\_\-SET\/}, or {\em CL\_\-COR\_\-OP\_\-DELETE\/}. \item[{\em CL\_\-EVENT\_\-ERR\_\-BAD\_\-HANDLE:}]When the {\em channel\-Handle\/} passed is zero.\end{description}
\end{Desc}
\begin{Desc}
\item[Description:]Using this function user can also subscribe to any of the operations such as Attribute Set, Object Create, and Object Delete.\end{Desc}
\begin{Desc}
\item[Library File:]Cl\-Cor\-Client\end{Desc}
\begin{Desc}
\item[Related Function(s):]\hyperlink{group__group13}{cl\-Cor\-Event\-Unsubscribe} \end{Desc}
\newpage


\subsection{clCorEventUnsubscribe}
\index{clCorEventUnsubscribe@{clCorEventUnsubscribe}}
\hypertarget{pagecor302}{}\paragraph{cl\-Cor\-Event\-Unsubscribe}\label{pagecor302}
\begin{Desc}
\item[Synopsis:]Unsubscribes for attribute change notification.\end{Desc}
\begin{Desc}
\item[Header File:]clCorNotifyApi.h\end{Desc}
\begin{Desc}
\item[Syntax:]

\footnotesize\begin{verbatim}       ClRcT clCorEventUnsubscribe(
                                CL_IN ClEvtChannelHandleT channelHandle,
                                CL_IN ClEvtSubscriptionIdT subscriptionId);
\end{verbatim}
\normalsize
\end{Desc}
\begin{Desc}
\item[Parameters:]
\begin{description}
\item[{\em channel\-Handle:}](in) Channel handle obtained when the COR channel was opened by the application. \item[{\em subscription\-Id:}](in) Subscription ID, which was passed to COR while Subscribing.\end{description}
\end{Desc}
\begin{Desc}
\item[Return values:]\end{Desc}
\begin{Desc}
\item[Return values:]
\begin{description}
\item[{\em CL\_\-OK:}]The API is successfully executed. \par
 Value returned by {\em cl\-Event\-Un\-Subscribe\/}, on failure. \item[{\em CL\_\-EVENT\_\-ERR\_\-INIT\_\-NOT\_\-DONE:}]Event library has not been initialized. \item[{\em CL\_\-EVENT\_\-ERR\_\-BAD\_\-HANDLE:}]On passing an invalid handle. \item[{\em CL\_\-EVENT\_\-INTERNAL\_\-ERROR:}]An unexpected problem occurred within the Event Manager. \item[{\em CL\_\-EVENT\_\-ERR\_\-INVALID\_\-PARAM:}]On passing an invalid parameter. \item[{\em CL\_\-EVENT\_\-ERR\_\-NO\_\-MEM:}]On memory allocation failure.\end{description}
\end{Desc}
\begin{Desc}
\item[Description:]Using this function you can also unsubscribe from any of the operations such as Attribute Set, Object Create and Object Delete.\end{Desc}
\begin{Desc}
\item[Library File:]Cl\-Cor\-Client\end{Desc}
\begin{Desc}
\item[Related Function(s):]\hyperlink{group__group13}{cl\-Cor\-Event\-Subscribe} \end{Desc}
\newpage










\subsection{clCorServiceRuleAdd}
\index{clCorServiceRuleAdd@{clCorServiceRuleAdd}}
\hypertarget{pagecor135}{}\paragraph{cl\-Cor\-Service\-Rule\-Add}\label{pagecor135}
\begin{Desc}
\item[Synopsis:]Adds a new route rule entry.\end{Desc}
\begin{Desc}
\item[Header File:]clCorApi.h\end{Desc}
\begin{Desc}
\item[Syntax:]

\footnotesize\begin{verbatim}        ClRcT clCorServiceRuleAdd(
                            CL_IN ClCorMOIdPtrT moh,
                            CL_IN ClCorAddrT addr);
\end{verbatim}
\normalsize
\end{Desc}
\begin{Desc}
\item[Parameters:]
\begin{description}
\item[{\em CL\_\-IN}]moh:(in)  Handle of \hyperlink{struct_cl_cor_m_o_id}{Cl\-Cor\-MOId}. \item[{\em CL\_\-IN}]addr:(in)  COR Address. It contains card ID and EO ID.\end{description}
\end{Desc}
\begin{Desc}
\item[Return values:]
\begin{description}
\item[{\em CL\_\-OK:}]The function is executed successfully. \item[{\em CL\_\-COR\_\-ERR\_\-NO\_\-MEM:}]On memory allocation failure. \item[{\em CL\_\-COR\_\-ERR\_\-VERSION\_\-UNSUPPORTED:}]version is not supported. \item[{\em CL\_\-COR\_\-ERR\_\-NULL\_\-PTR:}]On passing a NULL pointer. \item[{\em CL\_\-COR\_\-ERR\_\-MAX\_\-DEPTH:}]Routes for a given Mo\-Id has reached the maximum limit.\end{description}
\end{Desc}
\begin{Desc}
\item[Description:]This function is used to add a new rule. This rule table overrides all the entries and other routing information present in the COR. The route rule addition corresponds to adding {\em mo\-Id\/} and station list in the route table. By default all the routes are enabled.\end{Desc}
\begin{Desc}
\item[Library File:]Cl\-Cor\-Client\end{Desc}
\begin{Desc}
\item[Note:]For one {\em mo\-Id\/} there can be many stations $<$ioc address, port address$>$ that can be added in the route list. The debug cli rm\-Show will show all the routelists added.\end{Desc}
\begin{Desc}
\item[Related Function(s):]\hyperlink{pagecor135}{cl\-Cor\-Service\-Rule\-Delete}\end{Desc}

\newpage

\subsection{clCorServiceRuleDelete}
\index{clCorServiceRuleDelete@{clCorServiceRuleDelete}}
\hypertarget{pagecor135}{}\paragraph{cl\-Cor\-Service\-Rule\-Delete}\label{pagecor135}
\begin{Desc}
\item[Synopsis:]Delete the station from the route list.\end{Desc}
\begin{Desc}
\item[Header File:]clCorApi.h\end{Desc}
\begin{Desc}
\item[Library Name:]Cl\-Cor\-Client\end{Desc}
\begin{Desc}
\item[Syntax:]

\footnotesize\begin{verbatim}        extern ClRcT clCorServiceRuleDelete(
                            CL_IN ClCorMOIdPtrT moh,
                            CL_IN ClCorAddrT addr);
\end{verbatim}
\normalsize
\end{Desc}
\begin{Desc}
\item[Parameters:]
\begin{description}
\item[{\em CL\_\-IN}]moh:(in)  Handle of \textit{ClCorMOId}. \item[{\em CL\_\-IN}]addr:(in)  COR Address. It contains slot id and EO id.\end{description}
\end{Desc}
\begin{Desc}
\item[Return values:]
\begin{description}
\item[{\em CL\_\-OK:}]The API is successfully executed. \item[{\em CL\_\-COR\_\-ERR\_\-NO\_\-MEM:}]Failed to allocate memory. \item[{\em CL\_\-COR\_\-ERR\_\-VERSION\_\-UNSUPPORTED:}]version is not supported. \item[{\em CL\_\-COR\_\-ERR\_\-NULL\_\-PTR:}]Mo\-Id pointer passed is NULL. \item[{\em CL\_\-COR\_\-ERR\_\-ROUTE\_\-NOT\_\-PRESENT:}]No Route list present for the given MOId.\end{description}
\end{Desc}
\begin{Desc}
\item[Description:]This API is used to delete the station address\mbox{[}addr\mbox{]} from the route list for a given MOId \mbox{[}moh\mbox{]}. The station address is added by the component in the route list to participate in a transaction incase of modification of the mo\-Id. Whenever a component goes down gracefully, it need to clean up its entry in the route list. For this the component needs to call this api in its EO finalize callback function, once this station address is deleted.\end{Desc}
\begin{Desc}
\item[Related Function(s):]\hyperlink{pagecor135}{cl\-Cor\-Service\-Rule\-Add} \end{Desc}
\newpage




\subsection{clCorServiceIdValidate}
\index{clCorServiceIdValidate@{clCorServiceIdValidate}}
\hypertarget{pagecor248}{}\paragraph{cl\-Cor\-Service\-Id\-Validate}\label{pagecor248}
\begin{Desc}
\item[Synopsis:]Validates {\em service\-Id\/} in the input argument.\end{Desc}
\begin{Desc}
\item[Header File:]clCorUtilityApi.h\end{Desc}
\begin{Desc}
\item[Syntax:]

\footnotesize\begin{verbatim}   ClRcT clCorServiceIdValidate(
                         CL_IN ClCorServiceIdT srvcId);
\end{verbatim}
\normalsize
\end{Desc}
\begin{Desc}
\item[Parameters:]
\begin{description}
\item[{\em srvc\-Id:}](in) Argument passed in {\em service\-Id\/}.\end{description}
\end{Desc}
\begin{Desc}
\item[Return values:]
\begin{description}
\item[{\em CL\_\-OK:}]The function executed successfully.\end{description}
\end{Desc}
\begin{Desc}
\item[Description:]This function is used to validate the {\em service\-Id\/} passed in the input argument.\end{Desc}
\begin{Desc}
\item[Library File:]Cl\-Cor\-Client\end{Desc}
\begin{Desc}
\item[Related Function(s):]\hyperlink{group__group13}{cl\-Cor\-Mo\-Id\-Service\-Get}, \hyperlink{group__group13}{cl\-Cor\-Mo\-Id\-Service\-Set} \end{Desc}
\newpage



\subsection{clCorAttrPathAlloc}
\index{clCorAttrPathAlloc@{clCorAttrPathAlloc}}
\hypertarget{pagecor255}{}\paragraph{cl\-Cor\-Attr\-Path\-Alloc}\label{pagecor255}
\begin{Desc}
\item[Synopsis:]Creates an attribute path.\end{Desc}
\begin{Desc}
\item[Header File:]clCorUtilityApi.h\end{Desc}
\begin{Desc}
\item[Syntax:]

\footnotesize\begin{verbatim}   ClRcT clCorAttrPathAlloc(
                         CL_INOUT ClCorAttrPathPtrT *pAttrPath);
\end{verbatim}
\normalsize
\end{Desc}
\begin{Desc}
\item[Parameters:]
\begin{description}
\item[{\em p\-Attr\-Path:}](in/out) Handle of the new attribute path.\end{description}
\end{Desc}
\begin{Desc}
\item[Return values:]
\begin{description}
\item[{\em CL\_\-OK:}]The function executed successfully. \item[{\em CL\_\-COR\_\-ERR\_\-NO\_\-MEM:}]On memory allocation failure.\end{description}
\end{Desc}
\begin{Desc}
\item[Description:]This function is used as a constructor for \textit{ClCorAttrPath} as it creates an attribute path. It initializes the memory and returns an empty COR path. \par
 \par
 By default, the value for both {\em index\/} field and the attribute ID is {\em -1\/}. The default depth for the attribute path is 20. This value is incremented dynamically whenever a new entry is added.\end{Desc}
\begin{Desc}
\item[Library File:]Cl\-Cor\-Client\end{Desc}
\begin{Desc}
\item[Related Function(s):]\hyperlink{pagecor256}{cl\-Cor\-Attr\-Path\-Initialize}, \hyperlink{pagecor257}{cl\-Cor\-Attr\-Path\-Free}, 
\hyperlink{pagecor258}{cl\-Cor\-Attr\-Path\-Truncate}, \hyperlink{pagecor259}{cl\-Cor\-Attr\-Path\-Set},
\hyperlink{pagecor260}{cl\-Cor\-Attr\-Path\-Append}, \hyperlink{pagecor261}{cl\-Cor\-Attr\-Path\-Depth\-Get}, 
\hyperlink{pagecor262}{cl\-Cor\-Attr\-Path\-Show}, \hyperlink{pagecor263}{cl\-Cor\-Attr\-Path\-To\-Attr\-Id\-Get}, 
\hyperlink{pagecor264}{cl\-Cor\-Attr\-Path\-Index\-Get}, \hyperlink{pagecor265}{cl\-Cor\-Attr\-Path\-Index\-Set},
\hyperlink{pagecor266}{cl\-Cor\-Attr\-Path\-Compare}, \hyperlink{pagecor267}{cl\-Cor\-Attr\-Path\-Clone} \end{Desc}
\newpage


\subsection{clCorAttrPathInitialize}
\index{clCorAttrPathInitialize@{clCorAttrPathInitialize}}
\hypertarget{pagecor256}{}\paragraph{cl\-Cor\-Attr\-Path\-Initialize}\label{pagecor256}
\begin{Desc}
\item[Synopsis:]Initializes the attribute path.\end{Desc}
\begin{Desc}
\item[Header File:]clCorUtilityApi.h\end{Desc}
\begin{Desc}
\item[Syntax:]

\footnotesize\begin{verbatim}   ClRcT clCorAttrPathInitialize(
                         CL_INOUT ClCorAttrPathPtrT pAttrPath);
\end{verbatim}
\normalsize
\end{Desc}
\begin{Desc}
\item[Parameters:]
\begin{description}
\item[{\em p\-Attr\-Path:}](in/out) Handle of the attribute path.\end{description}
\end{Desc}
\begin{Desc}
\item[Return values:]
\begin{description}
\item[{\em CL\_\-OK:}]The function executed successfully.\end{description}
\end{Desc}
\begin{Desc}
\item[Description:]This function is used to initialize the COR attribute path. It resets the information of the path, if present, and re-initializes it as an empty path.\end{Desc}
\begin{Desc}
\item[Library File:]Cl\-Cor\-Client\end{Desc}
\begin{Desc}
\item[Related Function(s):]\hyperlink{pagecor256}{cl\-Cor\-Attr\-Path\-Initialize}, \hyperlink{pagecor257}{cl\-Cor\-Attr\-Path\-Free}, 
\hyperlink{pagecor258}{cl\-Cor\-Attr\-Path\-Truncate}, \hyperlink{pagecor259}{cl\-Cor\-Attr\-Path\-Set},
\hyperlink{pagecor260}{cl\-Cor\-Attr\-Path\-Append}, \hyperlink{pagecor261}{cl\-Cor\-Attr\-Path\-Depth\-Get}, 
\hyperlink{pagecor262}{cl\-Cor\-Attr\-Path\-Show}, \hyperlink{pagecor263}{cl\-Cor\-Attr\-Path\-To\-Attr\-Id\-Get}, 
\hyperlink{pagecor264}{cl\-Cor\-Attr\-Path\-Index\-Get}, \hyperlink{pagecor265}{cl\-Cor\-Attr\-Path\-Index\-Set},
\hyperlink{pagecor266}{cl\-Cor\-Attr\-Path\-Compare}, \hyperlink{pagecor267}{cl\-Cor\-Attr\-Path\-Clone}\end{Desc}
\newpage


\subsection{clCorAttrPathFree}
\index{clCorAttrPathFree@{clCorAttrPathFree}}
\hypertarget{pagecor257}{}\paragraph{cl\-Cor\-Attr\-Path\-Free}\label{pagecor257}
\begin{Desc}
\item[Synopsis:]Deletes the COR attribute path handle.\end{Desc}
\begin{Desc}
\item[Header File:]clCorUtilityApi.h\end{Desc}
\begin{Desc}
\item[Syntax:]

\footnotesize\begin{verbatim}   ClRcT clCorAttrPathFree(
                         CL_INOUT ClCorAttrPathPtrT  pAttrPath);
\end{verbatim}
\normalsize
\end{Desc}
\begin{Desc}
\item[Parameters:]
\begin{description}
\item[{\em p\-Attr\-Path:}](in/out) Handle of \textit{ClCorAttrPath}.\end{description}
\end{Desc}
\begin{Desc}
\item[Return values:]
\begin{description}
\item[{\em CL\_\-OK:}]The function executed successfully.\end{description}
\end{Desc}
\begin{Desc}
\item[Description:]This function is used as a destructor for \textit{ClCorAttrPath} as it deletes the handle of the COR attribute path. It removes the handle and frees the memory associated with it.\end{Desc}
\begin{Desc}
\item[Library File:]Cl\-Cor\-Client\end{Desc}
\begin{Desc}
\item[Related Function(s):]\hyperlink{pagecor256}{cl\-Cor\-Attr\-Path\-Initialize}, \hyperlink{pagecor257}{cl\-Cor\-Attr\-Path\-Free}, 
\hyperlink{pagecor258}{cl\-Cor\-Attr\-Path\-Truncate}, \hyperlink{pagecor259}{cl\-Cor\-Attr\-Path\-Set},
\hyperlink{pagecor260}{cl\-Cor\-Attr\-Path\-Append}, \hyperlink{pagecor261}{cl\-Cor\-Attr\-Path\-Depth\-Get}, 
\hyperlink{pagecor262}{cl\-Cor\-Attr\-Path\-Show}, \hyperlink{pagecor263}{cl\-Cor\-Attr\-Path\-To\-Attr\-Id\-Get}, 
\hyperlink{pagecor264}{cl\-Cor\-Attr\-Path\-Index\-Get}, \hyperlink{pagecor265}{cl\-Cor\-Attr\-Path\-Index\-Set},
\hyperlink{pagecor266}{cl\-Cor\-Attr\-Path\-Compare}, \hyperlink{pagecor267}{cl\-Cor\-Attr\-Path\-Clone}\end{Desc}
\newpage


\subsection{clCorAttrPathTruncate}
\index{clCorAttrPathTruncate@{clCorAttrPathTruncate}}
\hypertarget{pagecor258}{}\paragraph{cl\-Cor\-Attr\-Path\-Truncate}\label{pagecor258}
\begin{Desc}
\item[Synopsis:]Removes node after specified level.\end{Desc}
\begin{Desc}
\item[Header File:]clCorUtilityApi.h\end{Desc}
\begin{Desc}
\item[Syntax:]

\footnotesize\begin{verbatim}   ClRcT clCorAttrPathTruncate(
                         CL_INOUT ClCorAttrPathPtrT pAttrPath,
                         CL_IN ClInt16T level);
\end{verbatim}
\normalsize
\end{Desc}
\begin{Desc}
\item[Parameters:]
\begin{description}
\item[{\em p\-Attr\-Path:}](in/out) Handle of the {\em cl\-Corattr\-Path\-T\/}. \item[{\em level:}](in) Level to which {\em Cl\-Cor\-Attr\-Path\-T\/} needs to be truncated.\end{description}
\end{Desc}
\begin{Desc}
\item[Return values:]
\begin{description}
\item[{\em CL\_\-OK:}]The function executed successfully. \item[{\em CL\_\-COR\_\-ERR\_\-INVALID\_\-DEPTH:}]If the level specified is invalid.\end{description}
\end{Desc}
\begin{Desc}
\item[Description:]This function is used to remove all the nodes and reset the Mo\-Id until the specified level is reached.\end{Desc}
\begin{Desc}
\item[Library File:]Cl\-Cor\-Client\end{Desc}
\begin{Desc}
\item[Related Function(s):]\hyperlink{pagecor256}{cl\-Cor\-Attr\-Path\-Initialize}, \hyperlink{pagecor257}{cl\-Cor\-Attr\-Path\-Free}, 
\hyperlink{pagecor258}{cl\-Cor\-Attr\-Path\-Truncate}, \hyperlink{pagecor259}{cl\-Cor\-Attr\-Path\-Set},
\hyperlink{pagecor260}{cl\-Cor\-Attr\-Path\-Append}, \hyperlink{pagecor261}{cl\-Cor\-Attr\-Path\-Depth\-Get}, 
\hyperlink{pagecor262}{cl\-Cor\-Attr\-Path\-Show}, \hyperlink{pagecor263}{cl\-Cor\-Attr\-Path\-To\-Attr\-Id\-Get}, 
\hyperlink{pagecor264}{cl\-Cor\-Attr\-Path\-Index\-Get}, \hyperlink{pagecor265}{cl\-Cor\-Attr\-Path\-Index\-Set},
\hyperlink{pagecor266}{cl\-Cor\-Attr\-Path\-Compare}, \hyperlink{pagecor267}{cl\-Cor\-Attr\-Path\-Clone}\end{Desc}
\newpage


\subsection{clCorAttrPathSet}
\index{clCorAttrPathSet@{clCorAttrPathSet}}
\hypertarget{pagecor259}{}\paragraph{cl\-Cor\-Attr\-Path\-Set}\label{pagecor259}
\begin{Desc}
\item[Synopsis:]Sets the attribute ID for a given node.\end{Desc}
\begin{Desc}
\item[Header File:]clCorUtilityApi.h\end{Desc}
\begin{Desc}
\item[Syntax:]

\footnotesize\begin{verbatim}   ClRcT clCorAttrPathSet(
                         CL_INOUT ClCorAttrPathPtrT pAttrPath,
                         CL_IN ClUint16T level,
                         CL_IN ClCorAttrIdT attrId,
                         CL_IN ClUint32T  index);
\end{verbatim}
\normalsize
\end{Desc}
\begin{Desc}
\item[Parameters:]
\begin{description}
\item[{\em p\-Attr\-Path:}](in/out) Handle of the attribute path. \item[{\em level:}](in) Level of the node. \item[{\em attr\-Id:}](in) Attribute ID to be
set. \item[{\em index:}](in) Index to be set.\end{description}
\end{Desc}
\begin{Desc}
\item[Return values:]
\begin{description}
\item[{\em CL\_\-OK:}]The function executed successfully. \item[{\em CL\_\-COR\_\-ERR\_\-INVALID\_\-DEPTH:}]If the level specified is invalid.\end{description}
\end{Desc}
\begin{Desc}
\item[Description:]This function is used to set the attribute ID and the index at a specified node or level.\end{Desc}
\begin{Desc}
\item[Library File:]Cl\-Cor\-Client\end{Desc}
\begin{Desc}
\item[Related Function(s):]\hyperlink{pagecor256}{cl\-Cor\-Attr\-Path\-Initialize}, \hyperlink{pagecor257}{cl\-Cor\-Attr\-Path\-Free}, 
\hyperlink{pagecor258}{cl\-Cor\-Attr\-Path\-Truncate}, \hyperlink{pagecor259}{cl\-Cor\-Attr\-Path\-Set},
\hyperlink{pagecor260}{cl\-Cor\-Attr\-Path\-Append}, \hyperlink{pagecor261}{cl\-Cor\-Attr\-Path\-Depth\-Get}, 
\hyperlink{pagecor262}{cl\-Cor\-Attr\-Path\-Show}, \hyperlink{pagecor263}{cl\-Cor\-Attr\-Path\-To\-Attr\-Id\-Get}, 
\hyperlink{pagecor264}{cl\-Cor\-Attr\-Path\-Index\-Get}, \hyperlink{pagecor265}{cl\-Cor\-Attr\-Path\-Index\-Set},
\hyperlink{pagecor266}{cl\-Cor\-Attr\-Path\-Compare}, \hyperlink{pagecor267}{cl\-Cor\-Attr\-Path\-Clone}\end{Desc}
\newpage


\subsection{clCorAttrPathAppend}
\index{clCorAttrPathAppend@{clCorAttrPathAppend}}
\hypertarget{pagecor260}{}\paragraph{cl\-Cor\-Attr\-Path\-Append}\label{pagecor260}
\begin{Desc}
\item[Synopsis:]Adds an entry to the attribute path.\end{Desc}
\begin{Desc}
\item[Header File:]clCorUtilityApi.h\end{Desc}
\begin{Desc}
\item[Syntax:]

\footnotesize\begin{verbatim}   ClRcT clCorAttrPathAppend(
                         CL_INOUT ClCorAttrPathPtrT pAttrPath,
                         CL_IN ClCorClassTypeT attrId,
                         CL_IN ClCorInstanceIdT index);
\end{verbatim}
\normalsize
\end{Desc}
\begin{Desc}
\item[Parameters:]
\begin{description}
\item[{\em p\-Attr\-Path:}](in/out) Handle of the COR attribute path. \item[{\em attr\-Id:}](in) ID of the attribute. \item[{\em index:}](in) Index of the attribute.\end{description}
\end{Desc}
\begin{Desc}
\item[Return values:]
\begin{description}
\item[{\em CL\_\-OK:}]The function executed successfully. \item[{\em CL\_\-COR\_\-ERR\_\-MAX\_\-DEPTH:}]If depth exceeded the maximum limit.\end{description}
\end{Desc}
\begin{Desc}
\item[Description:]This function is used to add an entry to the attribute path. You must explicitly specify the {\em attr\-Id\/} and the {\em index\/} for the entry.\end{Desc}
\begin{Desc}
\item[Library File:]Cl\-Cor\-Client\end{Desc}
\begin{Desc}
\item[Related Function(s):]\hyperlink{pagecor256}{cl\-Cor\-Attr\-Path\-Initialize}, \hyperlink{pagecor257}{cl\-Cor\-Attr\-Path\-Free}, 
\hyperlink{pagecor258}{cl\-Cor\-Attr\-Path\-Truncate}, \hyperlink{pagecor259}{cl\-Cor\-Attr\-Path\-Set},
\hyperlink{pagecor260}{cl\-Cor\-Attr\-Path\-Append}, \hyperlink{pagecor261}{cl\-Cor\-Attr\-Path\-Depth\-Get}, 
\hyperlink{pagecor262}{cl\-Cor\-Attr\-Path\-Show}, \hyperlink{pagecor263}{cl\-Cor\-Attr\-Path\-To\-Attr\-Id\-Get}, 
\hyperlink{pagecor264}{cl\-Cor\-Attr\-Path\-Index\-Get}, \hyperlink{pagecor265}{cl\-Cor\-Attr\-Path\-Index\-Set},
\hyperlink{pagecor266}{cl\-Cor\-Attr\-Path\-Compare}, \hyperlink{pagecor267}{cl\-Cor\-Attr\-Path\-Clone}\end{Desc}
\newpage


\subsection{clCorAttrPathDepthGet}
\index{clCorAttrPathDepthGet@{clCorAttrPathDepthGet}}
\hypertarget{pagecor261}{}\paragraph{cl\-Cor\-Attr\-Path\-Depth\-Get}\label{pagecor261}
\begin{Desc}
\item[Synopsis:]Returns the COR attribute path node depth.\end{Desc}
\begin{Desc}
\item[Header File:]clCorUtilityApi.h\end{Desc}
\begin{Desc}
\item[Syntax:]

\footnotesize\begin{verbatim}   ClInt16T clCorAttrPathDepthGet(
                            CL_IN ClCorAttrPathPtrT pAttrPath);
\end{verbatim}
\normalsize
\end{Desc}
\begin{Desc}
\item[Parameters:]
\begin{description}
\item[{\em p\-Attr\-Path:}](in) Handle of the COR attribute path.\end{description}
\end{Desc}
\begin{Desc}
\item[Return values:]{\em Cl\-Int16T\/}, the number of elements.\end{Desc}
\begin{Desc}
\item[Description:]This function is used to return the number of nodes in the hierarchy within the COR attribute path.\end{Desc}
\begin{Desc}
\item[Library File:]Cl\-Cor\-Client\end{Desc}
\begin{Desc}
\item[Related Function(s):]\hyperlink{pagecor256}{cl\-Cor\-Attr\-Path\-Initialize}, \hyperlink{pagecor257}{cl\-Cor\-Attr\-Path\-Free}, 
\hyperlink{pagecor258}{cl\-Cor\-Attr\-Path\-Truncate}, \hyperlink{pagecor259}{cl\-Cor\-Attr\-Path\-Set},
\hyperlink{pagecor260}{cl\-Cor\-Attr\-Path\-Append}, \hyperlink{pagecor261}{cl\-Cor\-Attr\-Path\-Depth\-Get}, 
\hyperlink{pagecor262}{cl\-Cor\-Attr\-Path\-Show}, \hyperlink{pagecor263}{cl\-Cor\-Attr\-Path\-To\-Attr\-Id\-Get}, 
\hyperlink{pagecor264}{cl\-Cor\-Attr\-Path\-Index\-Get}, \hyperlink{pagecor265}{cl\-Cor\-Attr\-Path\-Index\-Set},
\hyperlink{pagecor266}{cl\-Cor\-Attr\-Path\-Compare}, \hyperlink{pagecor267}{cl\-Cor\-Attr\-Path\-Clone}\end{Desc}
\newpage


\subsection{clCorAttrPathShow}
\index{clCorAttrPathShow@{clCorAttrPathShow}}
\hypertarget{pagecor262}{}\paragraph{cl\-Cor\-Attr\-Path\-Show}\label{pagecor262}
\begin{Desc}
\item[Synopsis:]Displays the COR attribute path in debug mode only.\end{Desc}
\begin{Desc}
\item[Header File:]clCorUtilityApi.h\end{Desc}
\begin{Desc}
\item[Syntax:]

\footnotesize\begin{verbatim}   void clCorAttrPathShow(
                 CL_IN ClCorAttrPathPtrT  pAttrPath);
\end{verbatim}
\normalsize
\end{Desc}
\begin{Desc}
\item[Parameters:]
\begin{description}
\item[{\em p\-Attr\-Path:}](in) Handle of the COR attribute path.\end{description}
\end{Desc}
\begin{Desc}
\item[Return values:]\end{Desc}
\begin{Desc}
\item[Description:]This function is used to display all the entries within the COR attribute path.\end{Desc}
\begin{Desc}
\item[Library File:]Cl\-Cor\-Client\end{Desc}
\begin{Desc}
\item[Related Function(s):]\hyperlink{pagecor256}{cl\-Cor\-Attr\-Path\-Initialize}, \hyperlink{pagecor257}{cl\-Cor\-Attr\-Path\-Free}, 
\hyperlink{pagecor258}{cl\-Cor\-Attr\-Path\-Truncate}, \hyperlink{pagecor259}{cl\-Cor\-Attr\-Path\-Set},
\hyperlink{pagecor260}{cl\-Cor\-Attr\-Path\-Append}, \hyperlink{pagecor261}{cl\-Cor\-Attr\-Path\-Depth\-Get}, 
\hyperlink{pagecor262}{cl\-Cor\-Attr\-Path\-Show}, \hyperlink{pagecor263}{cl\-Cor\-Attr\-Path\-To\-Attr\-Id\-Get}, 
\hyperlink{pagecor264}{cl\-Cor\-Attr\-Path\-Index\-Get}, \hyperlink{pagecor265}{cl\-Cor\-Attr\-Path\-Index\-Set},
\hyperlink{pagecor266}{cl\-Cor\-Attr\-Path\-Compare}, \hyperlink{pagecor267}{cl\-Cor\-Attr\-Path\-Clone}\end{Desc}
\newpage


\subsection{clCorAttrPathToAttrIdGet}
\index{clCorAttrPathToAttrIdGet@{clCorAttrPathToAttrIdGet}}
\hypertarget{pagecor263}{}\paragraph{cl\-Cor\-Attr\-Path\-To\-Attr\-Id\-Get}\label{pagecor263}
\begin{Desc}
\item[Synopsis:]Returns the attribute ID.\end{Desc}
\begin{Desc}
\item[Header File:]clCorUtilityApi.h\end{Desc}
\begin{Desc}
\item[Syntax:]

\footnotesize\begin{verbatim}   ClCorAttrIdT clCorAttrPathToAttrIdGet(
                   CL_IN ClCorAttrPathPtrT pAttrPath);
\end{verbatim}
\normalsize
\end{Desc}
\begin{Desc}
\item[Parameters:]
\begin{description}
\item[{\em p\-Attr\-Path:}](in) Handle of the COR attribute path.\end{description}
\end{Desc}
\begin{Desc}
\item[Return values:]This API returns {\em Cl\-Cor\-Attr\-Id\-T\/}, the attribute ID.\end{Desc}
\begin{Desc}
\item[Description:]This function is used to return the attribute ID within the COR attribute path. It refers to the attribute ID at the bottom of the hierarchy.\end{Desc}
\begin{Desc}
\item[Library File:]Cl\-Cor\-Client\end{Desc}
\begin{Desc}
\item[Related Function(s):]\hyperlink{pagecor255}{cl\-Cor\-Attr\-Path\-Alloc}, \hyperlink{pagecor256}{cl\-Cor\-Attr\-Path\-Initialize}, 
\hyperlink{pagecor257}{cl\-Cor\-Attr\-Path\-Free}, \hyperlink{pagecor258}{cl\-Cor\-Attr\-Path\-Truncate}, 
\hyperlink{pagecor259}{cl\-Cor\-Attr\-Path\-Set}, \hyperlink{pagecor260}{cl\-Cor\-Attr\-Path\-Append}, 
\hyperlink{pagecor261}{cl\-Cor\-Attr\-Path\-Depth\-Get}, \hyperlink{pagecor262}{cl\-Cor\-Attr\-Path\-Show}, 
\hyperlink{pagecor264}{cl\-Cor\-Attr\-Path\-Index\-Get}, \hyperlink{pagecor265}{cl\-Cor\-Attr\-Path\-Index\-Set}, 
\hyperlink{pagecor266}{cl\-Cor\-Attr\-Path\-Compare}, \hyperlink{pagecor267}{cl\-Cor\-Attr\-Path\-Clone} \end{Desc}
\newpage


\subsection{clCorAttrPathIndexGet}
\index{clCorAttrPathIndexGet@{clCorAttrPathIndexGet}}
\hypertarget{pagecor264}{}\paragraph{cl\-Cor\-Attr\-Path\-Index\-Get}\label{pagecor264}
\begin{Desc}
\item[Synopsis:]Returns the index of COR attribute path.\end{Desc}
\begin{Desc}
\item[Header File:]clCorUtilityApi.h\end{Desc}
\begin{Desc}
\item[Syntax:]

\footnotesize\begin{verbatim}   ClUint32T clCorAttrPathIndexGet(
                        CL_IN ClCorAttrPathPtrT pAttrPath);
\end{verbatim}
\normalsize
\end{Desc}
\begin{Desc}
\item[Parameters:]
\begin{description}
\item[{\em p\-Attr\-Path:}](in) Handle of the COR attribute path.\end{description}
\end{Desc}
\begin{Desc}
\item[Return values:]{\em Cl\-Uint32T\/} index of last entry in attribute path.\end{Desc}
\begin{Desc}
\item[Description:]This function is used to retrieve the index of COR attribute path. This instance refers to the attribute ID and its index at the bottom of the hierarchy.\end{Desc}
\begin{Desc}
\item[Library File:]Cl\-Cor\-Client\end{Desc}
\begin{Desc}
\item[Related Function(s):]\hyperlink{pagecor256}{cl\-Cor\-Attr\-Path\-Initialize}, \hyperlink{pagecor257}{cl\-Cor\-Attr\-Path\-Free}, 
\hyperlink{pagecor258}{cl\-Cor\-Attr\-Path\-Truncate}, \hyperlink{pagecor259}{cl\-Cor\-Attr\-Path\-Set},
\hyperlink{pagecor260}{cl\-Cor\-Attr\-Path\-Append}, \hyperlink{pagecor261}{cl\-Cor\-Attr\-Path\-Depth\-Get}, 
\hyperlink{pagecor262}{cl\-Cor\-Attr\-Path\-Show}, \hyperlink{pagecor263}{cl\-Cor\-Attr\-Path\-To\-Attr\-Id\-Get}, 
\hyperlink{pagecor264}{cl\-Cor\-Attr\-Path\-Index\-Get}, \hyperlink{pagecor265}{cl\-Cor\-Attr\-Path\-Index\-Set},
\hyperlink{pagecor266}{cl\-Cor\-Attr\-Path\-Compare}, \hyperlink{pagecor267}{cl\-Cor\-Attr\-Path\-Clone}\end{Desc}
\newpage


\subsection{clCorAttrPathIndexSet}
\index{clCorAttrPathIndexSet@{clCorAttrPathIndexSet}}
\hypertarget{pagecor265}{}\paragraph{cl\-Cor\-Attr\-Path\-Index\-Set}\label{pagecor265}
\begin{Desc}
\item[Synopsis:]Sets the index of COR attribute path.\end{Desc}
\begin{Desc}
\item[Header File:]clCorUtilityApi.h\end{Desc}
\begin{Desc}
\item[Syntax:]

\footnotesize\begin{verbatim}   ClRcT clCorAttrPathIndexSet(
                         CL_INOUT ClCorAttrPathPtrT pAttrPath,
                         CL_IN ClUint16T ndepth,
                         CL_IN ClUint32T newIndex);
\end{verbatim}
\normalsize
\end{Desc}
\begin{Desc}
\item[Parameters:]
\begin{description}
\item[{\em p\-Attr\-Path:}](in) Handle of the COR attribute path. \item[{\em ndepth:}](in) The depth at which new index is to be set. \item[{\em new\-Index:}]: The new index to be set.\end{description}
\end{Desc}
\begin{Desc}
\item[Return values:]{\em CL\_\-OK\/}, on success.\end{Desc}
\begin{Desc}
\item[Description:]This function is used to set the index field within the COR attribute hierarchy.\end{Desc}
\begin{Desc}
\item[Library File:]Cl\-Cor\-Client\end{Desc}
\begin{Desc}
\item[Related Function(s):]\hyperlink{pagecor256}{cl\-Cor\-Attr\-Path\-Initialize}, \hyperlink{pagecor257}{cl\-Cor\-Attr\-Path\-Free}, 
\hyperlink{pagecor258}{cl\-Cor\-Attr\-Path\-Truncate}, \hyperlink{pagecor259}{cl\-Cor\-Attr\-Path\-Set},
\hyperlink{pagecor260}{cl\-Cor\-Attr\-Path\-Append}, \hyperlink{pagecor261}{cl\-Cor\-Attr\-Path\-Depth\-Get}, 
\hyperlink{pagecor262}{cl\-Cor\-Attr\-Path\-Show}, \hyperlink{pagecor263}{cl\-Cor\-Attr\-Path\-To\-Attr\-Id\-Get}, 
\hyperlink{pagecor264}{cl\-Cor\-Attr\-Path\-Index\-Get}, \hyperlink{pagecor265}{cl\-Cor\-Attr\-Path\-Index\-Set},
\hyperlink{pagecor266}{cl\-Cor\-Attr\-Path\-Compare}, \hyperlink{pagecor267}{cl\-Cor\-Attr\-Path\-Clone}\end{Desc}
\newpage


\subsection{clCorAttrPathCompare}
\index{clCorAttrPathCompare@{clCorAttrPathCompare}}
\hypertarget{pagecor266}{}\paragraph{cl\-Cor\-Attr\-Path\-Compare}\label{pagecor266}
\begin{Desc}
\item[Synopsis:]Compares two \textit{ClCorAttrPath}\end{Desc}
\begin{Desc}
\item[Header File:]clCorUtilityApi.h\end{Desc}
\begin{Desc}
\item[Syntax:]

\footnotesize\begin{verbatim}   ClInt32T clCorAttrPathCompare(
                        CL_IN ClCorAttrPathPtrT pAttrPath,
                        CL_IN ClCorAttrPathPtrT cmp);
\end{verbatim}
\normalsize
\end{Desc}
\begin{Desc}
\item[Parameters:]
\begin{description}
\item[{\em p\-Attr\-Path:}](in) First handle of \textit{ClCorAttrPath}. \item[{\em cmp:}](in) Second handle of \textit{ClCorAttrPath}.\end{description}
\end{Desc}
\begin{Desc}
\item[Return values:]
\begin{description}
\item[{\em 0:}]If both attribute paths are same, even if they have same indices, like wildcards. \item[{\em -1:}]If the attribute paths mismatch. \item[{\em 1:}]If wildcards are same.\end{description}
\end{Desc}
\begin{Desc}
\item[Description:]This function is used to make the comparison between two \textit{ClCorAttrPath} and verify whether they are same. This comparison goes through till depth of both the paths.\end{Desc}
\begin{Desc}
\item[Library File:]Cl\-Cor\-Client\end{Desc}
\begin{Desc}
\item[Related Function(s):]\hyperlink{pagecor256}{cl\-Cor\-Attr\-Path\-Initialize}, \hyperlink{pagecor257}{cl\-Cor\-Attr\-Path\-Free}, 
\hyperlink{pagecor258}{cl\-Cor\-Attr\-Path\-Truncate}, \hyperlink{pagecor259}{cl\-Cor\-Attr\-Path\-Set},
\hyperlink{pagecor260}{cl\-Cor\-Attr\-Path\-Append}, \hyperlink{pagecor261}{cl\-Cor\-Attr\-Path\-Depth\-Get}, 
\hyperlink{pagecor262}{cl\-Cor\-Attr\-Path\-Show}, \hyperlink{pagecor263}{cl\-Cor\-Attr\-Path\-To\-Attr\-Id\-Get}, 
\hyperlink{pagecor264}{cl\-Cor\-Attr\-Path\-Index\-Get}, \hyperlink{pagecor265}{cl\-Cor\-Attr\-Path\-Index\-Set},
\hyperlink{pagecor266}{cl\-Cor\-Attr\-Path\-Compare}, \hyperlink{pagecor267}{cl\-Cor\-Attr\-Path\-Clone}\end{Desc}
\newpage


\subsection{clCorAttrPathClone}
\index{clCorAttrPathClone@{clCorAttrPathClone}}
\hypertarget{pagecor267}{}\paragraph{cl\-Cor\-Attr\-Path\-Clone}\label{pagecor267}
\begin{Desc}
\item[Synopsis:]Clones a particular \textit{ClCorAttrPath}.\end{Desc}
\begin{Desc}
\item[Header File:]clCorUtilityApi.h\end{Desc}
\begin{Desc}
\item[Syntax:]

\footnotesize\begin{verbatim}   ClRcT clCorAttrPathClone(
                         CL_IN ClCorAttrPathPtrT pAttrPath,
                         CL_OUT ClCorAttrPathPtrT *newH);
\end{verbatim}
\normalsize
\end{Desc}
\begin{Desc}
\item[Parameters:]
\begin{description}
\item[{\em p\-Attr\-Path:}](in) Handle of the COR attribute path. \item[{\em new\-H:}](out) Handle of the new clone.\end{description}
\end{Desc}
\begin{Desc}
\item[Return values:]
\begin{description}
\item[{\em CL\_\-OK:}]The function executed successfully. \item[{\em CL\_\-COR\_\-ERR\_\-NO\_\-MEM:}]On memory allocation failure.\end{description}
\end{Desc}
\begin{Desc}
\item[Description:]This function is used to clone a particular COR attribute path. It allocates and makes a copy of the contents of the given path to a new path.\end{Desc}
\begin{Desc}
\item[Library File:]Cl\-Cor\-Client\end{Desc}
\begin{Desc}
\item[Related Function(s):]\hyperlink{pagecor256}{cl\-Cor\-Attr\-Path\-Initialize}, \hyperlink{pagecor257}{cl\-Cor\-Attr\-Path\-Free}, 
\hyperlink{pagecor258}{cl\-Cor\-Attr\-Path\-Truncate}, \hyperlink{pagecor259}{cl\-Cor\-Attr\-Path\-Set},
\hyperlink{pagecor260}{cl\-Cor\-Attr\-Path\-Append}, \hyperlink{pagecor261}{cl\-Cor\-Attr\-Path\-Depth\-Get}, 
\hyperlink{pagecor262}{cl\-Cor\-Attr\-Path\-Show}, \hyperlink{pagecor263}{cl\-Cor\-Attr\-Path\-To\-Attr\-Id\-Get}, 
\hyperlink{pagecor264}{cl\-Cor\-Attr\-Path\-Index\-Get}, \hyperlink{pagecor265}{cl\-Cor\-Attr\-Path\-Index\-Set},
\hyperlink{pagecor266}{cl\-Cor\-Attr\-Path\-Compare}, \hyperlink{pagecor267}{cl\-Cor\-Attr\-Path\-Clone}\end{Desc}
\newpage


\subsection{clCorUtilMoAndMSOCreate}
\index{clCorUtilMoAndMSOCreate@{clCorUtilMoAndMSOCreate}}
\hypertarget{pagecor270}{}\paragraph{cl\-Cor\-Util\-Mo\-And\-MSOCreate}\label{pagecor270}
\begin{Desc}
\item[Synopsis:]Creates MO and MSO objects.\end{Desc}
\begin{Desc}
\item[Header File:]clCorUtilityApi.h\end{Desc}
\begin{Desc}
\item[Syntax:]

\footnotesize\begin{verbatim}   ClRcT clCorUtilMoAndMSOCreate(
                         CL_IN ClCorMOIdPtrT pMoId,
                         CL_OUT ClCorObjectHandleT *pHandle);
\end{verbatim}
\normalsize
\end{Desc}
\begin{Desc}
\item[Parameters:]
\begin{description}
\item[{\em p\-Mo\-Id:}](in) MOId of the MO to be created. \item[{\em p\-Handle:}](out) Handle to the MO object created.\end{description}
\end{Desc}
\begin{Desc}
\item[Return values:]
\begin{description}
\item[{\em CL\_\-OK:}]The function executed successfully. \item[{\em CL\_\-ERR\_\-NOT\_\-EXIST:}]If MO class type does not exist.\end{description}
\end{Desc}
\begin{Desc}
\item[Description:]This function is used to create MO and MSO objects which are associated with the MO.\end{Desc}
\begin{Desc}
\item[Library File:]Cl\-Cor\-Client\end{Desc}
\begin{Desc}
\item[Related Function(s):]\hyperlink{pagecor271}{cl\-Cor\-Util\-Mo\-And\-MSODelete} \end{Desc}
\newpage


\subsection{clCorUtilMoAndMSODelete}
\index{clCorUtilMoAndMSODelete@{clCorUtilMoAndMSODelete}}
\hypertarget{pagecor271}{}\paragraph{cl\-Cor\-Util\-Mo\-And\-MSODelete}\label{pagecor271}
\begin{Desc}
\item[Synopsis:]Deletes MO and MSO objects.\end{Desc}
\begin{Desc}
\item[Header File:]clCorUtilityApi.h\end{Desc}
\begin{Desc}
\item[Syntax:]

\footnotesize\begin{verbatim}   ClRcT clCorUtilMoAndMSODelete(
                         CL_IN ClCorMOIdPtrT pMoId);
\end{verbatim}
\normalsize
\end{Desc}
\begin{Desc}
\item[Parameters:]
\begin{description}
\item[{\em p\-Mo\-Id:}](in) {\em MOId\/} of the MO to be deleted.\end{description}
\end{Desc}
\begin{Desc}
\item[Return values:]
\begin{description}
\item[{\em CL\_\-OK:}]The function executed successfully. \item[{\em CL\_\-ERR\_\-NOT\_\-EXIST:}]If MO class type does not exist.\end{description}
\end{Desc}
\begin{Desc}
\item[Description:]This function is used to delete MO and MSO objects which are associated with the MO.\end{Desc}
\begin{Desc}
\item[Library File:]Cl\-Cor\-Client\end{Desc}
\begin{Desc}
\item[Related Function(s):]\hyperlink{pagecor270}{cl\-Cor\-Util\-Mo\-And\-MSOCreate} \end{Desc}
\newpage


\section{COR Transaction APIs}
\begin{Desc}
\item[Overview]\end{Desc}
COR Transaction APIs are used by components that participate in a transaction. A component participating in a transaction obtains a transaction handle ( {\em txn\-Handle\/}) through the callback functions for validate,commit and rollback which can be used in the COR transaction APIs to validate the transaction. A transaction can contain various COR operations related to one or more Managed Objects (MO). The operations related to an MO are considered as one transaction job. Within a transaction job there can be multiple COR operations which can be obtained by using the COR-Transaction APIs.

A transaction can consist of one or more object-create/set/delete operations or a combination of these operations as well. COR creates one COR-Transaction ID for a combination of MOID and {\em attrpath\/}. For a given COR-Transaction ID, COR creates one COR-Job ID for each {\em attribute\-Id\/}. (that is being set for a particular MOID + attr\-Path). These jobs are linked together in the COR-job-list.

Following steps are performed to obtain COR-job-list and to perform a walk operation.

\begin{enumerate}
\item When a transaction validate/commit/rollback is called, the COR-job-list ID can be obtained from the COR-transaction ID via the API 
\textit{clCorTxnJobHandleToCorTxnIdGet()}. This function returns {\em Cl\-Cor\-Txn\-Id\-T\/}.\item
\textit{clCorTxnJobWalk()} function performs the walk operation corresponding to above COR-transaction ID. This function
invokes a callback for every COR-job in the COR-job-list.\item The callback function takes in COR-Transaction ID and COR-job ID as parameters. COR 
provides a function to obtain information about the {\em MOID\/} if it is a create/delete operation. COR provides an API to obtain information about the 
attribute ({\em MOID\/},{\em attrpath.index\/}, {\em size\/}, {\em value\/}) for a set operation, based on COR-Transaction ID and COR-job ID. For 
instance, the API \textit{clCorTxnJobSetParamsGet()} returns {\em attr\-Id\/}, {\em index\/}, {\em value\/} and size
corresponding to a COR-Transaction ID and COR-job ID.\end{enumerate}


\newpage 
\subsection{clCorTxnJobWalk}
\index{clCorTxnJobWalk@{clCorTxnJobWalk}}
\hypertarget{pagecor501}{}\paragraph{cl\-Cor\-Txn\-Job\-Walk}\label{pagecor501}
\begin{Desc}
\item[Synopsis:]Walk through the transaction Jobs.\end{Desc}
\begin{Desc}
\item[Header File:]clCorTxnApi.h\end{Desc}
\begin{Desc}
\item[Syntax:]

\footnotesize\begin{verbatim}       ClRcT    clCorTxnJobWalk(
                             CL_IN ClCorTxnIdT  pThis,
                             CL_IN ClCorTxnFuncT funcPtr,
                             CL_IN void  *cookie);
\end{verbatim}
\normalsize
\end{Desc}
\begin{Desc}
\item[Parameters:]
\begin{description}
\item[{\em p\-This:}](in) Pointer to the Transaction ID. \item[{\em func\-Ptr:}](in) Callback function for handling each job. \item[{\em cookie:}](in) Pointer to the user data.\end{description}
\end{Desc}
\begin{Desc}
\item[Return values:]
\begin{description}
\item[{\em CL\_\-OK:}]The API executed successfully. \item[{\em CL\_\-COR\_\-TXN\_\-ERR\_\-ZERO\_\-JOBS:}]When there are no jobs to walk.\end{description}
\end{Desc}
\begin{Desc}
\item[Description:]This function is used to walk through the transaction jobs. The callback function is used which operate on each of the jobs in the transaction.\end{Desc}
\begin{Desc}
\item[Library Name:]Cl\-Cor\-Client\end{Desc}
\begin{Desc}
\item[Related Function(s):]\hyperlink{pagecor510}{cl\-Cor\-Txn\-Job\-Mo\-Id\-Get}, \hyperlink{pagecor508}{cl\-Cor\-Txn\-Job\-Attr\-Path\-Get}, \hyperlink{group__group13}{cl\-Cor\-Txn\-Job\-Object\-Handle\-Get}, \hyperlink{group__group13}{cl\-Cor\-Txn\-Job\-Set\-Params\-Get} \end{Desc}
\newpage


\subsection{clCorTxnSessionCommit}
\index{clCorTxnSessionCommit@{clCorTxnSessionCommit}}
\hypertarget{pagecor502}{}\paragraph{cl\-Cor\-Txn\-Session\-Commit}\label{pagecor502}
\begin{Desc}
\item[Synopsis:]Commits an active transaction session.\end{Desc}
\begin{Desc}
\item[Header File:]clCorTxnApi.h\end{Desc}
\begin{Desc}
\item[Syntax:]

\footnotesize\begin{verbatim}    ClRcT    clCorTxnSessionCommit(
                          CL_IN ClCorTxnSessionIdT txnSessionId);
\end{verbatim}
\normalsize
\end{Desc}
\begin{Desc}
\item[Parameters:]
\begin{description}
\item[{\em txn\-Session\-Id:}](in) Transaction session ID.\end{description}
\end{Desc}
\begin{Desc}
\item[Return values:]
\begin{description}
\item[{\em CL\_\-OK:}]The API executed successfully. \item[{\em CL\_\-COR\_\-TXN\_\-ERR\_\-ZERO\_\-JOBS:}]If there are no jobs to commit.\end{description}
\end{Desc}
\begin{Desc}
\item[Description:]This function is used to commit the transaction whose session ID is passed through this API.\end{Desc}
\begin{Desc}
\item[Library Name:]Cl\-Cor\-Client\end{Desc}
\begin{Desc}
\item[Related Function(s):]\hyperlink{pagecor503}{cl\-Cor\-Txn\-Session\-Cancel} \end{Desc}
\newpage


\subsection{clCorTxnSessionCancel}
\index{clCorTxnSessionCancel@{clCorTxnSessionCancel}}
\hypertarget{pagecor503}{}\paragraph{cl\-Cor\-Txn\-Session\-Cancel}\label{pagecor503}
\begin{Desc}
\item[Synopsis:]Cancels transaction session.\end{Desc}
\begin{Desc}
\item[Header File:]clCorTxnApi.h\end{Desc}
\begin{Desc}
\item[Syntax:]

\footnotesize\begin{verbatim}       ClRcT    clCorTxnSessionCancel(
                            CL_IN    ClCorTxnSessionIdT  txnSessionId);
\end{verbatim}
\normalsize
\end{Desc}
\begin{Desc}
\item[Parameters:]
\begin{description}
\item[{\em txn\-Session\-Id:}](in) Transaction session ID.\end{description}
\end{Desc}
\begin{Desc}
\item[Return values:]
\begin{description}
\item[{\em CL\_\-OK:}]The API executed successfully.\end{description}
\end{Desc}
\begin{Desc}
\item[Description:]This function is used to cancel the active transaction sessions whose session ID is passed through this function.\end{Desc}
\begin{Desc}
\item[Library Name:]Cl\-Cor\-Client\end{Desc}
\begin{Desc}
\item[Related Function(s):]\hyperlink{pagecor502}{cl\-Cor\-Txn\-Session\-Commit} \end{Desc}
\newpage


\subsection{clCorTxnJobHandleToCorTxnIdGet}
\index{clCorTxnJobHandleToCorTxnIdGet@{clCorTxnJobHandleToCorTxnIdGet}}
\hypertarget{pagecor504}{}\paragraph{cl\-Cor\-Txn\-Job\-Handle\-To\-Cor\-Txn\-Id\-Get}\label{pagecor504}
\begin{Desc}
\item[Synopsis:]Get the transaction ID from the transaction Job handle.\end{Desc}
\begin{Desc}
\item[Header File:]clCorTxnApi.h\end{Desc}
\begin{Desc}
\item[Library Name:]Cl\-Cor\-Client\end{Desc}
\begin{Desc}
\item[Syntax:]

\footnotesize\begin{verbatim}  ClRcT clCorTxnJobHandleToCorTxnIdGet(
                    CL_IN    ClTxnJobDefnHandleT   jobDefn,
                    CL_IN    ClSizeT               size,
                    CL_OUT   ClCorTxnIdT          *pTxnId);
\end{verbatim}
\normalsize
\end{Desc}
\begin{Desc}
\item[Parameters:]
\begin{description}
\item[{\em job\-Defn:}](in) Transaction job handle. \item[{\em size:}](in) Size of the transaction. \item[{\em p\-Txn\-Id:}](out) Transaction ID .\end{description}
\end{Desc}
\begin{Desc}
\item[Return values:]
\begin{description}
\item[{\em CL\_\-OK:}]The API executed successfully.\end{description}
\end{Desc}
\begin{Desc}
\item[Description:]This function is used to retrieve the COR Transaction ID by using the transaction job handle.\end{Desc}
\begin{Desc}
\item[Related Function(s):]\hyperlink{pagecor505}{cl\-Cor\-Txn\-Id\-Txn\-Free}, \hyperlink{pagecor501}{cl\-Cor\-Txn\-Job\-Walk}, 
\hyperlink{pagecor510}{cl\-Cor\-Txn\-Job\-Mo\-Id\-Get}, \hyperlink{pagecor508}{cl\-Cor\-Txn\-Job\-Attr\-Path\-Get}, 
\hyperlink{group__group13}{cl\-Cor\-Txn\-Job\-Set\-Params\-Get}, \hyperlink{pagecor506}{cl\-Cor\-Txn\-Job\-Attribute\-Type\-Get} \end{Desc}
\newpage


\subsection{clCorTxnIdTxnFree}
\index{clCorTxnIdTxnFree@{clCorTxnIdTxnFree}}
\hypertarget{pagecor505}{}\paragraph{cl\-Cor\-Txn\-Id\-Txn\-Free}\label{pagecor505}
\begin{Desc}
\item[Synopsis:]Frees the data for the transaction\-ID .\end{Desc}
\begin{Desc}
\item[Header File:]clCorTxnApi.h\end{Desc}
\begin{Desc}
\item[Syntax:]

\footnotesize\begin{verbatim}       ClRcT clCorTxnIdTxnFree(
                         CL_IN    ClCorTxnIdT    corTxnId);
\end{verbatim}
\normalsize
\end{Desc}
\begin{Desc}
\item[Parameters:]
\begin{description}
\item[{\em cor\-Txn\-Id:}](in) COR transaction ID.\end{description}
\end{Desc}
\begin{Desc}
\item[Return values:]
\begin{description}
\item[{\em CL\_\-OK:}]The API executed successfully.\end{description}
\end{Desc}
\begin{Desc}
\item[Description:]This function is used to free the data corresponding to the transaction\-ID .\end{Desc}
\begin{Desc}
\item[Library Name:]Cl\-Cor\-Client\end{Desc}
\begin{Desc}
\item[Related Function(s):]\hyperlink{pagecor504}{cl\-Cor\-Txn\-Job\-Handle\-To\-Cor\-Txn\-Id\-Get} \end{Desc}
\newpage


\subsection{clCorTxnJobAttributeTypeGet}
\index{clCorTxnJobAttributeTypeGet@{clCorTxnJobAttributeTypeGet}}
\hypertarget{pagecor506}{}\paragraph{cl\-Cor\-Txn\-Job\-Attribute\-Type\-Get}\label{pagecor506}
\begin{Desc}
\item[Synopsis:]Get the attribute type information.\end{Desc}
\begin{Desc}
\item[Header File:]clCorTxnApi.h\end{Desc}
\begin{Desc}
\item[Syntax:]

\footnotesize\begin{verbatim}      ClRcT clCorTxnJobAttributeTypeGet(
                         CL_IN   ClCorTxnIdT          txnId,
                         CL_IN   ClCorTxnJobIdT       jobId,
                         CL_OUT  ClCorAttrTypeT      *pAttrType,
                         CL_OUT  ClCorTypeT          *pAttrDataType);
\end{verbatim}
\normalsize
\end{Desc}
\begin{Desc}
\item[Parameters:]
\begin{description}
\item[{\em txn\-Id:}](in) Transaction ID. \item[{\em job\-Id:}](in) ID of the job in the transaction. \item[{\em p\-Attr\-Type:}](out) Attribute type that is 
whether it is simple or array or association. \item[{\em p\-Attr\-Data\-Type:}](out) Basic type of the attribute.\end{description}
\end{Desc}
\begin{Desc}
\item[Return values:]
\begin{description}
\item[{\em CL\_\-OK:}]The API executed successfully. \item[{\em CL\_\-COR\_\-TXN\_\-ERR\_\-INVALID\_\-JOB\_\-ID:}]Invalid job ID. \item[{\em CL\_\-COR\_\-ERR\_\-NULL\_\-PTR:}]On passing a NULL pointer. \item[{\em CL\_\-COR\_\-ERR\_\-INVALID\_\-PARAM:}]On passing an invalid parameter.\end{description}
\end{Desc}
\begin{Desc}
\item[Description:]This function is used to extract the attribute type and its basic type from the transaction job. In case of a simple attribute (which corresponds to the Cl\-Cor\-Type\-T) the basic type is indicated by {\em p\-Attr\-Data\-Type\/}. In case of an array type, {\em p\-Attr\-Data\-Type\/} contains the basic type.\end{Desc}
\begin{Desc}
\item[Library Name:]Cl\-Cor\-Client\end{Desc}
\begin{Desc}
\item[Related Function(s):]\hyperlink{group__group13}{cl\-Cor\-Txn\-Job\-Set\-Params\-Get} \end{Desc}
\newpage


\subsection{clCorTxnJobSetParamsGet}
\index{clCorTxnJobSetParamsGet@{clCorTxnJobSetParamsGet}}
\hypertarget{pagecor507}{}\paragraph{cl\-Cor\-Txn\-Job\-Set\-Params\-Get}\label{pagecor507}
\begin{Desc}
\item[Synopsis:]Get all the information necessary for setting the attribute.\end{Desc}
\begin{Desc}
\item[Header File:]clCorTxnApi.h\end{Desc}
\begin{Desc}
\item[Syntax:]

\footnotesize\begin{verbatim}       ClRcT clCorTxnJobSetParamsGet(
                         CL_IN   ClCorTxnIdT          txnId,
                         CL_IN   ClCorTxnJobIdT       jobId,
                         CL_OUT  ClCorAttrIdT        *pAttrId,
                         CL_OUT  ClInt32T            *pIndex,
                         CL_OUT  void               **pValue,
                         CL_OUT  ClUint32T           *pSize);
\end{verbatim}
\normalsize
\end{Desc}
\begin{Desc}
\item[Parameters:]
\begin{description}
\item[{\em txn\-Id}]:(in)  Transaction ID . \item[{\em job\-Id}]:(in)  Transaction job ID . \item[{\em p\-Attr\-Id}]: (out) Pointer to attribute ID . 
\item[{\em p\-Index}]: (out) Pointer index of the attribute ( CL\_\-COR\_\-INVALID\_\-ATTR\_\-IDX in case of simple attributes). 
\item[{\em p\-Value:}](out) Pointer to the pointer to the value. \item[{\em p\-Size}]: (out) Size of the attribute value to be set.\end{description}
\end{Desc}
\begin{Desc}
\item[Return values:]
\begin{description}
\item[{\em CL\_\-OK:}]The API executed successfully. \item[{\em CL\_\-COR\_\-ERR\_\-NULL\_\-PTR:}]On passing a NULL pointer. \item[{\em CL\_\-COR\_\-ERR\_\-INVALID\_\-PARAM:}]On passing an invalid parameter. \item[{\em CL\_\-COR\_\-TXN\_\-ERR\_\-INVALID\_\-JOB\_\-ID:}]Job ID passed is NULL.\end{description}
\end{Desc}
\begin{Desc}
\item[Description:]This function is used to extract the set parameters that were passed when the set function is called on the client side. The set parameters are the following: \begin{itemize}
\item {\em attr\-Id\/} \item index of the attribute \item pointer to the value \item size of the attribute value\end{itemize}
\end{Desc}
\begin{Desc}
\item[Library Name:]Cl\-Cor\-Client\end{Desc}
\begin{Desc}
\item[Note:]This function is used only when the operation type is {\em set\/}.\end{Desc}
\begin{Desc}
\item[Related Function(s):]\hyperlink{pagecor509}{cl\-Cor\-Txn\-Job\-Operation\-Get}, 
\hyperlink{pagecor508}{cl\-Cor\-Txn\-Job\-Attr\-Path\-Get} \end{Desc}
\newpage


\subsection{clCorTxnJobAttrPathGet}
\index{clCorTxnJobAttrPathGet@{clCorTxnJobAttrPathGet}}
\hypertarget{pagecor508}{}\paragraph{cl\-Cor\-Txn\-Job\-Attr\-Path\-Get}\label{pagecor508}
\begin{Desc}
\item[Synopsis:]Retrieves the {\em attrpath\/} from the transaction.\end{Desc}
\begin{Desc}
\item[Header File:]clCorTxnApi.h\end{Desc}
\begin{Desc}
\item[Syntax:]

\footnotesize\begin{verbatim}       ClRcT  clCorTxnJobAttrPathGet(
                          CL_IN   ClCorTxnIdT     txnId,
                          CL_IN   ClCorTxnJobIdT  jobId,
                          CL_OUT  ClCorAttrPathT      **pAttrPath);
\end{verbatim}
\normalsize
\end{Desc}
\begin{Desc}
\item[Parameters:]
\begin{description}
\item[{\em txn\-Id}]:(in)  Transaction ID . \item[{\em job\-Id}]:(in)  Transaction Job ID . \item[{\em p\-Attr\-Path:}](out) Pointer to pointer of attr\-Path.\end{description}
\end{Desc}
\begin{Desc}
\item[Return values:]
\begin{description}
\item[{\em CL\_\-OK:}]The API executed successfully. \item[{\em CL\_\-COR\_\-ERR\_\-NULL\_\-PTR:}]On passing a NULL pointer. \item[{\em CL\_\-COR\_\-ERR\_\-INVALID\_\-PARAM:}]On passing an invalid parameter. \item[{\em CL\_\-COR\_\-TXN\_\-ERR\_\-INVALID\_\-JOB\_\-ID:}]Returned when the operation type is not set.\end{description}
\end{Desc}
\begin{Desc}
\item[Description:]This function is used to extract the {\em attrpath\/} or the containment path for the {\em set\/} operation. If the operation type is not 'set', an error is returned.\end{Desc}
\begin{Desc}
\item[Library Name:]Cl\-Cor\-Client\end{Desc}
\begin{Desc}
\item[Related Function(s):]\hyperlink{pagecor507}{cl\-Cor\-Txn\-Job\-Set\-Params\-Get}, 
\hyperlink{pagecor509}{cl\-Cor\-Txn\-Job\-Operation\-Get} \end{Desc}
\newpage


\subsection{clCorTxnJobOperationGet}
\index{clCorTxnJobOperationGet@{clCorTxnJobOperationGet}}
\hypertarget{pagecor509}{}\paragraph{cl\-Cor\-Txn\-Job\-Operation\-Get}\label{pagecor509}
\begin{Desc}
\item[Synopsis:]Get the operation type.\end{Desc}
\begin{Desc}
\item[Header File:]clCorTxnApi.h\end{Desc}
\begin{Desc}
\item[Syntax:]

\footnotesize\begin{verbatim}       ClRcT  clCorTxnJobOperationGet(
                          CL_IN     ClCorTxnIdT    txnId,
                          CL_IN     ClCorTxnJobIdT jobId,
                          CL_OUT    ClCorOpsT      *op);
\end{verbatim}
\normalsize
\end{Desc}
\begin{Desc}
\item[Parameters:]
\begin{description}
\item[{\em txn\-Id:}](in) Transaction ID . \item[{\em job\-Id:}](in) Transaction Job ID . \item[{\em op:}](out) Operation Type. {\tt }(CL\_\-COR\_\-CREATE/SET/DELETE)\end{description}
\end{Desc}
\begin{Desc}
\item[Return values:]
\begin{description}
\item[{\em CL\_\-OK:}]The API executed successfully. \item[{\em CL\_\-COR\_\-ERR\_\-NULL\_\-PTR:}]On passing a NULL pointer. \item[{\em CL\_\-COR\_\-ERR\_\-INVALID\_\-PARAM}]: On passing an invalid parameter.\end{description}
\end{Desc}
\begin{Desc}
\item[Description:]This function is used to extract the operation type from the transaction job. The transaction job type can be one of the following: \begin{itemize}
\item {\tt CL\_\-COR\_\-CREATE} \item {\tt CL\_\-COR\_\-DELETE} \item {\tt CL\_\-COR\_\-SET} \end{itemize}
\end{Desc}
\begin{Desc}
\item[Library Name:]Cl\-Cor\-Client\end{Desc}
\begin{Desc}
\item[Related Function(s):]\hyperlink{pagecor510}{cl\-Cor\-Txn\-Job\-Mo\-Id\-Get}, \hyperlink{pagecor508}{cl\-Cor\-Txn\-Job\-Attr\-Path\-Get}, 
\hyperlink{pagecor507}{cl\-Cor\-Txn\-Job\-Set\-Params\-Get} \end{Desc}
\newpage


\subsection{clCorTxnJobMoIdGet}
\index{clCorTxnJobMoIdGet@{clCorTxnJobMoIdGet}}
\hypertarget{pagecor510}{}\paragraph{cl\-Cor\-Txn\-Job\-Mo\-Id\-Get}\label{pagecor510}
\begin{Desc}
\item[Synopsis:]Get the Mo\-Id From the transaction.\end{Desc}
\begin{Desc}
\item[Header File:]clCorTxnApi.h\end{Desc}
\begin{Desc}
\item[Syntax:]

\footnotesize\begin{verbatim}       ClRcT  clCorTxnJobMoIdGet(
                          CL_IN     ClCorTxnIdT  txnId,
                          CL_OUT    ClCorMOIdT  *pMOId);
\end{verbatim}
\normalsize
\end{Desc}
\begin{Desc}
\item[Parameters:]
\begin{description}
\item[{\em txn\-Id}]:(in)  Transaction ID . \item[{\em p\-MOId}]:(out) Pointer to the Mo\-Id for the transaction.\end{description}
\end{Desc}
\begin{Desc}
\item[Return values:]
\begin{description}
\item[{\em CL\_\-OK:}]The API executed successfully. \item[{\em CL\_\-COR\_\-ERR\_\-NULL\_\-PTR:}]On passing a Null Pointer. \item[{\em CL\_\-COR\_\-ERR\_\-INVALID\_\-PARAM:}]On passing an invalid parameter.\end{description}
\end{Desc}
\begin{Desc}
\item[Description:]This function is used to extract the unique {\em Mo\-ID\/} associated with each transaction ID . The transaction having multiple jobs are broken into jobs having mo\-Id, attr\-Path (if used) and operation. So while we walk we get these jobs and we can operate on these data one by one.\end{Desc}
\begin{Desc}
\item[Library Name:]Cl\-Cor\-Client\end{Desc}
\begin{Desc}
\item[Related Function(s):]\hyperlink{group__group13}{cl\-Cor\-Txn\-Job\-Attr\-Path\-Get}, \hyperlink{group__group13}{cl\-Cor\-Txn\-Job\-Set\-Params\-Get} \end{Desc}
\newpage


\subsection{clCorTxnJobObjectHandleGet}
\index{clCorTxnJobObjectHandleGet@{clCorTxnJobObjectHandleGet}}
\hypertarget{pagecor511}{}\paragraph{cl\-Cor\-Txn\-Job\-Object\-Handle\-Get}\label{pagecor511}
\begin{Desc}
\item[Synopsis:]Get the Object handle from the transaction.\end{Desc}
\begin{Desc}
\item[Header File:]clCorTxnApi.h\end{Desc}
\begin{Desc}
\item[Syntax:]

\footnotesize\begin{verbatim}       ClRcT  clCorTxnJobObjectHandleGet(
                          CL_IN     ClCorTxnIdT  txnId,
                          CL_OUT    ClCorObjectHandleT  *pObjHandle);
\end{verbatim}
\normalsize
\end{Desc}
\begin{Desc}
\item[Parameters:]
\begin{description}
\item[{\em txn\-Id}]:(in)  Transaction ID . \item[{\em p\-Obj\-Handle}]: (out) Pointer to the Object handle.\end{description}
\end{Desc}
\begin{Desc}
\item[Return values:]
\begin{description}
\item[{\em CL\_\-OK:}]The API executed successfully. \item[{\em CL\_\-COR\_\-ERR\_\-NULL\_\-PTR:}]On passing a NULL pointer. \item[{\em CL\_\-COR\_\-ERR\_\-INVALID\_\-PARAM:}]On passing an invalid parameter.\end{description}
\end{Desc}
\begin{Desc}
\item[Description:]This function is used to extract the Object handle from the transaction. The transaction will contain the Object handle that it passed when the transaction was prepared.\end{Desc}
\begin{Desc}
\item[Library Name:]Cl\-Cor\-Client\end{Desc}
\begin{Desc}
\item[Related Function(s):]\hyperlink{pagecor510}{cl\-Cor\-Txn\-Job\-Mo\-Id\-Get}, \hyperlink{pagecor508}{cl\-Cor\-Txn\-Job\-Attr\-Path\-Get}, 
\hyperlink{pagecor507}{cl\-Cor\-Txn\-Job\-Set\-Params\-Get} \end{Desc}
\newpage


\subsection{clCorTxnFirstJobGet}
\index{clCorTxnFirstJobGet@{clCorTxnFirstJobGet}}
\hypertarget{pagecor512}{}\paragraph{cl\-Cor\-Txn\-First\-Job\-Get}\label{pagecor512}
\begin{Desc}
\item[Synopsis:]Get the first job.\end{Desc}
\begin{Desc}
\item[Header File:]clCorTxnApi.h\end{Desc}
\begin{Desc}
\item[Syntax:]

\footnotesize\begin{verbatim}       ClRcT  clCorTxnFirstJobGet(
                          CL_IN     ClCorTxnIdT          txnId,
                          CL_OUT    ClCorTxnJobIdT  *pJobId);
\end{verbatim}
\normalsize
\end{Desc}
\begin{Desc}
\item[Parameters:]
\begin{description}
\item[{\em txn\-Id}]:(in)  Transaction ID . \item[{\em p\-Job\-Id}]: (out) Pointer to the transaction job ID .\end{description}
\end{Desc}
\begin{Desc}
\item[Return values:]
\begin{description}
\item[{\em CL\_\-OK:}]The API executed successfully. \item[{\em CL\_\-COR\_\-ERR\_\-NULL\_\-PTR:}]On passing a NULL pointer. \item[{\em CL\_\-COR\_\-ERR\_\-INVALID\_\-PARAM:}]On passing an invalid parameter. \item[{\em CL\_\-COR\_\-TXN\_\-ERR\_\-ZERO\_\-JOBS:}]No jobs in the transaction.\end{description}
\end{Desc}
\begin{Desc}
\item[Description:]This function is used to get the first job in the transaction jobs when a walk is being performed.\end{Desc}
\begin{Desc}
\item[Library Name:]Cl\-Cor\-Client\end{Desc}
\begin{Desc}
\item[Related Function(s):]\hyperlink{pagecor514}{cl\-Cor\-Txn\-Next\-Job\-Get}, \hyperlink{pagecor515}{cl\-Cor\-Txn\-Previous\-Job\-Get}, 
\hyperlink{pagecor513}{cl\-Cor\-Txn\-Last\-Job\-Get} \end{Desc}
\newpage


\subsection{clCorTxnLastJobGet}
\index{clCorTxnLastJobGet@{clCorTxnLastJobGet}}
\hypertarget{pagecor513}{}\paragraph{cl\-Cor\-Txn\-Last\-Job\-Get}\label{pagecor513}
\begin{Desc}
\item[Synopsis:]Get the last job in the transaction.\end{Desc}
\begin{Desc}
\item[Header File:]clCorTxnApi.h\end{Desc}
\begin{Desc}
\item[Syntax:]

\footnotesize\begin{verbatim}       ClRcT  clCorTxnLastJobGet(
                          CL_IN     ClCorTxnIdT       txnId,
                          CL_OUT    ClCorTxnJobIdT   *pJobId);
\end{verbatim}
\normalsize
\end{Desc}
\begin{Desc}
\item[Parameters:]
\begin{description}
\item[{\em txn\-Id:}](in) Transaction ID . \item[{\em p\-Job\-Id:}](out) Pointer to the transaction Job ID .\end{description}
\end{Desc}
\begin{Desc}
\item[Return values:]
\begin{description}
\item[{\em CL\_\-OK:}]The API executed successfully. \item[{\em CL\_\-COR\_\-ERR\_\-NULL\_\-PTR:}]On passing a NULL pointer. \item[{\em CL\_\-COR\_\-ERR\_\-INVALID\_\-PARAM:}]On passing an invalid parameter. \item[{\em CL\_\-COR\_\-TXN\_\-ERR\_\-ZERO\_\-JOBS:}]No jobs in the transaction.\end{description}
\end{Desc}
\begin{Desc}
\item[Description:]This function is used to retrieve the last job in the transaction. The operation type for transaction job is {\em set\/}.\end{Desc}
\begin{Desc}
\item[Library Name:]Cl\-Cor\-Client\end{Desc}
\begin{Desc}
\item[Related Function(s):]\hyperlink{pagecor515}{cl\-Cor\-Txn\-Previous\-Job\-Get} \end{Desc}
\newpage



\subsection{clCorTxnNextJobGet}
\index{clCorTxnNextJobGet@{clCorTxnNextJobGet}}
\hypertarget{pagecor514}{}\paragraph{cl\-Cor\-Txn\-Next\-Job\-Get}\label{pagecor514}
\begin{Desc}
\item[Synopsis:]Get the next job in the transaction.\end{Desc}
\begin{Desc}
\item[Header File:]clCorTxnApi.h\end{Desc}
\begin{Desc}
\item[Syntax:]

\footnotesize\begin{verbatim}       ClRcT  clCorTxnNextJobGet(
                          CL_IN     ClCorTxnIdT       txnId,
                          CL_IN     ClCorTxnJobIdT    currentJobHdl,
                          CL_OUT    ClCorTxnJobIdT   *pNextJobHdl);
\end{verbatim}
\normalsize
\end{Desc}
\begin{Desc}
\item[Parameters:]
\begin{description}
\item[{\em txn\-Id}]:(in)  Transaction ID . \item[{\em current\-Job\-Hdl:}](in) Current transaction job handle. \item[{\em p\-Next\-Job\-Hdl:}](out) Pointer to the next transaction job handle.\end{description}
\end{Desc}
\begin{Desc}
\item[Return values:]
\begin{description}
\item[{\em CL\_\-OK:}]The API executed successfully. \item[{\em CL\_\-COR\_\-ERR\_\-NULL\_\-PTR:}]On passing a NULL pointer. \item[{\em CL\_\-COR\_\-ERR\_\-INVALID\_\-PARAM:}]On passing an invalid parameter. \item[{\em CL\_\-COR\_\-TXN\_\-ERR\_\-ZERO\_\-JOBS:}]No jobs in the transaction. \item[{\em CL\_\-COR\_\-TXN\_\-ERR\_\-LAST\_\-JOB:}]If the current job is the last job.\end{description}
\end{Desc}
\begin{Desc}
\item[Description:]This function is used to retrieve the next job queued in the transaction list. This function can be used after getting the first job in the transaction list.\end{Desc}
\begin{Desc}
\item[Library Name:]Cl\-Cor\-Client\end{Desc}
\begin{Desc}
\item[Related Function(s):]\hyperlink{pagecor512}{cl\-Cor\-Txn\-First\-Job\-Get} \end{Desc}
\newpage


\subsection{clCorTxnPreviousJobGet}
\index{clCorTxnPreviousJobGet@{clCorTxnPreviousJobGet}}
\hypertarget{pagecor515}{}\paragraph{cl\-Cor\-Txn\-Previous\-Job\-Get}\label{pagecor515}
\begin{Desc}
\item[Synopsis:]Get the previous job in the transaction.\end{Desc}
\begin{Desc}
\item[Header File:]clCorTxnApi.h\end{Desc}
\begin{Desc}
\item[Syntax:]

\footnotesize\begin{verbatim}     ClRcT  clCorTxnPreviousJobGet(
                        CL_IN     ClCorTxnIdT      txnId,
                        CL_IN     ClCorTxnJobIdT   currentJobId,
                        CL_OUT    ClCorTxnJobIdT  *pNextJobId);
\end{verbatim}
\normalsize
\end{Desc}
\begin{Desc}
\item[Parameters:]
\begin{description}
\item[{\em txn\-Id:}](in) Transaction ID . \item[{\em current\-Job\-Hdl:}](in) Current transaction job handle. \item[{\em p\-Next\-Job\-Hdl}]: (out) Pointer to the next transaction job handle.\end{description}
\end{Desc}
\begin{Desc}
\item[Return values:]
\begin{description}
\item[{\em CL\_\-OK:}]The API executed successfully. \item[{\em CL\_\-COR\_\-ERR\_\-NULL\_\-PTR:}]On passing a NULL pointer. \item[{\em CL\_\-COR\_\-ERR\_\-INVALID\_\-PARAM:}]On passing an invalid parameter. \item[{\em CL\_\-COR\_\-TXN\_\-ERR\_\-ZERO\_\-JOBS:}]No jobs in the transaction. \item[{\em CL\_\-COR\_\-TXN\_\-ERR\_\-FIRST\_\-JOB:}]If the current job is the first job.\end{description}
\end{Desc}
\begin{Desc}
\item[Description:]This function is used to get the previous job from the current job handle. This function must not be used while the current handle is pointing towards the first job in the transaction job list.\end{Desc}
\begin{Desc}
\item[Library Name:]Cl\-Cor\-Client\end{Desc}
\begin{Desc}
\item[Related Function(s):]\hyperlink{pagecor514}{cl\-Cor\-Txn\-Next\-Job\-Get} \end{Desc}
\newpage



\subsection{clCorTxnFailedJobGet}
\index{clCorTxnFailedJobGet@{clCorTxnFailedJobGet}}
\hypertarget{pagecor516}{}\paragraph{cl\-Cor\-Txn\-Failed\-Job\-Get}\label{pagecor516}
\begin{Desc}
\item[Synopsis:] Gets the information about the failed transaction job for a particular transaction ID.\end{Desc}
\begin{Desc}
\item[Header File:]clCorTxnApi.h\end{Desc}
\begin{Desc}
\item[Syntax:]

\footnotesize\begin{verbatim}     ClRcT clCorTxnFailedJobGet( 
					      CL_IN  ClCorTxnSessionIdT txnSessionId, 
			                      CL_IN  ClCorTxnInfoT *pPrevTxnInfo, 
			                      CL_OUT ClCorTxnInfoT *pNextTxnInfo)

\end{verbatim}
\normalsize
\end{Desc}
\begin{Desc}
\item[Parameters:]
\begin{description}
\item[{\em txn\-Session\-Id:}](in) The transaction session Id for which the transaction had failed.
\item[{\em p\-Prev\-Txn\-Info:}](in) This parameter if given as "NULL". Get the first transaction information about 
                                   the failed transaction in the second parameter else get the next
                                   entry with respect to it.
\item[{\em p\-Next\-Txn\-Info:}](out) The transaction information about the failed transaction is filled in this parameter.\end{description}
\end{Desc}
\begin{Desc}
\item[Return values:]
\begin{description}
\item[{\em CL\_\-OK:}]The API executed successfully. 
\item[{\em CL\_\-COR\_\-TXN\_\-ERR\_\-FAILED\_\-JOB\_\-GET:}]For the transaction Id there is no failed job info available.
\end{description}
\end{Desc}
\begin{Desc}
\item[Description:] Applications that initiate complex transaction to perform MO create/delete and attribute set operation, need
 to use this API to obtain  information regarding failures. Any component that participates in a transaction can
 report failures through their agent callback functions. This function retrieves all such failures reported.
 \par
 In case of multiple failures (on different components), there would be more than one  error
 entries added in COR for a particular transaction. All the error entries are obtained through the
 following mechanism: 
 \begin{enumerate}
 \item
 Obtain First Entry:  Call the function with the parameter \textit{pPrevTxnInfo} specified as NULL. This function
     would return the first error record in \textit{pNextTxnInfo} parameter.
 \item
 Obtain SubSequent Entries: Copy the  content of  \textit{pNextTxnInfo}, obtained in previous invocation of this function, 
 and assign it to \textit{pPrevTxnInfo} and call \textit{clCorTxnFailedJobGet}.
\end{enumerate}
\end{Desc}
\begin{Desc}
\item[Note:]
After getting all failed jobs, the api \textit{clCorTxnSessionCancel} should be called to free the memory.  
\end{Desc}
\begin{Desc}
\item[Library Name:]Cl\-Cor\-Client\end{Desc}
\begin{Desc}
\item[Related Function(s):]\hyperlink{pagecor503}{clCorTxnSessionCancel} \end{Desc}
\newpage


\subsection{clCorTxnJobStatusSet}
\index{clCorTxnJobStatusSet@{clCorTxnJobStatusSet}}
\hypertarget{pagecor517}{}\paragraph{cl\-Cor\-Txn\-Job\-Status\-Get}\label{pagecor517}
\begin{Desc}
\item[Synopsis:]Set the status of a particular job.\end{Desc}
\begin{Desc}
\item[Header File:]clCorTxnApi.h\end{Desc}
\begin{Desc}
\item[Syntax:]

\footnotesize\begin{verbatim}     ClRcT clCorTxnJobStatusSet(
		                        CL_IN   ClCorTxnIdT         txnId,
		                        CL_IN   ClCorTxnJobIdT      jobId,
		                        CL_IN   ClUint32            jobStatus);
\end{verbatim}
\normalsize
\end{Desc}
\begin{Desc}
\item[Parameters:]
\begin{description}
\item[{\em txn\-Id:}](in) Transaction ID . 
\item[{\em job\-Id:}](in) Job ID in the transaction.
\item[{\em Job\-Status:}](in) Status to be set. 
\end{description}
\end{Desc}
\begin{Desc}
\item[Return values:]
\begin{description}
\item[{\em CL\_\-OK:}]The API executed successfully. 
\item[{\em CL\_\-COR\_\-TXN\_\-ERR\_\-ZERO\_\-JOBS:}]No jobs in the transaction. 
\item[{\em CL\_\-COR\_\-TXN\_\-ERR\_\-INVALID\_\-JOB\_\-ID:}]The jobId passed is invalid. 
\end{description}
   This function is used to set the status of a particular job, given its txnId and jobId. The status will 
   be a 32-bit value.
\end{Desc}
\begin{Desc}
\item[Description:]

\end{Desc}
\begin{Desc}
\item[Library Name:]Cl\-Cor\-Client\end{Desc}
\begin{Desc}
\item[Related Function(s):]\hyperlink{pagecor519}{clCorTxnJobStatusGet} \end{Desc}
\newpage






\subsection{clCorTxnJobStatusGet}
\index{clCorTxnJobStatusGet@{clCorTxnJobStatusGet}}
\hypertarget{pagecor519}{}\paragraph{cl\-Cor\-Txn\-Job\-Status\-Get}\label{pagecor519}
\begin{Desc}
\item[Synopsis:]Get the status of a particular job.\end{Desc}
\begin{Desc}
\item[Header File:]clCorTxnApi.h\end{Desc}
\begin{Desc}
\item[Syntax:]

\footnotesize\begin{verbatim}     ClRcT clCorTxnJobStatusGet(
		                        CL_IN     ClCorTxnIdT     txnId,
		                        CL_IN     ClCorTxnJobIdT  jobId,
		                        CL_OUT    ClUint32T*      jobStatus);
\end{verbatim}
\normalsize
\end{Desc}
\begin{Desc}
\item[Parameters:]
\begin{description}
\item[{\em txn\-Id:}](in) Transaction ID . 
\item[{\em job\-Id:}](in) Job ID in the transaction.
\item[{\em Job\-Status:}](out) Status to be set. 
\end{description}
\end{Desc}
\begin{Desc}
\item[Return values:]
\begin{description}
\item[{\em CL\_\-OK:}]The API executed successfully. 
\item[{\em CL\_\-COR\_\-TXN\_\-ERR\_\-ZERO\_\-JOBS:}]No jobs in the transaction. 
\item[{\em CL\_\-COR\_\-TXN\_\-ERR\_\-INVALID\_\-JOB\_\-ID:}]The jobId passed is invalid. 
\end{description}
This function is used to get the status of a particular job, given its txnId and jobId. The status will 
be a 32-bit value.
\end{Desc}
\begin{Desc}
\item[Description:]

\end{Desc}
\begin{Desc}
\item[Library Name:]Cl\-Cor\-Client\end{Desc}
\begin{Desc}
\item[Related Function(s):]\hyperlink{pagecor517}{clCorTxnJobStatusSet} \end{Desc}
\newpage



\subsection{clCorTxnJobDefnHandleUpdate}
\index{clCorTxnJobDefnHandleUpdate@{clCorTxnJobDefnHandleUpdate}}
\hypertarget{pagecor520}{}\paragraph{cl\-Cor\-Txn\-Job\-Defn\-Handle\-Update}\label{pagecor520}
\begin{Desc}
\item[Synopsis:]This function is used to pack the transaction information and 
  update it in the given job definition handle.
 \end{Desc}
\begin{Desc}
\item[Header File:]clCorTxnApi.h\end{Desc}
\begin{Desc}
\item[Syntax:]

\footnotesize\begin{verbatim}    ClRcT clCorTxnJobDefnHandleUpdate(
		                        CL_OUT  ClTxnJobDefnHandleT  jobDefnHandle, 
		                        CL_IN     ClCorTxnIdT          corTxnId);
\end{verbatim}
\normalsize
\end{Desc}
\begin{Desc}
\item[Parameters:]
\begin{description}
\item[{\em job\-Defn\-Handle:}](out) Handle for the job definition. 
\item[{\em cor\-Txn\-Id:}](in) Transaction Id.
\end{description}
\end{Desc}
\begin{Desc}
\item[Return values:]
\begin{description}
\item[{\em CL\_\-OK:}]The API executed successfully. 
\end{description}
   This function is used to copy the updated transaction information into the job definition handle
   given in the agent callback functions. It will be called by the agents when the given job
   fails.
\end{Desc}
\begin{Desc}
\item[Description:]

\end{Desc}
\begin{Desc}
\item[Library Name:]Cl\-Cor\-Client\end{Desc}
\begin{Desc}
\item[Related Function(s):]None. \end{Desc}
\newpage

\chapter{Service Notifications}
\section{Logs}
\textit{IM creation failure. Configuration attribute [\textbf{CLASSID:ATTRID}] has incorrect qualifiers [\textbf{ATTRIBUTE-FLAGS}]}.
\begin{itemize}
\item
The configuration attribute is specified either as R/O,Non Cached, Non Persistent.
\item
This message is logged with ERROR severity.
\end{itemize}



\textit{IM creation failure. RunTIme  attribute [\textbf{CLASSID:ATTRID}]  has incorrect qualifiers [\textbf{ATTRIBUTE-FLAGS}]}.
\begin{itemize}
\item The Run time attribute is specified as R/W.
\item This message is logged with ERROR severity.
\end{itemize}

\textit{IM creation failure. Operational  attribute [\textbf{CLASSID:ATTRID}]  has incorrect qualifiers [\textbf{ATTRIBUTE-FLAGS}]}.
\begin{itemize}
\item	The Operational attribute is specified as R/O, Cached or Persistent.
\item	This message is logged with ERROR severity.
\end{itemize}



\textit{IM creation failure. Operational  attribute [\textbf{CLASSID:ATTRID}]  has incorrect qualifiers [\textbf{ATTRIBUTE-FLAGS}]}.
\begin{itemize}
\item
The Operational attribute is specified as R/O, Cached or Persistent.
\item
This message is logged with ERROR severity.
\end{itemize}


\textit{Failed while adding the OI [\textbf{Node-Address:Port-Id}] in the OI-List}.
\begin{itemize}
\item
	Failed while registering a readOI.
	\item
	This message is logged with ERROR severity.
	\end{itemize}
	
	
	\textit{MO [\textbf{Moid}]does  not have any Read-OI configured}.
	\begin{itemize}
	\item
	This message is logged with ERROR severity
	\end{itemize}



\textit{Preprocessing failed for the get operation. [\textbf{MOID}]}.
	\begin{itemize}
	\item
	Failed while validating the session jobs on the COR server.
	\item
	This message is logged with ERROR severity
	\end{itemize}
	
	
	
\textit{Attribute Id- [\textbf{ATTRID}] do not exist for ClassId[\textbf{CLASSID}]. rc]}.
	\begin{itemize}
	\item
	Failed while getting the attribute handle for the given attribute.
	\item
	This message is logged with ERROR severity.
\end{itemize}


\chapter{Debug CLI}
\section{COR}
The debug CLI for COR has been enhanced in-order to test the functionality of bulk get. The session initialization, job en-queue, apply and 
finalize commands have been added. The command wise description is given below:


\subsection{clCorBundleInitialize}
\begin{Desc}
\item
[Parameters:]
\begin{description}
\item[{\em 1:}] For transactional bundle.
\item[{\em 2:}] For non-transactional bundle.

\end{Desc}
\begin{Desc}
\item
[Description:]
This command creates a session and returns a session handle. This handle is unique throughout a given session. It should be used to add
jobs and to get all the values associated with the job. Using debug CLI only one session can be created at one time. 
\end{Desc}
\begin{Desc}
\item
[Displayed Values:]
\begin{itemize}
\item
In case of successful session initialization the session Id will be displayed.
\item
In case of failure the corresponding error code will be displayed. The error codes will be based on the error retuned by 
clCorReadSessionInitialize API.
\end{itemize}
\end{Desc}

\subsection{clCorReadSessionJobEnqueue}
\begin{Desc}
\item
[Parameters:]
\begin{description}
\item[{\em session\-Handle:}] Session Handle Obtained from the clCorReadSessionInitialize CLI.
\item[{\em Mo\-ID:}] MoID for the object. Ex: \\Chassis:0\\SysController:0 or \\0x10001:0\\0x10002:0 
\item[{\em Svc\-Id:}] Service Id of the MSO. SvcId should be 3 for PROV MSO and 2 for alarm MSO.
\item[{\em attr\-Path:}] Attribute Path should be NULL for PROV MSOs.
\item[{\em Attr\-Id:}] Attribute Id of the attribute for which get is performed.
\end{description}
\end{Desc}
\begin{Desc}
\item
[Description:] 
This command should be used to en-queue all the jobs for which get operation have to be done. 
After enqueueing the jobs, clCorReadSessionApply needs to be called.
\end{Desc}	
\begin{Desc}
\item
[Output:] 
\begin{itemize}
\item
In case of successful enqueue no message will be printed.
\item
In case of failure an appropriate error code will be displayed. The error codes will be based on the error retuned by 
clCorReadSessionJobEnqueue API.
\end{itemize}
\end{Desc}


\subsection{clCorReadSessionApply}
\begin{Desc}
\item
[Parameters:] 
\begin{description}
\item[{\em session\-Handle:}] Handle to the session obtained from clCorReadSessionInitalize CLI.
\end{description}
\end{Desc}
\begin{Desc}
\item
[Description:]
This command should be used to apply the session to obtain  values for job in this session.
This command is synchronous, i.e upon return of the command the job-values and its job-status will be printed.
\end{Desc}
\begin{Desc}
\item
[Output:]
\begin{itemize}
\item
The data obtained is displayed on the console in the form of attribute id and value. If it is an array attribute then it would display value for all 
of its indexes. 
\item
If there was any error while processing a job, then it would display the moId and attribute id of that job along with the error code. This error
code would take the value of  readJobStatus parameter of clCorReadSessionJobEnqueue.
\end{itemize}
\end{Desc}



\subsection{clCorReadSessionFinalize}
\begin{Desc}
\item
[Parameters:] 
\begin{description}
\item[{\em session\-Handle:}] Session Handle obtained through calling clCorReadSessionInitialize CLI.
\end{description}
\end{Desc}
\begin{Desc}
\item
[Description:] 
This command should be used to finalize a session. Upon finalization the sessionHandle is no more valid to be used.
\end{Desc}
\begin{Desc}
\item
[Output:] 
\begin{itemize}
\item In case of  success no error will be output.
\item In case of failure an appropriate error code will be displayed. The error codes will be based on the error retuned by 
clCorReadSessionFinalize API.
\end{itemize}
\end{Desc}


\subsection{objectShow}
\begin{Desc}
\item
[Parameters:] 
\begin{description}
\item[{\em Mo\-ID:}] MoID of the object. { Ex: \\Chassis:0\\SysController:0 or \\0x10001:0\\0x10002:0 }
\item[{\em Svc\-Id:}] Service Id of the MSO or -1 if it is a MO.
\item[{\em attr\-Path:}] Attribute Path of the MO if is a contained MO. Ex: 
	\\attr1:0\\attr2:2 or \\0x1001:0\\0x1002:1 otherwise it is NULL.
\end{description}
\end{Desc}
\begin{Desc}
\item
[Description:]
The command has been upgraded to show the value of  RunTime, Configuration and Operation attributes.
\end{Desc}
\begin{Desc}
\item
[Output:]
\begin{itemize}
\item	Displays the value of all the attributes.
\item Error message will be printed for attributes whose read has failed. The error code in these messages will correspond to  
readSessionJobStatus parameter of clCorReadSessionJobEnqueue.
\end{itemize}
\end{Desc}




\subsection{dmShow}
\begin{Desc}
\item
[Parameters:]
\begin{description} 
\item[{\em Class\-Id:}]Class Identifier.
\end{description}
\end{Desc}
\begin{Desc}
\item
[Description:]
This command when supplied with the class Id will show the details of  
Class  and its attributes. If no argument is specified it will show all    
classes that are part of COR.
\par
\par
For the classId specified following attributes will be displayed attribute Id, offset, size , type and user flags. The user flags 
that are displayed for each attribute will have following combination of characters to give detail about its type:
\begin{itemize}
\item
CF 	- Configuration attribute.
\item
RT	- Runtime attribute.
\item
OP	- Operational attribute.
\item
R-O	- Read only attribute 
\item
R-W	- Read Write attribute..
\item
C\$	- Cached attribute. 
\item
N-C\$	- Non cached attribute. 
\item
PERS	- Persistent attribute.
\item
N-PERS - Non-persistent attribute.
\item
WRONCE - Writable
\item
INITED -  Initialized
\end{itemize}
\end{Desc}





\chapter*{Glossary}
Job: A job is an abstract entity that represents an MO or an MO+attrId. For e.g. when a read performed on a job, the job is associated with
MO+attriID. When a creation is performed on an MO, a job is associated with MO. 
\end{flushleft}